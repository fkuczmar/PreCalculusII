\documentclass{ximera}
\title{Right Triangle Trigonometry, Part 2}

\newcommand{\pskip}{\vskip 0.1 in}

\begin{document}
\begin{abstract}
More right triangle tirg.
\end{abstract}
\maketitle


\section{Describing Position on the Earth}

The \emph{latitude} of a point $P$ on the surface of the earth is the angle the radius $OP$ from the earth's center to $P$ makes with the plane of the equator. Latitude, often represented by the Greek letter $\phi$ (phi), lies between $-\pi/2$ (at the south pole) and $\pi/2$ at the north pole. It is positive in the northern hemisphere, negative in the southern hemisphere.

Click on the circular icon to the left of Line 2 in the worksheet below to hide the earth. The two marked angles measure the latitude of the blue point (unlabeled) on the sphere.

\begin{onlineOnly}
    \begin{center}
\desmosThreeD{1zzigifhrx}{800}{600}         %zk06s3k6q4
\end{center}
\end{onlineOnly}

\href{https://www.desmos.com/3d/1zzigifhrx}{142: Circle of Latitude}


\begin{question} \label{Qfksadfsdt4e4}
\begin{enumerate}
\item How fast are we moving due to the rotation of the earth about its axis? Assume Shoreline is at a latitude of $47.75^\circ$N and take the radius of the earth to be $4000$ miles. Find the exact speed \emph{without} a calculator. Then use a calculator to approximate the speed to the nearest mile/hour.

{\bf Do not rely on a formula.} Use logic and common sense instead. You will need to know the definition of $\pi$. Can you remember this from the first day of class?

\item At what latitude(s) are people moving at a speed of $200$ miles/hour due to the rotation of the earth about its axis? Find the exact latitude(s). Then use a calculator to approximate their radian and degree measures.

Start by defining your unknown. Then write an equation and solve for the unknown.
\end{enumerate}
\end{question}

\begin{question}  \label{QODfefEREr}
Assume for this problem that Shoreline and Spokane are both at a latitude of $47.75^\circ$N. Suppose also that the cities have respective longitudes $122.3^\circ$W and $117.4^\circ$W. 

Click the camera icon at the left of Line 1 to hide the sphere.

\begin{onlineOnly}
    \begin{center}
\desmosThreeD{rjb4gcf8lo}{800}{600}         %zk06s3k6q4
\end{center}
\end{onlineOnly}

\href{https://www.desmos.com/3d/rjb4gcf8lo}{142: Seattle to Spokane}


\begin{enumerate}
\item Find the distance between the two cities measured along their common circle of latitude.

\item Compare this distance with the driving distance along I-90, about $270$ miles.

\end{enumerate}

\end{question}


\section{Other Questions}

\begin{question} \label{QdfRERER}
You wrap a rubber band around a jar lid of radius $R$ cm and stretch the band as shown below.

Express the length (measured in cm) of the band in terms of $R$ and the distance $h$ between the point $P$ and the center of the lid, also measured in cm.


\begin{onlineOnly}
    \begin{center}
\desmos{wz5qiyj4od}{800}{600}         %zk06s3k6q4
\end{center}
\end{onlineOnly}

\href{https://www.desmos.com/calculator/wz5qiyj4od}{142: Jar Lid}

\end{question}


\begin{question} \label{QddtgHHERER}
Sensors on the ground at points $A$ and $B$ below measure the angles of elevation to a plane to be $\theta_1$ and $\theta_2$, respectively, as shown below.

Express the altitdue of the plane (in km) in terms of the distance $s$ between the sensors (also measured in km).

\begin{onlineOnly}
    \begin{center}
\desmos{zbktegkke4}{800}{600}         %zk06s3k6q4
\end{center}
\end{onlineOnly}

\href{https://www.desmos.com/calculator/zbktegkke4}{142: Helicopter}




\end{question}


\end{document}