\documentclass{ximera}
\title{Quiz 10B}

\newcommand{\pskip}{\vskip 0.1 in}

\begin{document}
\begin{abstract}
Right triangle trig, the laws of sines and cosines, other topics.
\end{abstract}
\maketitle


\begin{question} \label{Q898dfbfdgdsg}
In $\Delta MAT$, angle $\angle MAT$ has measure $\pi/3$ radians. Sides $AM$ and $AT$ have respective lengths $2$ and $5$ inches. Determine the exact measure of angle $\angle AMT$. Draw a reasonably accurate picture. Do \emph{not} use a calculator.
\end{question}

\begin{question} \label{QLMVVEVEeddf}
A tree leans precariously with its trunk inclined at an angle of $\pi/3$ radians to the ground. The top of a seven-foot ladder rests against the tree and the bottom of the ladder lies eight feet from the tree's base. How far is the top of the ladder from the base of the tree? Do not use a calculator except for arithmetic. Use the worksheet below to help visualize the possibilities, but do \emph{not} use it to answer the question.

\begin{onlineOnly}
    \begin{center}
\desmos{sjmjseyqyp}{900}{600}
\end{center}
\end{onlineOnly}

Desmos activity available at \href{https://www.desmos.com/calculator/sjmjseyqyp}{142: Ladder and Tree 45}


\end{question}


\begin{question}  \label{QKLKLddfgkghg}
Suppose for this problem that the earth is a ball with uniform density of radius 4000 miles. Now imagine drilling a straight tunnel through the earth from the North Pole to the South Pole. 

A rock dropped from rest at the north pole falling through the tunnel would then oscillate in simple harmonic motion between the poles and return to the north pole every $84$ minutes. This means we can think of the rock as being dragged along by a point moving around the earth at constant speed as illustrated below. 

\begin{onlineOnly}
    \begin{center}
\desmos{ij8dqowgza}{900}{600}
\end{center}
\end{onlineOnly}

Desmos activity available at \href{https://www.desmos.com/calculator/ij8dqowgza}{142: Simple Harmonic Motion}

\begin{enumerate}

\item Activate the \emph{Protractor} folder in Line 15 to see the protractor, where consecutive tick marks subtend equal angles of $\pi/100$ radians about the earth's center. Use the protractor and circular interpolation, but \emph{not} any trigonometry to estimate the first two times the rock is $1000$ miles from the South Pole. Measure the times in minutes since the rock was released. 

%\begin{enumerate}
%\item Estimate the distance of the rock from the South Pole at time $t=16.8$ minutes after the rock is released.

%\item Estimate the first two times when the rock is $1000$ miles from the South Pole.
%\end{enumerate} 

\item Sketch by hand one period of the graph of the function
\[
 s = f(t) \, \, \, t\geq 0 ,
\]
that expresses the distance of the rock (measured in thousands of miles) from the South Pole in terms of the number of minutes since the rock was released. %Assume the rock is dropped at midnight on July 1, 2085.

\item Find an expression for the function $s=f(t)$. Start by expressing the distance $s$ in terms of the polar angle $\theta$ marked above.

%\item Use your function to determine the exact distance between the rock and the South Pole at time $t=16.8$ minutes past noon. Then use a calculator to approximate this distance to the nearest mile and compare it with your estimate in part (i).

\item Use  your function to find the first two times when the rock is $1000$ miles from the South Pole. Do this by first finding the first two positive polar angles when the rock is $1000$ miles from the South Pole. Give exact values without using a calculator. Then use a calculator to approximte these times to the nearest tenth of a minute. Compare these times with your estimates from part (a).

%\item Find the (exact) first two times when the rock is $1000$ miles from the South Pole. Then use a calculator to estimate these clock times to the nearest minute

%\item Find all (exact) times between 8pm and 12pm, July 1, 2085 when the rock is $1000$ miles from the South Pole. Then use a calculator to estimate these clock times to the nearest minute

\end{enumerate}
\end{question}




\begin{question} \label{QPPLFKEbd}
You measure the angle of elevation to the top of a tree to be $\theta_1$ radians. You then walk an additional $c$ feet directly away from the tree and measure the angle of elevation to be $\theta_2$ radians. 
\begin{enumerate}
\item Express the height of the tree above eye level in terms of $\theta_1$, $\theta_2$, and $c$.

\item Check that your expression has the correct units.
\end{enumerate} 
\end{question}



\end{document}