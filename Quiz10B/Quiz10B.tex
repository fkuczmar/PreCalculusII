\documentclass{ximera}
\title{Quiz 10B}

\newcommand{\pskip}{\vskip 0.1 in}

\begin{document}
\begin{abstract}
Right triangle trig, the laws of sines and cosines, other topics.
\end{abstract}
\maketitle


\begin{question} \label{QLMVVEVEeddf}
A tree leans precariously with its trunk inclined at an angle of $\phi=\arccos(3/4)$ radians to the ground. The top of a seven-foot ladder rests against the tree and the bottom of the ladder lies eight feet from the tree's base. %How far is the top of the ladder from the base of the tree? %Do not use a calculator except for arithmetic. Use the worksheet below to help visualize the possibilities, but do \emph{not} use it to answer the question.

\begin{onlineOnly}
    \begin{center}
\desmos{8bbcfyrita}{900}{600}
\end{center}
\end{onlineOnly}

Desmos activity available at \href{https://www.desmos.com/calculator/8bbcfyrita}{142: Ladder and Tree 47}

\begin{enumerate}
\item Drag the slider in Line 2 of the worksheet above to approximate the distance from the top of the ladder to the base of the tree when the bottom of the ladder is eight feet from the tree's base. Is there more than one possibility?

\item Use the law of cosines to find the exact distance(s) in part (a). Begin by defining an unknown, with units. Do not use a calculator or the quadratic formula. Complete the square instead.

\item Use a calculator (or desmos) to approximate the distances in part (b) to the nearest tenth of a foot. Compare these with your estimates.
\end{enumerate}
\end{question}

\begin{question}  \label{Q34dfefGfd}
At noon a rowboat is $10$ km due east of a sailboat. The sailboat travels at a consant speed of $v$ km/hour at a fixed bearing of $0.4$ radians (measured counterclockwise from the east). The rowboat travels half as fast as the sailboat at some fixed bearing. Sometime later the boats collide.

\pdfOnly{
Access Desmos interactives through the online version of this text at
 
\href{https://www.desmos.com/calculator/xjiwflhe2o}.
}
 
\begin{onlineOnly}
    \begin{center}
\desmos{muqgmov9lu}{900}{600}
\end{center}
\end{onlineOnly}

Access Desmos interactive at

\href{https://www.desmos.com/calculator/muqgmov9lu}{142: Sailboat and Rowboat 45}

\begin{enumerate}
\item Drag the slider $\phi$ (the rowboat's bearing) in Line 2 in the worksheet above to approximamte the possible bearing(s) of the rowboat.

\item Find the exact bearing(s) of the rowboat.

\item Use a calcluator to approximate the bearings to the nearest tenth of a radian and compare these with your estimates.

%(b) When do the boats collide?

%(c) Establish a rectangular coordinate system with the positive $x$-axis pointing due east, and the origin at the sailboat's position at noon. Then parameterize the motions of the rowboat and sailboat. Consider all possibilities. Include domains and do \emph{not} use inverse trigonometric functions in the parameterizations.

%(d) Follow the directions in the Desmos acitivity below to check your work.
\end{enumerate}

\end{question}

\begin{question} \label{QOdejgERE34rd}

Point $P$ moves counterclockwise around a circle of radius $4$ meters centered at the origin $O$ at a constant speed of $8$ m/sec, passing the point $(4,0)$ at noon.


One end of a rod $\overline{QP}$ of length $10$ meters is attached to $P$, while the other end $Q$ slides along the $x$-axis as shown below. 
 
\href{https://www.desmos.com/calculator/dra9jtsynq}{142: Polar Angle UCM}.

 
\begin{onlineOnly}
    \begin{center}
\desmos{dra9jtsynq}{900}{600}
\end{center}
\end{onlineOnly}


\begin{enumerate}

\item Use the animation to sketch by hand a graph of the function
\[
    \theta = a(t) \, , \, t\geq 0
\]
that expresses the polar angle (as marked) of the vector $\overrightarrow{QP}$ in terms of the number of seconds past noon.

\item Find an expression for the function $\theta = a(t)$.

\end{enumerate}
\end{question}



\begin{question} \label{Q9eergFDhgL}
The minute and hour hands of a clock have respective lengths $12$cm and $8$cm. 

\begin{onlineOnly}
    \begin{center}
\desmos{sl0k5pvb6h}{900}{600}
\end{center}
\end{onlineOnly}

Desmos activity available at \href{https://www.desmos.com/calculator/sl0k5pvb6h}{142: Hands of a Clock 4}

\begin{enumerate}
\item Drag the slider $u$ in Line 2 above and then sketch by hand a graph of the function
\[
    c = f(t) \, , t\geq 0 ,
\]
that expresses the distance (in cm) between the tips of the hands in terms of the number of minutes past noon.

\item Activate the folder in Line 3 to see the graph of the function $f$. How did you do?

\item Find an expression for the function $f$.

\item Use the graph of the function $f$ above to approximate the first two times after 12:00pm when the distance between the tips of the hands is $15$cm.

\item Find the exact times in part (d). Do \emph{not} use a calculator.

\item Use a calculator to approximate the times in part (e) to the nearest minute. Compare these with your estimates.
\end{enumerate}
\end{question}





\begin{question} \label{Q898dfbfdgdsg}
In $\Delta MAT$, angle $\angle MAT$ has measure $\pi/3$ radians. Sides $AM$ and $AT$ have respective lengths $2$ and $5$ inches. Determine the exact measure of angle $\angle AMT$. Draw a reasonably accurate picture. Do \emph{not} use a calculator.
\end{question}


\begin{question}  \label{QKLKLddfgkghg}
Suppose for this problem that the earth is a ball with uniform density of radius 4000 miles. Now imagine drilling a straight tunnel through the earth from the North Pole to the South Pole. 

A rock dropped from rest at the north pole falling through the tunnel would then oscillate in simple harmonic motion between the poles and return to the north pole every $84$ minutes. This means we can think of the rock as being dragged along by a point moving around the earth at constant speed as illustrated below. 

\begin{onlineOnly}
    \begin{center}
\desmos{ij8dqowgza}{900}{600}
\end{center}
\end{onlineOnly}

Desmos activity available at \href{https://www.desmos.com/calculator/ij8dqowgza}{142: Simple Harmonic Motion}

\begin{enumerate}

\item Activate the \emph{Protractor} folder in Line 15 to see the protractor, where consecutive tick marks subtend equal angles of $\pi/100$ radians about the earth's center. Use the protractor and circular interpolation, but \emph{not} any trigonometry to estimate the first two times the rock is $1000$ miles from the South Pole. Measure the times in minutes since the rock was released. 

%\begin{enumerate}
%\item Estimate the distance of the rock from the South Pole at time $t=16.8$ minutes after the rock is released.

%\item Estimate the first two times when the rock is $1000$ miles from the South Pole.
%\end{enumerate} 

\item Sketch by hand one period of the graph of the function
\[
 s = f(t) \, \, \, t\geq 0 ,
\]
that expresses the distance of the rock (measured in thousands of miles) from the South Pole in terms of the number of minutes since the rock was released. %Assume the rock is dropped at midnight on July 1, 2085.

\item Find an expression for the function $s=f(t)$. Start by expressing the distance $s$ in terms of the polar angle $\theta$ marked above.

%\item Use your function to determine the exact distance between the rock and the South Pole at time $t=16.8$ minutes past noon. Then use a calculator to approximate this distance to the nearest mile and compare it with your estimate in part (i).

\item Use  your function to find the first two times when the rock is $1000$ miles from the South Pole. Do this by first finding the first two positive polar angles when the rock is $1000$ miles from the South Pole. Give exact values without using a calculator. Then use a calculator to approximte these times to the nearest tenth of a minute. Compare these times with your estimates from part (a).

%\item Find the (exact) first two times when the rock is $1000$ miles from the South Pole. Then use a calculator to estimate these clock times to the nearest minute

%\item Find all (exact) times between 8pm and 12pm, July 1, 2085 when the rock is $1000$ miles from the South Pole. Then use a calculator to estimate these clock times to the nearest minute

\end{enumerate}
\end{question}




\begin{question} \label{QPPLFKEbd}
You measure the angle of elevation to the top of a tree to be $\theta_1$ radians. You then walk an additional $c$ feet directly away from the tree and measure the angle of elevation to be $\theta_2$ radians. 
\begin{enumerate}
\item Express the height of the tree above eye level in terms of $\theta_1$, $\theta_2$, and $c$.

\item Check that your expression has the correct units.
\end{enumerate} 
\end{question}



\end{document}