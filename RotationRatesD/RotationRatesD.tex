\documentclass{ximera}
\title{Rotation Rates Discussion Questions}

\newcommand{\pskip}{\vskip 0.1 in}

\begin{document}
\begin{abstract}
Measuring rates of rotation.
\end{abstract}
\maketitle






%\section{Discussion Questions}

\begin{question} \label{Qldsfsda4trgfvbvb}
You jog around a circular track of radius $40$ feet at the constant speed of $8$ ft/sec. As your friend jogs around a circular track of radius $50$ feet, she turns about the track's center at the constant rate of $0.17$ rad/sec.

\begin{enumerate}
\item Who is jogging faster?

\item Who is turning faster about the center of their track? 

\end{enumerate}
\end{question}

\begin{question} \label{Qodfdstr43}
An figure skater rotates at the constant rate of $10$ rad/sec. 
\begin{enumerate}
\item How long does it take her to turn through an angle of $30$ radians?

\item How long does it take her to turn turn through one revolution?

\end{enumerate}
\end{question}

\begin{question} \label{Q6721:Angles}
Kryptonites living at a latitude of $20^\circ$N move at a speed of $2000$ miles/hour due to the rotation of the planet about its axis.

Use the radian protractor below to help find approximate answers to the following questions.

\begin{enumerate}
\item  At what speed do the Kryptonites living at a latitude of $40^\circ$S move due to the rotation of the planet about its axis?
\begin{hint}
Start by converting the angle $40^\circ$ to radians.
\end{hint}

\item At what latitude(s) is the speed $500$ miles/hour?
\end{enumerate}


\begin{onlineOnly}
    \begin{center}
\desmos{fmjbqszyge}{900}{600}
\end{center}
\end{onlineOnly}

\href{https://www.desmos.com/calculator/fmjbqszyge}{142: Ships 22}
\end{question}



\begin{question} \label{QOLfdfdthnhgfg}

\begin{enumerate}
\item You and your friend jog in opposite directions around a circular track at the respective speeds of $3$ m/sec and $5$ m/s. You run in the inner lane, a circle with radius $40$ meters, while your friend runs in the outer lane, a circle with radius $44$ meters. Play the slider $u$ (time, measured in seconds) in the animation below to watch the motions.

\begin{onlineOnly}
\begin{center}
\desmos{a1jmg3x2mz}{900}{600}
\end{center}
\end{onlineOnly}

\href{https://www.desmos.com/calculator/a1jmg3x2mz}{142: Concentric Circles 2}

\begin{enumerate}
\item At what rate does the angle $\angle OAF$ subtended at the track's center by you and your friend increase?

\item How often do you and your friend pass each other?

\end{enumerate}


\item Now suppose you and your friend jog in the same direction around a circular track at the respective speeds of $3$ m/sec and $5$ m/s. You run in the inner lane, a circle with radius $40$ meters, while your friend runs in the outer lane, a circle with radius $44$ meters. Play the slider $u$ (time, measured in seconds) in the animation below to watch the motions.

\begin{onlineOnly}
\begin{center}
\desmos{s8ua80pw91}{900}{600}
\end{center}
\end{onlineOnly}

\href{https://www.desmos.com/calculator/s8ua80pw91}{142: Concentric Circles}

\begin{enumerate}
\item At what rate does the angle $\angle OAF$ subtended at the track's center by you and your friend increase?

\item How often do you and your friend pass each other?

\end{enumerate}
\end{enumerate}
\end{question}

\begin{question} \label{Q9df9sttr4rt}
\begin{enumerate}
\item You jog around a circlular track with radius $25$ meters while turning about the track's center at a constant rate of $0.3$ radians/sec. What is your speed? Explain your logic. Include units for all numbers in your calculation(s). 

\item Your friend jogs around the same track while turning about the track's center at a constant rate of $20^\circ$/sec. What is her speed?

\end{enumerate}
\end{question}



\begin{question}  \label{Q3245rghh5ty5t}
\begin{enumerate}

\item Play the slider $u$ in Line 1 of the worksheet below. Then \emph{without} stopping the motion, approximate the rate (in rad/sec) at which $P$ rotates about $O$. The clock shows the motion of a second hand in real time.

\begin{onlineOnly}
\begin{center}
\desmos{9on2xpcvta}{900}{600}
\end{center}
\end{onlineOnly}

\href{https://www.desmos.com/calculator/9on2xpcvta}{142: Rotation Rates 5}

\item Turn off the folder \emph{Point P Motion} in Line 19 by clicking off the camera tab at the left of the line. Then turn on the camera tab on Line 23 (Point Q Motion). Play the slider $u$ and approximate the rate (in rad/sec) at which $Q$ rotates about $O$ without stopping the slider.

\item Suppose the distance $OQ$ is $5$ cm and approximate the speed of $Q$ as it rotates about $A$.

\item Activate the folder \emph{Point R Motion} on Line 36. Find the exact speed of $R$ if the distance $AR$ is $3$ cm. 

\end{enumerate}

\end{question}


\begin{question} \label{QLkfdktrtrgvb}
Use the radian protractor below to help find approximate answers to the following question. Do \emph{not} use any trigonometry you might happen to know. Think proportionately instead.

\begin{enumerate}
\item At what speed are people on the earth at a latitude of $50^\circ$N moving due to the rotation of the earth about its axis? Take the earth to be a perfect sphere of radius $4000$ miles.

\item At what latitude in the northern hemisphere of the planet Krytpon are the Kryptonites moving twice as fast due to the rotation of the planet about its axis as they are at a latitude of $70^\circ$N?

\end{enumerate}


\begin{onlineOnly}
\begin{center}
\desmos{kjzez98bqn}{900}{600}
\end{center}
\end{onlineOnly}

\href{https://www.desmos.com/calculator/kjzez98bqn}{142: Radian Protractor 29}

\end{question}




\begin{question}  \label{QOFIGDIDeer3gg}
Assume for this problem that the planets in our solar system revolve about the sun in coplanar circular orbits, all with the same sense of rotation.

Then Kepler's third law states that the product $\omega^2 r^3$ is equal to a constant, where $\omega$ is the rotation rate of a planet about the sun and $r$ its orbital radius.

\begin{enumerate}
\item Find a function $P=f(r)$ that expresses the synodic period (measured in earth days) of a planet in terms of its orbital radius $r$ (measured in astronomical units.) Note that the earth has orbital radius $1$ AU.

The \emph{synodic period} is the time it takes for the planet and the earth to pass each other in their orbits.

\item Find the synodic period of Mars (with an orbtital radius of 1.524 AU).
\end{enumerate}

\end{question}








\end{document}
