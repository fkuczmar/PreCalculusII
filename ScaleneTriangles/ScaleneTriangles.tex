\documentclass{ximera}
\title{The Laws of Cosines and Sines}

\newcommand{\pskip}{\vskip 0.1 in}

\begin{document}
\begin{abstract}
The law of sines, the law of cosines.
\end{abstract}
\maketitle


\section{The Law of Sines and Relative Position}
We can use the law of sines to solve a new type of position problem.

\begin{example}  \label{Exer748gtgb}
Point $B$ is $6$ miles from point $A$ at a bearing of $0.3$ radians. Point $C$ lies at a bearing of $2.7$ radians relative to $B$ and at a bearing of $2.2$ radians relative to $A$. We wish to describe the positions of $C$ relative to $A$ and $B$.


(a) Sketch an accurate picture with points $A$, $B$, $C$ and the three vectors that describe the positions of $B$ relative to $A$, $C$ relative to $A$, and $C$ relative to $B$.

(b) Determine the exact distances from point $C$ to points $A$ and $B$.

\begin{explanation}
We first use the first demonstration below to draw an accurate picture. We begin by plotting point $A$ anywhere. We chose the origin. Then we center the radian protractor at $A$ with $r=6$ and plot point $B$.

To plot point $C$ is a bit trickier. The problem is that we do not know how far $C$ is from either $A$ or $B$. But we do know the bearings. So we center the protractor at $B$ and move point $B_1$ to make the ray $\overrightarrow{BB_1}$ have a bearing of $2.7$ radians (not shown). Then we center the protractor at $A$ and move $A_1$ to make the ray $\overrightarrow{AA_1}$ have a bearing of $2.2$ radians as shown below. The point where the rays $\overrightarrow{AA_1}$ and $\overrightarrow{BB_1}$ intersect tells us where to plot point $C$.  Then we drag $C$ to this position as shown in the second demonstration.

\pdfOnly{
Access Geogebra interactives through the online version of this text at
 
\href{https://www.geogebra.org/classic/sn2cfjc4}.
}
 
\begin{onlineOnly}
    \begin{center}
\geogebra{sn2cfjc4}{900}{600}
\end{center}
\end{onlineOnly}

Geogebra activity available at

\href{https://www.geogebra.org/classic/sn2cfjc4}{142: Relative Position 5}


We next calculate the angles of $\Delta ABC$. 

Angle $\angle BAC$ is the angle from the vector $\overrightarrow{AB}$ to $\overrightarrow{AC}$. So it has a radian measure 
\[
   \angle BAC = \text{polar angle of }\overrightarrow{AC} - \text{polar angle of }\overrightarrow{AB} = 2.2 - 0.3 = 1.9 .
\]


Angle $\angle BCA$ is the angle from the vector $\overrightarrow{AC}$ to the vector $\overrightarrow{BC}$. So it has radian measure
\[
   \angle BCA = \text{polar angle of }\overrightarrow{BC} - \text{polar angle of }\overrightarrow{AC} = 2.7 - 2.2 = 0.5 .
\]



%Finding the measure of $\angle ABC$ is a bit trickier. The angle $\angle ABC$ is the angle between the vectors $\overrightarrow{BA}$ and $\overrightarrow{BC}$. Since $\overrightarrow{BA}$ points in the opposite direction of $\overrightarrow{AB}$, the vector $\overrightarrow{BA}$ has polar angle (bearing)
%\[
%  \theta_{BA} = \pi + \text{polar angle of }\overrightarrow{AB}  = \pi + 0.3.
%\]
%So the angle $\angle ABC$ has radian meausure
%\[
%    \angle ABC = \text{polar angle of }\overrightarrow{BA} - \text{polar angle of }\overrightarrow{BC} = \pi + 0.3 - 2.7 = \pi -  2.4.
%\]

Finally, since the angles of $\Delta ABC$ sum to $\pi$, the angle $\angle ABC$ has radian measure
\[
  \angle ABC = \pi - (1.9 + 0.5) = \pi -2.4.
\]


\pdfOnly{
Access Geogebra interactives through the online version of this text at
 
\href{https://www.geogebra.org/classic/rbkfbjau}.
}
 
\begin{onlineOnly}
    \begin{center}
\geogebra{rbkfbjau}{900}{600}
\end{center}
\end{onlineOnly}

Geogebra activity available at

\href{https://www.geogebra.org/classic/rbkfbjau}{142: Relative Position 5b}



Now we use the law of sines in $\Delta ABC$. Since
\[
   \frac{|\overrightarrow{AB}|}{\sin (\angle BCA)} =\frac{|\overrightarrow{AC}|}{\sin (\angle ABC)} ,
\]
\[
     \frac{6}{\sin (0.5)} =\frac{|\overrightarrow{AC}|}{\sin (\pi - 2.4)} .
\]
So the the distance between $C$ and $A$ is
\[
   |\overrightarrow{AC}| = 6  \left( \frac{\sin (2.4)}{\sin (0.5)} \right)  \sim 8.45 \text{ miles}.
\]

\begin{question}  \label{Q34rg4t5rt}
Explain why we can replace $\sin (\pi -  2.4)$ with $\sin (2.4)$ in the computation above. Include a picture to help with your explanation.
\end{question}

\begin{question}   \label{Qd45r5643rs}
Find the exact distance between $C$ and $B$. Then find the approximate distance to the nearest hundredth of a mile.
\end{question}


\begin{question}  \label{Qerdef6645yu}
Explain how this problem differs from the other relative-position problems this quarter, not in terms of the solution, but in terms of the statement of the problem itself.
\end{question}


%Similarly, since
%\[
%   \frac{|\overrightarrow{AB}|}{\sin (\angle BCA)} =\frac{|\overrightarrow{BC}|}{\sin (\angle BAC)} ,
%\]
%\[
%     \frac{6}{\sin (0.5)} =\frac{|\overrightarrow{BC}|}{\sin (1.9)},
%\]
%and
%\[
%   |\overrightarrow{BC}| = 6  \left( \frac{\sin (1.9)}{\sin (0.5)} \right)  \sim 11.84 .
%\]

%So the approximate distances from $C$ to $A$ and from $C$ to $B$ are respectively $8.45$ miles and $11.84$ miles.

\end{explanation}


\end{example}


\section{More Applications of the Law of Sines}
\begin{question}  \label{Q343f44fd}
At noon a rowboat is $10$ km due east of a sailboat. The sailboat travels at a consant speed of $v$ km/hour at a fixed bearing of $0.4$ radians (measured counterclockwise from the east). The rowboat travels half as fast as the sailboat at some fixed bearing. Sometime later the boats collide.

(a) Find the possible bearing(s) of the rowboat.

(b) When do the boats collide?

(c) Establish a rectangular coordinate system with the positive $x$-axis pointing due east, and the origin at the sailboat's position at noon. Then parameterize the motions of the rowboat and sailboat. Consider all possibilities.

\end{question}





\begin{question}  \label{Qercg663}
At noon a motorboat is $10$ km due east of a sailboat. The sailboat travels at a consant speed of $v$ km/hour at a fixed bearing of $0.4$ radians (measured counterclockwise from the east). The motorboat travels twice as fast as the sailboat at some fixed bearing. Sometime later the boats collide.

(a) Find the possible bearing(s) of the motorboat.

(b) When do the boats collide?

(c) Establish a rectangular coordinate system with the positive $x$-axis pointing due east, and the origin at the sailboat's position at noon. Then parameterize the motions of the motorboat and sailboat. Consider all possibilities.

\end{question}



\section{The Law of Cosines and Relative Position}
Earlier in this class we solved problems about relative position by resolving vectors into their $x$ and $y$ components. Here we show how to take a different approach by working with triangles instead.

 \begin{example}  \label{Exergt43tgr}
Point $B$ is $10$ miles due east of point $A$. Point $C$ is $8$ miles from $A$ at a bearing of $0.6$ radians. Desrcribe the position of $C$ relative to $B$ by giving a distance and a bearing.

\begin{explanation}
We start by drawing an accurate picture.

\pdfOnly{
Access Geogebra interactives through the online version of this text at
 
\href{https://www.geogebra.org/classic/vsdrcnqm}.
}
 
\begin{onlineOnly}
    \begin{center}
\geogebra{vsdrcnqm}{900}{600}
\end{center}
\end{onlineOnly}


Geogebra activity available at

\href{https://www.geogebra.org/classic/vsdrcnqm}{142: Relative Position 4b}


\pskip \pskip

Next we'll work through the solution as usual, by working with the components of the vectors.

The vector $\overrightarrow{AB}$ has components
\[
   \overrightarrow{AB} = \langle 10, 0  \rangle
\]
measured in miles.

The vector $\overrightarrow{AC}$ has components
\[
    \overrightarrow{AC} = \langle 8 \cos (0.6), 8 \sin(0.6)  \rangle
\]
also measured in miles.

The vector $\overrightarrow{BC}$ that gives the position of $C$ relative to $B$ is
\begin{align*}
       \overrightarrow{BC} &= \overrightarrow{AC} - \overrightarrow{AB}    \\
                                    &= \langle 8 \cos (0.6), 8 \sin(0.6)  \rangle - \langle 10, 0  \rangle \\
                                    &= \langle 8 \cos (0.6) - 10, 8 \sin(0.6)  \rangle
\end{align*}

We'll now compute the \emph{square} of the distance from $B$ to $C$ as
\begin{align*}
         |  \overrightarrow{BC} |^2 &= |    \langle 8 \cos (0.6) - 10, 8 \sin(0.6)  \rangle    |^2  \\
                                             &= (8 \cos (0.6) - 10)^2 + (8 \sin(0.6))^2   \\
                                             &=  8^2 + 10^2 - 2(8)(10) \cos(0.6)
\end{align*}

\begin{question}  \label{Q4r345fg}
Fill in all the missing steps between the second and third lines in the above computation. 
\end{question}

\begin{question}  \label{Qer55ft4w}
Show how the above computation for $|\overrightarrow{BC}|^2$ can be arrived at more directly by using the law of cosines in $\Delta ABC$.
\end{question}

So the main point here is that we could have skipped the step of writing the vector $\overrightarrow{AC}$ in terms of its components and used the law of cosines in $\Delta ABC$ right from the start to find the distance $BC$. 

So we'll start the solution again by using the law of cosines in $\Delta ABC$. This tells us that
\begin{align*}
   |  \overrightarrow{BC} |^2 &=  |\overrightarrow{AB}|^2 + |\overrightarrow{AC}|^2 - 2|\overrightarrow{AB}||\overrightarrow{AC}| \cos (\angle BAC)  \\
                                          &= 8^2 + 10^2 - 2(8)(10) \cos(0.6)  \\
                                          &= 164-160 \cos(0.6) .
\end{align*}
And so the distance from $C$ to $B$ is
\[
   |  \overrightarrow{BC} | = \sqrt{164-160 \cos(0.4)} \sim 5.652 \text{ miles} .
\]

The second step in describing the position of $C$ relative to $B$ is to determine the polar angle of the vector $\overrightarrow{BC}$. To do this we first find the radian measure of angle $\angle ACB$ in $\Delta ABC$.

We'll use the law of cosines again. Then since
\[
    |\overrightarrow{AB}|^2 =  |\overrightarrow{AC}|^2 + |\overrightarrow{BC}|^2 - 2|\overrightarrow{AC}||\overrightarrow{BC}| \cos (\angle ACB)  ,
\]
\begin{align*}
   \cos(\angle ACB) &= \frac{|\overrightarrow{AC}|^2 + |\overrightarrow{BC}|^2 - |\overrightarrow{AB}|^2}{2|\overrightarrow{AC}||\overrightarrow{BC}|} \\
                           &\sim -0.0448 .
\end{align*}
 
So
\[
    \angle ACB \sim \arccos(-0.0448) \sim 1.616 .
\]

Now to measure the polar angle of $\overrightarrow{BC}$, imagine standing at $C$ facing the postive $x$-axis (due east). First turn counterclockwise through the angle $\angle CAB = 0.6$ radians and you'll be facing in the direction of the vector $\overrightarrow{AC}$. Then turn couterclockwise through the angle $1.616$ and you'll be facing in the direction of the vector $\overrightarrow{BC}$. So the bearing of $C$ relative to $B$ is about
\[
    0.6 + 1.616 \sim 2.22 \text{ radians} .
\]

\begin{question}  \label{Qertr452}
Suppose instead we had used the law of sines in $\Delta ABC$ to determine the measure of $\angle ACB$. Carry out this computation and explain the difficulty you run into.
\end{question}

\end{explanation}

\end{example}


\section{A Few Problems}

\begin{question}  \label{Qwereduy764}
Point $A$ is $4$ miles from point $C$ at a bearing of $3\pi/2$ radians. Point $B$ lies at a bearing of $2.5$ radians relative to $C$ and at a bearing of $2$ radians relative to $A$. 

(a) Use the application below to draw an accurate picture with the appropriate vectors. Explain your method.

(b) Find the exact distances from $B$ to $C$ and from $B$ to $A$. Do not use a calculator.

(c) Use a calculator to approximate the distances in part (b) to the nearest tenth of a mile.

\pdfOnly{
Access Geogebra interactives through the online version of this text at
 
\href{https://www.geogebra.org/classic/fus9f4qf}.
}
 
\begin{onlineOnly}
    \begin{center}
\geogebra{fus9f4qf}{900}{600}
\end{center}
\end{onlineOnly}


\end{question}






\section{Skip This Section for Now}

\begin{exploration}
The law of cosines, like the Pythagorean theorem, is really a statement about areas. But perhaps the simplest way to think about the law of cosines might be in terms of lengths. 

The figure below shows a way to visualize the law of cosines. To avoid fractions, we scale $\Delta ABC$ with side lengths $a$, $b$, $c$ , by the factor $c$, so that the scaled triangle (still called $\Delta ABC$) has side lengths $ac$, $bc$, and $c^2$.

\pdfOnly{
Access Desmos interactives through the online version of this text at
 
\href{https://www.desmos.com/calculator/vpq0crsvw3}.
}
 
\begin{onlineOnly}
    \begin{center}
\desmos{vpq0crsvw3}{900}{600}
\end{center}
\end{onlineOnly}


Now the idea is to partition $\overline{AB}$ into three segments by finding points $C_A$ and $C_B$ on $\overline{AB}$ so that
\[ 
    \Delta ACC_A \sim \Delta C BC_B \sim \Delta ABC .
\]
Then in $\Delta ACC_A$,
\[
    AC_A = (b/c)(bc) = b^2
\]
and
\[
  CC_A = (b/c)(ac) = ab.
\]
And in $\Delta C BC_B $,
\[
    BC_B = (a/c)(ac) = a^2
\]
and
\[
  CC_B = (a/c)(bc) = ab.
\]

To determine the length of $CA_A C_B$, let $2\theta$ be the measure of $\angle C_A C C_B$. Then
\[
     2\theta + \alpha + \beta = \gamma ,
\]
and since
\[
    \alpha + \beta = \pi - \gamma ,
\]
\[
   \theta = \gamma - \pi/2.
\]
So with $M$ as the midpoint of $\overline{C_A C_B}$,
\[
   C_A C_B = 2 C_A M = 2ab \sin \theta = -2ab\cos\gamma .
\]

That's all we need, for since
\[
    AB = BC_B + C_B C_A + C_AA ,
\]
\[
  c^2 = a^2 - 2ab \cos \gamma + b^2 .
\]

But perhaps since $AB = BM +AM$ it might be better to write the law of cosines as
\[
   c^2 = (a^2 -ab\cos\gamma) + (b^2- ab\cos \gamma) .
\]

This suggests another, more abstract way of getting at the law of cosines. Because $AM = bc \cos \alpha$ and $BM = ac \cos\beta$,
\[
   c^2 = bc \cos \alpha + ac \cos \beta .
\]
Now the expressions 
\[
   f(a,b,c) = bc \cos\alpha
\]
and
\[
   g(a,b,c) = ac \cos \beta
\]
must be quadratic functions of $a$,$b$, $c$ and symmetric in the sense that
\[
  g(a,b,c) = f(b,a,c) .
\]
So if
\[
  f(a,b,c) = \lambda_1 a^2 + \lambda_2 b^2 + \lambda_3 c^2 + \lambda_4 ab + \lambda_5 ac + \lambda_6 bc ,
\]
then
\[
  g(a,b,c) = \lambda_1 b^2 + \lambda_2 a^2 + \lambda_3 c^2 + \lambda_4 ab + \lambda_5 bc + \lambda_6 ac  .
\]

\end{exploration}







\section{Exercises}

\begin{question} \label{Q0:SineCosine}
A loading ramp that makes an angle of $\pi/9$ radians with the ground is replaced with a new ramp that makes an angle of $\pi/18$ radinas with the ground. The new ramp is what percent longer than the original? Round your answer to the nearest percent.

The new ramp is approximately 
\[
   \answer{97}\% \text{ longer than the original.}
\]


\end{question}


\begin{question} \label{Q00:SineCosine}
A tree leans precariously (or maybe impossibly) with its trunk making at an angle of $\pi/6$ radians with the ground. The bottom end of a 10-foot ladder is 16 feet from the base of the trunk and the top end rests against the trunk. How far is the top end from the trunk's base? 

Solve this problem twice, once using the law of sines and then again using the law of cosines.

Use the exploration below to check your work.

\begin{exploration}

\pdfOnly{
Access Desmos interactives through the online version of this text at
 
\href{https://www.desmos.com/calculator/yryupe6lkt}.
}
 
\begin{onlineOnly}
    \begin{center}
\desmos{yryupe6lkt}{900}{600}
\end{center}
\end{onlineOnly}
\end{exploration} 

\end{question}


\begin{question} \label{Q1:SineCosine}
You stand an unknown distance from the base of a building and measure the angle of elevation to the top of the building to be $\theta_1$ radians. You then walk an additional $s$ meters directly away from the building and measure the angle of elevation to its top to be $\theta_2$ radians. 

(a) Use right triangle trigonometry to express the height of the building above eye level (in meters) in terms of $s$, $\theta_1$, and $\theta_2$.


The height above eye level is 
\[
       \answer{\frac{s}{\cot\theta_2-\cot \theta_1}} \text{ feet}.
\]

(b) Use the law of sines to express the height of the building above eye level (in meters) in terms of $s$, $\theta_1$, and $\theta_2$.

The height above eye level is 
\[
       \answer{\frac{s \sin \theta_1 \sin\theta_2}{\sin(\theta_2-\theta_1)}} \text{ feet}.
\]

(c) Compare your answers to parts (a) and (b) to express $\sin(\theta_1 - \sin\theta_2)$ in terms of $\sin\theta_1$, $\cos\theta_1$, $\sin\theta_2$, and $\cos\theta_2$.

\end{question}

\begin{question} \label{Q2:SineCosine}
At noon a sailboat is $10$ km due east of a motorboat. The motorboat travels at a constant speed of $v$ km/hour at a fixed bearing of $\theta_1$ radians (measured counterclockwise from the east), with $0<\theta_1 <\pi/2$. The sailboat travels in a fixed bearing at a constant speed of $kv$ km/hr, where $0<k<1$ Sometime later the boats collide.

(a) Express the two possible bearings $\theta_2$ and $\theta_3$ of the sailboat in terms of $v$, $k$, and $\theta_1$.

(b) Express the times of the collision in terms of $v$, $k$, $\theta_1$, $\theta_2$, and $\theta_3$.

(c) Under what condition do the boats collide?

(d) Under what condition is there a unique collision time?

(e) Establish a rectangular coordinate system with the positive $x$-axis pointing due east, and the origin at the motorboat's position at noon. Then parameterize the motions of the motorboat and sailboat. Consider all possibilities.

(f) Fill in the missing lines in the Exploration below. Then play the slider $u$ to check that you are correct. Include at least two screenshots.

\begin{exploration}

\pdfOnly{
Access Desmos interactives through the online version of this text at
 
\href{https://www.desmos.com/calculator/pxuajk6zyv}.
}
 
\begin{onlineOnly}
    \begin{center}
\desmos{pxuajk6zyv}{900}{600}
\end{center}
\end{onlineOnly}
\end{exploration} 

\end{question}


\begin{question} \label{Q3:SineCosine}
The minute and hour hands of a clock have respective lengths $a$ and $b$ centimeters.

(a) Find a function 
\[
   s = f(t) \, , 0\leq t \leq 12/11 ,
\]
that expresses the distance (in centimeters) between the tips of the hands in terms of the number of hours past noon.

\end{question}


\begin{question}  \label{Q4:SineCosine}
On the vernal equinox at a location on the equator, the sun rises due east at 6am, passes directly overhead at noon, and sets due west at 6pm. 

A building $h$ feet high on the equator casts a shadow on a {\bf spherical} earth on the vernal equinox. 

(a) Find a function
\[
   s = g(t) \, , 0\leq t \leq A ,
\]
that expresses the length of the shadow (in feet) in terms of the number of hours past noon. Take the earth to be a perfect sphere of radius $R=3960$ miles. Note that there are 5,280 feet in one mile. Consider only the times when the entire building casts a shadow on the ground as in the exploration below.

(b) Find the value of $A$ above. That is, find the latest time at which the entire building casts a shadow on the ground.

(c) Take $h=100$ and compare the function in part (a) with the function 
\[
  s = f(t) = 100 \tan \left( \frac{\pi}{12}t \right) \, , 0\leq t < 6 ,
\]
that expresses the length of the shadow of a 100-foot building on a flat earth in terms of the number of hours past noon. Include graphs of the functions $f$ and $g$ as part of your analysis.

\begin{exploration}

\pdfOnly{
Access Desmos interactives through the online version of this text at
 
\href{https://www.desmos.com/calculator/9i7e00owtr}.
}
 
\begin{onlineOnly}
    \begin{center}
\desmos{9i7e00owtr}{900}{600}
\end{center}
\end{onlineOnly}
\end{exploration} 

\end{question}




https://www.desmos.com/3d/d2xv3dwc



\pdfOnly{
Access Desmos interactives through the online version of this text at
 
\href{https://www.desmos.com/3d/fbaa38f918}.
}
 
\begin{onlineOnly}
    \begin{center}
\geogebra{d2xv3dwc}{900}{600}
\end{center}
\end{onlineOnly}



\end{document}