\documentclass{ximera}
\title{The Laws of Cosines and Sines}

\newcommand{\pskip}{\vskip 0.1 in}

\begin{document}
\begin{abstract}
The law of sines, the law of cosines.
\end{abstract}
\maketitle


\pdfOnly{
Access Desmos interactives through the online version of this text at
 
\href{https://www.desmos.com/calculator/9fubvhv8wj}.
}
 
\begin{onlineOnly}
    \begin{center}
\desmos{9fubvhv8wj}{900}{600}
\end{center}
\end{onlineOnly}





\begin{question} \label{Q0:SineCosine}
A loading ramp that makes an angle of $\pi/9$ radians with the ground is replaced with a new ramp that makes an angle of $\pi/18$ radinas with the ground. The new ramp is what percent longer than the original? Round your answer to the nearest percent.

The new ramp is approximately 
\[
   \answer{97}\% \text{ longer than the original.}
\]


\end{question}


\begin{question} \label{Q00:SineCosine}
A tree leans precariously (or maybe impossibly) with its trunk making at an angle of $\pi/6$ radians with the ground. The bottom end of a 10-foot ladder is 16 feet from the base of the trunk and the top end rests against the trunk. How far is the top end from the trunk's base? 

Solve this problem twice, once using the law of sines and then again using the law of cosines.

Use the exploration below to check your work.

\begin{exploration}

\pdfOnly{
Access Desmos interactives through the online version of this text at
 
\href{https://www.desmos.com/calculator/yryupe6lkt}.
}
 
\begin{onlineOnly}
    \begin{center}
\desmos{yryupe6lkt}{900}{600}
\end{center}
\end{onlineOnly}
\end{exploration} 

\end{question}


\begin{question} \label{Q1:SineCosine}
You stand an unknown distance from the base of a building and measure the angle of elevation to the top of the building to be $\theta_1$ radians. You then walk an additional $s$ meters directly away from the building and measure the angle of elevation to its top to be $\theta_2$ radians. 

(a) Use right triangle trigonometry to express the height of the building above eye level (in meters) in terms of $s$, $\theta_1$, and $\theta_2$.


The height above eye level is 
\[
       \answer{\frac{s}{\cot\theta_2-\cot \theta_1}} \text{ feet}.
\]

(b) Use the law of sines to express the height of the building above eye level (in meters) in terms of $s$, $\theta_1$, and $\theta_2$.

The height above eye level is 
\[
       \answer{\frac{s \sin \theta_1 \sin\theta_2}{\sin(\theta_2-\theta_1)}} \text{ feet}.
\]

(c) Compare your answers to parts (a) and (b) to express $\sin(\theta_1 - \sin\theta_2)$ in terms of $\sin\theta_1$, $\cos\theta_1$, $\sin\theta_2$, and $\cos\theta_2$.

\end{question}

\begin{question} \label{Q2:SineCosine}
At noon a sailboat is $10$ km due east of a motorboat. The motorboat travels at a constant speed of $v$ km/hour at a fixed bearing of $\theta_1$ radians (measured counterclockwise from the east), with $0<\theta_1 <\pi/2$. The sailboat travels in a fixed bearing at a constant speed of $kv$ km/hr, where $0<k<1$ Sometime later the boats collide.

(a) Express the two possible bearings $\theta_2$ and $\theta_3$ of the sailboat in terms of $v$, $k$, and $\theta_1$.

(b) Express the times of the collision in terms of $v$, $k$, $\theta_1$, $\theta_2$, and $\theta_3$.

(c) Under what condition do the boats collide?

(d) Under what condition is there a unique collision time?

(e) Establish a rectangular coordinate system with the positive $x$-axis pointing due east, and the origin at the motorboat's position at noon. Then parameterize the motions of the motorboat and sailboat. Consider all possibilities.

(f) Fill in the missing lines in the Exploration below. Then play the slider $u$ to check that you are correct. Include at least two screenshots.

\begin{exploration}

\pdfOnly{
Access Desmos interactives through the online version of this text at
 
\href{https://www.desmos.com/calculator/pxuajk6zyv}.
}
 
\begin{onlineOnly}
    \begin{center}
\desmos{pxuajk6zyv}{900}{600}
\end{center}
\end{onlineOnly}
\end{exploration} 

\end{question}


\begin{question} \label{Q3:SineCosine}
The minute and hour hands of a clock have respective lengths $a$ and $b$ centimeters.

(a) Find a function 
\[
   s = f(t) \, , 0\leq t \leq 12/11 ,
\]
that expresses the distance (in centimeters) between the tips of the hands in terms of the number of hours past noon.

\end{question}


\begin{question}  \label{Q4:SineCosine}
On the vernal equinox at a location on the equator, the sun rises due east at 6am, passes directly overhead at noon, and sets due west at 6pm. 

A building $h$ feet high on the equator casts a shadow on a {\bf spherical} earth on the vernal equinox. 

(a) Find a function
\[
   s = g(t) \, , 0\leq t \leq A ,
\]
that expresses the length of the shadow (in feet) in terms of the number of hours past noon. Take the earth to be a perfect sphere of radius $R=3960$ miles. Note that there are 5,280 feet in one mile. Consider only the times when the entire building casts a shadow on the ground as in the exploration below.

(b) Find the value of $A$ above. That is, find the latest time at which the entire building casts a shadow on the ground.

(c) Take $h=100$ and compare the function in part (a) with the function 
\[
  s = f(t) = 100 \tan \left( \frac{\pi}{12}t \right) \, , 0\leq t < 6 ,
\]
that expresses the length of the shadow of a 100-foot building on a flat earth in terms of the number of hours past noon. Include graphs of the functions $f$ and $g$ as part of your analysis.

\begin{exploration}

\pdfOnly{
Access Desmos interactives through the online version of this text at
 
\href{https://www.desmos.com/calculator/9i7e00owtr}.
}
 
\begin{onlineOnly}
    \begin{center}
\desmos{9i7e00owtr}{900}{600}
\end{center}
\end{onlineOnly}
\end{exploration} 

\end{question}




\end{document}