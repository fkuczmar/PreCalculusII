\documentclass{ximera}
\title{Motion, Part 1}

\newcommand{\pskip}{\vskip 0.1 in}

\begin{document}
\begin{abstract}
An introduction to parameterizing uniform cirular motion and motion with a constant velocity.
\end{abstract}
\maketitle

\section{Uniform Circlular Motion}

You already know at least two ways to describe a curve algebraically - as  the graph of a function like $y=x^2$, or as the graph of a relation like $x^2 + y^2 = 3$. In this section, we introduce a third way and describe curves parametrically. 

To describe a curve in the plane \emph{parametrically} means to express the $x$ and $y$-coordinates as functions of a third variable called the \emph{parameter}. Often, but not always, the parameter measures time, and in this case the parameterization describes a \emph{motion}. It tells us the position of a point as a function of time. As such, the parameterization gives more information than just the path itself.% of the motion.

For example, it is relatively easy to describe the path of the earth around the sun; the path is an ellipse with one focus at the sun, having a certain shape and size. But it is much more difficult to parameterize the motion of the earth because this would require knowing the position of the earth at any time in its orbit.

In this class we will parameterize several different types of motions:

a. uniform circular motion

\pskip

b. motion at a constant velocity

\pskip

c. motions around an ellipse

\pskip


d. projectile motion in a uniform gravitational field

\pskip

This chapter focuses on the first two.


\section{Uniform Circular Motion}

\begin{exploration}

To move with \emph{uniform circular motion} means to move around a circle at a constant speed.

\pdfOnly{
Access Desmos interactives through the online version of this text at
 
\href{https://www.desmos.com/calculator/lhlfgnrulb}.
}
 
\begin{onlineOnly}
    \begin{center}
\desmos{lhlfgnrulb}{900}{600}
\end{center}
\end{onlineOnly}
\end{exploration}






\end{document}