\documentclass{ximera}
\title{Motion, Part 1}

\newcommand{\pskip}{\vskip 0.1 in}

\begin{document}
\begin{abstract}
An introduction to parameterizing uniform cirular motion and motion with a constant velocity.
\end{abstract}
\maketitle

\section{Uniform Circlular Motion}

You already know at least two ways to describe a curve algebraically - as  the graph of a function like $y=x^2$, or as the graph of a relation like $x^2 + y^2 = 3$. In this section, we introduce a third way and describe curves parametrically. 

To describe a curve in the plane \emph{parametrically} means to express the $x$ and $y$-coordinates as functions of a third variable called the \emph{parameter}. Often, but not always, the parameter measures time, and in this case the parameterization describes a \emph{motion}. It tells us the position of a point as a function of time. As such, the parameterization gives more information than just the path itself.% of the motion.

For example, it is relatively easy to describe the path of the earth around the sun; the path is an ellipse with one focus at the sun, having a certain shape and size. But it is much more difficult to parameterize the motion of the earth because this would require knowing the position of the earth at any time in its orbit.

In this class we will parameterize several different types of motions:

a. uniform circular motion

\pskip

b. motion at a constant velocity

\pskip

c. motions around an ellipse

\pskip


d. projectile motion in a uniform gravitational field

\pskip

This chapter focuses on the first two.


\section{Uniform Circular Motion}

\begin{exploration}

To move with \emph{uniform circular motion} means to move around a circle at a constant speed. The animation below shows an example where a beetle crawls counterclockwise at a constant speed around a circle of radius $20$cm centered at the origin.

\pdfOnly{
Access Desmos interactives through the online version of this text at
 
\href{https://www.desmos.com/calculator/xxnetzog4j}.
}
 
\begin{onlineOnly}
    \begin{center}
\desmos{xxnetzog4j}{900}{600}
\end{center}
\end{onlineOnly}
\end{exploration}

To parameterize the motion we need more information than just the path. We need just enough to determine the beetle's position at any time.

So we'll assume that the beetle starts from the point $(20,0)$ at time $t=0$ seconds past noon and that it crawls with a constant speed of $4$ cm/sec. We'll also assume that the beetle crawls around the circle for two minutes. Our problem now is to express the coordinates of the beetle as functions of time, say $t$, measured as the number of seconds past noon. That is, we wish to find \emph{coordinate functions}
\[
   x = f(t) \text{ and } y=g(t) , \text{ for } 0\leq t \leq 120 ,
\]
that express the coordinates of the beetle (measured in centimeters) in terms of the number of seconds past noon.

The key to finding these functions is to realize that the defintions
\[
    \cos \theta = \frac{x}{r}
\]
\[
    \sin\theta = \frac{y}{r}
\]
of the sine and cosine functions tell us the coordinates $(x,y)$ of a point $B$ on the circle of radius $r$ centered at the origin. They are 
\[
    x = r\cos\theta
\]
and 
\[
   y= r\sin \theta .
\]
Remember that $\theta$ is the \emph{polar angle}. It is meaured (in radians) counterclockwise from the positive $x$-axis to the segment $\overline{OB}$ from the origin to $B$.

So to parameterize the beetle's motion, where $r=20$ cm, we need only express the polar angle as a function of time. For this, stop the motion in the animation above by pausing the slider $u$. 

Now recall that the beetle crawled with a constant speed of $4$ cm/s.  So during the first $t$ seconds of its motion, the beetle crawls a distance of
\[
    s = \left( \frac{4 \text{ cm}}{\text{sec}} \right) (t \text{ sec}) = 4t \text{ cm}  
\]
around the circle. And during this time it turned counterclockwise about the origin through the angle
\[
    \Delta \theta = \frac{s}{r} = \frac{4t \text{ cm}}{20 \text{ cm}} = \frac{1}{5}t \text{ radians}.
\]

\begin{question}  \label{Qgf45365:Motion1}
(a) What are the units of the fraction $1/5$ in the above expression for $\Delta \theta$?

(b) Explain the meaning of the fraction $1/5$.
\end{question}


The last step is to realize that since the beetle started on the positive $x$-axis, the change $\Delta \theta$  is in fact the polar angle $\theta$ that encodes the beetle's position. So the functions
\[
    x = r\cos \theta = 20 \cos \left( \frac{1}{5}t \right) , 0\leq t \leq 120
\]
\[
    y = r\cos \theta = 20 \sin \left( \frac{1}{5}t \right) , 0\leq t \leq 120
\]
give the beetle's coordinates at time $t$ seconds past noon.


\begin{question}  \label{Q345tds5:Motion1}
An ant crawls clockwise with a constant speed of $3$ cm/sec around a circle of radius $40$ cm centered at the origin. The ant starts at the point $(40,0)$ at time $t=0$ seconds past noon and crawls for $3$ minutes.

Find functions
\[
   x = f(t) \text{ and } y=g(t) , \text{ for } 0\leq t \leq 180 ,
\]
that express the ant's coordinates (measured in centimeters) in terms of the number of seconds past noon. {\it Hint:} Keep in mind that the ant crawls clockwise around its path.



\end{question}


\end{document}