\documentclass{ximera}
\title{Solving Trigonometric Equations, CW}

\newcommand{\pskip}{\vskip 0.1 in}

\begin{document}
\begin{abstract}
Introduction to solving trigonometric equations.
\end{abstract}
\maketitle

\section{Exercises}

\begin{exercise}  \label{Ex:44r4nj3r33g3e}

A block at the end of a spring oscillates in simple harmonic motion between positions $x=\pm 3$ as shown below. The block reaches the extreme position $x=3$m at time $t=2$ seconds and completes one oscillation about the origin every 5 seconds. The block makes four complete oscillations before a dog swallows the block and the problem ends.

\begin{onlineOnly}
    \begin{center}
\desmos{83lf5m446b}{900}{600}
\end{center}
\end{onlineOnly}

\href{https://www.desmos.com/calculator/83lf5m446b}{142: Simple Harmonic motion 54B}

\begin{enumerate}

\item Find a function 
\[
  x=f(\theta)  %\, , \, 0 \leq \theta \leq \answer{8\pi}
\]
that express the position (ie. the $x$-coordinate) of the block in terms of the polar angle $\theta$ of the vector $\overrightarrow{OP}$ from the origin to the point $P$ that drives the simple harmonic motion.

The function (\emph{leave the domain blank for now}) is
\[
  x = f(\theta) = \answer{3\cos\theta} \, , \, \answer{-4\pi/5} \leq \theta \leq \answer{36\pi/5}.
\]


\item Find a function 
\[
 \theta = a(t) 
\]
that expesses the  polar angle in terms of the number of seconds since the mass was released.

The function is
\[
   \theta = a(t) =  \answer{\frac{2\pi}{5} \left(  t-2 \right)} \, , \, 0 \leq t \leq \answer{20} .
\]

\item Use your expression for the function $\theta = a(t)$ to find the domain of the function $x=f(\theta)$ in part (a).

\item Then find a function
\[
     x = g(t) \, , \, 0 \leq t \leq \answer{20} ,
\]
that expresses the position (in meters) of the mass in terms of the number of seconds since the mass was released.

The function is
\[
      x = g(t) = \answer{3\cos(\frac{2\pi}{5} \left(  t-2 \right))} \, , \, 0 \leq t \leq \answer{20}.
\]

\item What equation would you solve to find all times when the block has position $x=1.4$ meters?

\item To solve the equation in part (e), first make the substitution 
\[
        \theta =  \frac{2\pi}{5} \left(  t-2 \right)
\]
and find the \emph{exact} polar angles of the driving point when the block has position $x=1.4$ meters. Do \emph{not} use a calculator.

The polar angles are
\[
   \theta = \answer{\arccos ( 1.4/3 )} + 2\pi n \, , \, n=\answer{0} , \answer{1} , \answer{2}, \answer{3} 
\]
or
\[
     \theta = \answer{-\arccos ( 1.4/3 )} + 2\pi n \, , \, n=1, 2, 3, 4.
\]

\item Then find the \emph{exact} times when the block has position $x=1.4$ meters. Do \emph{not} use a calculator.

The times are 
\[
   t = \answer{2+ \frac{5}{2\pi}\arccos ( 1.4/3 )} + \answer{5}n \, , \, n=\answer{0} , \answer{1} , \answer{2}, \answer{3} 
\]
or
\[
     t = \answer{2-\frac{5}{2\pi}\arccos ( 1.4/3 )} + \answer{5}n \, , \, n=1, 2, 3, 4.
\]



\item Use the radian protractor above to approximate the two smallest angles in part (f). Then use these to approximate the first two times the block has position $x=1.4$ meters.

\item Use a calculator and the results of part (g) to approximate the first two times the block has position $x=1.4$ meters.

\item When does the block have position $x=1.4$ meters for the \emph{last} time? Give the exact time and then a decimal approximation.

%\begin{onlineOnly}
%    \begin{center}
%\geogebra{qnhtd4h8}{900}{600}
%\end{center}
%\end{onlineOnly}

\end{enumerate}

\end{exercise}


\begin{exercise}  \label{Ex:LmDmr3r33343}
A block at the end of a spring oscillates with a period of $8$ seconds in simple harmonic motion between positions $y=\pm 5$ meters along the $y$-axis as shown below. The block passes the origin on its way up at time $t=1$ second.  The block makes five complete oscillations before a dog intervenes and the problem ends.

\begin{onlineOnly}
    \begin{center}
\desmos{cpt96llf7e}{900}{600}
\end{center}
\end{onlineOnly}

\href{https://www.desmos.com/calculator/cpt96llf7e}{142: Simple Harmonic motion 54}

Repeat the parts of Question 1 for this scenario, finding polar angles and times (as in Question 1) when the block has position $y=-2$ meters.

\end{exercise}

\end{document}