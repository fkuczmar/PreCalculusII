\documentclass{ximera}
\title{Solving Trigonometric Equations, CW}

\newcommand{\pskip}{\vskip 0.1 in}

\begin{document}
\begin{abstract}
Introduction to solving trigonometric equations.
\end{abstract}
\maketitle

\section{Exercises}

\begin{exercise}  \label{Ex:44r4nj3r33g3e}

A block at the end of a spring oscillates in simple harmonic motion on a frictionless surface as shown below. The spring is stretched two meters from beyond its relaxed (ie. unstretched) length and then released from rest at $x=2$ m, completing one oscillation about the origin every 5 seconds. The block makes four complete oscillations before a dog intervenes and the problem ends.

\begin{onlineOnly}
    \begin{center}
\desmos{shbymucxq6}{900}{600}
\end{center}
\end{onlineOnly}

\href{https://www.desmos.com/calculator/shbymucxq6}{142: Simple Harmonic motion 53}

\begin{enumerate}

\item Find a function 
\[
  x=f(\theta) \, , \, 0 \leq \theta \leq \answer{8\pi}
\]
that express the position (ie. the $x$-coordinate) of the block in terms of the polar angle $\theta$ of the vector $\overrightarrow{OP}$ from the origin to the point $P$ that drives the simple harmonic motion.

\item Find a function 
\[
 \theta = a(t) \, , \, 0 \leq t \leq \answer{20} , 
\]
that expesses the  polar angle in terms of the number of seconds since the mass was released.

\item Then find a function
\[
     x = g(t) \, , \, 0 \leq t \leq \answer{20} ,
\]
that expresses the position (in meters) of the mass in terms of the number of seconds since the mass was released.

\item Use your function from part (a) to write an equation to find all possible polar angles of the driving point when the block has position $x=1.5$ meters. Solve this equation to find these exact angles \emph{without} a calculator.

\item Use the radian protractor above to approximate the first two angles in part (d).

\item U





\begin{onlineOnly}
    \begin{center}
\geogebra{qnhtd4h8}{900}{600}
\end{center}
\end{onlineOnly}


\end{enumerate}

\end{exercise}

\end{document}