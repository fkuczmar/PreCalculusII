\documentclass{ximera}
\title{Parametrically-defined Curves}


\newcommand{\pskip}{\vskip 0.1 in}

\begin{document}
\begin{abstract}
Motion with constant velocity
\end{abstract}
\maketitle

You already know at least two ways to describe a curve algebraically - as  the graph of a function like $y=x^2$, or as the graph of a relation like $x^2 + y^2 = 3$. In this section, we introduce a third way and describe curves parametrically. 

To describe a curve in the plane \emph{parametrically} means to express the $x$ and $y$-coordinates as functions of a third variable called the \emph{parameter}. Often, but not always, the parameter measures time, and in this case the parameterization describes a \emph{motion}. It tells us the position of a point as a function of time. As such, the parameterization gives us much more information than just the path.% of the motion.

For example, it is relatively easy to describe the \emph{path} of the earth around the sun as an ellipse with one focus at the sun, having a certain shape and size. But is would be much more difficult to parameterize the motion of the earth, as this would require knowing the position of the earth at any time in its orbit.

In this class we will parameterize several different types of motions:

a. motion at a constant velocity

\pskip

b. projectile motion in a uniform gravitational field

\pskip

c. uniform circular motion

\pskip

d. motions around an ellipse

\pskip

This chapter focuses on the first, motion with a constant velocity.

\section{Motion with a Constant Velocity}

In colloquial English, velocity is a synonym for speed, as in ``A 747 has a maximum velocity of 650 miles/hour." But this is incorrect, as velocity is properly a vector that describes both speed and direction. So we could describe the velocity of a car by saying that it moves due north at a speed of 60 miles/hour. For an object to move with a constant velocity means that it travels in a fixed direction at a constant speed.

\pskip

\noindent {\bf Example 1:} Between 11:58 am and 12:20pm a beetle crawls with a constant velocity, passing the point $A(-3,8)$ at 12:01pm and the point $B(10,-1)$ at 12:08pm, where the coordinates are measured in yards. Parameterize the beetle's motion.

\pskip

\noindent {\bf Solution:} The question is asking us to express the beetle's coordinates as functions of time, and we should start by precisely defining a variable that measures time. We'll do this by letting $t$ be the number of minutes past noon, so that for this problem $-2 \leq t \leq 20$. 

Now we need to find \emph{coordinate functions}
\[
   x = f(t) \text{ and } y = g(t) , \text{ for } -2\leq t \leq 20 ,
\]
that express the coordinates of the beetle, measured in yards, in terms of the number of minutes past noon.  Since the velocity is constant, the coordinate functions change at a constant rate and are therefore linear.

First the $x$-coordinate. The $x$-coordinate changes at the rate of 
\[
    \frac{\Delta x}{\Delta t} = \frac{10-(-3) \text{ yds}}{8-1 \text{ sec}} = \frac{13 \text{ yd}}{7 \text{ sec}}
\]
and (using point-slope) 
\[
     x = f(t) = -3 + \frac{13}{7}(t-1) , -2\leq t \leq 20 .
\]


\begin{question} 
Now make a similar computation for the $y$-coordinate function and input your expression in the box below. 
\[
 y=g(t) =  \answer{8-\frac{9}{7}(t-1)} 
\]
    \end{question}

In conclusion, the functions
\[
    x = f(t) = -3 + \frac{13}{7}(t-1) \text{ and }  y=g(t) = 8-\frac{9}{7}(t-1) , \text{ for } -2\leq t \leq 20 ,
\]
express the beetle's coordinates (measured in yards) in terms of the number of minutes past noon.

\begin{exploration}\label{exp:pc1}
The graphs of the beetle's coordinate functions $x=f(t)$ and $y=g(t)$, $-2\leq t \leq 20$, are shown below. Note the units and variable names on the axes. Drag the slider $s$ to see the coordinates of the beetle at different times during its journey.
 
\pdfOnly{
Access Desmos interactives through the online version of this text at
 
\href{https://www.desmos.com/calculator/0iruo192ir}.
}
 
\begin{onlineOnly}
    \begin{center}
\desmos{0iruo192ir}{900}{600}
\end{center}
\end{onlineOnly}
\end{exploration}

The next slide shows the motion of the beetle along its path.

\begin{exploration}\label{exp:pc1b}
Play the slider $s$ to see the beetle move along its path (the path is graphed in Line 4 and the motion in Line 5). Then click on the graph icon at the far left of Line 4 to hide the path. Play the slider again to see the beetle leave its tracks (coded in Line 6 - note the domain of the parameter $s$ there).
 
\pdfOnly{
Access Desmos interactives through the online version of this text at
 
\href{https://www.desmos.com/calculator/splcvc5xzq}.
}
 
\begin{onlineOnly}
    \begin{center}
\desmos{splcvc5xzq}{900}{600}
\end{center}
\end{onlineOnly}
\end{exploration}


Let's add another insect and suppose that between 11:58 am and 12:20pm an ant also crawls with a constant velocity, passing the point $C(-2,-10)$ at 12:03pm heading directly toward the point $D(30,6)$.

\pskip

\noindent {\bf Example 2:} Parameterize the motion of the ant to make it collide with the beetle.

\pskip

\noindent {\bf Solution:} The solution to this problem requires several steps and to get started it helps to think backwards. We can parameterize the ant's motion if we know its position at two times. We already know where it is at 12:02pm, so we just need to find its position at some other time. And here is where the requirement that the insects collide comes into play. We'll find both where and when they collide, and then use this additional information to parameterize the ant's motion.

We can find where the insects collide by first finding equations of their paths. Then once we find where they collide, we can find the collision time by using the parameterization of the beetle's motion.


\begin{question}  
To get started, first find a Cartestian equation (expressing $y$ in terms of $x$) of the ant's path and enter it in the line below.
 \[
     y=a(x) =  \answer{-10+\frac{1}{2}(x+2)}
\] 
   \end{question}

Next find a Cartestian equation of the beetle's path and enter it in the line below.

\begin{question}  
         $ y=b(x) =  \answer{8-\frac{9}{13}(x+3)}$  
    \end{question}

{\bf Use algebra} to find the {\bf exact} coordinates of the collision point and enter these (as an ordered pair) in the line below.

\begin{question}  
         $ (x_0,y_0) =  \answer{(388/31 , -85/31)}$
    \end{question}

Then find the time $t_0$ (measured in minutes past noon) of collision and enter it in the line below.

\begin{question}  
         $ t_0 =  \answer{290/31}$
    \end{question}

Finally, find expressions $x=f_1(t)$ and $y=g_1(t)$ for the ant's coordinate functions. Enter these twice. Once in the boxes below and again in the Desmos Activity that follows.

\begin{question}  
         $ x = f_1(t) =   \answer{-2+\frac{388/31+2}{290/31-3}(t-3)}$

        \hskip 0.93 in $y=g_1(t) =  \answer{-10+\frac{-85/31+10}{290/31-3}(t-3)}$
    \end{question}


\begin{exploration}\label{exp:pc1c}
Enter the correct expressions for the ant's coordinate functions $x=f_1(t)$ and $y=g_1(t)$ in Lines 15 and 16 below. Then play the slider $s$ to make sure that the insects collide.

\pdfOnly{
Access Desmos interactives through the online version of this text at
 
\href{https://www.desmos.com/calculator/pjci6nssqv}.
}
 
\begin{onlineOnly}
    \begin{center}
\desmos{pjci6nssqv}{900}{600}
\end{center}
\end{onlineOnly}
\end{exploration}


\noindent {\bf Example 3:} Parameterize the motion of a beetle given that between noon and 12:30pm it crawls with a constant speed of $1.7$ yds/min, that it always heads directly toward the point $A(-23,19)$, and that it passes the point $B(7,-1)$ at 12:23pm. 

\pskip

\noindent {\bf Solution:} As before, we'll let $t$ be the number of minutes past noon. Our goal is to find functions
\[
   x = f(t) \text{ and } y = g(t), \text{ for } 0\leq t \leq 30 ,
\]
that express the beetle's coordinates (measured in yards) in terms of the number of minutes past noon.

Unlike the previous problems, we are given the position only at a single time. But we can use the other information to compute the rates (with respect to time) at which the beetle's coordinates change as it moves along its path. We'll then use these rates along with our knowledge of the beetle's position at 12:23pm to find expressions for the coordinate functions. 

The rates of change in the beetle's coordinates are determined by the speed and the direction of travel. We'll first find the time it takes to crawl from $A$ to $B$. 

In crawling from $B(7,-1)$ to $A(-23,19)$, the beetle crawls a distance of
\begin{align*}
  \Delta s  &= \sqrt{(-23 - 7)^2 +   (19 - (-1))^2}  \text{ yds} \\
      & = 10\sqrt{13} \text { yds} .
\end{align*}

And to travel this distance at the constant speed of $v = 1.7$ yards/min takes
\[
       \Delta t = \frac{\Delta s}{v}   = \frac{10 \sqrt{13} \text { yds}}{1.7 \text{ yds/min}} = \frac{10 \sqrt{13}}{1.7 } \text{ min} .
\]

During this time, its $x$-coordinate changes by
\[
   \Delta x = (-23 - 7) \text{ yds} = -30 \text{ yds} ,
\]
giving an average rate of change of
\begin{align*}
  \frac{\Delta x}{\Delta t} &= \frac{-30 \text{ yds}}{\frac{10 \sqrt{13}}{1.7} \text{ min}}  \\
                                     & = \left( 1.7 \text{ yds/min} \right) \left( \frac{-30 \text{ yds}}{10\sqrt{13} \text{ yds}} \right) \label{Eq:RateofChange} \\
                                     & = \frac{-5.1}{\sqrt{13}} \text{ yds/min}  
\end{align*}
in the $x$-coordinate. 

Then since the beetle passes the point $B(7,-1)$ at 12:23pm,  the $x$-coordinate function for the motion is
\[
     x = f(t) = 7 - \frac{5.1}{\sqrt{13}}\left( t-23 \right) , \text{ for } 0\leq t \leq 30 .
\]

It is worth pointing out that the rate of change in the $x$-coordinate
\[
    \frac{\Delta x}{\Delta t} = \left( 1.7 \text{ yds/min} \right) \left( \frac{-30 \text{ yds}}{10\sqrt{13} \text{ yds}} \right)
\]
is the product of the beetle's speed and the dimensionless ratio 
\[
     \frac{\Delta x}{\Delta s} = \frac{-30 \text{ yds}}{10\sqrt{13} \text{ yds}}
\]
of the change in the $x$-coordinate to the distance between $A$ and $B$. This ratio is much like the slope $\Delta y / \Delta x$ of the path, where the change $\Delta s$ is positive (or negative) when we move along the path in the direction (or in the opposite direction) of the motion. If you have taken trigonometry before, you might recognize the ratio $\Delta x\/Delta s$ as the cosine of the counterclockwise angle from the positive $x$-axis to the direction of motion. But much more on this later.
 
%It is this ratio that encodes the direction of travel and if you have taken trigonometry before, you might recognize this as the cosine of a certain angle (from the positive $x$-axis counterclockwise to the direction of motion). It is much like the slope $\Delta y / \Delta x$ of the path, but encodes more information as the slope does not distinguish between the two possible directions of travel along the path. But much more on this later. 

\begin{question}
 Find an expression for the $y$-coordinate function  
\[
 y = g(t) =   \answer{-1+\frac{3.4}{\sqrt{13}}( t-23)}
\]
of the motion.
    \end{question}


\pskip \pskip


\noindent {\bf Example 4:} Let $h>0$ be a constant. Bug A leaves the point $(0,h)$ at time $t=0$ seconds and moves with constant speed $v$ m/s in the direction of the positive $x$-axis along the line $y=h$. At time $t=0$, Bug $B$ leaves the origin and moves with a constant velocity and constant speed $w$ m/s. Some time later the bugs collide.

\begin{question}
 Find an expression in terms of $h$, $v$, and $w$ for the time $t_0$ of the collision.  
\[
        t_0   =  \answer{\frac{h}{\sqrt{w^2-v^2}}}
\]
    \end{question}

\begin{question}
 Find an expression in terms of $h$, $v$, and $w$ for the coordinates of the collision point.  
\[
       (x_0, y_0)  =  \answer{\left( \frac{vh}{\sqrt{w^2-v^2}},h \right)}
\]
    \end{question}


\begin{question}
Parameterize the motion of Bug $B$ in terms of $h$, $v$, $w$ and $t$. Assume $t\geq 0$.
\[
        x = f(t)   =  \answer{v_0 t}
\]
\[
      y = g(t) = \answer{\sqrt{w^2-v^2}t}
\]
    \end{question}

\begin{exploration}\label{exp:pc1c}
Enter the correct expression for the collision time in Line 10 below. Then enter the correct expressions for Bug B's coordinate functions $x=f(t)$ and $y=g(t)$ in Lines 11 and 12. Play the slider $s$ to make sure that the bugs collide.

\pdfOnly{
Access Desmos interactives through the online version of this text at
 
\href{https://www.desmos.com/calculator/xuhjbr4m5r}.
}
 
\begin{onlineOnly}
    \begin{center}
\desmos{xuhjbr4m5r}{900}{600}
\end{center}
\end{onlineOnly}
\end{exploration}


\begin{example}
Test
\end{example}


\begin{explanation}
Test
\end{explanation}


\noindent {\bf Example 5:} 

\end{document}

