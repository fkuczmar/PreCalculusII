\documentclass{ximera}
\title{Parameterized Curves}


\newcommand{\pskip}{\vskip 0.1 in}

\begin{document}
\begin{abstract}
An introduction to describing lines and segments parametrically.
\end{abstract}
\maketitle

JUST A TEST ....... AGAIN .... AGAIN ... ONCE MORE ... AGAIN

You already know at least two ways to describe curves algebraically - as graphs of functions like $y=x^2$, or as graph of relations like $x^2 + y^2 = 3$. In this section, we introduce a third way and describe curves parametrically. 

To describe a curve in the plane \emph{parametrically} means to express the $x$ and $y$-coordinates as functions of a third variable called the \emph{parameter}. Often, but not always, the parameter measures time, and in this case the parameterization describes a \emph{motion}. It tells us the position of a point as a function of time. As such, the parameterization gives us much more information than just the path of the motion.

For example, it is relatively easy to describe the \emph{path} of the earth around the sun as an ellipse with one focus at the sun, having a certain shape and size. But is would be much more difficult to parameterize the motion of the earth around the sun, as this would require knowing the position of the earth at any time in its orbit.

In this class we will parameterize several different types of motions:

a. motion at a constant velocity

\pskip

b. projectile motion in a uniform gravitational field

\pskip

c. uniform circular motion

\pskip

d. motions around an ellipse

\pskip

This chapter focuses on the first, motion with a constant velocity.

\section{Motion with a Constant Velocity}

In colloquial English, velocity is a synonym for speed, as in ``A 747 has a maximum velocity of 650 miles/hour." But this is incorrect, as velocity is propery a vector that describes both speed and direction. So we might describe the velocity of a car by saying that it is moving due north at a speed of 60 miles/hour. For an object (in the plane) to move with a constant velocity means that it is travelling in a fixed direction at a constant speed.

\pskip

\noindent {\bf Example 1:} Between 11:58 am and 12:05pm a beetle crawls with a constant velocity, passing the point $A(-3,8)$ at 12:01pm and the point $B()$ at 12:04pm. Parameterize the ant's motion.

\end{document}

