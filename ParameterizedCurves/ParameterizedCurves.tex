\documentclass{ximera}
\title{Parametrically-defined Curves}


\newcommand{\pskip}{\vskip 0.1 in}

\begin{document}
\begin{abstract}
Motion with constant velocity
\end{abstract}
\maketitle

You already know at least two ways to describe a curve algebraically - as  the graph of a function like $y=x^2$, or as the graph of a relation like $x^2 + y^2 = 3$. In this section, we introduce a third way and describe curves parametrically. 

To describe a curve in the plane \emph{parametrically} means to express the $x$ and $y$-coordinates as functions of a third variable called the \emph{parameter}. Often, but not always, the parameter measures time, and in this case the parameterization describes a \emph{motion}. It tells us the position of a point as a function of time. As such, the parameterization gives more information than just the path.% of the motion.

For example, it is relatively easy to describe the path of the earth around the sun as an ellipse with one focus at the sun, having a certain shape and size. But it is much more difficult to parameterize the motion of the earth, as this would require knowing the position of the earth at any time in its orbit.

In this class we will parameterize several different types of motions:

a. motion at a constant velocity

\pskip

b. projectile motion in a uniform gravitational field

\pskip

c. uniform circular motion

\pskip

d. motions around an ellipse

\pskip

This chapter focuses on the first, motion with a constant velocity. But we'll start with a review of transformations and how they apply to parametrically-defined curves.

\section{Transforming Parametrically-Defined Curves}
\begin{example} \label{Ex0}
The \emph{coordinate functions}
\[
     x = f(t) , \text{ for } t_0 \leq t \leq t_1
\]
and
\[
    y = g(t) , \text{ for } t_0 \leq t \leq t_1 ,
\]
express the coordinates (in cm) of a beetle crawling in the $xy$-plane in terms of the number of seconds past noon. The graphs of these functions are shown below. Note that the coordinate axes are playing several roles. So, for the graph of the coordinate function $x=f(t)$ (green), the input $t$ (along the horizontal axis) is measured in seconds and the output $x$ (along the vertical axis) is measured in centimeters. 



\begin{exploration}\label{exp:pc1c}
(a) Use the graphs below to appoximate the coordinates of four points on the path. Then use the graphs to sketch the path. Explain your reasoning.

Then check your work by clicking on the circle at the beginning of Line 13 to see the path. 

\pskip

(b) Play the slider $u$ on Line 31 to see the motion of the beetle along its path. Use this to sketch by hand a graph of the speed $v$ of the beetle (in cm/sec) as a function of $t$. Do not include a scale on the vertical axis. Explain your reasoning.


\pdfOnly{
Access Desmos interactives through the online version of this text at
 
\href{https://www.desmos.com/calculator/xmxuyctaib}.
}
 
\begin{onlineOnly}
    \begin{center}
\desmos{tetwsj8rki}{900}{600}
\end{center}
\end{onlineOnly}
\end{exploration}

We'll now define a new motion for an ant by transforming the beetle's coordinate functions, as below:
\[
   x = f_1(t) = a_1 + b_1 f(k(t-c_1))
\]
and
\[
   y =g_1(t) = a_2 + b_2 g(k(t-c_1)) .
\]

\begin{question}
What are the units of the constants $a_1$, $b_1$, $k$, and $c_1$? Explain your reasoning.
\end{question}

\begin{question}
Find the domain of the coordinate functions. Assume $k>0$.
\end{question}

Now we'll explore the effects the constants $a_1$, ... $c_1$ have on the graphs of the coordinate function, the path, and the motion. Possible ways to describe these tranformations include statements like:

\begin{itemize}
\item{A transation in the direction of the postive $x$-axis by 4 ??, where the ?? need to be replaced with the appropriate units.}

\item{A dilation by a factor of two about the positive $t$-axis.}

\item{The motion is speeded up by a factor of two. So at corresponding points on the paths, the ant always moves twice as fast as the beetle.}

\end{itemize}



\begin{exploration}\label{exp:pc1c}
Turn on the path folder by clicking the circle at the beginning of Line 13 as before. 

\pskip

Experiment with the sliders on Lines 32-37, by changing one parameter at a time, in order. Describe what effect each parameter has on the graphs of the coordinate function, the path, and the motion. Explain \emph{why} each parameter has the effect it does.

Take your time. There is quite a bit to notice here.


\pdfOnly{
Access Desmos interactives through the online version of this text at
 
\href{https://www.desmos.com/calculator/xmxuyctaib}.
}
 
\begin{onlineOnly}
    \begin{center}
\desmos{tetwsj8rki}{900}{600}
\end{center}
\end{onlineOnly}
\end{exploration}


\end{example}




\section{Motion with a Constant Velocity}

In colloquial English, velocity is a synonym for speed, as in ``A 747 has a maximum velocity of 650 miles/hour." But this is incorrect, as velocity is properly a vector that describes both speed and direction. So we could describe the velocity of a car by saying that it moves due north at a speed of 60 miles/hour. For an object to move with a constant velocity means that it travels in a fixed direction at a constant speed.

\pskip

\begin{example}   \label{Ex1}
 Between 11:58 am and 12:20pm a beetle crawls with a constant velocity, passing the point $A(-3,8)$ at 12:01pm and the point $B(10,-1)$ at 12:08pm, where the coordinates are measured in yards. Parameterize the beetle's motion.
\end{example}

\begin{explanation}
The question is asking us to express the beetle's coordinates as functions of time, and we should start by precisely defining a variable that measures time. We'll do this by letting $t$ be the number of minutes past noon, so that for this problem $-2 \leq t \leq 20$. 

Now we need to find \emph{coordinate functions}
\[
   x = f(t) \text{ and } y = g(t) , \text{ for } -2\leq t \leq 20 ,
\]
that express the coordinates of the beetle, measured in yards, in terms of the number of minutes past noon.  Since the velocity is constant, the coordinate functions change at a constant rate and are therefore linear.

First the $x$-coordinate. The $x$-coordinate changes at the rate of 
\[
    \frac{\Delta x}{\Delta t} = \frac{10-(-3) \text{ yds}}{8-1 \text{ sec}} = \frac{13 \text{ yd}}{7 \text{ sec}}
\]
and (using point-slope) 
\[
     x = f(t) = -3 + \frac{13}{7}(t-1) , -2\leq t \leq 20 .
\]


\begin{question} 
Now make a similar computation for the $y$-coordinate function and input your expression in the box below. 
\[
 y=g(t) =  \answer{8-\frac{9}{7}(t-1)} 
\]
    \end{question}

In conclusion, the functions
\[
    x = f(t) = -3 + \frac{13}{7}(t-1) \text{ and }  y=g(t) = 8-\frac{9}{7}(t-1) , \text{ for } -2\leq t \leq 20 ,
\]
express the beetle's coordinates (measured in yards) in terms of the number of minutes past noon.

\end{explanation}

\begin{exploration}\label{exp:pc1}
The graphs of the beetle's coordinate functions $x=f(t)$ and $y=g(t)$, $-2\leq t \leq 20$, are shown below. Note the units and variable names on the axes. Drag the slider $s$ to see the coordinates of the beetle at different times during its journey.
 
\pdfOnly{
Access Desmos interactives through the online version of this text at
 
\href{https://www.desmos.com/calculator/0iruo192ir}.
}
 
\begin{onlineOnly}
    \begin{center}
\desmos{0iruo192ir}{900}{600}
\end{center}
\end{onlineOnly}
\end{exploration}

The next slide shows the motion of the beetle along its path.

\begin{exploration}\label{exp:pc1b}
Play the slider $s$ to see the beetle move along its path (the path is graphed in Line 4 and the motion in Line 5). Then click on the graph icon at the far left of Line 4 to hide the path. Play the slider again to see the beetle leave its tracks (coded in Line 6 - note the domain of the parameter $s$ there).
 
\pdfOnly{
Access Desmos interactives through the online version of this text at
 
\href{https://www.desmos.com/calculator/splcvc5xzq}.
}
 
\begin{onlineOnly}
    \begin{center}
\desmos{splcvc5xzq}{900}{600}
\end{center}
\end{onlineOnly}
\end{exploration}


Let's add another insect and suppose that between 11:58 am and 12:20pm an ant also crawls with a constant velocity, passing the point $C(-2,-10)$ at 12:03pm heading directly toward the point $D(30,6)$.

\pskip

\begin{example} \label{Ex2}

 Parameterize the motion of the ant to make it collide with the beetle.

\end{example}

\begin{explanation}
The solution to this problem requires several steps and to get started it helps to think backwards. We can parameterize the ant's motion if we know its position at two times. We already know where it is at 12:02pm, so we just need to find its position at some other time. And here is where the requirement that the insects collide comes into play. We'll find both where and when they collide, and then use this additional information to parameterize the ant's motion.

We can find where the insects collide by first finding equations of their paths. Then once we find where they collide, we can find the collision time by using the parameterization of the beetle's motion.


\begin{question}  
To get started, first find a Cartestian equation (expressing $y$ in terms of $x$) of the ant's path and enter it in the line below.
 \[
     y=a(x) =  \answer{-10+\frac{1}{2}(x+2)}
\] 
   \end{question}

Next find a Cartestian equation of the beetle's path and enter it in the line below.

\begin{question}  
         $ y=b(x) =  \answer{8-\frac{9}{13}(x+3)}$  
    \end{question}

{\bf Use algebra} to find the {\bf exact} coordinates of the collision point and enter these (as an ordered pair) in the line below.

\begin{question}  
         $ (x_0,y_0) =  \answer{(388/31 , -85/31)}$
    \end{question}

Then find the time $t_0$ (measured in minutes past noon) of collision and enter it in the line below.

\begin{question}  
         $ t_0 =  \answer{290/31}$
    \end{question}

Finally, find expressions $x=f_1(t)$ and $y=g_1(t)$ for the ant's coordinate functions. Enter these twice. Once in the boxes below and again in the Desmos Activity that follows.

\begin{question}  
         $ x = f_1(t) =   \answer{-2+\frac{388/31+2}{290/31-3}(t-3)}$

        \hskip 0.93 in $y=g_1(t) =  \answer{-10+\frac{-85/31+10}{290/31-3}(t-3)}$
    \end{question}


\begin{exploration}\label{exp:pc1c}
Enter the correct expressions for the ant's coordinate functions $x=f_1(t)$ and $y=g_1(t)$ in Lines 15 and 16 below. Then play the slider $s$ to make sure that the insects collide.

\pdfOnly{
Access Desmos interactives through the online version of this text at
 
\href{https://www.desmos.com/calculator/pjci6nssqv}.
}
 
\begin{onlineOnly}
    \begin{center}
\desmos{pjci6nssqv}{900}{600}
\end{center}
\end{onlineOnly}
\end{exploration}

\end{explanation}


\begin{example} \label{Ex3}

Parameterize the motion of a beetle given that between noon and 12:30pm it crawls with a constant speed of $1.7$ yds/min, that it always heads directly toward the point $A(-23,19)$, and that it passes the point $B(7,-1)$ at 12:23pm. 

\end{example}

\begin{explanation}
As before, we'll let $t$ be the number of minutes past noon. Our goal is to find functions
\[
   x = f(t) \text{ and } y = g(t), \text{ for } 0\leq t \leq 30 ,
\]
that express the beetle's coordinates (measured in yards) in terms of the number of minutes past noon.

Unlike the previous problems, we are given the position only at a single time. But we can use the other information to compute the rates (with respect to time) at which the beetle's coordinates change as it moves along its path. We'll then use these rates along with our knowledge of the beetle's position at 12:23pm to find expressions for the coordinate functions. 

The rates of change in the beetle's coordinates are determined by the speed and the direction of travel. We'll first find the time it takes to crawl from $A$ to $B$. 

In crawling from $B(7,-1)$ to $A(-23,19)$, the beetle crawls a distance of
\begin{align*}
  \Delta s  &= \sqrt{(-23 - 7)^2 +   (19 - (-1))^2}  \text{ yds} \\
      & = 10\sqrt{13} \text { yds} .
\end{align*}

And to travel this distance at the constant speed of $v = 1.7$ yards/min takes
\[
       \Delta t = \frac{\Delta s}{v}   = \frac{10 \sqrt{13} \text { yds}}{1.7 \text{ yds/min}} = \frac{10 \sqrt{13}}{1.7 } \text{ min} .
\]

During this time, its $x$-coordinate changes by
\[
   \Delta x = (-23 - 7) \text{ yds} = -30 \text{ yds} ,
\]
giving an average rate of change of
\begin{align*}
  \frac{\Delta x}{\Delta t} &= \frac{-30 \text{ yds}}{\frac{10 \sqrt{13}}{1.7} \text{ min}}  \\
                                     & = \left( 1.7 \text{ yds/min} \right) \left( \frac{-30 \text{ yds}}{10\sqrt{13} \text{ yds}} \right) \label{Eq:RateofChange} \\
                                     & = \frac{-5.1}{\sqrt{13}} \text{ yds/min}  
\end{align*}
in the $x$-coordinate. 

Then since the beetle passes the point $B(7,-1)$ at 12:23pm,  the $x$-coordinate function for the motion is
\[
     x = f(t) = 7 - \frac{5.1}{\sqrt{13}}\left( t-23 \right) , \text{ for } 0\leq t \leq 30 .
\]

It is worth pointing out that the rate of change in the $x$-coordinate
\[
    \frac{\Delta x}{\Delta t} = \left( 1.7 \text{ yds/min} \right) \left( \frac{-30 \text{ yds}}{10\sqrt{13} \text{ yds}} \right)
\]
is the product of the beetle's speed and the dimensionless ratio 
\[
     \frac{\Delta x}{\Delta s} = \frac{-30 \text{ yds}}{10\sqrt{13} \text{ yds}}
\]
of the change in the $x$-coordinate to the distance between $A$ and $B$. This ratio is much like the slope $\Delta y / \Delta x$ of the path, where the change $\Delta s$ is positive (or negative) when we move along the path in the direction (or in the opposite direction) of the motion. If you have taken trigonometry before, you might recognize the ratio $\Delta x/\Delta s$ as the cosine of the counterclockwise angle from the positive $x$-axis to the direction of motion. But much more on this later.


%It is this ratio that encodes the direction of travel and if you have taken trigonometry before, you might recognize this as the cosine of a certain angle (from the positive $x$-axis counterclockwise to the direction of motion). It is much like the slope $\Delta y / \Delta x$ of the path, but encodes more information as the slope does not distinguish between the two possible directions of travel along the path. But much more on this later. 

\begin{question}
 Find an expression for the $y$-coordinate function  
\[
 y = g(t) =   \answer{-1+\frac{3.4}{\sqrt{13}}( t-23)}
\]
of the motion.
    \end{question}

\end{explanation}


\pskip


\begin{example} \label{Ex4}
 Let $h>0$ be a constant. Bug A leaves the point $(0,h)$ at time $t=0$ seconds and moves with constant speed $v$ m/s in the direction of the positive $x$-axis along the line $y=h$. At time $t=0$, Bug $B$ leaves the origin and moves with a constant velocity and constant speed $w$ m/s. How can we parameterize the motion of Bug B to make it collide with Bug A?
\end{example}

\begin{explanation}
The solution has several steps. Let's first think about what these should be.

We are given the speed of Bug B and its initial position. So to completely describe the motion we need to determine its direction. We'll do this by finding the coordinates of the collision point. 

Since we can easily paramterize the motion of Bug A, it is enough to first determine the time of the collision. We'll start by defining an unknown, call it $t_0$, that measures the number of seconds after time $t=0$ when the bugs collide. The key now is to draw right triangle $\Delta OAC$, where $O$ is the origin, $A$ is the point with coordinates $(0,h)$, and $C$ is the collision point. 

We can express the lengths of the sides of $\Delta AOC$ in terms of the parameters $h$, $v$, $w$ and the unknown $t_0$. Then we can use the Pythagorean theorem to write an equation and solve for $t_0$.

\begin{question}
 Find an expression in terms of $h$, $v$, and $w$ for the time $t_0$ of the collision.  
\[
        t_0   =  \answer{\frac{h}{\sqrt{w^2-v^2}}}
\]
    \end{question}

Now that we have the collision time, we can use what we know about the motion of Bug A to find the collision point.

\begin{question}
 Find an expression in terms of $h$, $v$, and $w$ for the coordinates of the collision point.  
\[
       (x_0, y_0)  =  \answer{\left( \frac{vh}{\sqrt{w^2-v^2}},h \right)}
\]
    \end{question}


\begin{question}
Parameterize the motion of Bug $B$ in terms of $h$, $v$, $w$ and $t$. Assume $t\geq 0$.
\[
        x = f(t)   =  \answer{v_0 t}
\]
\[
      y = g(t) = \answer{\sqrt{w^2-v^2}t}
\]
    \end{question}

\begin{exploration}\label{exp:pc1c}
Enter the correct expression for the collision time in Line 10 below. Then enter the correct expressions for Bug B's coordinate functions $x=f(t)$ and $y=g(t)$ in Lines 11 and 12. Play the slider $s$ to make sure that the bugs collide.

\pdfOnly{
Access Desmos interactives through the online version of this text at
 
\href{https://www.desmos.com/calculator/xuhjbr4m5r}.
}
 
\begin{onlineOnly}
    \begin{center}
\desmos{xuhjbr4m5r}{900}{600}
\end{center}
\end{onlineOnly}
\end{exploration}

\end{explanation}

\begin{example}   \label{Ex5}
Let $h,x_0>0$ be constants. Bug A leaves the point $(0,h)$ at time $t=0$ seconds and moves with constant speed $v$ m/s in the direction of the positive $x$-axis along the line $y=h$. Bug $B$ leaves the origin some time later and moves with a constant speed $w$ m/s directly toward the point $(x_0,h)$.  How can we parameterize the motion of Bug B to make it collide with Bug A?
\end{example}

\begin{explanation}
The key is to determine \emph{not} when the bugs collide, but rather when Bug B leaves the origin. We'll start by defining an unknown, $t_0$, to be the time (measured in seconds since time $t=0$) when Bug B leaves the origin. 

Now use the Pythagorean theorem to write an equation that relates the variables. Then solve your equation to express $t_0$ in terms of the parameters $h$, $x_0$, $v$, and $w$.


\begin{question}
 Find an expression for the time $t_0$ (measured in seconds since time $t=0$) that Bug B leaves the origin. %in terms of $h$, $v$, and $w$ for the time $t_0$ of the collision.  
\[
        t_0   =  \answer{\frac{x_0}{v} - \frac{\sqrt{x_0^2+h^2}}{w}}
\]
    \end{question}


\begin{question}
Parameterize the motion of Bug $B$ for $t\geq t_0$.
\[
        x = f(t)   =  \answer{\frac{wx_0}{\sqrt{x_0^2+h^2}}\left( t - t_0  \right)}
\]
\[
      y = g(t) =  \answer{\frac{wh}{\sqrt{x_0^2+h^2}}\left( t - t_0  \right)}
\]
    \end{question}

\begin{exploration}\label{exp:pc1c}
Enter the correct expression for the collision time in Line 16 below. Then enter the correct expressions for Bug B's coordinate functions $x=f(t)$ and $y=g(t)$ in Lines 17 and 18. Play the slider $s$ to make sure that the bugs collide.

\pdfOnly{
Access Desmos interactives through the online version of this text at
 
\href{https://www.desmos.com/calculator/ooylgfbvtx}.
}
 
\begin{onlineOnly}
    \begin{center}
\desmos{ooylgfbvtx}{900}{600}
\end{center}
\end{onlineOnly}
\end{exploration}


\end{explanation}

\begin{example} {\bf Target Practice:}
A balloon with radius $r$ cm is centered at the point with coordinates $(a,b)$. You shoot an arrow from the origin at time $t=0$ seconds in the direction of the point $(c, \sqrt{1-c^2})$, where $-1\leq c \leq 1$. The arrow travels with constant speed $v$ cm/sec and constant velocity as it would in a video game that ignores the force of gravity. 

Our goal is to make the balloon pop when (and if) it gets hit by the arrow. 
\end{example}

\begin{explanation}
The key is to determine when (if ever) the arrow punctures the balloon. This will happen \emph{the first time} (if ever) the distance between the arrow and the center of the balloon is equal to the balloon's radius. To find this time, start by parameterizing the motion of the balloon. Then write an equation that says the distance between the arrow and the balloon is equal to the balloon's radius. Finally, solve this quadratic equation by completing the square and enter the first of the two times below. 

\begin{question}

Find an expression for the time $t_0$ (measured in seconds since time $t=0$) when the arrow (modeled as a point) punctures the balloon. 
\[
   t_0 = \answer{\frac{1}{v} \left( a c + b \sqrt{1-c^2}-\sqrt{r^2 - \left( a \sqrt{1-c^2} - b c \right)^2}\right)}
\]

\end{question}

\begin{exploration}\label{exp:pc1c}
Enter the correct expression for $t_0$ in Line 19 below. Play the slider $s$ to check the balloon pops at the correct time. Drag the slider $c$ or the center of the balloon so that the arrow misses the balloon. Then play the slider $s$ to check that the balloon does not pop. Use the slider to change the value of $v$ and make sure everything still works.

\pdfOnly{
Access Desmos interactives through the online version of this text at
 
\href{https://www.desmos.com/calculator/tz7fde0di2}.
}
 
\begin{onlineOnly}
    \begin{center}
\desmos{tz7fde0di2}{900}{600}
\end{center}
\end{onlineOnly}
\end{exploration}


\end{explanation}



\begin{example} {\bf Billiards:}
A problem for later in the course. 
\end{example}


\begin{question}
The cue ball moves with unit speed from the origin. Parameterize the motions of the target ball and the cue ball after the collision.
\end{question}

\begin{exploration}\label{exp:pc1c}
Play the slider $s$. Control the direction of the cue ball with the slider $\phi$. 
%Enter the motions of the balls in the desmos activity in Lines ??-?? below.

\pdfOnly{
Access Desmos interactives through the online version of this text at
 
\href{https://www.desmos.com/calculator/8p4pp5ri6k}.
}
 
\begin{onlineOnly}
    \begin{center}
\desmos{8p4pp5ri6k}{900}{600}
\end{center}
\end{onlineOnly}
\end{exploration}








\end{document}

