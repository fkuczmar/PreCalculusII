\documentclass{ximera}
\title{Non-uniform Circular Motion}

\newcommand{\pskip}{\vskip 0.1 in}

\begin{document}
\begin{abstract}
Circular Interpolation
\end{abstract}
\maketitle


\section{A Pendulum-Like Motion}

\begin{question} \label{Q9Adfdsf}

The polar angle $\theta$ of the pendulum below oscillates in simple harmonic motion between $\theta = 5\pi/4$ and $\theta  =7\pi/4$ with a period of $16$ seconds. 

\begin{onlineOnly}
    \begin{center}
\desmos{646ae3ac03}{900}{600}
\end{center}
\end{onlineOnly}

\href{https://www.desmos.com/calculator/646ae3ac03}{142: Pendulum Like Motion}

\begin{enumerate}
\item Sketch by hand a graph of the polar angle function 
\[
      \theta=a(t) \, , \, t\geq 0,
\]
that expresses the polar angle of the vector $\overrightarrow{OP}$ in terms of the number of seconds since the motion began.

\item Find possible expression for the polar angle function.

\item Use part (b) to parameterize the pendulum's motion by finding functions
\[
    x = g(t) \text{ and } y = h(t) \, , \, t\geq 0,
\] 
that express the coordinates of the pendulum (in meters) in terms of the number of seconds since the pendulum was released.

\item Enter your expressions for the coordinate functions in Lines 3 and 4 of the worksheet to check your work.
\end{enumerate}

\end{question}


\begin{question} \label{PFRDF}
Do the same for the motion below.

\begin{onlineOnly}
    \begin{center}
\desmos{kdbl8xlwrw}{900}{600}
\end{center}
\end{onlineOnly}

\href{https://www.desmos.com/calculator/kdbl8xlwrw}{142: Ferris Wheel Motion}

\end{question}

\end{document}