\documentclass{ximera}
\title{Uniform Circular Motion}

\newcommand{\pskip}{\vskip 0.1 in}

\begin{document}
\begin{abstract}
Parameterizing motion around a circle at a constant speed.
\end{abstract}
\maketitle



\begin{example}  \label{Exp89dfe94tf4}

A beetle crawls around the circle of radius $40$ meters centered at the origin at a constant speed of $8$ meters/min, starting from the point $(40,0)$ at noon and making $5$ complete revolutions before stopping.

\begin{enumerate}

\item Use the protractor below to estimate the beetle's coordinates at 12:12pm. Explain your reasoning.

\begin{onlineOnly}
    \begin{center}
\desmos{lbkveixdno}{900}{600}
\end{center}
\end{onlineOnly}

\href{https://www.desmos.com/calculator/lbkveixdno}{142: Radian Protractor 2C}

\item Find the beetle's exact coordinates at 12:12pm \emph{without} using a calculator.

\item Use a calculator to approximate the coordinates from part (c) and compare this estimate with your original estimate from part (b).

\end{enumerate}

\end{example}


\begin{example} \label{Edfhg]hghgdfg}

\begin{enumerate}

\item Find your exact coordinates after you stop. Do \emph{not} use a calculator.

\item Use the radian protractor below to approximate your coordinates after you walk $220$ meters.
 
\item Use your answer from part (a) and a calculator to find a better approximation of your coordinates. Compare this with your approximation from part (b).



\end{enumerate}

\end{example}


\end{document}