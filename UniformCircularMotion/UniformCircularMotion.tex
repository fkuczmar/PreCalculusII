\documentclass{ximera}
\title{Uniform Circular Motion}

\newcommand{\pskip}{\vskip 0.1 in}

\begin{document}
\begin{abstract}
Parameterizing motion around a circle at a constant speed.
\end{abstract}
\maketitle



\begin{example}  \label{Exp89dfe94tf4}

A beetle crawls around the circle of radius $40$ meters centered at the origin at a constant speed of $8$ meters/min, starting from the point $(40,0)$ at noon and making $5$ complete revolutions before stopping.

\begin{enumerate}

\item Use the protractor below to estimate the beetle's coordinates at 12:12pm. Explain your reasoning.

\begin{onlineOnly}
    \begin{center}
\desmos{lbkveixdno}{900}{600}
\end{center}
\end{onlineOnly}

\href{https://www.desmos.com/calculator/lbkveixdno}{142: Radian Protractor 2C}

\item Find the beetle's exact coordinates at 12:12pm \emph{without} using a calculator.

\item Use a calculator to approximate the coordinates from part (b) and compare this estimate with your original estimate from part (a).

\end{enumerate}

\end{example}


\begin{example} \label{Edfhg]hghgdfg}
A beetle crawls counterclockwise around the circle of radius $40$ meters centered at the origin at a constant speed of $8$ meters/min, starting from the point $(40,0)$ (coordinates in meters) at noon and making $5$ complete revolutions before stopping.

\begin{enumerate}

\item Find a function
\[
    \theta = b(t)
\]
that expresses the beetle's polar angle in terms of the number of minutes past noon. Include the appropriate domain.

The polar angle function is
\[
   \theta = b(t) = \answer{t/5} \, , \, 0 \leq t \leq \answer{50\pi} .
\]

\item At what rate does the beetle turn about the origin?  $\answer{1/5}$ rad/min



\item Find functions
\[
   x = f_1(t) \text{ and } y=g_1(t)
\]
that express the beetle's coordinates (in meters)  in terms of the number of minutes past noon. Include the appropriate domains.

The coordinate functions are
\[
    x = f_1(t) = \answer{40\cos(\frac{1}{5}t)} \, , \, 0 \leq t \leq \answer{50\pi} 
\]
and 
\[
    y = f_1(t) = \answer{40\sin(\frac{1}{5}t)} \, , \, 0 \leq t \leq \answer{50\pi} .
\]

\item Enter these coordinate functions on Lines 3 and 4 of the worksheet below. Then play the slider $u$ (the number of minutes past noon) to check if they are correct.

\begin{onlineOnly}
    \begin{center}
\desmos{pygmcbcogi}{900}{600}
\end{center}
\end{onlineOnly}

\href{https://www.desmos.com/calculator/pygmcbcogi}{142: Beetle 1}



An ant crawls \emph{clockwise} about the same circle at the constant speed of $5$ meters/min. It starts from the point $(0,-40)$  (coordinates in meters) and stops when the beetle does.

\item Find a function
\[
    \theta = a(t)
\]
that expresses the ant's polar angle in terms of the number of minutes past noon. Include the appropriate domain.

The polar angle function is
\[
   \theta = a(t) = \answer{t/8} \, , \, 0 \leq t \leq \answer{50\pi} .
\]

\item At what rate does the ant turn about the origin?  $\answer{1/8}$ rad/min



\item Find functions
\[
   x = f_2(t) \text{ and } y=g_2(t)
\]
that express the ant's coordinates (in meters)  in terms of the number of minutes past noon. Include the appropriate domains.

%The coordinate functions are
%\[
%    x = f_2(t) = \answer{40\cos(\frac{1}{8}t)} \, , \, 0 \leq t \leq \answer{50\pi} 
%\]
%and 
%\[
 %   y = f_2(t) = \answer{40\sin(\frac{1}{8}t)} \, , \, 0 \leq t \leq \answer{50\pi} .
%\]

\item Enter these coordinate functions on Lines 5 and 6 of the worksheet above. Then play the slider $u$ (the number of minutes past noon) to check if they are correct.

\item When do the insects meet for the first time. Give an exact time Do this \emph{without} relying on the animation and \emph{without} using a calculator. Then use a calculator to approximate this time and use the animation above as a check.

\item \emph{Where} do the insects meet for the first time. Give \emph{exact} coordinatges \emph{without} relying on the animation and \emph{without} using a calculator. Then use a calculator to approximate the coordinates and use the animation above as a check.

\item After the first meeting, how often do the insects meet? Give an exact time interval \emph{without} relying on the animation and \emph{without} using a calculator. Then use a calculator to approximate the time interval and use the animation above as a check.

\item Find a function 
\[
    t = h(n)
\]
that gives the exact time when the bugs meet for the $n$th time. Include the appropriate domain.

\item Find an expression for the \emph{counter} function 
\[
      n = c(t) \, 0\leq t \leq 50\pi , 
\]
that records the number of times the bugs meet between noon and time $t$ minutes past noon. Enter this on Line 9.

\end{enumerate}
\end{example}


\section{Discussion Questions}

\begin{question}  \label{Qrtt54tghhhgfd}
Between 12:30pm and 1:00pm A beetle crawls along the $x$-axis in the positive direction at a constant speed of $5$ meters/min. It passes the point $(8,0)$ at 12:41pm. 

\begin{enumerate}
\item Find the coordinates of the beetle at 12:52pm \emph{without} finding its coordinates at noon (which is not possible anyway).

\item Graph the function 
\[
    x = f(t) \, , \, 30\leq t \leq 60 ,
\] 
that expresses the beetle's $x$-coordinate as a function of the number of minutes past noon.

\item Find expressions for the functions
\[
    x = f(t) \, , \, 30\leq t \leq 60 ,
\] 
and 
\[
    y = g(t) \, , \, 30\leq t \leq 60 ,
\] 
that express the beetle's coordinates as a functions of time.

\item Enter the expressions for your coordinate functions in the worksheet below. Then play the slider $u$ (another name for $t$) to watch the motion.

\end{enumerate}

\begin{onlineOnly}
    \begin{center}
\desmos{bxq8ehlo0t}{900}{600}
\end{center}
\end{onlineOnly}

\href{https://www.desmos.com/calculator/bxq8ehlo0t}{142:Beetle on X axis}


\end{question}






\begin{question}  \label{Q9dfrtfg94tf4}

Between 12:15pm and 1:00pm a beetle crawls counterclockwise around the circle of radius $20$ meters centered at the origin at a constant speed of $4$ meters/min. It passes the point $(0,-30)$ at 12:17 pm.

\begin{enumerate}

\item Use the protractor below to estimate the beetle's coordinates at 12:37pm. 

\begin{onlineOnly}
    \begin{center}
\desmos{lbkveixdno}{900}{600}
\end{center}
\end{onlineOnly}

\href{https://www.desmos.com/calculator/lbkveixdno}{142: Radian Protractor 2C}

\item Find the beetle's exact coordinates at 12:37pm.
%\item Use a calculator to approximate the coordinates from part (b) and compare this estimate with your original estimate from part (a).

\item Graph a possible function
\[
 \theta = a(t) \, , \, 15\leq t \leq 60 ,
\]
that expresses the beetle's polar angle (measured in radians) in terms of the number of minutes past noon. Label the axes with the appropriate variable names and units.

\item Find an expression for the polar angle function $\theta = a(t)$.

\item Find functions
\[
     x = f(t) \, , \, 15 \leq t \leq 60,
\]
and
\[
     y = g(t) \, , \, 15 \leq t \leq 60,
\]
that express the beetle's coordinates (in meters) in terms of the number of minutes past noon.
\end{enumerate}

\end{question}



\begin{question}  \label{QLDFg4tbbfg}
Between 12:15pm and 1:00pm a beetle crawls clockwise around a circle centered at the origin at a constant speed. It passes the point $(-40,40)$ (coordinates in cm) at 12:40 pm. The beetle completes one revolution every $5$ minutes.

\begin{enumerate}
\item Graph a possible function
\[
 \theta = a(t) \, , \, 15\leq t \leq 60 ,
\]
that expresses the beetle's polar angle (measured in radians) in terms of the number of minutes past noon. Label the axes with the appropriate variable names and units.

\item Find an expression for the polar angle function $\theta = a(t)$.

\item Find functions
\[
     x = f(t) \, , \, 15 \leq t \leq 60,
\]
and
\[
     y = g(t) \, , \, 15 \leq t \leq 60,
\]
that express the beetle's coordinates (in cm) in terms of the number of minutes past noon.

\end{enumerate}

\end{question}

\end{document}