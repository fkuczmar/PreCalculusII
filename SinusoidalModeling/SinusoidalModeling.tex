\documentclass{ximera}
\title{Sinusoidal Modeling}

\newcommand{\pskip}{\vskip 0.1 in}

\begin{document}
\begin{abstract}
Using the cosine function to model osciallatory behavior.
\end{abstract}
\maketitle

\begin{example} \label{Efdgu7u67}
Suppose that over the course of a 24-hour period, from midnight October 29 to midnight October 30, the depth of the water at the Edmonds Pier is a sinusoidal function of time. Suppose also that a high tide of 21 feet occurs at 2:00am and the following low tide of 5 feet occurs at 8:00am. 


\begin{explanation}
Our aim is to find a function
\[
    h = f(t) , 0\leq t \leq 24, 
\]
that expresses the depth of the water (in feet) in terms of the number of hours past midnight, October 29. 

\pskip

Note that to say ``sinusoidal function'' means that the graph of $f$ is generated by uniform circular motion. But the graphs of the sine and cosine functions are both generated this way, so it is ok to express $f(t)$ in terms of the cosine function, and we will do just that.

Here are the steps.


(a) First we'll uUse the information above to sketch by hand a graph of the function $f$. Label the axes with the appropriate variable names and units. Label the coordinates of two key points on the graph.

Desmos activity available at:

\href{https://www.desmos.com/calculator/x2kocpkcfm}{142: Edmonds Pier}.

 
\begin{onlineOnly}
    \begin{center}
\desmos{x2kocpkcfm}{900}{600}
\end{center}
\end{onlineOnly}



(b) Next we'll compute the mean depth of the water over the 24 hours and the maximum deviation of the depth from this mean. Including units in our computation, the mean depth is 
\[
    h_{avg} = 0.5 ( 21 \text{ ft } + 5 \text{ ft }) = 13 \text{ ft} . 
\]
And the maximum deviation from the mean is
\[
     21 \text{ ft } - 13 \text{ ft } = 8 \text{ ft}.
\]

Activate the Amplitude folder on Line 5 of the above demonstration to draw the horizontal line showin the average depth. Note this line is labeled with its equation. There is also a vertical line that shows the the maximum deviation from the mean.

(c) Next we'll use the graph to find the period of oscillation. Since high tide occurs at 2:00am and low tide at 8:00am, the period (the time between succesive) high (or low) tides is
\[
    2(8 \text{ hours } - 2 \text{ hours }) = 12 \text{ hours}
\]
Activate the Period folder on Line 11 to show the period on the graph.

(d) Now we'll compute the rotation rate of a uniform circular motion that generates the sinusoidal variation in the depth of the water. This rotation rate is
\[
  \omega = \frac{2\pi \text{ radians}}{12 \text{ hours}} = \frac{\pi}{6} \text{ radians/hr} .
\]
Activate the Uniform Circular Motion folder on Line 18 to show the period on the graph.

(e) Next we'll use parts (a)-(d) to find an expression for the function 
\[
   h = f(t) ,  0\leq t \leq 24, 
\]
that gives the depth of the water (in feet) at time $t$ hours past midnight. Use the cosine function. Keeping in mind that high tide occrs at 2:00am, our function is 
\[
   h  = f(t) = 13 + 8 \cos \left(  \frac{\pi}{6} \left( t - 2 \right) \right),  0\leq t \leq 24.
\]



(f) To check that our function is correct, we'll use the given information that the depth of the water is $21$ feet at 2:00am and $5$ feet at 8:00am.

Substituting $t=2$ gives the depth at 2:00am (in feet) as
\begin{align*}
   f(2)   & = 13 + 8 \cos \left(  \frac{\pi}{6} \left( 2 - 2 \right) \right) \\ 
           & = 13 + 8 \cos (0)  \\ 
           & = 13 + 8 \\
           & = 21 .
\end{align*}

Substituting $t=2$ gives the depth at 2:00am (in feet) as
\begin{align*}
   f(8)   & = 13 + 8 \cos \left(  \frac{\pi}{6} \left(8 - 2 \right) \right) \\ 
           & = 13 + 8 \cos (\pi)  \\ 
           & = 13 + 8(-1) \\
           & = 5 .
\end{align*}
These check out.

(g) Finally, we'll use our function to estimate the depth (in feet) of the water at 5:30am, October 29 to be
\begin{align*}
   f(5.5)   & = 13 + 8 \cos \left(  \frac{\pi}{6} \left(5.5 - 2 \right) \right) \\ 
           & = 13 + 8 \cos (7\pi / 12)  \\ 
           &  \sim  10.92 \\
         \end{align*}

\begin{question} \label{Qdgtr6y7y}
Enter the coordinates of the appropriate point in Line 27 of the Desmos Activity above to check that the depth at 5:30am is reasonable.
\end{question}
\end{explanation}

\end{example}

\end{document}