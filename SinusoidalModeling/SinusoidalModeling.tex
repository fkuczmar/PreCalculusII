\documentclass{ximera}
\title{Sinusoidal Modeling}

\newcommand{\pskip}{\vskip 0.1 in}

\begin{document}
\begin{abstract}
Using the cosine function to model osciallatory behavior.
\end{abstract}
\maketitle

\begin{example} \label{Efdgu7u67}
Suppose that over the course of a 24-hour period, from midnight October 29 to midnight October 30, the depth of the water at the Edmonds Pier is a sinusoidal function of time. Suppose also that a high tide of 21 feet occurs at 2:00am and the following low tide of 5 feet occurs six hours later. 


\begin{explanation}
Our aim is to find a function
\[
    h = f(t) , 0\leq t \leq 24, 
\]
that expresses the depth of the water (in feet) in terms of the number of hours past midnight, October 29. 

\pskip

Note that to say ``sinusoidal function'' means that the graph of $f$ is generated by uniform circular motion. But the graphs of the sine and cosine functions are both generated this way, so it is ok to express $f(t)$ in terms of the cosine function, and we will do just that.

Here are the steps.


(a) First we'll uUse the information above to sketch by hand a graph of the function $f$. Label the axes with the appropriate variable names and units. Label the coordinates of two key points on the graph.

Desmos activity available at:

\href{https://www.desmos.com/calculator/x2kocpkcfm}{142: Edmonds Pier}.

 
\begin{onlineOnly}
    \begin{center}
\desmos{x2kocpkcfm}{900}{600}
\end{center}
\end{onlineOnly}



(b) Compute the mean depth of the water over the 24 hours and the maximum deviation of the depth from this mean. Include all units in your computations. Draw on your graph the horizontal line that shows the average depth. Label this line with its equation. Draw also a vertical line that shows the the maximum deviation from the mean.

(c) Use the graph to find the period of oscillation. Then compute the rotation rate of a uniform circular motion that generates the sinusoidal variation in the depth of the water. Include all units.

(d Use parts (a)-(c) to find an expression for the function 
\[
   h = f(t) ,  0\leq t \leq 24, 
\]
that gives the depth of the water (in feet) at time $t$ hours past midnight. Use the cosine function.

(e) Check that your function is correct by using the information given in the problem.

(f) Use your function to estimate the depth of the water at midnight, October 29 to the nearest tenth of a foot.

\end{explanation}

\end{example}

\end{document}