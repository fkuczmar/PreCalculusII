\documentclass{ximera}
\title{Practice Quiz 1}

\newcommand{\pskip}{\vskip 0.1 in}

\begin{document}
\begin{abstract}
First practice quiz, Weeks 1-2
\end{abstract}
\maketitle

%\section{Directions}
\emph{Directions:}

\begin{enumerate}
\item Show all work.

\item Give brief explanations for each problem. Include these explanations in the flow of the solution.

\item Show all units in all computations.
\end{enumerate}


\section{Part 1}

\begin{question}  \label{Qdgfg4thhnn}
Explain what it means to measure in angle in radians. Include a picture to help with your explanation.
\end{question}

\begin{question}  \label{Q5543hghhjmmm}
 As you run around a circular track with radius $30$ meters you turn about the track's center at the constant rate of $0.2$ rad/sec.

\begin{enumerate}
\item At what speed are you running?

\item Explain the \emph{logic} behind your computation in part (a).

\end{enumerate}
\end{question}


\begin{question}  \label{Qcvfg44f}
Explain the difference between \emph{speed} and \emph{linear speed}.
\end{question}

\begin{question}  \label{Qghhybyhyy}
Use a four-function calculator (if necessary) and the radian protractor below to approximate each of the following.

\begin{enumerate}
\item $\cos(-4.3)$

\item $\sin(5.2)$

\item $\sin(90)$

\item $\cos(\pi^\circ)$

\end{enumerate} 

\begin{onlineOnly}
    \begin{center}
\desmos{lbkveixdno}{900}{600}
\end{center}
\end{onlineOnly}

\end{question}


\section{Part 2}

\begin{question}  \label{Q9dfe94tf4}

Between noon and 1:00 pm a beetle crawls counterclockwise around the circle of radius $20$ meters centered at the origin at a constant speed of $4$ meters/min. It passes the point $(0,-30)$ at 12:17 pm.

\begin{enumerate}

\item Use the protractor below to estimate the beetle's coordinates at 12:37pm. 

\begin{onlineOnly}
    \begin{center}
\desmos{lbkveixdno}{900}{600}
\end{center}
\end{onlineOnly}

\href{https://www.desmos.com/calculator/lbkveixdno}{142: Radian Protractor 2C}

\item Find the beetle's exact coordinates at 12:37pm.
%\item Use a calculator to approximate the coordinates from part (b) and compare this estimate with your original estimate from part (a).

\end{enumerate}

\end{question}

\begin{question}  \label{Qghggnmnree}
Let $Q$ be the point on the circle of radius $50$ meters that is $90$ meters from the point $A(50,0)$  (coordinates measured in meters). The distance is \emph{measured counterclockwise around the circle} from $A$ to $B$. 

Between 12:03pm and 1:00pm a beetle crawls clockwise around this circle at a constant speed of $7$ meters/min, passing the point $Q$ at 12:13pm.

\item Find functions 
\[
    x = f(t)
\]
and 
\[
 y = g(t)
\]
that express the beetle's coordinates (measured meters) as a function of the number of minutes past noon. Include a domain for 
each function.

\begin{explanation}
\begin{enumerate}
\item The first step is to find a function
\[
     \theta = a(t) \, , \, 3\leq t \leq 60,
\]
that expresses the beetle's polar angle (measured in radians) in terms of the number of minutes past noon. This really amounts to using the point-slope equation of a line, but we'll go through the logic in detail.

At 12:13pm we can measure the polar angle as
\[
  \theta = a(13) = \frac{90\text{ meters}}{50 \text{ meters}} = 1.8 \text{ radians}.
\]

Now the beetle rotates about the origin at the \emph{signed} rate of 
\[
  \omega = - \frac{7 \text{ meters/min}}{50\text{ meters}} = -0.14 \text{ rad/min} .
\]
The \emph{signed} rate is negative because as the beetle turns \emph{clockwise} about the origin its polar angle \emph{decreases}.

So if we wanted to find, for example, the polar angle at 12:30pm, we would first find the change
\[
  \Delta \theta = a(30) - a(13)
\] 
in the polar angle going forward in time from 12:13pm to 12:30pm. Since the time interval has length 
\[
   \Delta t = 30\text{ min} - 13\text{ min} = 17\text{ min} ,
\]
\[
      \Delta \theta = \left( -0.14 \text{ rad/min}  \right)\left( 17 \text{ min} \right) = -2.38\text{ rad} .
\]
Adding this change to the polar angle $a(13) = 1.8$ radians at 12:13pm gives us the polar angle (in radians)
\[
   \theta = a(30) = 1.8 - 2.38 = -0.58
\]
at 12:30pm.

In other words (and omitting the units), 
\begin{align*}
       a(\textcolor{red}{30}) &= a(13) - 0.14(\textcolor{red}{30} - 13)   \\
                &= 1.8  - 0.14(\textcolor{red}{30} - 13) .
\end{align*}

The same logic tells us the polar angle function is
\begin{align*}
 \theta &= a(\textcolor{red}{t}) \\
           &=   a(13) - 0.14(\textcolor{red}{t} - 13)   \\
            &= 1.8  - 0.14(\textcolor{red}{t} - 13)   \, , \, 3\leq t \leq 30 .
\end{align*}

\item Now that we have the polar angle function, it's quick work to find the coordinate functions 
\[
       x = f(t) \hskip 0.2in  \text{and} \hskip 0.2in  y=g(t) \, , \, 3\leq t \leq 30 ,
\] 
that express the beetle's position (in meters) in terms of the number of minutes past noon.

Since 
\[
    \cos\theta = \frac{x}{r} = \frac{x}{50} ,
\]
\begin{align*}
    x        &= 50\cos\theta \\
       &= 50\cos \left( 1.8  - 0.14(t - 13) \right)  \, , \, 3\leq t \leq 30  .
\end{align*}

Similarly,
\begin{align*}
    y        &= 50\sin\theta \\
       &= 50\sin \left( 1.8  - 0.14(t - 13) \right)  \, , \, 3\leq t \leq 30  .
\end{align*}

\item We can check all this by playing the slider $u$ in Line 4 in the demonstration below. Be sure to check that the beetle has the correct position at 12:13pm and that the rotation rate is correct.

\begin{onlineOnly}
    \begin{center}
\desmos{h0ojebdzqy}{900}{600}
\end{center}
\end{onlineOnly}

\href{https://www.desmos.com/calculator/h0ojebdzqy}{142: Uniform Circular Motion 8}


\end{enumerate}
\end{explanation}


\end{question}


\begin{question}  \label{Q9dfdfhhhfe94tf4}

Between noon and 12:16 pm an ant and a beetle crawl around the circle of radius $40$ meters centered at the origin.  

The graphs of the functions
\[
  \theta = a(t) \, , \, 0\leq t \leq 16
\]
and
\[
   \theta = b(t) \, , \, 0\leq t \leq 16
\]
expressing their respective polar angles (in radians) in terms of the number of minutes past noon are shown below.

\begin{onlineOnly}
    \begin{center}
\desmos{dilqmqqrt4}{900}{600}
\end{center}
\end{onlineOnly}

\href{https://www.desmos.com/calculator/dilqmqqrt4}{142: Ant and Beetle Polar Angles}


\begin{enumerate}

\item Use the graphs above and the slider $u$ to approximate the first time when the bugs pass one another. 

\item Use algebra to find the \emph{exact} time when the bugs first pass one another.

\item Find \emph{where} the bugs pass one another for the first time. Give the exact coordinates..

\end{enumerate}

\end{question}


\begin{question} \label{Qdgbhyy7777}
\begin{enumerate} 
\item Use the animation below to approximate the first two times after 7:00pm when the minute and hour hands of a clock are perpendicular.

\begin{onlineOnly}
    \begin{center}
\desmos{rdqfuflcmq}{900}{600}
\end{center}
\end{onlineOnly}

\href{https://www.desmos.com/calculator/rdqfuflcmq}{142: Clock 32}


\item Find a function 
\[
  \theta = f(t) \, , \, t\in \mathbb{R} ,
\]
that expresses the angle between the hands (measured in radians) in terms of the number of hours past 7:00pm.

\item Use your function from part (b) to determine the \emph{exact} times in part (a).
\end{enumerate}

\end{question}





\end{document}