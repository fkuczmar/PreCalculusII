\documentclass{ximera}
\title{Graphs of the Cosine and Sine Functions Classwork}

\newcommand{\pskip}{\vskip 0.1 in}

\begin{document}
\begin{abstract}
Going backwards.
\end{abstract}
\maketitle



\section{Going Forwards}

\begin{question} \label{QLDfef344333}
Let
\[
     y = f(t) = 4\sin\left( \frac{1}{3}t \right) \, , \, t\geq 0,  
\] 
where $y$ is measured in cm and $t$ in seconds.

\begin{enumerate}
\item What are the units of the factor $4$? How do you know?

\item What are the units of the factor $1/3$? How do you know?

\item What is the meaning of the factor $1/3$?

\item What is meaning of the factor $4$?

\end{enumerate}

The point here is that we can interpret the function 
\[
     y = f(t) = 4\sin\left( \frac{1}{3}t \right) \, , \, t\geq 0,  
\] 
as expressing the $y$-coordinate of a point moving counterclockwise around a circle centered at a point on the $y=0$ with polar angle function
\[
  \theta = a(t) = \frac{1}{3}t \, , \, t \geq 0.
\]

So to draw the graph, we can first picture the motion as illustrated below, where the slider $u$ is another name for $t$.


 
\begin{onlineOnly}
    \begin{center}
\desmos{ddidb9vl04}{900}{600}
\end{center}
\end{onlineOnly}

\href{https://www.desmos.com/calculator/ddidb9vl04}{142: Simple Harmonic Motion Universal G}

\item Use the motion above to help draw the graph of the function $y=f(t)$. Label the coordinates of at least three key points on the graph. Key points means either turning points or intercepts. Then activate the folder \emph{Graph} in Line 2 to see how you did.

\end{question}





\begin{question} \label{QLDfef432234333}
Let
\[
     x = f(t) =3+  4\cos\left( \frac{\pi}{3}t \right) \, , \, t\geq 0,  
\] 
where $x$ is measured in cm and $t$ in seconds.

\begin{enumerate}
\item What are the units of the factor $4$? How do you know?

\item What are the units of the factor $\pi/3$? How do you know?

\item What is the meaning of the factor $\pi/3$?

\item What is meaning of the factor $4$?

\item What about the term $3$?

\end{enumerate}

The point here is that we can interpret the function 
\[
     x = f(t) = 3  + 4\cos\left( \frac{1}{3}t \right) \, , \, t\geq 0,  
\] 
as expressing the $x$-coordinate of a point moving counterclockwise around a circle centered at a point on the $x=3$ with polar angle function
\[
  \theta = a(t) = \frac{\pi}{3}t \, , \, t \geq 0.
\]

So to draw the graph, we can first picture the motion as illustrated below, where the slider $u$ is another name for $t$.


 
\begin{onlineOnly}
    \begin{center}
\desmos{ipytya7ami}{900}{600}
\end{center}
\end{onlineOnly}

\href{https://www.desmos.com/calculator/ipytya7ami}{142: Simple Harmonic Motion Universal I}

\item Use the motion above to help draw the graph of the function $y=f(t)$. Label the coordinates of at least three key points on the graph. Key points means either turning points or intercepts. Then activate the folder \emph{Graph} in Line 2 to see how you did.

\end{question}






\section{Going Backwards}

\begin{exercise} \label{ELDKFe5ytef}
Assume each of the motions below are at a constant speed around a circle with a radius measured in meters. Assume also that time is measured in seconds.

For each motion do the following:

\begin{enumerate}

\item Start the motion by playing the slider $u$ (time, measured in seconds) in Line 1.

\item Sketch by hand a graph of the function $s=g(\theta)$ that expresses the $s$-coordinate of the point $P$ as a function of the marked angle $\theta$, measured counterclockwise in radians. Label the coordinates of the axes with the appropriate variable names and units.

\item Sketch by hand a graph of the function $s=f(t)$ that expresses the $s$-coordinate of the point $P$ as a function of time measured in seconds as recorded by the parameter $u$ in Line 1 of the worksheet. Label the coordinates of the axes with the appropriate variable names and units. Then activate the folder \emph{Graph} in Line 2 to see how you did


\item Using the graph of the function $s=f(t)$ shown in the worksheet, describe the uniform circular motion that drives the oscillation about the mean as follows.
 
\begin{enumerate}
\item Find the radius of the circular path.

\item Determine the rotation rate about the path's center.

\item Determine the $s$-coordinate of the path's center.

\end{enumerate}

\item Express the $s$-coordinate of the point $Q$ on the graph in terms of the marked polar angle $\theta$.

\item Express the polar angle $\theta$ in terms of $t$. Start by graphing the polar angle function $\theta = f(t)$.

\item Use the results of parts (b) and (c) to find an expression for the function $s=f(t)$. Check your work by substituting two of the given inputs.



\end{enumerate}

\begin{enumerate}

\item \href{https://www.desmos.com/calculator/ngsoxa0rwj}{142:Simple Harmonic Motion 34B}

%\href{https://www.desmos.com/calculator/o37ufo2ev5}{142:Simple Harmonic Motion 34}.


 
\begin{onlineOnly}
    \begin{center}
\desmos{ngsoxa0rwj}{900}{600}
\end{center}
\end{onlineOnly}


\item 

\href{https://www.desmos.com/calculator/xht0nibjy8}{142:Simple Harmonic Motion 45B}.

 
\begin{onlineOnly}
    \begin{center}
\desmos{xht0nibjy8}{900}{600}      %3t8bev0ify
\end{center}
\end{onlineOnly}

\item

\href{https://www.desmos.com/calculator/gpjavrbrwk}{142:Simple Harmonic Motion 37B}.

 
\begin{onlineOnly}
    \begin{center}
\desmos{gpjavrbrwk}{900}{600}
\end{center}
\end{onlineOnly}


\item 

\href{https://www.desmos.com/calculator/mjihwit0dn}{142:Simple Harmonic Motion UniversalC}.

\begin{onlineOnly}
    \begin{center}
\desmos{mjihwit0dn}{900}{600}
\end{center}
\end{onlineOnly}


\item 

\href{https://www.desmos.com/calculator/ox9gyiz2r4}{142:Simple Harmonic Motion UniversalD}.

\begin{onlineOnly}
    \begin{center}
\desmos{ox9gyiz2r4}{900}{600}
\end{center}
\end{onlineOnly}


\item 

\href{https://www.desmos.com/calculator/fulfprkp11}{142:Simple Harmonic Motion UniversalB}.

\begin{onlineOnly}
    \begin{center}
\desmos{fulfprkp11}{900}{600}
\end{center}
\end{onlineOnly}


\end{enumerate}

\end{exercise}



\end{document}

