\documentclass{ximera}
\title{Relative Motion}

\newcommand{\pskip}{\vskip 0.1 in}

\begin{document}
\begin{abstract}
An introduction to relative motion.
\end{abstract}
\maketitle

\section{Einstein's Railway Carriage}

I stand at the window of a railway carriage which is travelling uniformly, and drop a stone on the embankment without throwing it ....

From Albert Einstein's \it{Relativity: The Special and General Theory}, 1916.


\begin{exploration} \label{Edsss4tDE}

Access Desmos interactive at
 
\href{https://www.desmos.com/calculator/vyxsrjx0kp}{Railway Car}

 
\begin{onlineOnly}
    \begin{center}
\desmos{vyxsrjx0kp}{900}{600}
\end{center}
\end{onlineOnly}


\end{exploration}



\section{The Retrograde Motion of Mars}

\begin{exploration}  \label{Ed6t6yhh}
Assume for this problem that Earth and Mars rotate about the sun at constant rates in circular orbits, that Mars has an orbital period of 687 earth days and an orbital radius of $1.524$ astronomical units.

Assume the sun to be at the origin of a rectangular coordinate system....

Access Desmos interactive at
 
\href{https://www.desmos.com/calculator/6qjy0mzuis}{Mars Retrograde Motion}

 
\begin{onlineOnly}
    \begin{center}
\desmos{6qjy0mzuis}{900}{600}
\end{center}
\end{onlineOnly}


\end{exploration}


\section{Circles Rolling on Circles}

\begin{exploration}  \label{Edst4hnt}
(a) We first explore the composition of two circular motions.

Given the following information:
\begin{itemize}

\item{Point $C$ rotates counterclockwise  at the constant rate of $\omega_1$ rad/sec about a circle of radius $R$ centered at $O$.}

\item{Point $P$ rotates counterclockwise at the constant rate of $\omega_2$ rad/sec about a circle of radius $r_3$ centered at $C$.}


\item{Point $P$ has $xy$-coordinates $(R , 0)$ at time $t=0$ seconds past noon.}

\end{itemize}


We'll parameterize the motion of $P$ by expressing its coordinates in terms of the number of seconds past noon. Here's the method:

\begin{itemize}

\item{Find the components of the vector $\overrightarrow{OC}$ at time $t$ seconds past noon.}

\item{Find the components of the vector $\overrightarrow{CP}$ at time $t$ seconds past noon.}


%\item{What vector gives the position of $P$ relative to $O$?}

\item{Express the vector $\overrightarrow{OP}$ in terms of the vectors $\overrightarrow{OC}$ and $\overrightarrow{CP}$.}

%\item{Find the components of the vector $\overrightarrow{CP}$ at time $t$ seconds past noon.}

%\item{Find the components of the vector $\overrightarrow{OC}$ at time $t$ seconds past noon.}

\item{Then find the $xy$-coordinates of $P$ at time $t$ seconds past noon.}

\end{itemize}


Check that your parameterization is correct by inputing the coordinate functions in Lines 11 and 12 of the demonstration below. Then play the animation (Line 1). Vary the parameters and see how the trajectory of $P$ varies.

Access Desmos interactive at
 
\href{https://www.desmos.com/calculator/c2i8y3oaxe}{Epicycloid}

 
\begin{onlineOnly}
    \begin{center}
\desmos{c2i8y3oaxe}{900}{600}
\end{center}
\end{onlineOnly}


(b) Next we'll put double-rotation in the context of one circle rolling around another.

To see the circles, turn on the Circles folder in Line 14 of the animation.

\begin{itemize}

\item{The blue circle is centered at the origin $O$ and has radius $r_1$.}

\item{The red circle is centered at $C$ and has radius $r_2$.}
 
\item{The distance $R = |\overrightarrow{OC}|$ is equal to the sum $r_1+r_2$ of the two radii.}

\end{itemize}

Our goal is to make the red circle roll around the blue circle {\bf without} slipping. 

\begin{itemize}

\item{ Keep $r_1$, $r_2$, and $\omega_1=1$ all fixed. Then try to find the value of $\omega_2$ that makes the red circle roll around the blue circle without slipping. Gather some data, recording the values of $r_1$, $r_2$, and $\omega_2$.}

\item{Try to find a pattern in the data. That is, try to find non-slipping formula that expresses $\omega_2$ in terms of $r_1$ and $_2$ when $\omega_1=1$.}

\item{Experiment with the sliders, keeping $\omega_1=1$ fixed to check if your condition is correct.}

\item{Modify your expression for $\omega_2$ so that it works for any value of $\omega_1$. That is, find a non-slipping condition that expresses $\omega_2$ in terms of $r_1$, $r_2$, and $\omega_1$ (ie. to make the red circle roll without slipping.)}

\item{Experiment with the sliders to see if your condition is correct.}

\end{itemize}


\end{exploration}


\end{document}