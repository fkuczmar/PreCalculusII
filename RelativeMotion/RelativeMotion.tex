\documentclass{ximera}
\title{Relative Motion}

\newcommand{\pskip}{\vskip 0.1 in}

\begin{document}
\begin{abstract}
An introduction to relative motion.
\end{abstract}
\maketitle

\section{Einstein's Railway Carriage}

I stand at the window of a railway carriage which is travelling uniformly, and drop a stone on the embankment without throwing it ....

From Albert Einstein's \it{Relativity: The Special and General Theory}, 1916.


\begin{exploration} \label{Edsss4tDE}

Access Desmos interactive at
 
\href{https://www.desmos.com/calculator/vyxsrjx0kp}{Railway Car}

 
\begin{onlineOnly}
    \begin{center}
\desmos{vyxsrjx0kp}{900}{600}
\end{center}
\end{onlineOnly}


\end{exploration}



\section{The Retrograde Motion of Mars}

\begin{exploration}  \label{Ed6t6yhh}
Assume for this problem that Earth and Mars rotate about the sun at constant rates in circular orbits, that Mars has an orbital period of 687 earth days and an orbital radius of $1.524$ astronomical units.

Assume the sun to be at the origin of a rectangular coordinate system....

Access Desmos interactive at
 
\href{https://www.desmos.com/calculator/6qjy0mzuis}{Mars Retrograde Motion}

 
\begin{onlineOnly}
    \begin{center}
\desmos{6qjy0mzuis}{900}{600}
\end{center}
\end{onlineOnly}


\end{exploration}


\section{Circles Rolling on Circles}

\begin{exploration}  \label{Edst4hnt}
The radius of the blue circle is $r_1$ and the radius of the red circle is $r_2$.
 
The center $C$ of the red circle rotates counterclockwise about the origin $O$ of the $xy$-plane at the constant rate of $\omega_1$ rad/sec.

Relative to the $x^\prime y^\prime$ coordinate system, point $P$ rotates counterclockwise about the origin $C$ at the constant
of $\omega_2$ rad/sec. The distance $|\overrightarrow{CP}|$ is equal to $r_3$.

Ponit $P$ has $xy$-coordinates $(r_1 + r_2 , 0)$ at time $t=0$ seconds past noon.

We wish to parameterize the motion of $P$ relative to the $xy$-coordinate system.

We proceed as follows:

\begin{itemize}

\item{Find the components of the vector $\overrightarrow{CP}$ at time $t$ seconds past noon.}

\item{Find the components of the vector $\overrightarrow{OC}$ at time $t$ seconds past noon.}

%\item{What vector gives the position of $P$ relative to $O$?}

\item{Express the vector $\overrightarrow{OP}$ in terms of the vectors $\overrightarrow{OC}$ and $\overrightarrow{CP}$.}

%\item{Find the components of the vector $\overrightarrow{CP}$ at time $t$ seconds past noon.}

%\item{Find the components of the vector $\overrightarrow{OC}$ at time $t$ seconds past noon.}

\item{Then find the $xy$-coordinates of $P$ at time $t$ seconds past noon.}

\end{itemize}

Then experiment with the sliders keeping $\omega_1 = 1$ rad/sec. Observe the types of curves traced by $P$.

Then with $r_1$, $r_2$, and $\omega_1=1$ all fixed, try to find the value of $\omega_2$ that makes the red circle roll around the blue circle {\bf without slipping}. Gather some data and try to find a non-slipping formula that expresses $\omega_2$ in terms of $r_1$ and $_2$ when $\omega_1=1$. How would you modify this formula if $\omega_1$ were allowed to vary? That is, try to express $\omega_2$ in terms of $r_1$, $r_2$, and $\omega_1$ to make the red circle roll without slipping.

Access Desmos interactive at
 
\href{https://www.desmos.com/calculator/5b5u77otbu}{Epicycloid}

 
\begin{onlineOnly}
    \begin{center}
\desmos{5b5u77otbu}{900}{600}
\end{center}
\end{onlineOnly}

https://www.desmos.com/calculator/5b5u77otbu


\end{exploration}


\end{document}