\documentclass{ximera}
\title{Vectors}

\newcommand{\pskip}{\vskip 0.1 in}

\begin{document}
\begin{abstract}
An introduction to vectors in the plane.
\end{abstract}
\maketitle

\section{Position Vectors}

Position is relative. You describe where something is not in absolute terms but only in relation to something else.

We might say, for example, that Bellevue is 4 miles east and 3 miles north of Akron. Then in a rectangular coordinate system with the $x$-axis pointing due east, the $y$-axis pointing due north, and the coordinates measured in miles, we can represent the position of Bellevue (at point $B$) relative to Akron (at point $A$) by the vector
\[
   \overrightarrow{AB} = \langle 4, 3 \rangle .
\]
This vector tells how to get from Akron to Bellevue. 


Suppose that in our rectangular coordinate sytstem Charleston (point $C$) has coordinates $(-5,-2)$ and Dayton (point $D$) has coordinates $(-1,1)$. Then to get from Charleston to Dayton we would travel east
\[
     ( -1 - (-5) ) \text{ miles} = 4 \text{ miles}
\]
and north
\[
     ( 1 - (-2)) \text{ miles} = 3\text{ miles} .
\]
So we can represent the position of Dayton relative to Charleston by the vector
\[
  \overrightarrow{CD} =       \langle -1 - (-5) , 1- (-2)  \rangle   =  \langle 4, 3 \rangle .
\] 

So the directions that tell us how to get from Akron to Bellevue also tell us how to get from Charleston to Dayton. The
vectors $\overrightarrow{AB}$ and $\overrightarrow{CD}$ are equal and we write 
\[
      \overrightarrow{AB} = \overrightarrow{CD} .
\]

To go the other way and get from Bellevue to Akron or from Dayton to Charleston, we would go 4 miles west and 3 miles south. So the  position of Akron relative to Bellevue, or of Charleston relative to Dayton, is given by the vector
\[ 
   \overrightarrow{BA} = - \overrightarrow{AB} = - \langle 4,3 \rangle =  \langle -4,-3 \rangle.
\]

The distance between Akron and Bellvue is just the length of the vector $\overrightarrow{AB}$, written as
\[
    | \overrightarrow{AB}  | =  | \langle 3, 4 \rangle  | \text{ miles} = \sqrt{3^2 + 4^2} \text{miles} = 5 \text{ miles} .
\] 


\section{Adding and Subtracting Vectors}

\begin{question}    \label{Q234234:Vectors}
Continuing with our example above, suppose that Frankfort (point $F$) is $2$ miles east and $5$ miles south of Bellevue. Then the vector
\[
    \overrightarrow{BF}  = \langle \answer{2}, \answer{-5}\rangle
\]
gives the position of Frankfort relative to Bellvue. 
\end{question}

\begin{question} \label{Q233:Vectors}
(a) Give directions to get from Akron to Frankfort.

(b) Fill in the vector and its components that gives the position of Franfort relative to Akron.
\[
       \overrightarrow{\answer{A} \answer{F}}    =  \langle  \answer{6} , \answer{-2}  \rangle 
\]

(c) Which of the following are correct? Choose all that apply.
\begin{selectAll}  
\choice[correct]{$\overrightarrow{AF}   = \overrightarrow{AB} + \overrightarrow{BF}$}  
\choice{$\overrightarrow{AF}   = \overrightarrow{AB} - \overrightarrow{BF} $}  
\choice[correct]{$\overrightarrow{AF}   = \overrightarrow{BF} + \overrightarrow{AB} $}  
\end{selectAll} 


\end{question}


\begin{question}  \label{Q543:Vectors}
Let ${\bf v}$ be the vector that gives the position of point $K$ relative to point $G$ and ${\bf w}$ the vector that gives the position of $M$ relative to $K$. 

\begin{enumerate}

\item Express in terms of ${\bf v}$ and ${\bf w}$ each of the following vectors: 

\begin{enumerate}

\item The position of $M$ relative to $G$:  $\overrightarrow{\bf \answer{v}} +  \overrightarrow{\bf \answer{w}} $

\item The position of $G$ relative to $M$:  $\overrightarrow{\bf \answer{-v}} +  \overrightarrow{\bf \answer{-w}} $

\end{enumerate}

\item Draw vectors ${\bf v}$ and ${\bf w}$ tail-to-tail. Any choices will do, just make sure the vectors are not parallel. Then draw the vector ${\bf v} + {\bf w}$.

\item Draw vectors ${\bf v}$ and ${\bf w}$ tail-to-tail. Any choices will do, just make sure the vectors are not parallel. Then draw the vector ${\bf v} - {\bf w}$.

\item Explain the meaning of each of the following in the context of this scenario. Include units. 

\begin{enumerate}

\item $|  {\bf v} |$

\item $|  {\bf w} |$

\item $|  {\bf v+w} |$

\item $|  {\bf v} | + |  {\bf w} | - |  {\bf v+w} |$

\end{enumerate}

\item When would the expression in part (iv) above be positive? Negative? Zero? Explain your reasoning. 

\end{enumerate}

\end{question}


\begin{question}  \label{Qsdfsdt4r3:Vectors}
Suppose
\[
 \overrightarrow{AB} = \langle  -3,5 \rangle
\]
and
\[
   \overrightarrow{AC} = \langle  4,2 \rangle ,
\]
where the components are measured in miles.

(a) Give directions to get from $B$ to $C$.

(b) What vector gives the position of $C$ relative to $B$? Choose all that apply.
\begin{multipleChoice} 
 \choice[correct]{$\langle  7, -3  \rangle$}  
 \choice{$\langle  -7, 3  \rangle$}
\choice{$\overrightarrow{CB}$}  
\choice[correct]{$\overrightarrow{BC}$} 
\choice{$\overrightarrow{AB} -\overrightarrow{AC} $} 
\choice[correct]{$\overrightarrow{AC} -\overrightarrow{AB} $}
\end{multipleChoice} 

\end{question}





\begin{question}  \label{Q234r3:Vectors}
Let ${\bf v}$ be the vector that gives the position of point $Q$ relative to point $R$ and ${\bf w}$ the vector that gives the position of $S$ relative to $R$. 

(a) Which of the following express the position of $S$ relative to $Q$? Choose all that apply.

\begin{selectAll}  
\choice{${\bf v} + {\bf w}$}  
\choice{${\bf v} - {\bf w}$}  
\choice[correct]{${\bf w} - {\bf v}$}  
\end{selectAll}

(b) Draw a picture illustrating part (a). Include the vectors ${\bf v}$ and ${\bf w}$ tail-to-tail, and the vector that gives the position of $S$ relative to $R$. That's it. Do not draw a fourth vector like $-{\bf v}$ or $-{\bf w}$.


\end{question}



IGNORE THIS

\pdfOnly{
Access Desmos interactives through the online version of this text at
 
\href{https://www.www.geogebra.org/classic/e6rsvnsz}.
}
 
\begin{onlineOnly}
    \begin{center}
\geogebra{e6rsvnsz}{900}{600}
\end{center}
\end{onlineOnly}



\section{Computing Coordinates With Vectors}

\begin{question} \label{Q45340:Vectors}
(a) Explain why the circles with equations
\[
   (x-2)^2 + (y+1)^2  = 49
\]
and
\[
    (x-5)^2 + (y-3)^2 = 4
\]
are tangent to each other.

(b) Use vector arithmetic, not algebra, to determine the coordinates of the point of tangency.


\end{question}


\begin{question} \label{Q45323240:Vectors}
Points $A$ and $B$ have respective coordinates $(5,1)$ and $(17,-15)$. 

(a) Use vectors, not algebra, to determine the coordinates of the point $P$ on segment $\overline{AB}$ that is exactly 17 units from $A$.

(b) Use vectors, not algebra, to determine the coordinates of another point $P$ on the line through $A$ and $B$ that is exactly 17 units from $A$.

(c) Solve questions (a) and (b) simultaneously using algebra, not vectors.

\end{question}


\begin{question} \label{Q4df8240:Vectors}
(a) Use vector arithmetic, not algebra, to determine the coordinates of the centers of both circles with radius 6 that are tangent to the line
\[
    2x + 5y = 13
\]
at the point $P(-1,3)$.

(b) Use the method of part (a) to determine the coordinates of the centers of all circles with radius $r$ that are tangent to the same line at the same point.

(c) Follow the directions in the demonstration below to check your work.

\pdfOnly{
Access Desmos interactives through the online version of this text at
 
\href{https://www.desmos.com/calculator/aifh0gprxt}.
}
 
\begin{onlineOnly}
    \begin{center}
\desmos{aifh0gprxt}{900}{600}
\end{center}
\end{onlineOnly}

\end{question}


\begin{question}   \label{Q6547:Vectors}
(a) Use vectors, not trigonometry, to help find equations of the two lines that bisect the angles between the lines $y=x$ and $y=3x$.

(b) Use part (a) to find equations of the two lines that bisect the angles between the lines $y=x+3$ and $y=3x-5$.


\end{question}


\section{Position Vectors and Bearing}

Bearings are typically measured clockwise from due north. So you might report a fire to be $10$ miles from Miner's Ridge Lookout at a bearing of $100^\circ$. But we'll continue to meaure our angles counterclockwise (in radians) from the direction of the positive $x$-axis. We'll use polar angle and bearing interchangeably to mean the same thing.

\begin{example}  \label{Edsa5454v:Vectors}
Point $B$ is 4 miles from $A$ at a bearing of $3$ radians. Point $C$ is $7$ miles from $B$ at a bearing of $4.1$ radians. We wish to describe the position of $C$ relative to $A$ in two ways:

i) By giving north-south-east direction. For example, by saying that $C$ is $2$ miles east and $6$ miles south of $A$. We'll call this the NESW description.

ii) By giving a distance and a bearing. For example by saying that $C$ is $5$ miles from $A$ at a bearing (polar angle) of $5.3$ radians. We'll call this the {\bf distance-bearing} description.

Neither of these descriptions is correct for our problem. But we'll now determine the correct descriptions.

\begin{explanation}

(a) We'll start by using the demonstration below to draw a reasonably accurate picture. %Use the radian protractor and varying its radius $r$ with the slider and checking the correct boxes for vectors, we get the following picture


Access Geogebra interactives through the online version of this demonstation at
 
\href{https://www.geogebra.org/classic/bhdsgxtx}{142: Relative Position}.

 
\begin{onlineOnly}
    \begin{center}
\geogebra{bhdsgxtx}{900}{600}
\end{center}
\end{onlineOnly}


(b) We are given enough information about the vector $\overrightarrow{AB}$ to find its components (measured in miles) in the rectangular coordinate system above, where the unlabeled, positive $x$-axis points due east and the positive $y$-axis due north.

Because the distance from $A$ to $B$ is $r=  |\overrightarrow{AB}| = 4$ miles and $\overrightarrow{AB}$ is inclined at an angle of $3$ radians relative to the positive $x$-axis, the vector
\[
       \overrightarrow{AB} = \langle 4 \cos 3, 4 \sin 3\rangle .
\]
gives the position of $B$ relative to $A$. Its components are measured in miles.

Similarly, the vector
\[
   \overrightarrow{BC} = \langle 7 \cos 4.1, 7 \sin 4.1\rangle
\]
gives the position of $C$ relative to $B$ (in miles).

To find the position of $C$ relative to $A$, we add these vectors. So the position of $C$ relative to $A$ (in miles) is 
\begin{align*}
\overrightarrow{AC} &= \overrightarrow{AB} + \overrightarrow{BC} \\
                              &= \langle 4 \cos 3, 4 \sin 3\rangle + \langle 7 \cos 4.1, 7 \sin 4.1\rangle \\
                              & =    \langle 4 \cos 3 + 7 \cos 4.1 , 4 \sin 3 + 7 \sin  4.1\rangle   \\
                              & \sim \langle  -7.98 ,  -5.16  \rangle 
\end{align*}

So to get from point $A$ to point $C$ you would walk
\[
     (4 \cos 3 + 7 \cos 4.1) \text{  miles} \sim 7.98 \text{ miles} 
\]
due west and 
\[
   (4 \sin 3 + 7 \sin 4.1) \text{  miles} \sim 5.16 \text{ miles} 
\]
due south.

\pskip

(c) For the distance-bearing description of how to get from $A$ to $C$, we'll fiirst compute the exact distance between $A$ and $C$. This distance in miles is
\begin{align*}
   | \overrightarrow{AB}| &= |   \langle 4 \cos 3 + 7 \cos 4.1 , 4 \sin 3 + 7 \sin  4.1\rangle   |   \\ \\
                                   & = \sqrt{( 4 \cos 3 + 7 \cos 4.1)^2 + (4 \sin 3 + 7 \sin  4.1)^2}  \\ \\
                                  & =  \sqrt{16 + 49 + 56 (\cos(3) \cos(4.1) + \sin(3)\sin(4.1))}  \\ \\
                                  & \sim 9.46
\end{align*}

We do not yet know enough to get the exact bearing from $A$ to $C$ (we'll need to learn about inverse trig first), but for now we'll just use the radian protractor to estimate the bearing. 

Let's drag the protractor's center to $A$ and first check the distance between $A$ and $C$ by dragging the slider $r$ that controls the protractor's radius. Check that $r\sim 9.4$. Then read off the polar angle of the vector $\overrightarrow{AC}$ from the protractor. 

\begin{question}  \label{Q43656ff:Motion}
The bearing from $A$ to $C$ is about 
\begin{multipleChoice}  
\choice{$3$ radians}  
\choice[correct]{$3.7$ radians}  
\choice{$0.7$ radians}  
\end{multipleChoice} 

\end{question} 


So to get directly from $A$ to $C$ we should walk about $9.5$ miles at a bearing of approximately $3.7$ radians (actually closer to $3.6$ radians).

\end{explanation}


\end{example}


\begin{question} \label{Qdgg4tnh:Vectors}
Point $B$ is 5 miles from $A$ at a bearing of $2$ radians. Point $C$ is $8$ miles from $A$ at a bearing of $5$ radians. We wish to describe the position of $C$ relative to $B$.

\begin{enumerate}

\item First express the vector $\overrightarrow{BC}$ in terms of the vectors $\overrightarrow{AB}$ and $\overrightarrow{AC}$. Draw a picture to help with your explanation.

\item Then describe  the position of $C$ relative to $B$ in the two ways of Example \ref{Edsa5454v:Vectors}:

\begin{enumerate}
\item the NESW description (ie. the component form of a vector) and

\item the distance-bearing description (the polar form of a vector)

\end{enumerate}

Follow all the steps in Example \ref{Edsa5454v:Vectors}, making modifications as necessary. Create an accurate picture using the geogebra worksheek in Example \ref{Edsa5454v:Vectors}.

\end{enumerate}
\end{question}


\begin{question}  \label{Qdgbht66t554434}
Point $C$ has coordinates $(2,-3)\text{ meters}$. Point $B$ is $6$ meters from $C$ at a bearing of $4.2$ radians. We wish to determine the coordinates of $B$.

\begin{enumerate}
\item Let $O$ be the origin. Express the vector $\overrightarrow{OB}$ in terms of the vectors $\overrightarrow{OC}$ and $\overrightarrow{CB}$. 

\item Follow the steps of Example \ref{Edsa5454v:Vectors} to find the NESW description of the position of $B$ relative to $O$.

\item Use the result of part (b) to find the coordinates (exact and approximate) of point $B$.
\end{enumerate}
\end{question}


\begin{question}  \label{Qgtsygh:Vectors}
Starting from the point $A$ with coordinates $(-4 \text{ miles},-2 \text{ miles})$, you walk $8$ miles directly toward the point $Q(6,8)$ and stop at point $B$. Still facing point $Q$, you then turn clockwise through an angle of $7\pi/12$ radians, walk $10$ miles, and stop at point $C$. Describe the position of the origin relative to point $C$ in the same two ways as in Example 10. Follow all the steps in that example, making modifications as necessary. Use the {\bf exact} values for the trigonometric ratios (the two given bearings are both common angles).  Create an accurate picture using the geogebra worksheed in Example 10.


\end{question}

\section{Discussion Questions}

\begin{question}  \label{QPoDfdlLDCZXZ}
Points $A$ and $B$ have respective coordinates $A(-2,3)$ and $B(-4,-1)$, measured in meters.

Points $C$ and $D$ have respective coordinates $C(1,-2)$ and $D(-5,-14)$, measured in meters

\begin{enumerate}
\item Find the components of the vector that gives the position of $B$ relative to $A$.

\item Find the components of the vector that gives the position of $C$ relative to $D$.

\item Express the vector $\overrightarrow{CD}$ in terms of the vector $\overrightarrow{AB}$.

\item Express the vector $\overrightarrow{BA}$ in terms of the vector $\overrightarrow{CD}$.

\item Express $|\overrightarrow{BA}|$ in terms of $|\overrightarrow{CD}|$.
\end{enumerate}
\end{question}


\begin{question} \label{Q944tfghhnvbfc}
Points $A$, $B$, $C$ have respective coordinates $A(1,3)$, $B(5,-2)$, and $C(-6,1)$, measured in meters. Point $O$ is the origin.
\begin{enumerate}
\item Point $D$ is twice as far from $C$ as $A$ is from $B$. To get from $C$ to $D$ you walk in the same direction as you would to get from $A$ to $B$.

\begin{enumerate}

\item Express the vector that gives the position of $D$ relative to $C$ in terms of the vector $\overrightarrow{AB}$. Then find its components.

\item Express the vector $\overrightarrow{OD}$ in terms of the vectors $\overrightarrow{OC}$ and $\overrightarrow{CD}$.

\item Find the components of the vector $\overrightarrow{OD}$.

\item Find the coordinates of $D$.
\end{enumerate}

\item Point $E$ is $5$ meters from $C$. To get from $C$ to $E$ you walk in the \emph{opposite} direction as you would to get from $A$ to $B$. 

\begin{enumerate}
\item Repeat parts (i)-(iv) above to find the exact coordinates of $E$. Do \emph{not} use a calculator.

\item Then follow the directions in the worksheet below to see if you are correct.

\begin{onlineOnly}
    \begin{center}
\desmos{bk4xda8g08}{900}{600}
\end{center}
\end{onlineOnly}

\href{https://www.desmos.com/calculator/bk4xda8g08}{142: Vectors 23}

\end{enumerate}
\end{enumerate}

\end{question}


\begin{question} \label{QLdfreFefrer}
Points $O$, $A$, $Q$, $E$, and $G$ lie in a plane. Points $Q$ and $A$ are $20$ meters apart. Points $E$ and $G$ are $15$ meters apart. To get from $G$ to $E$ you walk in the same direction as you would to get from $A$ to $Q$.

\begin{enumerate}
\item Express the vector that gives the position of $E$ relative to $G$ in terms of the vetor $\overrightarrow{AQ}$.

\item Express the vector that gives the position of $E$ relative to $O$ in terms of the vectors $\overrightarrow{AQ}$ and $\overrightarrow{OG}$.
\end{enumerate}
\end{question}

\begin{question} \label{QLddferFefrer}
Points $O$, $A$, $Q$, $E$, and $G$ lie in a plane. Points $E$ and $G$ are $40$ meters apart. To get from $G$ to $E$ you walk in the same direction as you would to get from $A$ to $Q$.

\begin{enumerate}
\item Express the vector that gives the position of $E$ relative to $G$ in terms of the vetor $\overrightarrow{AQ}$.

\item Express the vector that gives the position of $E$ relative to $O$ in terms of the vectors $\overrightarrow{AQ}$ and $\overrightarrow{OG}$.
\end{enumerate}
\end{question}


\begin{question}  \label{QMNDVBed43}
Points $A$, $B$, $C$ have respective coordinates $(1,3)$, $(5,4)$, and $(6,1)$, measured in meters.
\begin{enumerate}
\item Find the components of the vectors $\overrightarrow{AB}$, $\overrightarrow{BC}$, and $\overrightarrow{AC}$.

\item Explain the meanings of the three vectors in part (a) in terms of relative position.

\item Express $\overrightarrow{AC}$ in terms of $\overrightarrow{AB}$, $\overrightarrow{BC}$. Draw a picture.
\end{enumerate}

\end{question}

\begin{question} \label{QDFEUJZXFedd}
Points $Q$, $T$, $W$ have respective coordinates $(-1,-3)$, $(5,4)$, and $(1,2)$, measured in meters.

\begin{enumerate}
\item Find the components of the vectors $\overrightarrow{QT}$, $\overrightarrow{QW}$, and $\overrightarrow{WT}$.

\item Explain the meanings of the three vectors in part (a) in terms of relative position.

\item Express $\overrightarrow{WT}$ in terms of $\overrightarrow{QT}$, $\overrightarrow{QW}$. Draw a picture.
\end{enumerate}

\end{question}


\begin{question}  \label{QodfderrewEWR}
Let $Q$, $R$, and $S$ be points in the plane.

\begin{enumerate}
\item What vector ${\bf v}$ expresses the position of $Q$ relative to $R$?

\item What vector ${\bf w}$ expresses the position of $S$ relative to $R$?

\item Express the position of $S$ relative to $Q$ in terms of the vectors ${\bf v}$ and ${\bf w}$.
\end{enumerate}
\end{question}

\begin{question}  \label{Q343g0dce}
Use the figure below to sketch the vectors ${\bf u}+{\bf v}$ and ${\bf u} - {\bf v}$.

\href{https://www.geogebra.org/classic/xunxnnyh}{142: Vector Arithmetic}.

 
\begin{onlineOnly}
    \begin{center}
\geogebra{xunxnnyh}{900}{600}
\end{center}
\end{onlineOnly}

\end{question}

\begin{question}  \label{QOKDlerdfsre}
Point $S$ and $T$ have respective coordinates $S(1,2)$ and $T(10,14)$, measured in meters.

An ant starts at $S$, crawls $7$ meters directly toward $T$ and stops at point $A$.

\begin{enumerate}
\item Find the components of the vector $\overrightarrow{SA}$.

\item Find the components of the vector $\overrightarrow{OA}$ from the origin to $A$.

\item Find the coordinates of $A$.
\end{enumerate}

\end{question}


\begin{question}  \label{QDfgegsE}
A beetle starts from the point $A(8,3)$, crawls $s$ meters counterclockwise around the circle centered at $Q(2,3)$ and stops at point $B$. All coordinates measured in meters.

\begin{enumerate}
\item Express the components of the vector $\overrightarrow{QB}$ in terms of $s$.

\item Express the vector $\overrightarrow{OB}$ in terms of the vectors $\overrightarrow{OQ}$ and $\overrightarrow{QB}$, where $O$ is the origin.

\item Express the coordinates of $B$ in terms of $s$.
\end{enumerate}
\end{question}


\begin{question} \label{QIDIFDfeD}
A beetle starts at point $A(3,5)$ (coordinates in meters), crawls $s$ meters around the circle centered at the origin $O$ and stops at $B$.

\begin{enumerate}
\item Express the vector $\overrightarrow{OB}$ in terms of $s$. No inverse trig.

\item Express the coordinates of $B$ in terms of $s$.
\end{enumerate}
\end{question}



\begin{question}  \label{Q9dfrwDSFD}
Point $A$ is $5$ meters due south of point $B$. Point $C$ lies $8$ meters from $A$ at a bearing of $3.8$ radians (the bearing is measured counterclockwise from the east).

Our goal is to describe the postion of $C$ relative to $B$ in two ways:
\begin{enumerate}
\item By giving the exact components of the relative position vector in a rectangular coordinate system with the positive $x$-axis pointing due east.

\item By giving the exact distance from $B$ to $C$ and an approximate bearing (measured counterclockwise from the east). For the bearing we will use the worksheet below. But it would be best to click the link just above the worksheet and work in Geogebra directly. That way you'll be able to zoom in and out.
\end{enumerate}



Here's the process:

\begin{enumerate}

\item The first step is to draw a reasonably accurate picture with the points $A$, $B$, and $C$ plotted in the rectangular coordinate system described above.

\item The next step is to draw the two given relative position vectors and the unknown relative position vector. Remember that vectors have arrows to indicate their directions.
%\item Next you should express the unknown vector in terms of the two given vectors. Do this \emph{without} using any of the given information. 

\item Now express the unknown vector giving the position of $C$ relative to $B$ in terms of the known vectors giving the positions of $A$ relative to $B$ and of $C$ relative to $A$. Do this \emph{without} using any of the given information.

\item Without using a calculator, find the exact components of three vectors in part (c).

\item Without using a calculator, find the exact distance between $B$ and $C$. Simplify your expression as much as possible.

\item Use the worksheet below to approximate the bearing  of the vector that gives the position of $C$ relative to $B$. Best to click on the link so that you can zoom in and out.

\href{https://www.geogebra.org/classic/bhdsgxtx}{142: Relative Position}.

 
\begin{onlineOnly}
    \begin{center}
\geogebra{bhdsgxtx}{900}{600}
\end{center}
\end{onlineOnly}
\end{enumerate}
\end{question}

\begin{question}  \label{Q9dferevdeewDSFD}
Point $A$ is $5$ meters due west of point $B$. Point $C$ lies $8$ meters from $B$ at a bearing of $5.5$ radians (the bearing is measured counterclockwise from the east).

Follow the steps in the previous question to describe the position of $C$ relative to $A$.  Use the worksheet below, or rather its link, to approximate the bearing of the vector that gives the position of $C$ relative to $A$.

\href{https://www.geogebra.org/classic/bhdsgxtx}{142: Relative Position}.

 
\begin{onlineOnly}
    \begin{center}
\geogebra{bhdsgxtx}{900}{600}
\end{center}
\end{onlineOnly}

\end{question}

\begin{question}  \label{QERERER435r}
Suppose in the previous question that $A$ has coordinates $(10,3)$ meters. Find the coordinates of $C$ and $B$.

\end{question}



\end{document}