\documentclass{ximera}
\title{Circles, Transformations, Ellipses}

\newcommand{\pskip}{\vskip 0.1 in}

\begin{document}
\begin{abstract}
Transformations
\end{abstract}
\maketitle


\section{Circles Not Centered at the Origin}
This section is about parameterizing circles centered at points other than the origin.

\begin{question}  \label{Ex:sdfdsfdfL}

Parameterize a circle with radius $3$ cm centered at the point $C(5,4)$ (coordinates measured in cm) in terms of the polar angle of the vector from $C$ to a point $P(x,y)$ on the circle.

\begin{onlineOnly}
    \begin{center}
\desmos{pcndel98wg}{900}{600}
\end{center}
\end{onlineOnly}

\href{https://www.desmos.com/calculator/pcndel98wg}{142: Translating Circles}


\begin{explanation} 

Here are the steps, with the details left for you to fill in.

We'll start by letting $P$ be a point on the circle with coordinates $(x,y)$ and $\theta$ the polar angle of the vector $\overrightarrow{CP}$ giving the position of $P$ relative to $C$.

\begin{enumerate}
\item First express the components of the vector $\overrightarrow{CP}$ in terms its polar angle $\theta$.

\item Next express the vector $\overrightarrow{OP}$ giving the position of $P$ relative to the origin in terms of $\overrightarrow{CP}$ and one other vector.

\item Use the results of parts (a) and (b) to express the components of the vector $\overrightarrow{OP}$ in terms of $\theta$.

\item Use the result of part (c) to find functions
\[
    x = f(\theta)  \hskip 0.5 in \text{and} \hskip 0.5in  y=g(\theta) ,
\]
that express the coordinates of $P$ in terms of $\theta$. Include appropriate domains for these functions. 

\item One last question. Find equations in Cartesian coordinates of 

\begin{enumerate}
\item the unit circle centered at the origin and 

\item The circle of radius $3$ centered at  $C(5,4)$.

These equations express relationships between the $x$ and $y$ coordinates of points on the circles. They do not involve the parameter $\theta$. 

\end{enumerate}
\end{enumerate}

\end{explanation}
\end{question}


\begin{question}  \label{QDF09dfdfLKDD}
\begin{enumerate}

\item Parametrize the circle in Question \ref{Ex:sdfdsfdfL} by the signed arclength from $A$ of a point $P(x,y)$ on the circle. This means to express the coordinates $(x,y)$ of $P$ (measured in cm) in terms of its signed distance from $A$. Measure the signed arclength $s$ (in cm)  from the point $A(8,0)$ along the circle and take the counterclockwise direction to be positive. 

\begin{onlineOnly}
    \begin{center}
\desmos{llrzcv5ckf}{900}{600}
\end{center}
\end{onlineOnly}

\href{https://www.desmos.com/calculator/llrzcv5ckf}{142: Circle Arclength Parameterization}



The coordinate functions (in cm) are 
\[
  x = f_1(s) = \answer{5 + 3 \cos(s/3)} \, , \, s\in \mathbb{R}
\]
and
\[
   y = g_1(s) = \answer{4 + 3 \sin(s/3)} \, , \, s\in \mathbb{R} .
\]

\item Use part (a) to find the exact coordinates of the point on the circle  $20$ cm from $A$ as measured counterclockwise along the circle. 

\item Use the worksheet above to approximate these coordinates and check if your coordinates look reasonable.

\item Between noon and 12:02pm a beetle crawls counterclockwise around the circle at a constant speed of $6$cm/sec, passing the point $A(8,0)$ at $23$ seconds past noon. Find coordinate functions
\[
   x = f_2(t)  \hskip 0.5 in \text{and} \hskip 0.5in  y=g_2(t) ,
\] 
that express the beetle's coordinates (in cm) in terms of the number of seconds past noon.

\end{enumerate}
\end{question}

\section{Ellipses}
\begin{example} \label{Ex:DFDrefdfobbzx}
Start with the standard parameterization 
\[
    (x,y) = (f(\theta) , g(\theta)) =   ( \cos \theta , \sin \theta ) \, , \, 0\leq\theta < 2\pi ,
\]
of the unit circle centered at the origin. Here $\theta$ is the polar angle of the vector $\overrightarrow{OP}$ from the origin to a point $P(x,y)$ on the circle.

Now replace $x$ with $x/4$ and $y$ with $y/2$ to get a parameterization
\[
  \frac{x}{4} = f(\theta) = \cos\theta \, , \, 0\leq\theta < 2\pi ,
\]
\[
  \frac{y}{2} = g(\theta) = \sin \theta \, , \, 0\leq\theta < 2\pi 
\]
of a new curve. How is this new curve related to the unit circle?

We might start by rewriting the new parameterization as
\[
      x = 4 f(\theta) = 4\cos\theta \, , \, 0\leq\theta < 2\pi ,
\]
\[ 
   y = 2 g(\theta) = 2\sin \theta \, , \, 0\leq\theta < 2\pi .
\]

What this means is that to get the coordinates of points on the new curve we quadruple the $x$-coordinates and double the $y$-coordinates of the points on the unit circle. Geometrically, this has the effect of dilating (stretching) the unit circle horizontally by a factor of four about the $y$-axis and vertically by a factor of two about the $x$-axis. 


\begin{onlineOnly}
    \begin{center}
\desmos{3ibj708prx}{900}{600}
\end{center}
\end{onlineOnly}

\href{https://www.desmos.com/calculator/3ibj708prx}{142: Ellipse 1}

\begin{enumerate}
\item Watch these stretching actions by dragging the dilation factors $k_1$ and $k_2$ in the worksheet above from $k_1=1$ to $k_1=4$ and $k_2=1$ to $k_2 = 2$.

\item If the point $P$ on the unit circle has coordinates $(a,b)$, what are the coordinates of its image $P^\prime$ after these dilations?

After the two dilations $P^\prime$ has coordinates
\[
    (a^\prime, b^\prime) = (  \answer{4a} , \answer{2b} ) .
\]

\item Find an equation in Cartesian coordinates of the new curve (an ellipse). This equation gives a relationship between the $x$ and $y$ coordinates of a point $P(x,y)$ on the ellipse. It does not involve the parameter $\theta$.

\item Find the exact coordinates of the points where the line $x=3$ intersects the ellipse. Rely on the worksheet only to check your coordinates.

\item Find the exact coordinates of the points where the line $y=2x$ intersects the ellipse. Rely on the worksheet only to check your coordinates.

\item Working from the parameterization 
\[ 
     x = f_2 (\theta) = 4\cos\theta \, , \, 0\leq \theta <2\pi ,
\]
\[
   y = g_2 (\theta) = 2\sin\theta \, , \, 0\leq \theta <2\pi ,
\]
of the ellipse, let's replace $x$ with $x-5$ and $y$ with $y-4$ to get a parameterization of a new curve.

\begin{enumerate}
\item Sketch by hand the new curve.

\item Find an equation of the new curve in Cartesian coordinates.

\item Remember our point $P$ with coordinates $(a,b)$ on the unit circle. Replacing $x$ with $x/4$ and $y$ with $y/2$ in the parameterization of the unit circle sent the point $P$ to the point $P^\prime$ with coordinates $(4a, 2b)$ on the ellipse
\[
  \left( \frac{x}{4} \right)^2 + \left( \frac{y}{2} \right)^2 = 1.
\]
What are the points of the corresponding point $P^{\prime\prime}$ on the new curve?

The coordinates are 
\[
   (a^{\prime\prime} , b^{\prime\prime}) = (\answer{4a+5} , \answer{2b+4} ) .
\]


\end{enumerate}

\end{enumerate}
\end{example}

\begin{example}
Returning now to the ellipse with paramterization
\[
      (x,y) = (4\cos \theta , 2\sin\theta) \, , \, 0\leq \theta <2\pi
\]
and Cartesian equation
\[
     \left( \frac{x}{4} \right)^2 + \left( \frac{y}{2} \right)^2 = 1.
\]
\end{example}

Our aim in this problem is to parameterize this ellipse by the polar angle $\phi$ of the vector $\overrightarrow{OP^\prime}$ from the origin to the point $P^\prime$ on the ellipse.

\begin{onlineOnly}
    \begin{center}
\desmos{nbdmgb1nbv}{900}{600}
\end{center}
\end{onlineOnly}

\href{https://www.desmos.com/calculator/nbdmgb1nbv}{142: Ellipse 2}


\begin{enumerate}
\item First drag the slider $u$ on Line 2 of the worksheet above to convince yourself that the parameter $\theta$ in the parameterization 
\[
    (x,y) = (\cos \theta , \sin\theta ) \, , \, 0 \leq \theta <2\pi .
\]
is \emph{not} the polar angle of the vector $\overrightarrow{OP^\prime}$. Explain how you know.

\begin{freeResponse}
\end{freeResponse}



\item Now let $\phi$ be the polar angle of the vector $\overrightarrow{OP^\prime}$ from the origin to the point $P^\prime$ with coordinates $(x,y)$ on the ellipse (see the worksheet below). To parameterize the ellipse in terms of $\phi$ means to express $x$ and $y$ in terms of $\phi$. To do this, we need to first express the polar radius  
\[
   r = \Big| \overrightarrow{OP^\prime}  \Big|
\]
in terms of $\phi$. Here are the steps:

\begin{onlineOnly}
    \begin{center}
\desmos{r8kijowlcm}{900}{600}
\end{center}
\end{onlineOnly}

\href{https://www.desmos.com/calculator/r8kijowlcm}{142: Ellipse 3}


r8kijowlcm

\begin{enumerate}
\item First express the coordinates of $P^\prime$ in terms of $r$ and $\phi$.

\item Then use algebra and the Cartesian equation of the ellipse to find an equation relating $r$ and $\phi$.

\item Next solve this equation for $r$ in terms of $\phi$ keeping in mind that $r>0$.

The result is that
\[
  r = \answer{\frac{8}{\sqrt{4\cos^2\phi + 16\sin^2\phi }}} .
\]

\item Finally, use the results of parts (i) and (iii) to parameterize the ellipse in terms of the polar angle $\phi$.
\end{enumerate}


\end{enumerate}


\end{document}