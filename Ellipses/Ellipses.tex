\documentclass{ximera}
\title{Circles, Transformations, Ellipses}

\newcommand{\pskip}{\vskip 0.1 in}

\begin{document}
\begin{abstract}
Transformations
\end{abstract}
\maketitle


\section{Circles Not Centered at the Origin}
This section is about parameterizing circles centered at points other than the origin.

\begin{question}  \label{Ex:sdfdsfdfL}

Parameterize a circle with radius $3$ cm centered at the point $C(5,4)$ (coordinates measured in cm) in terms of the polar angle of the vector from $C$ to a point $P(x,y)$ on the circle.

\begin{onlineOnly}
    \begin{center}
\desmos{pcndel98wg}{900}{600}
\end{center}
\end{onlineOnly}

\href{https://www.desmos.com/calculator/pcndel98wg}{142: Translating Circles}


\begin{explanation} 

Here are the steps, with the details left for you to fill in.

We'll start by letting $P$ be a point on the circle with coordinates $(x,y)$ and $\theta$ the polar angle of the vector $\overrightarrow{CP}$ giving the position of $P$ relative to $C$.

\begin{enumerate}
\item First express the components of the vector $\overrightarrow{CP}$ in terms its polar angle $\theta$.

\item Next express the vector $\overrightarrow{OP}$ giving the position of $P$ relative to the origin in terms of $\overrightarrow{CP}$ and one other vector.

\item Use the results of parts (a) and (b) to express the components of the vector $\overrightarrow{OP}$ in terms of $\theta$.

\item Finally, use the result of part (c) to find functions
\[
    x = f(\theta)  \hskip 0.5 in \text{and} \hskip 0.5in  y=g(\theta) ,
\]
that express the coordinates of $P$ in terms of $\theta$. Include appropriate domains for these functions. 
\end{enumerate}

\end{explanation}
\end{question}


\begin{question}  \label{QDF09dfdfLKDD}
\begin{enumerate}

\item Parametrize the circle in Question \ref{Ex:sdfdsfdfL} by the signed arclength from $A$ of a point $P(x,y)$ on the circle. Measure the signed arclength $s$ from the point $A(8,0)$ along the circle and take the counterclockwise direction to be positive. 

\begin{onlineOnly}
    \begin{center}
\desmos{llrzcv5ckf}{900}{600}
\end{center}
\end{onlineOnly}

\href{https://www.desmos.com/calculator/llrzcv5ckf}{142: Circle Arclength Parameterization}



The coordinate functions (in cm) are 
\[
  x = f_1(s) = \answer{5 + 3 \cos(s/3)} \, , \, s\in \mathbb{R}
\]
and
\[
   y = g_1(s) = \answer{4 + 3 \sin(s/3)} \, , \, s\in \mathbb{R} .
\]

\item Use part (a) to find the exact coordinates of the point on the circle  $20$ cm from $A$ as measured counterclockwise along the circle. 

\item Use the worksheet above to approximate these coordinates and check if your coordinates look reasonable.

\item Between noon and 12:02pm a beetle crawls counterclockwise around the circle at a constant speed of $6$cm/sec, passing the point $A(8,0)$ at $23$ seconds past noon. Find coordinate functions
\[
   x = f_2(t)  \hskip 0.5 in \text{and} \hskip 0.5in  y=g_2(t) ,
\] 
that express the beetle's coordinates (in cm) in terms of the number of seconds past noon.

\end{enumerate}
\end{question}

\section{Ellipses}
\begin{example} \label{Ex:DFDrefdfobbzx}
Start with the standard parameterization (in terms of the polar angle of the vector $\overrightarrow{OP}$ from the origin to a point $P(x,y)$ on the circle)
\[
    (x,y) = (f(\theta) , g(\theta)) =   ( \cos \theta , \sin \theta ) \, , \, 0\leq\theta < 2\pi 
\]
of the unit circle centered at the origin.

Now replace $x$ with $x/4$ and $y$ with $y/2$ to get a parameterization
\[
  \frac{x}{4} = f(\theta) = \cos\theta \, , \, 0\leq\theta < 2\pi 
\]
\[
  \frac{y}{2} = g(\theta) = \sin \theta \, , \, 0\leq\theta < 2\pi 
\]
of a new curve. How is this new curve related to the unit circle?

We might start by rewriting the new parameterization as
\[
      x = 4 f(\theta) = 4\cos\theta \, , \, 0\leq\theta < 2\pi 
\]
\[ 
   y = 2 g(\theta) = 3\sin \theta \, , \, 0\leq\theta < 2\pi .
\]

What this means is that to get the coordinates of the points on the new curve we quadruple the $x$-coordinates and triple the $y$-coordinates of the points on the unit circle. Geometrically, this has the effect of dilating (stretching) the unit circle horizontally by a factor of four about the $y$-axis and vertically by a factor of three about the $x$-axis. 


\begin{onlineOnly}
    \begin{center}
\desmos{3ibj708prx}{900}{600}
\end{center}
\end{onlineOnly}

\href{https://www.desmos.com/calculator/3ibj708prx}{142: Ellipse 1}

\begin{enumerate}
\item Watch these stretching actions by dragging the dilation factors $k_1$ and $k_2$ in the worksheet above from $k_1=0$ to $k_1=4$ and $k_2=0$ to $k_2 = 4$.

\item If the point $P$ on the unit circle has coordinates $(a,b)$, what are the coordinates of its image $P^\prime$ under the two dilations?

After the two dilations, $P^\prime$ has coordinates
\[
    (x,y) = (  \answer{4a} , \answer{2b} ) .
\]
\end{enumerate}




\end{example}





\end{document}