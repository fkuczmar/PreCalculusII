\documentclass{ximera}
\title{Circles, Transformations, Ellipses}

\newcommand{\pskip}{\vskip 0.1 in}

\begin{document}
\begin{abstract}
Transformations
\end{abstract}
\maketitle


\section{Circles Not Centered at the Origin}
This section is about parameterizing circles centered at points other than the origin.

\begin{question}  \label{Ex:sdfdsfdfL}

Parameterize a circle with radius $3$ cm centered at the point $C(5,4)$ (coordinates measured in cm) in terms of the polar angle of the vector from $C$ to a point $P(x,y)$ on the circle.

\begin{onlineOnly}
    \begin{center}
\desmos{pcndel98wg}{900}{600}
\end{center}
\end{onlineOnly}

\href{https://www.desmos.com/calculator/pcndel98wg}{142: Translating Circles}


\begin{explanation} 

Here are the steps, with the details left for you to fill in.

We'll start by letting $P$ be a point on the circle with coordinates $(x,y)$ and $\theta$ the polar angle of the vector $\overrightarrow{CP}$ giving the position of $P$ relative to $C$.

\begin{enumerate}
\item First express the components of the vector $\overrightarrow{CP}$ in terms its polar angle $\theta$.

\item Next express the vector $\overrightarrow{OP}$ giving the position of $P$ relative to the origin in terms of $\overrightarrow{CP}$ and one other vector.

\item Use the results of parts (a) and (b) to express the components of the vector $\overrightarrow{OP}$ in terms of $\theta$.

\item Finally, use the result of part (c) to express the functions
\[
    x = f(t)  \hskip 0.5 in \text{and} \hskip 0.5in  y=g(t) ,
\]
that express the coordinates of $P$ in terms of $\theta$. Include appropriate domains for these functions. 
\end{enumerate}

\end{explanation}
\end{question}


\begin{question}  \label{QDF09dfdfLKDD}
\begin{enumerate}

\item Parametrize the circle in Question \ref{Ex:sdfdsfdfL} by the signed arclength of a point $P(x,y)$ on the circle. Measure the signed arclength from the point $(8,0)$ around the circle and take the counterclockwise direction to be positive.

\item sdf

\end{enumerate}

\end{question}


\end{document}