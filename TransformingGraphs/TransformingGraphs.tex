\documentclass{ximera}
\title{Graphs of the Cosine and Sine Functions}

\newcommand{\pskip}{\vskip 0.1 in}

\begin{document}
\begin{abstract}
An introduction to relative motion.
\end{abstract}
\maketitle


So far we have used the cosine and sine functions to parameterize uniform circular motion. This same type of motion is useful in modeling oscillatory behavior, like the motion of a sound wave through the air, the temperature over the course of a day, or the position of a small object (like a molecule or a mass on a spring) as it oscillates.


\begin{exploration}   \label{Ertr64yhd}

The animation below shows how uniform circular motion generates oscillatory motion (simple harmonic motion) about the center of the circular path.

Play the slider $u$ and watch how the constant speed motion of point $P$ makes point $Q$ oscillate horizontally about the origin. Then activate the Vertical Motion folder (Line 9) to see the same oscillatory motion in the vertical direction. The two oscillations are essentially identical, just in different directions and out of sync (phase) with each other.

Access Desmos interactive at
 
\href{https://www.desmos.com/calculator/ng2ed0t2jh}{142: Oscillatory Motion}

 
\begin{onlineOnly}
    \begin{center}
\desmos{ng2ed0t2jh}{900}{600}
\end{center}
\end{onlineOnly}

\end{exploration}


We can exploit this oscillatory behavior to model something like the number of hours of daylight/day in Seattle over the course of a year. The key point again is that uniform (constant speed) circular motion generates the oscillatory behavior. This idea is illustrated below.

Access Desmos interactive at
 
\href{https://www.desmos.com/calculator/4qf3zb54vd}{142: Hours of Daylight}

\begin{onlineOnly}
    \begin{center}
\desmos{4qf3zb54vd}{900}{600}
\end{center}
\end{onlineOnly}


\begin{onlineOnly}
    \begin{center}
\desmos{dmukniaa0x}{900}{600}
\end{center}
\end{onlineOnly}


\section{Graphing Transformations of the Sine Function}
We really just need either the sine or the cosine function to model simple oscillations. In most cases, it's easier to use the cosine function, but we'll start by graphing the sine function because it's easier to see how the graph relates to the circular motion.


\begin{exploration}  \label{Esdfsatnh4}

Key points about the graph of the function $f(\theta)=\sin\theta$.

\begin{itemize}

\item{The input to the function $f(\theta) = \sin\theta$ is the dimensionless angle $\theta$. But by working on the unit circle, we can interpret $\theta$ as a dimensionless arclength (blue) measured, as usual, counterclockwise from the positive $x$-axis.} 

\item{The output to the function $f(\theta) = \sin\theta$ is the dimensionless ratio $y/r$. But by working on the unit circle, we can interpret $\sin\theta$ as a dimensionless $y$-coordinate (red).}

\item{The function $f(\theta) = \sin\theta$ is periodic and the graph repeats with period is $2\pi$. This means that 
\[
   \sin (\theta + 2\pi) = \sin \theta \text{ for all } \theta \in \mathbb{R}  
\]
and that $2\pi$ is the smallest positive number we can add to any input of the sine function to leave the output unchanged.}

\item{The function $y=\sin \theta$ is graphed only over the domain $0\leq \theta \leq 2\pi$, but its domain is the set of all real numbers.} 

\item{The range of the function $y=\sin\theta$ is the set
\[
     \{  y \, | \, -1\leq y \leq 1 \}.
\]}


\end{itemize}


Access Desmos interactives through the online version of this text at
 
\href{https://www.desmos.com/calculator/lurdcthxqu}{142: Sine Graph}

 
\begin{onlineOnly}
    \begin{center}
\desmos{lurdcthxqu}{900}{600}
\end{center}
\end{onlineOnly}
\end{exploration}






\end{document}