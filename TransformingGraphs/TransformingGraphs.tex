\documentclass{ximera}
\title{Graphs of the Cosine and Sine Functions}

\newcommand{\pskip}{\vskip 0.1 in}

\begin{document}
\begin{abstract}
Graphing the sine and cosine functions and their transformations.
\end{abstract}
\maketitle


So far we have used the cosine and sine functions to parameterize uniform circular motion. This same type of motion is useful in modeling oscillatory behavior, like the motion of a sound wave through the air, the temperature over the course of a day, or the position of a small object (like a molecule or a mass on a spring) as it oscillates.


\begin{exploration}   \label{Ertr64yhd}

The animation below shows how uniform circular motion generates oscillatory motion (simple harmonic motion) about the center of the circular path.

Play the slider $u$ and watch how the constant speed motion of point $P$ makes point $Q$ oscillate horizontally about the origin. Then activate the Vertical Motion folder (Line 9) to see the same oscillatory motion in the vertical direction. The two oscillations are essentially identical, just in different directions and out of sync (phase) with each other.

Access Desmos interactive at
 
\href{https://www.desmos.com/calculator/ng2ed0t2jh}{142: Oscillatory Motion}

 
\begin{onlineOnly}
    \begin{center}
\desmos{ng2ed0t2jh}{900}{600}
\end{center}
\end{onlineOnly}

\end{exploration}


We can exploit this oscillatory behavior to model something like the number of hours of daylight/day in Seattle over the course of a year. The key point again is that uniform (constant speed) circular motion generates the oscillatory behavior. This idea is illustrated below.

Access Desmos interactive at
 
\href{https://www.desmos.com/calculator/4qf3zb54vd}{142: Hours of Daylight}

\begin{onlineOnly}
    \begin{center}
\desmos{4qf3zb54vd}{900}{600}
\end{center}
\end{onlineOnly}


\begin{onlineOnly}
    \begin{center}
\desmos{dmukniaa0x}{900}{600}
\end{center}
\end{onlineOnly}


\section{Graphing the Sine Function}
We really just need either the sine or the cosine function to model simple oscillations. In most cases, it's easier to use the cosine function, but we'll start by graphing the sine function because it's easier to see how its graph relates to uniform circular motion.


\begin{exploration}  \label{Esdfsatnh4}

Key points about the function $f(\theta)=\sin\theta$.

\begin{itemize}

\item{The input to the function $f(\theta) = \sin\theta$ is the dimensionless angle $\theta$. But by working on the unit circle, we can interpret $\theta$ as a dimensionless arclength (blue) measured, as usual, counterclockwise from the positive $x$-axis.} 

\item{The output to the function $f(\theta) = \sin\theta$ is the dimensionless ratio $y/r$. But by working on the unit circle, we can interpret $\sin\theta$ as a dimensionless $y$-coordinate (red).}

\item{The function $f(\theta) = \sin\theta$ is periodic and the graph repeats with period $2\pi$. This means that 
\[
   \sin (\theta + 2\pi) = \sin \theta \text{ for all } \theta \in \mathbb{R}  
\]
and that $2\pi$ is the smallest positive number we can add to any input of the sine function to leave the output unchanged.}

\item{The function $y=\sin \theta$ is graphed only over the domain $0\leq \theta \leq 2\pi$, but its domain is the set of all real numbers.} 

\item{The range of the function $y=\sin\theta$ is the set
\[
     \{  y \, | \, -1\leq y \leq 1 \}.
\]}

\item{The function $f(\theta) = \sin\theta$ is odd. This means that 
\[
   f(-\theta) = - f(\theta) \text{ for all } \theta \in \mathbb{R} ,
\]
and here that 
\[
    \sin (-\theta) = - \sin \theta \text{ for all } \theta \in \mathbb{R} .
\]
Explain why this is true directly from the definition of the sine function.
}
\end{itemize}


Access Desmos interactives through the online version of this text at
 
\href{https://www.desmos.com/calculator/lurdcthxqu}{142: Sine Graph}

 
\begin{onlineOnly}
    \begin{center}
\desmos{lurdcthxqu}{900}{600}
\end{center}
\end{onlineOnly}
\end{exploration}


\section{Graphing Transformations of the Sine Function}
\begin{example} \label{Ed7ugg}

Our next step is to graph tranformations of the sine function, like
\[
    g(t) = 1.5\sin (2t) \, , \, t\in \mathbb{R} .
\]
We'll take two approaches. One is to relate the function to uniform circular motion. The other is to transform the graph of the sine function $f(t) = \sin t$. 

Either way, we'll suppose the input $t$ is time, meausured in seconds, and the output is measured in meters. 


\emph{ Solution 1:} For the first approach, we'll interpret $g(t) = 1.5 \sin (2t)$ as giving the $y$-coordinate of a point moving counterclockwise around a circle at a constant speed and ask ourselves three questions.

\begin{itemize}
\item{
\begin{question} \label{Qdst656}
What is the radius of the circular path?  
\[
\answer{1.5} \text{ meters}
\]
\end{question}
}

\item{
\begin{question}  \label{Qdfsatt56}
What is the rotation rate about the center of the circular path?
\[
 \answer{2} \text{ rad/sec}
\]
\end{question}
}

\item{
\begin{question}  \label{Qdfsdsg56}
What is the period of the motion? That is, how long does it take the point to move once around its path?

For this, we just divide $2\pi$, the number of radians in one revolution, by the rotation rate. So for our example, the period of the motion is
\[
    \frac{2\pi \text{ radians}}{\answer{2}\text{ rad/sec}}   = \answer{\pi} \text{ seconds}
\]
\end{question}
}
\end{itemize}

We now have enough information to graph the function 
\[
    g(t) = 1.5 \sin (2t) \, , \, t \in \mathbb{R} 
\]
as shown below.

Access Desmos interactives through the online version of this text at
 
\href{https://www.desmos.com/calculator/txuiehcciu}{142:Sine Graphs 2}.

 
\begin{onlineOnly}
    \begin{center}
\desmos{txuiehcciu}{900}{600}
\end{center}
\end{onlineOnly}

\pskip

{\bf Solution 2:} For the second approach, we think about the transformations that take the graph of 
\[
      y = f(t) = \textcolor{teal}{1} \sin (\textcolor{red}{1}t) \, , \, t\in \mathbb{R}
\]
to the graph of
\[
  y = g(t) = 1.5 \sin (2t) \, , \, t\in \mathbb{R} .
\]

But first we should clear up a common misconception about transformations. Suppose, for example, we start with an equation 
\[
   x^2 + y^2 = 1
\]
of the unit circle and replace $x$ with $x/3$ and $y$ with $y/4$. This gives us the equation
\[
    \left( \frac{x}{3} \right)^2 + \left( \frac{y}{4} \right)^2 = 1 .
\]
How is the graph of this new relation related to the graph of the original? Well, setting $y=0$ and solving for $x$ tells us that the $x$-intercepts of the new curve are $(\pm 3, 0)$. Similarly, the $y$-intercepts are $(0,\pm 4)$. This suggests that to go from the unit circle to the new curve (an ellipse) we triple the $y$-coordinate and quadruple the $x$-coordinate of each point on the unit circle. Geometrically, this means we stretch the unit circle horizontally about the $y$-axis by a factor of $3$ and vertically about the $x$-axis by a factor of $4$. This is the effect of substituting $x$ with $x/3$ and $y$ with $y/4$.

Now returning to the problem at hand. We get from the function $y=f(t) = \sin t$ to the function $y=g(t) = 1.5 \sin (2t)$ by replacing $y$ with $y/1.5$ and $t$ with $2t$, like this: 
\[
   y = \sin t  \longrightarrow \frac{y}{1.5} = \sin (2t) .
\]
And multiplying both sides of the second equation by $1.5$, we get
\[
        y = 1.5 \sin (2t) = g(t) .
\]
Now the effect of replacing $y$ with $y/1.5$ is to multiply each $y$-coordinate by 1.5. Geometrically, we stretch the graph of $y=f(t)$ about the horizontal ($t$) axis by a factor of $1.5$. And replacing $t$ with 
\[
   2t = \frac{t}{1/2}
\]
halves the $x$-coordinates. It stretches the graph $y=f(t)$ about the vertical ($y$) axis by a factor of $1/2$ (or we can say compresses the graph about the vertical axis by a factor of $2$.)

That's all we need to graph $y=1.5 \sin (2t)$. Instead of the outputs oscillating between $y=-1$ and $y=1$ (as they do for $f(t)=\sin t$), they oscillate between $y=-1.5$ and $y=1.5$. And because the graph of $y=\sin t$ gets compressed horizontally about the $y$-axis by a factor of $2$, the function $y=1.5 \sin (2t)$ has period 
\[
  \frac{2 \pi \text{ radians}}{2 \text{ rad/sec}} = \pi \text{ sec}.
\]

 Drag the sliders $b$ and $w$ in the demonstation below to see the transformation of the graph of
\[
  y = \sin (t) \text{ where } a=0, b=1, \text{ and }\omega = 1
\] 
to the graph of 
\[
  y = 1.5 \sin (2t) \text{ where } a=0, b=1.5, \text{ and }\omega = 2 .
\] 

Access Desmos interactives through the online version of this text at
 
\href{https://www.desmos.com/calculator/qrkgbi3gtr}{142: Transform Sine}.

 
\begin{onlineOnly}
    \begin{center}
\desmos{qrkgbi3gtr}{900}{600}
\end{center}
\end{onlineOnly}
\end{example}

\begin{question}\label{Qdgt4jh65}
Use the methods of Example 3 to graph the function
\[
   h(t) = 3\sin\left( \frac{\pi}{4}t \right) , t\in \mathbb{R}.
\]
\end{question}


\section{Graphing the Cosine Function}

\begin{example}  \label{ExPDlreDFer}
To relate the graph of the cosine function to uniform circular motion it helps to draw the unit circle with the postive $x$-axis pointing upward and the positive $y$-axis pointing to the left as shown below.

\href{https://www.desmos.com/calculator/ro4dbqifwe}{142: Cosine Graph}.

 
\begin{onlineOnly}
    \begin{center}
\desmos{ro4dbqifwe}{900}{600}
\end{center}
\end{onlineOnly}
\end{example}





\end{document}