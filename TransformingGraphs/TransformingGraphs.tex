\documentclass{ximera}
\title{Graphs of the Cosine and Sine Functions}

\newcommand{\pskip}{\vskip 0.1 in}

\begin{document}
\begin{abstract}
An introduction to relative motion.
\end{abstract}
\maketitle


So far we have used the cosine and sine functions to parameterize uniform circular motion. This same type of motion is useful in modeling oscillatory behavior, like the motion of a sound wave through the air, the temperature over the course of a day, or the position of a small object (like a molecule or a mass on a spring) as it oscillates.


\begin{exploration}   \label{Ertr64yhd}

The animation below shows how uniform circular motion generates oscillatory motion (simple harmonic motion) about the center of the circular path.

Play the slider $u$ and watch how the constant speed motion of point $P$ makes point $Q$ oscillate horizontally about the origin. Then activate the Vertical Motion folder (Line 9) to see the same oscillatory motion in the vertical direction. The two oscillations are essentially identical, just in different directions and out of sync (phase) with each other.

Access Desmos interactive at
 
\href{https://www.desmos.com/calculator/ng2ed0t2jh}{142: Oscillatory Motion}

 
\begin{onlineOnly}
    \begin{center}
\desmos{ng2ed0t2jh}{900}{600}
\end{center}
\end{onlineOnly}

\end{exploration}


We can exploit this oscillatory behavior to model something like the number of hours of daylight/day in Seattle over the course of a year. The key point again is that the oscillatory behavior is generated by uniform (constant speed) circular motion. This idea is illustrated below.

Access Desmos interactive at
 
\href{https://www.desmos.com/calculator/4qf3zb54vd}{142: Hours of Daylight}

\begin{onlineOnly}
    \begin{center}
\desmos{4qf3zb54vd}{900}{600}
\end{center}
\end{onlineOnly}


\begin{onlineOnly}
    \begin{center}
\desmos{dmukniaa0x}{900}{600}
\end{center}
\end{onlineOnly}



\end{document}