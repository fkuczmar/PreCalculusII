\documentclass{ximera}
\title{Graphs of the Cosine and Sine Functions}

\newcommand{\pskip}{\vskip 0.1 in}

\begin{document}
\begin{abstract}
An introduction to relative motion.
\end{abstract}
\maketitle


So far we have used the cosine and sine functions to parameterize uniform circular motion. This same type of motion is useful in modeling oscillatory behavior, like the motion of a sound wave through the air, the temperature over the course of a day, or the position of a small object (like a molecule or a mass on a spring) as it vibrates or oscillates.

Here's a particularly useful image to keep in mind.

\begin{exploration}   \label{Ertr64yhd}

Access Desmos interactive at
 
\href{https://www.desmos.com/calculator/ng2ed0t2jh}{142: Oscillatory Motion}

 
\begin{onlineOnly}
    \begin{center}
\desmos{ng2ed0t2jh}{900}{600}
\end{center}
\end{onlineOnly}


\end{exploration}


\end{document}