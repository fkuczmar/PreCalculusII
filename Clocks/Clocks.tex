\documentclass{ximera}
\title{Thinking About Clocks}

\newcommand{\pskip}{\vskip 0.1 in}

\begin{document}
\begin{abstract}
A few clock problems.
\end{abstract}
\maketitle

\begin{question} \label{Qdfr4tfds3}

\begin{enumerate}
\item Find the exact and approximate (to the nearest tenth) radian measure of the angle between the minute and hour hands of a clock at 1:00pm.

\item Find the exact rate at which the angle between the minute and hour hands is changing at 1:00pm.

\item Use the worksheet below to approximat the radian measure of the angle between the minute and hour hands at 1:20pm.

\pdfOnly{
Access Desmos interactives through the online version of this text at
\href{https://www.desmos.com/calculator/j0lwwthevs}.
}
 
\begin{onlineOnly}
    \begin{center}
\desmos{tkoaenjw2z}{900}{600}
\end{center}
\end{onlineOnly}

\href{https://www.desmos.com/calculator/tkoaenjw2z}{142:Clock 55}

\item Use parts (a) and (b) to find the exact and approximate (to the nearest tenth) radian measure of the angle between the minute and hour hands at 1:20pm.
\end{enumerate}

\pdfOnly{
Access Desmos interactives through the online version of this text at
\href{https://www.desmos.com/calculator/j0lwwthevs}.
}
 
\begin{onlineOnly}
    \begin{center}
\desmos{tkoaenjw2z}{900}{600}
\end{center}
\end{onlineOnly}

\href{https://www.desmos.com/calculator/tkoaenjw2z}{142:Clock 55}

\end{question}



\begin{question}   \label{Q54: Angles4544}
(a) Find an increasing function $\theta = a(t)$ that expresses the radian measure of the angle between the minute and hour hands of a clock in terms of the number of hours past noon. Measure the angle from the hour hand to the minute hand, taking the clockwise sense to be positive. %Note that for a function to be \emph{monotonic} means that it is always increasing or always decreasing.

(b) Use the clock below to estimate the radian measure of the \emph{acute} angle between the minute and hour hands at 12:53pm. An \emph{acute} angle has a measure between $0$ and $\pi/2$ radians.

(c) Use your function from part (a) to help find the exact radian measure of the \emph{acute} angle between the minute and hour hands at 12:53pm. Then find the approximate radian measure, correct to the nearest hundredth. Compare with your estimate from part (b).

(d) Use the clock below to estimate the first two times after 12:00pm when the minute and hour hands are perpendicular.

(e) Use your function from part (a) to help find the first two times after 12:00pm when the minute and hour hands of a clock are perpendicular. Round these times to the nearest second and compare them with your estimate from part (d).

\pdfOnly{
Access Desmos interactives through the online version of this text at
 
\href{https://www.desmos.com/calculator/vt6utnkfve}.
}
 
\begin{onlineOnly}
    \begin{center}
\desmos{vt6utnkfve}{900}{600}
\end{center}
\end{onlineOnly}

\end{question}


\begin{question}   \label{Q54B:Angles23434}
(a) Find an increasing function $\theta = a(t)$ that expresses the radian measure of the angle between the minute and hour hands of a clock in terms of the number of hours past 3:00pm. Measure the angle from the hour hand to the minute hand, taking the clockwise sense to be positive. %Note that for a function to be \emph{monotonic} means that it is always increasing or always decreasing.

(b) Use the clock above to estimate the first two times after 3:00pm when the minute and hour hands make an angle of $2\pi/3$ radians with each other.

(c) Use your function from part (a) to help find the first two times after 3:00pm when the minute and hour hands of a clock make an angle of $2\pi/3$ radians with each other. Round these times to the nearest second and compare them with your estimate from part (b).

\end{question}




\end{document}