\documentclass{ximera}
\title{Quiz 10A}

\newcommand{\pskip}{\vskip 0.1 in}

\begin{document}
\begin{abstract}
Right triangle trig and the law of sines.
\end{abstract}
\maketitle



\emph{Directions:}
\begin{enumerate}
\item  Use right triangle trigonometry if possible. 

\item Otherwise, try to law of sines. 

\item Do \emph{not} use the law of cosines.
\end{enumerate}


\begin{question} \label{ExKdfdKREGER}
An astronaut above the surface of a planet sees only a fraction of the surface as suggested by the figure below.

\begin{onlineOnly}
    \begin{center}
\desmos{8shf1msp4m}{450}{600}  
\end{center}
\end{onlineOnly}

\href{https://www.desmos.com/calculator/8shf1msp4m}{151: Distance to Horizon 44}

The visible part of the surface is a spherical disk with spherical radius $BC$ above. We can think of this distance (an arc of a circle) as the distance to the horizon.

Find a function 
\[
   s = f(h) \, , \, h\geq 0,
\]
that expresses the distance to the horizon on a planet of radius $R$ kilometers in terms of the altitude of the astronaut (in km) above the surfarce.

\end{question}



\begin{question} \label{QdfRDEfeERER}
You wrap a rubber band around a jar lid of radius $R$ cm and stretch the band as shown below.

Express the length (measured in cm) of the band in terms of $R$ and the distance $h$ between the point $P$ and the center of the lid, also measured in cm.


\begin{onlineOnly}
    \begin{center}
\desmos{wz5qiyj4od}{800}{600}         %zk06s3k6q4
\end{center}
\end{onlineOnly}

\href{https://www.desmos.com/calculator/wz5qiyj4od}{142: Jar Lid}

\end{question}


\begin{question} \label{Q454ddtgHHERER}
Sensors on the ground at points $A$ and $B$ below measure the angles of elevation to a plane to be $\theta_1$ and $\theta_2$, respectively, as shown below.

Express the altitdue of the plane (in km) in terms of the distance $s$ between the sensors (also measured in km). Solve this problem twice.

\begin{enumerate}
\item Once without using the law of sines.

\item Again with the law of sines.
\end{enumerate}

Compare your expressions from the two methods. Comments?


\begin{onlineOnly}
    \begin{center}
\desmos{zbktegkke4}{800}{600}         %zk06s3k6q4
\end{center}
\end{onlineOnly}

\href{https://www.desmos.com/calculator/zbktegkke4}{142: Helicopter}

\end{question}


\begin{question} \label{Q444f44tgHHERER}
In $\Delta VUW$, side $\overline{VU}$ has length $10$cm, side $\overline{UW}$ has length $6$cm, and  $\angle A$ has measure $\arccos(9/10)$.

\begin{onlineOnly}
    \begin{center}
\desmos{a2m3iexwfx}{800}{600}         
\end{center}
\end{onlineOnly}

\href{https://www.desmos.com/calculator/a2m3iexwfx}{142: Law of Cosinses 53}

\begin{enumerate}
\item Drag slider $u$ (the length of side $\overline{VW}$) in Line 4 above to change the position of $W$. Keep your eye on Line 6 (showing the length of $\overline{UW}$) to first determine how many triangles satisfy the given conditions and second to approximate the possible lengths of side $\overline{VW}$.

\item Use right triangle trigonometry to determine the exact length of $\overline{VW}$ and the measure of $\angle W$ in all possible triangles satisfying the given conditions. Activate the \emph{Hint} folder on Line 7 (click the circle at the left of the line) for a hint.

\item Use the law of sines to answer question (b).

 
\end{enumerate}

\end{question}

\begin{question}  \label{Q343f44fd}
At noon a rowboat is $10$ km due east of a sailboat. The sailboat travels at a consant speed of $v$ km/hour at a fixed bearing of $0.4$ radians (measured counterclockwise from the east). The rowboat travels half as fast as the sailboat at some fixed bearing. Sometime later the boats collide.

(a) Find the possible bearing(s) of the rowboat.

(b) When do the boats collide?

(c) Establish a rectangular coordinate system with the positive $x$-axis pointing due east, and the origin at the sailboat's position at noon. Then parameterize the motions of the rowboat and sailboat. Consider all possibilities. Include domains and do \emph{not} use inverse trigonometric functions in the parameterizations.

(d) Follow the directions in the Desmos acitivity below to check your work.

\pdfOnly{
Access Desmos interactives through the online version of this text at
 
\href{https://www.desmos.com/calculator/xjiwflhe2o}.
}
 
\begin{onlineOnly}
    \begin{center}
\desmos{xjiwflhe2o}{900}{600}
\end{center}
\end{onlineOnly}

Access Desmos interactive at

\href{https://www.desmos.com/calculator/xjiwflhe2o}{142: Sailboat and Rowboat}

\end{question}




\end{document}
