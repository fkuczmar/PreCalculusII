\documentclass{ximera}
\title{Quiz 10A}

\newcommand{\pskip}{\vskip 0.1 in}

\begin{document}
\begin{abstract}
Right triangle trig and the law of sines.
\end{abstract}
\maketitle



\emph{Directions:}
\begin{enumerate}
\item  Use right triangle trigonometry if possible. 

\item Otherwise, try to law of sines, but \emph{not} the law of cosines. 

\end{enumerate}


\begin{question} \label{ExKdfdKREGER}
An astronaut above the surface of a planet sees only a fraction of the surface as illustrated in the figure below.

\begin{onlineOnly}
    \begin{center}
\desmos{8shf1msp4m}{450}{600}  
\end{center}
\end{onlineOnly}

\href{https://www.desmos.com/calculator/8shf1msp4m}{151: Distance to Horizon 44}

The visible part of the surface is a spherical disk with spherical radius $BC$ above. We can think of this distance (an arc of a circle) as the distance to the horizon.

This problem is about the function 
\[
   s = f(h) \, , \, h\geq 0,
\]
that expresses the distance to the horizon on a planet of radius $R$ kilometers in terms of the altitude of the astronaut (in km) above the surface. 

There are no numbers here, but $R$ is a constant. If it makes you more comfortable you can choose a specific value for $R$, but it would be best to work with $R$ directly. Keep in mind that $h$, the altitude above the surface, is the input to the function $f$. It is a variable.

\begin{enumerate}
\item Drag point $A$ in the worksheet above to get a sense of how the distance to the horizon varies with altitude. Use this to sketch by hand a rough graph of the function $s=f(h)$.

\item Find an expression for the function $s=f(h)$.

\item Without making a computation, approximate $f(100R)$ as some multiple of $R$. Then use your function to get a better approximation. 

\end{enumerate} 

\end{question}



\begin{question} \label{QdfRDEfeERER}
You wrap a rubber band around a jar lid of radius $R$ cm and stretch the band as shown below.

Express the length $L$ (measured in cm) of the band in terms of $R$ and the distance $h$ between the point $P$ and the center of the lid, also measured in cm.


\begin{onlineOnly}
    \begin{center}
\desmos{wz5qiyj4od}{800}{600}         %zk06s3k6q4
\end{center}
\end{onlineOnly}

\href{https://www.desmos.com/calculator/wz5qiyj4od}{142: Jar Lid}

\end{question}


\begin{question} \label{Q454ddtgHHERER}
Sensors on the ground at points $A$ and $B$ below measure the angles of elevation to a plane to be $\theta_1$ and $\theta_2$, respectively, as marked below.

Express the altitdue $h$ of the plane (in km) in terms of the distance $s$ between the sensors (also measured in km). Solve this problem twice.

\begin{enumerate}
\item Once without using the law of sines. 

\emph{Hint:} Draw segment $\overline{PQ}$ from $P$ perpendicular to the ground to split $\Delta APB$ into two right triangles $\Delta AQP$ and $\Delta BQP$. The problem now is that you do not know any side lengths in either of these triangles. We can get around this by reversing the question and expressing $s = \text{dist}(A,B)$ in terms of $h$, $\theta_1$, and $\theta_2$. Do this and then solve your equation for $h$.

\item Again with the law of sines. For this, erase segment $\overline{PQ}$ from part (a) and work directly with $\Delta APB$.
\end{enumerate}

Compare your expressions from the two methods. Comments?


\begin{onlineOnly}
    \begin{center}
\desmos{zbktegkke4}{800}{600}         %zk06s3k6q4
\end{center}
\end{onlineOnly}

\href{https://www.desmos.com/calculator/zbktegkke4}{142: Helicopter}

\end{question}


\begin{question} \label{Q444f44tgHHERER}
In $\Delta VUW$, side $\overline{VU}$ has length $10$cm, side $\overline{UW}$ has length $6$cm, and  $\angle V$ has measure $\arccos(9/10)$.

\begin{onlineOnly}
    \begin{center}
\desmos{a2m3iexwfx}{800}{600}         
\end{center}
\end{onlineOnly}

\href{https://www.desmos.com/calculator/a2m3iexwfx}{142: Law of Cosinses 53}

\begin{enumerate}
\item Drag slider $u$ (the length of side $\overline{VW}$) in Line 4 above to change the position of $W$. Keep your eye on Line 6 (showing the length of $\overline{UW}$) to first determine how many triangles satisfy the given conditions and second to approximate the possible lengths of side $\overline{VW}$.

\item Use right triangle trigonometry to determine the \emph{exact} lengths of $\overline{VW}$ and the measures of $\angle W$ in \emph{all} possible triangles satisfying the given conditions. Then use a calculator to approximate the lengths to the nearest hundredth of a centimeter and the angles to the nearest hundredth of a degree. Activate the \emph{Hint} folder on Line 7 (click the circle at the left of the line) for a hint.

\item Work directly with $\Delta VUW$ and use the law of sines to solve part (b) again.

 
\end{enumerate}

\end{question}

\begin{question}  \label{Q343f44fd}
At noon a rowboat is $10$ km due east of a sailboat. The sailboat travels at a consant speed of $v$ km/hour at a fixed bearing of $0.4$ radians (measured counterclockwise from the east). The rowboat travels half as fast as the sailboat at some fixed bearing. Sometime later the boats collide.

(a) Find the possible bearing(s) of the rowboat.

(b) When do the boats collide?

(c) Establish a rectangular coordinate system with the positive $x$-axis pointing due east, and the origin at the sailboat's position at noon. Then parameterize the motions of the rowboat and sailboat. Consider all possibilities. Include domains and do \emph{not} use inverse trigonometric functions in the parameterizations.

(d) Follow the directions in the Desmos acitivity below to check your work.

\pdfOnly{
Access Desmos interactives through the online version of this text at
 
\href{https://www.desmos.com/calculator/xjiwflhe2o}.
}
 
\begin{onlineOnly}
    \begin{center}
\desmos{xjiwflhe2o}{900}{600}
\end{center}
\end{onlineOnly}

Access Desmos interactive at

\href{https://www.desmos.com/calculator/xjiwflhe2o}{142: Sailboat and Rowboat}

\end{question}




\end{document}
