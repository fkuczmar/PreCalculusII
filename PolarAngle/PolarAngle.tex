
\documentclass{ximera}
\title{The Polar Angle}

\newcommand{\pskip}{\vskip 0.1 in}

\begin{document}
\begin{abstract}
Measuring the polar angle.
\end{abstract}
\maketitle

\section{Clock Problems}


\begin{question}   \label{Q54: Angles}
(a) Find an increasing function $\theta = a(t)$ that expresses the radian measure of the angle between the minute and hour hands of a clock in terms of the number of hours past noon. Measure the angle from the hour hand to the minute hand, taking the clockwise sense to be positive. %Note that for a function to be \emph{monotonic} means that it is always increasing or always decreasing.

(b) Use the clock below to estimate the radian measure of the \emph{acute} angle between the minute and hour hands at 12:53pm. An \emph{acute} angle has a measure between $0$ and $\pi/2$ radians.

(c) Use your function from part (a) to help find the exact radian measure of the \emph{acute} angle between the minute and hour hands at 12:53pm. Then find the approximate radian measure, correct to the nearest hundredth. Compare with your estimate from part (b).

(d) Use the clock below to estimate the first two times after 12:00pm when the minute and hour hands are perpendicular.

(e) Use your function from part (a) to help find the first two times after 12:00pm when the minute and hour hands of a clock are perpendicular. Round these times to the nearest second and compare them with your estimate from part (d).

\pdfOnly{
Access Desmos interactives through the online version of this text at
 
\href{https://www.desmos.com/calculator/vt6utnkfve}.
}
 
\begin{onlineOnly}
    \begin{center}
\desmos{vt6utnkfve}{900}{600}
\end{center}
\end{onlineOnly}

\end{question}


\begin{question}   \label{Q54B: Angles}
(a) Find an increasing function $\theta = a(t)$ that expresses the radian measure of the angle between the minute and hour hands of a clock in terms of the number of hours past 3:00pm. Measure the angle from the hour hand to the minute hand, taking the clockwise sense to be positive. %Note that for a function to be \emph{monotonic} means that it is always increasing or always decreasing.

(b) Use the clock above to estimate the first two times after 3:00pm when the minute and hour hands make an angle of $2\pi/3$ radians with each other.

(c) Use your function from part (a) to help find the first two times after 3:00pm when the minute and hour hands of a clock make an angle of $2\pi/3$ radians with each other. Round these times to the nearest second and compare them with your estimate from part (b).

\end{question}






\section{Thinking Proportionately, Part 2}

\begin{question}  \label{Q4P:Angles}
Use the radian protractor below to help you find approximate answers to the following questions. Keep the radius of the protractor at $7$ m.

\pskip

(a) You stand at the origin facing the direction of the postive $x$ axis. Estimate your coordinates after you turn counterclockwise through an angle of 4 radians and then walk 100 meters directly away from the origin.

\begin{explanation}
We first use the radian protractor to estimate our coordinates if we had walked $7$ meters directly away from the origin. To do this, we frist drag point $P$ on the radian protractor to the 4 radian mark as shown. Next we approximate the coordinates of $P$ by estimating where the vertical and horizontal lines through $P$ intersect the coordinate axes. It looks like the coordinates of $P$ are about $(-4.7, -5.3){\text m}$.

So if we had walked $7$ meters directly away from the origin, we would have ended up at the point $P$ with coordinates about $(-4.7, -5.3)\text{ meters}$. To approximate our coordinates if we walked $100$ meters instead of $7$ meters, we just need to scale these coordinates by a factor of $(100\text{m})/7(\text{m}) = 100/7$. Then our coordinates $(x,y)$ would be about
\[
   x \sim (-4.7\text{ m}) \left( \frac{100\text{ m}}{7\text{ m}} \right) = (100 \text{ m})\left( \frac{-4.7\text{ m}}{7\text{ m}}\right) \sim  -67.1 \text{ m}.
\]
and
\[
   y   \sim (-5.3\text{ m}) \left( \frac{100\text{ m}}{7\text{ m}} \right) = (100 \text{ m})\left( \frac{-5.7\text{ m}}{7\text{ m}}\right) \sim   -75.7 \text{ m}.
\]

\pdfOnly{
Access Desmos interactives through the online version of this text at
 
\href{https://www.desmos.com/calculator/vb1kvzkzlq}.
}
 
\begin{onlineOnly}
    \begin{center}
\desmos{vb1kvzkzlq}{900}{600}
\end{center}
\end{onlineOnly}


\end{explanation}

(b) You start from the point with coordinates $(100,0)$ meters and walk counterclockwise around the circle of radius $100$ meters centered at the origin. Estimate your coordinates after you have walked $230$ meters. {\bf Click the Hint tab at the beginning of this question for a hint.}

\begin{hint}
First compute the radian measure of the angle through which you turn about the circle's center.
\end{hint}

(c) Use your estimate from part (b) to approximate your coordinates if you had walked $230$ meters clockwise around the same circle.

(d) Repeat part (b) if you walk $2$ meters instead of $230$ meters.

(e) You start on the postive $x$-axis and walk counterclockwise around a circle centered at the origin until you reach the point with coordinates $(7,-24)$ meters. Estimate the distance you walked.  



(f) A ferris wheel has a radius of $50$ feet and its center is $60$ feet above the ground. The wheel rotates at a constant rate, making one revolution every two minutes.

(i) Approximate your height above the ground $10$ seconds after you get on the ferris wheel. 

(ii) Approximate your height above the ground $70$ seconds after you get on the ferris wheel. 

(iii) Approximately how long after you board are you $100$ feet above the ground for the first time.

(iv) Use your approximation from part (iii) to approximate the second time you are $100$ feet above the ground.


\begin{exploration}
\pdfOnly{
Access Desmos interactives through the online version of this text at
 
\href{https://www.desmos.com/calculator/kdakzcloqr}.
}
 
\begin{onlineOnly}
    \begin{center}
\desmos{vb1kvzkzlq}{900}{600}
\end{center}
\end{onlineOnly}
\end{exploration}

\end{question}


\begin{question}  \label{Q941F:Angles}
(a) Find the distance between two ships on the equator at longitudes $20^\circ$W and $50^\circ$W. Measure the distance along the shorter arc of the equator between the points. Take the radius of the earth to be $3960$ miles.

(b) Use the radian protractor above to approximate the distance between two ships on the circle of latitude $55^\circ$N at longitudes $20^\circ$W and $50^\circ$W. Measure the distance along the shorter arc of the circle of latitude through the ships.  Take the radius of the earth to be $3960$ miles.
\end{question}

















\section{Measuring the Polar Angle}


%\begin{exploration}  \label{Q57:Radians}

%\pdfOnly{
%Access Desmos interactives through the online version of this text at
 
%\href{https://www.desmos.com/calculator/2dlgnpeqsm}.
%}
 
%\begin{onlineOnly}
%    \begin{center}
%\geogebra{cgpad5nc}{900}{600}
%\end{center}
%\end{onlineOnly}
%\end{exploration}



\begin{exploration}\label{exp:angles1}
Suppose you stand in place at point $A$ and face point $B$ in the desmos activity below. 

(a) Now turn counterclockwise until you face $C$ for the first time. Use the radian protractor to approximate the angle through which you turn during this time. Show a screenshot.

(b) Now turn back to face $B$ again (sitll standing at $A$). Then turn clockwise and stop when you face $C$ for the fifth time. Through approximately what angle do you turn during this time? 

(c) Turn back to face $B$ again (still standing at $A$). Then turn counterclockwise at the constant rate of $0.45$ rad/sec until you face $C$ for the tenth time. Approximately how long does this take? 


\pdfOnly{
Access Desmos interactives through the online version of this text at
 
\href{https://www.desmos.com/calculator/mwh3wwfrqv}.
}
 
\begin{onlineOnly}
    \begin{center}
\desmos{mwh3wwfrqv}{900}{600}
\end{center}
\end{onlineOnly}
\end{exploration}









\begin{example} \label{Ex1:Angles}
At noon an ant is in the second quadrant at the point on the circle $x^2+y^2=400$ (the coordinates $(x,y)$ measured in centimeters) that is 45 cm from $(20,0)$, as measured along the circle. It crawls counterclockwise around the circle at a constant speed of $5$ cm/sec.

At noon a beetle is in the first quardrant on the same circle at a point that is 10 cm from $(20,0)$, this distance also measured along the circle. It crawls clockwise around the circle at a constant speed of $8$ cm/sec.

The insects crawl for 100 seconds.

\pskip

(a) Find a function
\[
    \theta = a(t) , 0\leq t \leq 100,
\]
that expresses the angle (in radians) from the postive $x$-axis to the segment $OA$ running from the origin to the ant in terms of the number of seconds past noon. Take the counterclockwise sense of rotation to correspond to a postive angle.

\pskip

(b) Find a function
\[
    \theta = b(t) , 0\leq t \leq 100,
\]
that expresses the angle (in radians) from the postive $x$-axis to the segment $OB$ running from the origin to the beetle in terms of the number of seconds past noon. Take the counterclockwise sense of rotation to correspond to a postive angle.

\pskip


(c) Use your functions from (a) and (b) to determine the first time the insects pass each other. Give an exact time, then an approximation to the nearest tenth of a second.

\pskip

(d) Use your functions from (a) and (b) to find an expression for the $n$th time the insects pass each other. 

\pskip

(e) Find a function $c(t)$ that counts the number of times the insects pass each other during the first $t$ seconds of their motions.

\pskip

(f) When is the last time the insects pass each other? Give an exact time, then an approximation to the nearest tenth of a second.

\begin{exploration}\label{exp:angles2}
Use the demonstration below to check your work.


\pdfOnly{
Access Desmos interactives through the online version of this text at
 
\href{https://www.desmos.com/calculator/btopul9ji9}.
}
 
\begin{onlineOnly}
    \begin{center}
\desmos{btopul9ji9}{900}{600}
\end{center}
\end{onlineOnly}
\end{exploration}


\end{example}


\begin{example} \label{Ex2:Angles}
Make up and solve your own version of Example 3. Use the exploration below to check your work, but not to help with the computations in any way. 


\begin{exploration}\label{exp:angles2}

\pdfOnly{
Access Desmos interactives through the online version of this text at
 
\href{https://www.desmos.com/calculator/632iayap2b}.
}
 
\begin{onlineOnly}
    \begin{center}
\desmos{632iayap2b}{900}{600}
\end{center}
\end{onlineOnly}
\end{exploration}


\end{example}


\end{document}