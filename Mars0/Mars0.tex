\documentclass{ximera}
\title{Mars and the Outer Planets, Part 1}

\newcommand{\pskip}{\vskip 0.1 in}

\begin{document}
\begin{abstract}
Thinking about planetary motion.
\end{abstract}
\maketitle

\begin{question}  \label{QOdfthhDIDeer3gg}
We'll assume for this problem that the planets in our solar system revolve about the sun in coplanar circular orbits, all with the same sense of rotation.

Then Kepler's third law states that the product $\omega^2 r^3$ is equal to a constant, where $\omega$ is the rotation rate of a planet about the sun and $r$ its orbital radius.

With just this information we can parameterize the motion of the planets about the sun. We'll assume the earth has an orbital radius of 1 astronomical unit (AU) and an orbit period of $365$ days.  


\begin{enumerate}

\item Parameterize the motion of earth about the sun in terms of the number of earth days. Assume

\begin{itemize}

\item{The sun is at the origin.}

\item{The planets rotate counterclockwise at constant rates about the sun.} 

\item{The earth has coordinates $(1,0)$ a time $t=0$ days.}

\end{itemize}


\item Find a function $P=f(r)$ that expresses the synodic period (measured in earth days) of a planet in terms of its orbital radius $r$ (measured in astronomical units.) Note that the earth has orbital radius $1$ AU.

The \emph{synodic period} is the time it takes for the planet and the earth to pass each other in their orbits.

\item Find the synodic period of Mars (with an orbtital radius of 1.524 AU).
\end{enumerate}

\end{question}


\end{document}