\documentclass{ximera}
\title{Inverse Trig Review}

\newcommand{\pskip}{\vskip 0.1 in}

\begin{document}
\begin{abstract}
Applications of inverse trig, review.
\end{abstract}
\maketitle

\section{Relative Position}

\begin{question} \label{QKDfe3rde3rr3}
Point $A$, different from the origin, has coordinates $(a,b)$. 

Find an expression for the polar angle $\theta$ of the vector $\overrightarrow{OA}$ giving the position of $A$ relative to the origin. Do this three times, each time making sure that $0\leq \theta <2\pi$.

\begin{enumerate}
\item First assume $b>0$.

\item Then assume $a<0$.

\item Next assume $b<0$.

\end{enumerate}

\end{question}

\begin{question} \label{QLFefm33eee3d}
Point $A$ is $6$ miles due west of point $B$. Point $C$ is $8$ miles from $B$ at a bearing of $3.8$ radians.

\begin{enumerate}

\item Draw a picture by hand showing the points and the two given position vectors.

\item Find expressions for the two given position vectors and find their exact components.

\item Express the unknown position vector in terms of the two given position vectors (without referring to components).

\item Find the exact components (do \emph{not} use a calculator) of the vector giving the position of $C$ relative to $A$.

\item Give directions for how to get directly from $A$ to $C$ by giving

\begin{enumerate}
\item A simplified expression for the exact distance from $A$ to $C$. Do not use a caculator.

\item The exact bearing of the direct path from $A$ to $C$. Do not use a calculator. Measure the bearing counterclockwise from the east to be between $0$ and $2\pi$. Keep in mind that you do not know the sign of $x$-component of the vector $\overrightarrow{AC}$.

\item Then use a calculator to approximate the distance (to the nearsest tenth of a mile) and bearing (to the nearest tenth of a radian).

\item Use the worksheet below to check your work. But best to click on the link so you can zoom in and out.

\href{https://www.geogebra.org/classic/bhdsgxtx}{142: Relative Position}.

 
\begin{onlineOnly}
    \begin{center}
\geogebra{bhdsgxtx}{900}{600}
\end{center}
\end{onlineOnly}

\end{enumerate}

\end{enumerate}
\end{question}

\section{Simple Harmonic Motion}

\begin{question} \label{Q9f3fr3rnnm}
Between times $t=0$ and $t=10$ seconds, a mass oscillates in simple harmonic motion along the $x$-axis between positions $x=16$cm and $x=-4$cm with a period of $2.5$ seconds. The mass passes position $x=-1$cm at time $t=4$ seconds while moving to the right (ie. in the direction of the positive $x$-axis).

\begin{enumerate}
\item Find a function
\[
    x = f(t) \, , \, 0\leq t \leq 10,
\] 
that expresses the position of the mass relative to the origin (in cm) in terms of the number of seconds since time $t=0$.  Do this as follows.

\begin{enumerate}
\item Use vectors to express the position $\overrightarrow{OP}$ of the driving point $P$ relative to the origin in terms of the polar angle $\theta$ of the vector $\overrightarrow{CP}$ from the center of the circle of motion to $P$.

\item Use inverse trigonometry to help find a function 
\[
    \theta = a(t) \, , \, 0\leq t \leq 10,
\]
that expresses the polar angle $\theta$ in terms of the number of seconds since time $t=0$. Start by sketching a graph of the polar angle function.

\item Check  your work by following the directions in Line 1 of the worksheet below.

\begin{onlineOnly}
    \begin{center}
\desmos{wu1dhe2xck}{900}{600}
\end{center}
\end{onlineOnly}

\href{https://www.desmos.com/calculator/wu1dhe2xck}{151: Simple Harmonic Motion 29}

\end{enumerate}

\item Find another expression for the same relative position function $x=f(t)$ \emph{without} using inverse trig.

\item Compare your expressions in parts (a) and (b).
\end{enumerate}

\end{question}


\end{document}
