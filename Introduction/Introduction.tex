\documentclass{ximera}
\title{Introduction}

\newcommand{\pskip}{\vskip 0.1 in}

\begin{document}
\begin{abstract}
An overview of our class.
\end{abstract}
\maketitle

Trigonometry is all about ratios. It might not seem that way, particularly if you're familiar with triangle trigonometry. But even triangle trigonometry is primarily about shapes and the ratios of lengths. Knowing all the ratios and just one length is enough to compute all lengths in a figure.

If you've spent a winter in Seattle, you might have noticed how long the days take to get longer. On the winter solstice (the first day of winter, usually on December 21) we get our minimum of about 8 hours of daylight/day. It's not until the vernal equinox
(the first day of spring, usually on March 21) that we, like the rest of the planet, get 12 hours of daylight/day.

But what's really striking is that halfway through the winter we get only about 9.2 hours of daylight.day. It is not until $2/3$ of the winter has passed that we get of 10 hours of daylight/day. The function that expresses the number of daylight hours/day in terms of time is decidedly non-linear.

In Fairbanks, Alaska the situation is even more extreme, where there is only about 4 hours of daylight on the winter solstice. But, and this is the key point, it still takes $2/3$ of the winter for Fairbanks to reach their halfway point and get
\[
  ( 4 \text{ hours of daylight/day}) + ( 12 \text{ hours of daylight/day}) =  8 \text{ hours of daylight/day}.
\]  

\begin{question} \label{Q1:Introduction}
Use ratios and the information about Seattle, to approximate the number of hours of daylight/day in Fairbanks on the mid-winter day. Round your answer to the nearest tenth of an hour and enter it below.

On the mid-winter day, Fairbanks gets about 
\[
    \answer{6.4} \text{ hours of daylight/day.}
\]
\end{question}

We can observe the same phenomenon as the moon goes through its phases. You might have noticed that the full moon seems to linger and looks full, at least to me, for several days. But the third quarter moon, occuring seven days after the full moon when the half the lunar disk is illuminated, passes by quickly. One day later, the lunar disk looks noticeably less than half illuminated. And 3.5 days after the full moon and halfway to the third quarter, it still looks like about $85\%$ of the lunar disk is illuminated, signifcantly more than the $75\%$ that actually is. You might move the slider below to verify this and also try to  determine about how long it takes the lunar disk to look like it is $75\%$ illuminated.

%Because the moon revolves about the earth at a nearly constant rate, the function that takes as an input the day of the lunar month and returns as an output the fraction of the lunar surface that is both illuminated and visible to us is piecewise linear. But because we see the moon as a flat disk and not as a hemisphere, the \emph{apparent} fraction of the disk that it is illuminated 

%It takes about 7 days for the full moon to reach the third quarter. But after half that time (3.5 days) it still looks like about $85\%$ of the lunar disk is illuminated, signifcantly more than the $75\%$ that actually is. You might play around with the slider below to verify this and try to  determine about how long it takes the lunar disk look like it is $75\%$ illuminated.

\pdfOnly{
Access Desmos interactives through the online version of this text at
 
\href{https://www.desmos.com/3d/kwus2c0n24}.
}
 
\begin{onlineOnly}
    \begin{center}
\desmos{mvdruejzoy}{900}{600}
\end{center}
\end{onlineOnly}


A key point of this class is that these kinds of models, for the number of daylight hours and the phases of the moon, can be generated by uniform (constant speed) circular motion. This idea is illustrated below.

\begin{onlineOnly}
    \begin{center}
\desmos{4qf3zb54vd}{900}{600}
\end{center}
\end{onlineOnly}


\begin{onlineOnly}
    \begin{center}
\desmos{dmukniaa0x}{900}{600}
\end{center}
\end{onlineOnly}







\end{document}