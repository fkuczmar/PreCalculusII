\documentclass{ximera}
\title{Introduction}

\newcommand{\pskip}{\vskip 0.1 in}

\begin{document}
\begin{abstract}
An overview of our class.
\end{abstract}
\maketitle

Trigonometry is all about ratios. It might not seem that way, particularly if you're familiar with triangle trigonometry. But even triangle trigonometry is primarily about shapes and the ratios of lengths. Knowing all the ratios and just one length is enough to compute all lengths in a figure.

If you've spent a winter in Seattle, you might have noticed how long the days take to get longer. On the winter solstice (the first day of winter, usually on December 21) we get our minimum of about 8 hours of daylight/day. It's not until the vernal equinox
(the first day of spring, usually on March 21) that we, like the rest of the planet, get 12 hours of daylight/day.

But what's really striking is that halfway through the winter we get only about 9.2 hours of daylight.day. It is not until $2/3$ of the winter has passed that we reach the halfway point and get of 10 hours of daylight/day. The function that expresses the number of daylight hours/day in terms of time is decidedly non-linear.

In Fairbanks, Alaska the situation is even more grim. They get only about 4 hours of daylight/day on the winter soltice. But, and this is the key point, it still takes $2/3$ of the winter for Fairbanks to reach their halfway point and get
\[
  ( 4 \text{ hours of daylight/day}) + ( 12 \text{ hours of daylight/day}) =  8 \text{ hours of daylight/day}.
\]  

\begin{question} \label{Q1:Introduction}
Use ratios and the information about Seattle, to approximate how many hours of daylight/day Fairbanks gets on the mid-winter day. Round your answer to the nearest tenth of an hour and enter it below.

On the mid-winter day, Fairbanks gets about 
\[
    \answer{6.4} \text{ hours of daylight/day.}
\]
\end{question}

We can observe the same phenomenon as the moon goes through its phases. You might have noticed that it can be difficult to tell exactly when the moon is full. It looks full, at least to me, for at least a few days. But when the moon reaches the first or third quarter (when half of the side of the moon facing us is illuminated), you can see the very next day that the moon has clearly passed that phase.

It takes about 7 days for the full moon to reach the third quarter (when half of the side of the moon facing us is illuminated). But after half that time (3.5 days) it looks like $87\%$ of the lunar disk is illuminated, signifcantly more than $75\%$ that actually is. You might play around with the slider below and see if you can determine about how long after the full moon it takes for the lunar disk to look like it is $75\%$ illuminated.

\pdfOnly{
Access Desmos interactives through the online version of this text at
 
\href{https://www.desmos.com/3d/kwus2c0n24}.
}
 
\begin{onlineOnly}
    \begin{center}
\desmos{kwus2c0n24}{900}{600}
\end{center}
\end{onlineOnly}



\end{document}