\documentclass{ximera}
\title{Introduction to Circular Interpolation}

\newcommand{\pskip}{\vskip 0.1 in}

\begin{document}
\begin{abstract}
Circular Interpolation
\end{abstract}
\maketitle

\section{Circular Interpolation with a Variable Protractor}

There are many other applications of simple harmonic motion other than describing the oscillation of a mass attached to the end of a spring. Over a short period of time, the depth of the water at a particular location, for example, varies approximately sinusoidally about its mean. We can think of a point moving around a circle at a constant speed as dragging the surface of th water to varying depths as illustrated below.  

\begin{onlineOnly}
    \begin{center}
\desmos{enxgzmsxri}{900}{600}
\end{center}
\end{onlineOnly}

\href{https://www.desmos.com/calculator/enxgzmsxri}{142: Edmonds Pier 22}

So the ideas in this chapter are not new, but are just different applications of circular interpolation and simple harmonic motion.



\begin{example}  \label{Ex:KDreDFerdfdsa}
We're near the autumnal equinox and you may have noticed how quickly we're losing daylight. It seems not too long ago when we were getting our maximum of $16$ hours/day of daylight on the summer solstice (June 21). And it won't be too long before we'll get our minimum of $8$ hours/day of daylight on the winter solstice (December 21). One of the main goals of this class is to model oscillations like this with trigonometric functions.

We'll introduce this idea now graphically, \emph{without} using any trig functions. So if you've had this class before, try to forget what you know and start from scratch.

One way to model the number of hours/day of daylight over the course of a year in Shoreline would be with a piece-wise linear function. For simplicity, we'll assume each month has 30 days, for a total of 360 days in the year. Then the graph of the function $h=f(t)$ giving the number of hours of daylight/day in terms of the number of months since the summer solstice would look like this.

\begin{onlineOnly}
    \begin{center}
\desmos{swvhkxim4w}{900}{600}
\end{center}
\end{onlineOnly}

\href{https://www.desmos.com/calculator/swvhkxim4w}{142: Hours of Daylight per Day Linear 2}

\begin{question}  \label{Q98dfrdedsfdsf}

Assuming the piecewise-linear function above, use algebra (\emph{not} the worksheet above) to predict 
\begin{enumerate}
\item the number of hours of daylight/day on July 21, August 21, and May 21.

\item the day(s) of the year when we get $10$ hours of daylight/day.
\end{enumerate}

But this kind of piece-wise linear model is not accurate. Unlike now, when near the fall equinox  we're losing daylight at a rapid rate, near the summer solstice, the long days linger. So during the summer our function should not decrease at a constant rate but at a decreasing rate.

A more accurate way to model the number of hours of daylight/day would be to use \emph{circular interpolation} instead. To do this geometrically, we start by drawing a circle of radius 
\[
     (16 - 12) \text{ hours of daylight/day} = 4 \text{ hours of daylight/day}
\]
centered at a point with $h$-coordinate $h=12$ as shown below. To approximate, for example, the number of hours of daylight/day on July 21, we move $1/3$ of the way along the circle from point $M$ (corresponding to June 21, when we get $16$ hours of daylight/day) to point $A$ (corresponding to September 21, when we get $12$ hours of daylight/day). Do this by dragging point $B$ in the worksheet below so that $P$ lies $1/3$ of the way around the circle from $M$ to $A$ and approximate how many hours of daylight/day we get on July 21.

 
\begin{onlineOnly}
    \begin{center}
\desmos{dhzo6hhjw1}{900}{600}
\end{center}
\end{onlineOnly}

\href{https://www.desmos.com/calculator/dhzo6hhjw1}{142: Circular Interpolation}

\begin{enumerate}
\item Use the protractor above to approximate the number of hours of daylight/day we get on April 11, August 21, and March 1. Explain your reasoning.
\begin{freeResponse} 
\end{freeResponse}

\item Use the protractor above to approximate what days of the year we get $9$ hours of daylight/day. Explain your reasoning.
\begin{freeResponse}
\end{freeResponse}
\end{enumerate}

This kind of circular interpolation is more accurate and gives a smoother function $h=g(t)$ for the number of hours of daylight/day. Graph this function by activating the Folder \emph{Circular interpolation} in the first worksheet above.

\end{question}
\end{example}


\iffalse

\begin{example}  \label{Ex:LLL}
We just completed the first third of spring and you've probably noticed how many more hours of daylight we get than we did than just a few months ago. On the vernal equinox (the first day of spring, usually March 21), Seattle gets 12 hours of daylight/day. And on the summer solstice (the first day of summer) we'll get just over 16 hours of daylight/day. 

The question is this. On April 21st, one-third of the way into spring, about how many hours of daylight per day do we get? Or more to the point, thinking proportionally and asking a question that would have the almost same answer for all latitudes, when we're one-third of the way through spring, through what fraction of the way from 12 hours/day to the maximum number of hours of daylight/day are we? One-third? Halfway? Two-thirds? What do you think?

The simplest way to approach this problem would be with a linear interpolation. Then when we're one-third of the way through spring in Seattle, we would be 1/3 of the way from $12$ to $16$ hours of daylight/day. So on April 21st we would expect to get about
\[
    12 \text{ hrs/day} + \frac{1}{3}(4 \text{ hrs/day}) = 13 \frac{1}{3}\text{ hrs/day}
\]
hours of daylight/day as illustrated in the graph below.


\begin{onlineOnly}
    \begin{center}
\desmos{esj3yytaug}{900}{600}
\end{center}
\end{onlineOnly}

\href{https://www.desmos.com/calculator/esj3yytaug}{142: Hours of Daylight per Day}

But this kind of piece-wise linear thinking does not seem quite right. We should not expect such abrupt transitions on the winter and summer solstices.

Instead of linear interpolation, we would do better to use \emph{circular interpolation}. For this, draw a circle of radius 
\[
     (16 - 12) \text{ hours of daylight/day} = 4 \text{ hours of daylight/day}
\]
centered at a point with $h$-coordinate $h=12$ as shown below. Now go $1/3$ of the way along the circle from point $A$ (corresponding to March 21, when we get $12$ hours of daylight/day) to point $M$ (corresponding to June 21, when we get $16$ hours of daylight/day).

 
\begin{onlineOnly}
    \begin{center}
\desmos{nfsifbppzz}{900}{600}
\end{center}
\end{onlineOnly}

\href{https://www.desmos.com/calculator/nfsifbppzz}{142: Circular Interpolation}

\begin{enumerate}
\item Drag point $B$ above so that point $P$ lies $1/3$ of the way around the circle from $A$ to $M$. Then approximate how many hours of daylight/day we get on April 21st.
\begin{freeResponse} 
\end{freeResponse}

\item Use the worksheet above to approximate what days of the year we get $15$ hours of daylight/day. Explain your reasoning.
\begin{freeResponse}
\end{freeResponse}
\end{enumerate}

This kind of circular interpolation is more accurate and gives a smoother function $h=g(t)$ for the number of hours of daylight/day. Graph this function by activating the Folder \emph{Circular interpolation} in the first worksheet.

\end{example}

%What is usually called \emph{sinusoidal modeling} is kind of a misnomer. First, it's usually easier to model osciallatory behavior with a cosine function. Second, and more importantly, the name \emph{sinusuoidal modeling} fails to convey the main idea of replacing linear interpolation with circular interpolation.

\fi

Here's a similar example where you'll need to work a bit harder.

\begin{example} \label{Ex:PDFIRgv4rr}
We'll make the same assumptions as in Example 1, as well as assuming the number of hours of daylight/day in Fairbanks AK oscillates between $4$ hours/day and $20$ hours/day.

\begin{onlineOnly}
    \begin{center}
\desmos{nfihqizmdm}{900}{600}
\end{center}
\end{onlineOnly}

\href{https://www.desmos.com/calculator/nfihqizmdm}{142: Fairbanks 1}

Use \emph{only the protractor above} to help answer the following questions. \emph{Do not change its radius.}

\begin{enumerate}
\item Approximatlely how many hours of daylight/day does Fairbanks get on January 21st?

\item On approximately what days of the year does Fairbanks get $18$ hours of daylight/day? 

\item Approximately what fraction of the summer does Fairbanks get at least $16$ hours of daylight/day?

\item Approximately what fraction of the winter does Fairbanks get a most $8$ hours of daylight/day?

\end{enumerate}

\begin{explanation}


In Fairbanks the number of hours of daylight/day oscillates with a maximum variation of
\[
  \frac{1}{2}\left(20 \text{ hrs} - 4\text{ hrs}  \right) = 8 \text{ hrs}
\]
about a mean of
\[
       \frac{1}{2}\left(4 \text{ hrs} + 20\text{ hrs}  \right) = 12 \text{ hrs} .    
\]

Ideally, we would like a circle of radius $8$ hours centered at $(0,12)$ (or at some other point with the same $y$-coordinate). We'll need to scale the information we can read from the given circle of radius $10$cm centered at the origin by the factor
\[
   \frac{8 \text{ hrs/day}}{10 \text{ cm}} .
\] 

\begin{enumerate}

\item To determine how many hours of daylight/day Fairbanks gets in January 21st, we first use the protractor to approximate the variation 
\[
    \Delta h = h - 12
\]
from the mean in the number of hours of daylight/day on January 21st. In the equation above $h$ is what we're looking for; it is the number of hours of daylight/day on January 21st.

We'll think of the low point $(0,-10)$ of the protractor as representing the winter solstice (when Fairbanks gets $4$ hours of daylight/day). Now January 21st is one month, or $1/12$ of a year after the winter solstice. So we drag point $P$ to one of the two points $1/12 = 2/24$ of a revolution from $(0,-10)$ as shown below.


\begin{onlineOnly}
    \begin{center}
\desmos{4xsc9i7i4s}{900}{600}
\end{center}
\end{onlineOnly}

\href{https://www.desmos.com/calculator/4xsc9i7i4s}{142: Fairbanks 2}

Now point $R$ (the projection of $P$ onto the $y$-axis) has approximate $y$-coordinate $y\sim-8.6$ cm. Using the factor above, we convert this distance to hours. This tells us that
\[
  \Delta h \sim (-8.6 \text{ cm})\left( \frac{8 \text{ hrs/day}}{10 \text{ cm}} \right) \sim -6.9\text{ hrs/day}.
\] 

In other words, on January 21st, Fairbanks gets about $6.9$ hours/day of daylight less than its mean of $12$ hours/day. So on this date, Fairbanks gets about
\[
   h = 12  + \Delta h \sim 5.1 \text{ hrs/day} 
\]
of daylight.

\item To determine on what days of the year Fairbanks gets $18$ hours/day of daylight we reverse the above computation. 

First, with $h=18$ hours/day of daylight the variation from the mean is
\[
  \Delta h = (18-12)\text{ hrs/day} = 6 \text{ hours/day} .
\]

Now use the conversion factor to convert this to centimeters on our protractor, 
\[
       (6 \text{ hours/day}) \left( \frac{10 \text{ cm}}{8 \text{ hrs/day}} \right) = 7.5\text{ cm}. 
\]

So we drag point $P$ so that $R$ has $y$-coordinate $y=7.5$ cm as shown below.

\begin{onlineOnly}
    \begin{center}
\desmos{xvy6s6aw41}{900}{600}
\end{center}
\end{onlineOnly}

\href{https://www.desmos.com/calculator/xvy6s6aw41}{142: Fairbanks 3}

Measured from the top point $(0,10)$ of the circle (corresponding to the summer solstice when the number of hours of daylight/day is a maximum), $P$ is about
\[
      \frac{2.8\text{ tics}}{24\text{ tics/rev}} \sim 0.12\text{ rev}
\]
around the circle. Since one revolution corresponds to a year (360 days), Fairbanks gets $18$ hours/day of daylight on the days that are approximately
\[ 
   ( 0.12\text{ rev}) ( 360 \text{ days/rev}   )  \sim 42\text{ days}
\]
from June 21. These are about May 9 or August 3.

\end{enumerate}
\end{explanation}

\end{example}


Here's another example.


\begin{example}  \label{E888b0bbdsdsf}
Suppose that over the course of a 24-hour period, from midnight October 29 to midnight October 30, the depth of the water at the Edmonds Pier is a sinusoidal function of time. Suppose also that a high tide of 21 feet occurs at 2:00am and the following low tide of 5 feet occurs at 8:00am. 

We'll ask two questions:

\begin{enumerate}
\item Use circular interpolation with the protractor in \emph{Example 3} (\emph{Do not change its radius.}) to approximate the depth of the water at 5:30am on October 29. \emph{Use the protractor below only if you get stuck.}

\item Use the circular protractor  in \emph{Example 3} (\emph{Do not change its radius.}) to approximate the first two times after midnight when the water is $18$ feet deep.  \emph{Use the protractor below only if you get stuck.}

Explain your reasoning and include screenshots to help with your explanations.
\begin{freeResponse}
\end{freeResponse}

\begin{onlineOnly}
    \begin{center}
\desmos{0wxwmkzvky}{900}{600}
\end{center}
\end{onlineOnly}

\href{https://www.desmos.com/calculator/0wxwmkzvky}{142: Circular Interpolation Tides}

\emph{Keep in mind that high tide occurs at 2:00am and low tide at 8:00am.}

\end{enumerate}
\end{example}

\begin{example} \label{Ex:L3rdsfr43r}
The graph below shows the variation of the tides at the Edmonds Pier on Sunday, June 1, 2025.
\begin{onlineOnly}
    \begin{center}
\desmos{5921617274}{900}{600}
\end{center}
\end{onlineOnly}

\href{https://www.desmos.com/calculator/5921617274}{142: Edmonds Tide Graph Real}

\emph{Use only the protractor of Example 3} (\emph{Do not change its radius.}) to answer the following questions.

\begin{enumerate}
\item Using the high and low times at 9am and 3:30pm, estimate the depth of the water at noon. Then compare your estimate with the graph above.

\item Using the high and low times at 9am and 3:30pm, estimate when the water is two feet deep. Then compare your estimates with the graph above.
\end{enumerate}



\end{example}




\end{document}