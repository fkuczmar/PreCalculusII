\documentclass{ximera}
\title{The Inverse Tangent Function}

\newcommand{\pskip}{\vskip 0.1 in}

\begin{document}
\begin{abstract}
Introduction to the tangent and inverse tangent functions.
\end{abstract}
\maketitle



\section{Discussion Questions}

\begin{question} \label{Q45dsfTRETFGree}
Use the radian protractor below to approximate each of the following.
\begin{enumerate}
\item $\tan(-2)$

\item $\arctan(-2)$

\item $\tan(7/10)$

\item $\arctan(7/10)$
\end{enumerate}

\href{https://www.desmos.com/calculator/tb0hegzfdv}{142: Radian Protractor 4}
 
\begin{onlineOnly}
    \begin{center}
\desmos{tb0hegzfdv}{900}{600}
\end{center}
\end{onlineOnly}
\end{question}


\begin{question} \label{QDFrLMDndvvV}
Evaluate each of the following expressions without technology. Draw pictures to help with your explanations. Do \emph{not} work on the unit circle.

\begin{enumerate}
\item $\tan(\pi/4)$

\item $\tan(3\pi/4)$

\item $\tan(5\pi/4)$

\item $\tan(53\pi + 5\pi/4)$

\item $\tan(\pi/2)$

\item $\tan(0)$.

\end{enumerate}
\end{question}

\begin{question} \label{Q89DfegvVDE}
Draw pictures to help with your explanations.
\begin{enumerate}
\item Suppose $\tan \theta = 5$ and evaluate $\tan(\theta + \pi)$.

\item Express $\tan(\theta+\pi)$ in terms of $\tan \theta$.

\item Suppose $\cos \theta = 2/3$ and evaluate $\cos(\theta + \pi)$.

\item Express $\cos(\theta+\pi)$ in terms of $\cos \theta$.

\item Suppose $\sin \theta = 2/3$ and evaluate $\sin(\theta + \pi)$.

\item Express $\sin(\theta+\pi)$ in terms of $\sin \theta$.

\item Express $\tan(\theta+\pi/2)$ in terms of $\tan \theta$.

\item Express $\tan(\theta -\pi/2)$ in terms of $\tan \theta$.
\end{enumerate}
\end{question}



\begin{question}  \label{Q9DDFegbb}
For each of the vectors below, do the following.

\begin{enumerate}
\item Use the radian protractor in the Question 1 to estimate the polar angle. Meaure the angle to be between $0$ and $2\pi$.

\item Use the inverse sine, inverse cosine, and inverse tangent functions to find three expressions for each polar angle. Draw pictures to help with your explanations.

\item Use a calculator to check your expressions.

\end{enumerate}

Here are the vectors.

\begin{enumerate}
\item $\overrightarrow{AB} = \langle -4, 7 \rangle$

\item $\overrightarrow{CD} = \langle -4, -7 \rangle$

\item $\overrightarrow{CD} = \langle 4, -7 \rangle$

\end{enumerate}
\end{question}

\begin{question} \label{QEggdR34rRER}
Point $Q$ is $5$ miles due north of point $D$. Point $F$ is $10$ miles from $D$ at a bearing of $4$ radians (measured counterclockwise from the east). Express the bearing of $F$ relative to $Q$ by giving the exact distance (simplified, no calculators) between $F$ and $Q$ and the exact bearing of the direct path from $Q$ to $F$. Give the bearing in three ways:

\begin{enumerate}
\item using the inverse cosine function

\item using the inverse sine function

\item using the inverse tangent function

\end{enumerate}
\end{question}

\begin{question} \label{QOgggERE34rd}

Point $P$ moves counterclockwise around a circle of radius $4$ meters centered at the origin $O$ at a constant speed of $8$ m/sec, passing the point $(4,0)$ at noon.


One end of a rod $\overline{QP}$ of length $10$ meters is attached to $P$, while the other end $Q$ slides along the $x$-axis as shown below. 
 
\href{https://www.desmos.com/calculator/dra9jtsynq}{142: Polar Angle UCM}.

 
\begin{onlineOnly}
    \begin{center}
\desmos{dra9jtsynq}{900}{600}
\end{center}
\end{onlineOnly}


\begin{enumerate}

\item Use the animation to sketch by hand a graph of the function
\[
    \theta = a(t) \, , \, t\geq 0
\]
that expresses the polar angle (as marked) of the vector $\overrightarrow{QP}$ in terms of the number of seconds past noon.

\item Find an expression for the function $\theta = a(t)$.

\end{enumerate}
\end{question}


\begin{question} \label{QPODferFDRE}

This question is almost identical to the previous one. The only difference is that the lower end $Q$ of the rod slides along the negative $y$-axis as shown below. The questions remain the same. 

\begin{enumerate}

\item Use the animation to sketch by hand a graph of the function
\[
    \theta = a(t) \, , \, t\geq 0
\]
that expresses the polar angle (as marked) of the vector $\overrightarrow{QP}$ in terms of the number of seconds past noon.

\item Find an expression for the function $\theta = a(t)$.

\end{enumerate}

\href{https://www.desmos.com/calculator/jcyu1r8vwn}{142: Polar Angle UCM}.

 
\begin{onlineOnly}
    \begin{center}
\desmos{jcyu1r8vwn}{900}{600}
\end{center}
\end{onlineOnly}


\end{question}


\begin{question} \label{Q45rghREERwer}
Approximate each of the following expressions \emph{without} using technology.
\begin{enumerate}
\item $\tan \left(\frac{\pi}{2} - 0.01\right)$

\item $\tan \left(\frac{\pi}{2} + 0.01 \right)$

\item $\arctan(2000)$

\item $\arctan(-2000)$
\end{enumerate}

\end{question}

\end{document}