\documentclass{ximera}
\title{Inverse Trigonometric Functions CW}

\newcommand{\pskip}{\vskip 0.1 in}

\begin{document}
\begin{abstract}
Introduction to inverse trig.
\end{abstract}
\maketitle

\section{Inverse Trig Functions, the Basics}


\begin{question} \label{EJDFefe33}

\begin{enumerate}
\item Explain what it means for two functions to be inverses of each other.

\item Give some examples of pairs of inverse functions.
\end{enumerate}

\end{question}

\begin{question} \label{QOerrgbxxxe3}
\begin{enumerate}
\item Explain the meaning of $\sqrt{831}$.

\item Approximate $\sqrt{831}$ with a \emph{four-function} calculator.
\end{enumerate}
\end{question}

\begin{question} \label{Q87dfe3RE}
\begin{enumerate}

\item True or False: $\sin^{-1} (x) = 1/(\sin x)$?

\item True or False: The inverse sine function is the inverse of the sine function.

\item What is the inverse of the function $\theta = \arcsin(y)$?

\item True or False: The inverse cosine function is the inverse of the cosine function.

\item What is the inverse of the function $\theta = \arccos(x)$?

\end{enumerate}

\end{question}



\begin{question} \label{QKEDf3err3}
\begin{enumerate}
\item Explain the meaning of $\arccos(-0.32) = \cos^{-1}(-0.32)$

\item Would you expect $\arccos(-0.32)$ to be postive or negative? Explain.

\item Use the radian protractor below to approximate the value of $\arccos(-0.32)$.

\pdfOnly{
Access Desmos interactives through the online version of this text at
 
\href{https://www.desmos.com/calculator/kf29wdiej0}.
}
 
\begin{onlineOnly}
    \begin{center}
\desmos{kf29wdiej0}{900}{600}
\end{center}
\end{onlineOnly}

\href{https://www.desmos.com/calculator/kf29wdiej0}{142: Radian Protractor 3}

\item Would it have been possible to define $\arccos(-0.32)$ as the angle between $\pi/2$ and $3\pi/2$ whose cosine is $(-0.32)$? Explain.

\item Would it have been possible to define $\arccos(-0.32)$ as the angle between $\pi$ and $2\pi$ whose cosine is $(-0.32)$? Explain.

\item Repeat parts (a)-(e) for $\arcsin(-0.32) = \sin^{-1}(-0.32)$.

\end{enumerate}
\end{question}


\begin{question} \label{QKDferr33rr}
Work on a circle of radius $2$ cm to evaluate each of the following expressions.

\begin{enumerate}
\item $\arccos 0$

\item $\arccos 1$

\item $\arccos(-1)$

\item $\arccos(-1/2)$

\item $\arcsin 0$

\item $\arcsin 1$

\item $\arcsin(-1)$

\item $\arcsin(-1/2)$

\end{enumerate}
\end{question}



\iffalse

******************************************************************************************************

\begin{question} \label{QPfer3rr3fd}

Let
\[
    f(x) = x^3 .
\]

\begin{enumerate}

\item Solve the equation
\[
   f(x) =25 .
\]

\item Find an expression for the inverse of the function
\[
   f(x) = x^3 .
\]



\item Describe the relationship between parts (a) and (b).

\item Repeat parts (a)-(c) for the function
\[
  f(x) = 2^x .
\]

\end{enumerate}
\end{question}


\begin{question} \label{QF3err3wefr}
Let 
\[
    f(x) = x^2.
\]
\begin{enumerate}

\item Solve the equation
\[
  f(x) = 5 .
\]

\item Is the inverse of the function
\[
    f(x) = x^2
\]
also a function? Explain.

\item Find the inverse of the function
\[
      g(x) = x^2 \, , \, x\geq 0.
\]

\item  Find the inverse of the function
\[
      h(x) = x^2 \, , \, x\leq 0.
\]
\end{enumerate}
\end{question}

\fi

***********************************************************************************


\section{The Inverse Sine and Inverse Cosine Functions}

\begin{question} \label{Ex1dsfgt4hh:Inverse}
For each of the following expressions, do the following.

\begin{itemize}

\item Explain its meaning in a complete sentence.

\item Use the radian protractor below to approximate the expression. Explain your reasoning. 

\end{itemize}

\begin{enumerate}
\item $\cos^{-1}(0.64)$

\item $\cos^{-1}(-0.64)$

\item $\sin^{-1}(0.64)$

\item $\sin^{-1}(-0.64)$

\item $\sin^{-1}(-4)$

\end{enumerate}

\begin{exploration}\label{Exp3:Comp}

\pdfOnly{
Access Desmos interactives through the online version of this text at
 
\href{https://www.desmos.com/calculator/kf29wdiej0}.
}
 
\begin{onlineOnly}
    \begin{center}
\desmos{kf29wdiej0}{900}{600}
\end{center}
\end{onlineOnly}
\end{exploration}

\end{question}

\iffalse

*********************************************************************************************************

\begin{question} \label{Q34d4r34tgFG}
For each of the following expressions:
\begin{itemize}
\item Explain its meaning in a complete sentence.

\item Evaluate the expression without using a caclulator. Draw a picture to help with your explanation.
\end{itemize}

\begin{enumerate}
\item $\cos (\arccos(-3/8))$

\item $\arccos (\cos (4\pi/5))$

\item $\arccos (\cos (7\pi))$

\item $\arccos (\cos (7\pi/5))$

\item $\arcsin (\sin (7\pi/5))$

\item $\arccos (\cos (16\pi/9))$

\item $\arcsin (\sin (16\pi/9))$

\item $\arcsin(\sin (23\pi /7))$

\item $\arcsin(\sin (-23\pi /7))$

\item $\arccos(\sin (\pi /7))$

\item $\arccos(\sin (12\pi /7))$

\item $\sin(\arccos(-3/4))$

\item $\cos(\arcsin(-3/4))$
\end{enumerate}

\begin{explanation}
(d) The expression $\arccos (\cos (7\pi/5))$ is the angle between $0$ and $\pi$ that has the same cosine as the angle $7\pi/5$.

To evaluate this expression, we'll let 
\[
   \theta = \arccos (\cos (7\pi/5)).
\] 

From our description, we know
\begin{enumerate}
\item $0 \leq \theta \leq \pi$, and

\item $\cos \theta =  \cos(7\pi/5))$.
\end{enumerate}

We should point out that because $7\pi/5$ is \emph{not} between $0$ and $\pi$,
\[
   \theta =   \arccos (\cos (7\pi/5)) \neq 7\pi/5 .
\]

To determine the value of $\theta$, we'll work on the unit circle and start by stetching the point $P$ on the unit circle with coordinates $(\cos (7\pi/5), \sin(7\pi/5))$ and polar angle 
\[
     \angle AOP =  \frac{7\pi}{5} = \pi + \frac{2\pi}{5} .
\]

See the picture below.

\href{https://www.desmos.com/calculator/wzxssez3fu}{142: Inverse Trig Composition}

 
\begin{onlineOnly}
    \begin{center}
\desmos{wzxssez3fu}{900}{600}
\end{center}
\end{onlineOnly}

Now there are infinitely many other choices for the polar angle of $P$ (or to be more precise, for the polar angle of $\overrightarrow{OP}$), but because $P$ is in the third quadrant, none of these choices lies between $0$ and $\pi$. 

So we must find another family of angles with the same cosine as $7\pi/5$. To do this we draw the vertical line
\[
   x = \cos(7\pi/5)
\]
through $P$ and see where it intersects the unit circle a second time. We'll call this point $Q$. Now since $P$ and $Q$ share the same $x$-coordinate (and because $\cos\theta = x/r$), w know that
\[
 \cos (\angle AOP) = \cos (\angle AOQ).
\]
The angle $\angle AOQ$ is the one we're looking for. It has the same cosine as $\angle AOP = 7\pi/5$ and its measure is between $0$ and $\pi$. 


%This vertical line intersects the circle in two points, $P$ and $Q$. Because $Q$ is in the third quadrant, we cannot measure the polar angle of the vector $\overrightarrow{OP}$ to be between $0$ and $\pi$.

%So we should focus our attention on point $Q$, the other point where the line $x = \cos(7\pi/5)$ intersects the unit circle. We have infinitely many choices for the polar angle of the vector $\overrightarrow{OQ}$, but because $P$ and $Q$ have the same $x$-coordinate, each of these choices has the same cosine as the angle $7\pi/5$. We need to choose the one angle, the marked angle $\angle AOQ$, that is between $0$ and $\pi$. We can do this because $Q$ is in the second quadrant.

Because $P$ and $Q$ are symmetric about the $x$-axis, this measure of angle $\angle AOQ$ is
\[
   \angle AOQ = 2\pi - \frac{7\pi}{5} = \frac{3\pi}{5}.
\]
 
So our conclusion is that
\[
     \arccos (\cos (7\pi/5)) = \frac{3\pi}{5}.
\]

If we wanted to summarize our reasoning, we could give more detail like this:
\begin{align*}
 \arccos (\cos (7\pi/5)) &= \arccos (\cos (2\pi - 7\pi/5)) \\
                                 &= \arccos (\cos (3\pi/5)) \\
                                 & = 3\pi/5 .
\end{align*}


(e) The expression $\arcsin (\sin (7\pi/5))$ means the angle between $-\pi/2$ and $\pi/2$ that has the same sine as the angle $7\pi/5$.

To find this angle we use reasoning much like we did to evaluate $\arccos (\cos (7\pi/5))$. But we need to make some adjustments to account for replacing the cosine and arccosine functions with sine and cosine.

I'll leave the details for you to fill in, but here's the picture we need.

\href{https://www.desmos.com/calculator/ilztfm9da1}{142: Inverse Trig Composition 2}

 
\begin{onlineOnly}
    \begin{center}
\desmos{ilztfm9da1}{900}{600}
\end{center}
\end{onlineOnly}

You should finally conclude that
\begin{align*}
 \arcsin (\sin (7\pi/5)) &= \arcsin (\sin (\pi - 7\pi/5)) \\
                                 &= \arcsin (\sin (-2\pi/5)) \\
                                 & = -2\pi/5 .
\end{align*}

\end{explanation}

\end{question}

\fi

***********************************************************************************************

\section{Back to Bearings}

\begin{question} \label{QOker343}
Points $A$ and $B$ have respective coordinates $A(-3,1)$ and $B(-8,-2)$. The coordinates are measured in miles.

\begin{enumerate}
\item What vector gives the position of $A$ relative to $B$? Sketch this vector.

\item Use the worksheet below to approximate 
\begin{enumerate}
\item the distance from $B$ to $A$ and

\item the bearing of the direct path from $B$ to $A$, measured counterclockwise from the east.

 
\begin{onlineOnly}
    \begin{center}
\geogebra{bhdsgxtx}{900}{600}
\end{center}
\end{onlineOnly}

\href{https://www.geogebra.org/classic/bhdsgxtx}{142: Relative Position}.

\end{enumerate}

\item Give directions to get \emph{directly} from $B$ to $A$. Do this  \emph{without a calculator} by computing the exact distance and the bearing (measured counterclockwise from the east). Then use a calculator to approximate the distance (to the nearest tenth of a mile) and the bearing (to the nearest tenth of a radian).

Give \emph{two} expressions for the bearing:

\begin{enumerate}
\item One using the inverse cosine function

\item Another using the inverse sine function.
\end{enumerate}

\item Repeat parts (a) and (b) for the vector giving the position of $B$ relative to $A$.

\end{enumerate}
\end{question}

\begin{question} \label{QOker333243}
Points $A$ and $B$ have respective coordinates $A(-3,1)$ and $B(-8,5)$. The coordinates are measured in miles.

\begin{enumerate}
\item What vector gives the position of $A$ relative to $B$? Sketch this vector.

\item Give directions to get \emph{directly} from $B$ to $A$. Do this  \emph{without a calculator} by computing the exact distance and the bearing (measured counterclockwise from the east). Then use a calculator to approximate the distance (to the nearest tenth of a mile) and the bearing (to the nearest tenth of a radian).

Use the components of the vector for the position of $A$ relative to $B$ to give \emph{two} expressions for the bearing:

\begin{enumerate}
\item One using the inverse cosine function

\item Another using the inverse sine function.
\end{enumerate}

\item Repeat parts (a) and (b) for the vector giving the position of $B$ relative to $A$.

\end{enumerate}
\end{question}


\begin{question} \label{QKDfe3r3rr3}
Point $A$, different from the origin, has coordinates $(a,b)$. 

Find an expression for the polar angle $\theta$ of the vector $\overrightarrow{OA}$ giving the position of $A$ relative to the origin. Do this four times, each time making sure that $0\leq \theta <2\pi$.

\begin{enumerate}
\item First assume $a>0$.

\item Then assume $b>0$.

\item Next assume $a<0$.

\item Finally assume $b<0$.
\end{enumerate}

\end{question}


\begin{question} \label{QLFefm333d}
Point $A$ is $5$ miles due north of point $B$. Point $C$ is $10$ miles from $B$ at a bearing of $2.7$ radians.

\begin{enumerate}

\item Draw a picture.

\item Find the exact components (do \emph{not} use a calculator) of the vector giving the position of $C$ relative to $A$.

\item Give directions for how to get directly from $A$ to $C$ by giving

\begin{enumerate}
\item A simplified expression for the exact distance from $A$ to $C$. Do not use a caculator.

\item The exact bearing of the direct path from $A$ to $C$. Do not use a calculator. \emph{Hint:} Use Question 9.

\item Then use a calculator to approximate the distance (to the nearsest mile) and bearing (to the nearest tenth of a radian).
\end{enumerate}

\end{enumerate}
\end{question}


\section{Finding the Inverse of a Function}

\begin{question} \label{Q25:InverseTrig45}
The function
\[
  h = f(t) = 2500 - 200 \cos \left(  \frac{\pi}{12}t  \right) \, , 0\leq t \leq 10 ,
\]
expresses the height (in feet) of a balloon in terms of the number of minutes past noon.

\pskip

(a) Find a function $T=u(h)$ that takes a height (in feet) as an input and returns as an output the time (measured in minutes past noon) when the balloon is at that height and on its way up. Include the correct domain.

(b) Find a function $t=d(h)$ that  takes a height (in feet) as an input and returns as an output the time (measured in minutes past noon) when the balloon is at that height and on its way down. Include the correct domain.

(c) Explain the meaning of the composition $u\circ f$. What does the function take as an input? What does it return as an output? Then simplify the composition as much as possible (there should be no trig or inverse trig functions in your final result). Finally graph the compostion over its domain.

(d) Repeat part (d) for the composition $d\circ f$.

\end{question}



\begin{question} \label{Q28:InverseTrig56}
The function
\[
  h = f(t) = 2500 - 200 \cos \left(  \frac{\pi}{6} \sqrt{t}  \right) \, , 0\leq t \leq 81 ,
\]
expresses the height (in feet) of a balloon in terms of the number of minutes past noon.

\pskip

Answer parts (a)-(d) of the previous question for this new height function.

\end{question}

\end{document}

