\documentclass{ximera}
\title{The Cosine and Sine Fucntions}

\newcommand{\pskip}{\vskip 0.1 in}

\begin{document}
\begin{abstract}
An introduction to circular trigonometry.
\end{abstract}
\maketitle

You might have noticed at the end of the last chapter that we did not parameterize the motion of the bugs crawling around a circle at a constant speed. That was because we did not have the cosine and sine functions and these are needed to parameterize uniform circular motion.

Imagine standing at the origin facing the positive $x$-axis. Now rotate counterclockwise through an angle of $\theta$ radians and walk any distance, say $r$ meters in this direction, and stop at the point $P$ with coordinates $(x,y)$.

Now the cosine function takes as an input the (dimensionless) angle $\theta$ measured in radians, and returns as an output the \emph{dimensionless} ratio $x/r$. We write this as 
\[
   \cos \theta = \frac{x \text{ cm}}{r \text{ cm}} = \frac{x}{r} .
\]

Similarly, the sine of the angle $\theta$ is defined as the \emph{dimensionless} ratio $y/r$, so that
\[
   \sin \theta = \frac{y \text{ cm}}{r \text{ cm}} = \frac{y}{r} .
\]

\begin{question} \label{Q1:Cosine}
Explain why the values of $\cos \theta$ and $\sin\theta$ do {\bf not} depend on $r$ (ie. on how far you walk away from the origin). 
\end{question}



Many of you are probably familiar with the unit circle. But working with the unit circle makes it seem like the outputs to these functions are lengths. It obscures the key point that the outputs are dimensionless. So do yourself a favor and {\bf forget about the unit circle.} It is also not necessary to waste time memorizing anything about the unit circle. Instead, knowing the definitions of the cosine and sine functions is enough to figure out all you need.

It's one thing to define these functions, but quite another to compute or approximate their values. Much effort and ingenuity has gone into this problem over many centuries. But we can start to get a feel for the sine and cosine functions by using a radian protractor to approximate some of their outputs.


\begin{exploration}\label{Exp1:CF}
Use the radian protractor below to approximate each of the following being sure {\bf not} to change the radius. Show a screenshot and give a complete explanation for each. Include all units in your computations. Then approximate each ratio to the nearest hundredth. You may use a {\bf four-function calculator} for the last two of these questions to do arithmetic.

(a) $\cos 5$, $\sin 5$  (no calculator)

(b) $\cos (-5)$,  $\sin (-5)$ (no calculator)

(c) $\cos 2.7$, $\sin 2.7$ (no calculator)

(d) $\cos (83\pi + 2.7)$, $\sin (83\pi-2.7)$ (no calculator)

(e) $\cos 360$, $\sin 360$ (four-function calculator)

(f) $\cos 200^\circ$, $\sin 200^\circ$ (four-function calculator)

\pdfOnly{
Access Desmos interactives through the online version of this text at
 
\href{https://www.desmos.com/calculator/kdakzcloqr}.
}
 
\begin{onlineOnly}
    \begin{center}
\desmos{kdakzcloqr}{900}{600}
\end{center}
\end{onlineOnly}
\end{exploration}

\begin{question} \label{Q2:Cosine}
Approximate the value of $\cos (\pi^\circ)$. Explain your reasoning and include a picture to help with your explanation.
\end{question}


\section{Parameterizing Motions}
From the definitions of $\cos\theta$ and $\sin \theta$, we know
\[
    x = r \cos \theta 
\]
and
\[
      y=r\sin \theta .
\]
These equations tell us how to compute the coordinates of point given its distance to the orign $r$ and its \emph{polar angle} $\theta$.

Regarding $\theta$ as a constant lets us parameterize a segment from the origin that makes the angle $\theta$ with the postive $x$-axis.

\begin{example} \label{Ex11:Cosine}
Starting at the origin facing the positive $x$-axis, you turn counterclockwise through an angle of 2 radians. You then walk $10$ meters directly away from the origin. Express the coordinates of a point along your path in terms of its distance $r$ from the origin.

\begin{explanation}
The coordinate functions
\[
     x = f(r) = r\cos 2  , \,\,   0\leq r \leq 10
\]
and
\[
  y =g(r) = r \sin 2  , \,\,   0\leq r \leq 10
\]
parameterize this path in terms of the distance to the origin.
\end{explanation}
\end{example}



Regarding $r$ as a constant, they also tell us how to paramterize the circle $x^2 + y^2 = r^2$ in terms of the polar angle $\theta$.

\begin{example} \label{Ex10:Cosine}
Parameterize the circle of radius $10$ centered at the point $(4,9)$.

\begin{explanation}
The coordinate functions
\[
     x = f(\theta) = 10\cos \theta  , \, \,   0\leq \theta \leq 2\pi
\]
and
\[
  y =g(\theta) = 10\sin \theta   , \,\, 0\leq \theta \leq 2\pi
\]
parameterize the circle of radius $10$ centered at the origin. Adding $4$ to the $x$-coordinates and $9$ to the $y$-coordinates translates this circle centered at $(0,0)$ to the circle of radius 10 centered at $(4,9)$. So the latter has parameterization
\[
     x = f(\theta) = 4+ 10\cos \theta ,  \,\,  0\leq \theta \leq 2\pi
\]
and
\[
  y = g(\theta) = 9 + 10\sin \theta  , \,\,   0\leq \theta \leq 2\pi .
\]
\end{explanation}
\end{example}




\begin{question}  \label{Q2:Cosine}
Parameterize each of the following motions from time $t=0$ seconds to time $t=20$ seconds.  All coordinates are measured in meters. Use the point-slope form of a linear function when necessary. Check that your parameterizations are correct in the Desmos worksheet  below. 

(a) A motion that rotates counter-clockwise at a constant rate of $20^\circ$/sec around the circle centered at the point $(-2,10)$ and passes the point $(5,10)$ at time $t=13$ seconds. 

(b) A motion that rotates clockwise around the circle centered at the point $(-2,10)$ at a constant speed of $3$m/s  and passes the point $(-2,0)$ at time $t=13$ seconds. 

(c) A motion that moves along a path inclined at the angle of $5\pi/8$ radians to the postive $x$-axis. The motion has a constant speed of $1/4$ m/s and leaves the origin at time $t=0$ seconds.

(d) A motion in a fixed direction with a constant speed that passes the point $A(3,-2)$ at time $t=6$ seconds and the point $B(-1,8)$ at time $t=10$ seconds.

(e) Any clockwise motion about the ellipse
\[
   \frac{x^2}{25} + \frac{y^2}{15} = 1 .
\]

(f) A counterclockwise motion around the circle $x^2+y^2=r^2$ that rotates about the point $(a,0)$ at a constant rate. Assume $-r<a<r$.

\pdfOnly{
Access Desmos interactives through the online version of this text at
 
\href{https://www.desmos.com/calculator/3r6ncnf3fc}.
}
 
\begin{onlineOnly}
    \begin{center}
\desmos{3r6ncnf3fc}{900}{600}
\end{center}
\end{onlineOnly}


\end{question}


\end{document}
