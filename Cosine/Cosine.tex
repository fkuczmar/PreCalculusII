\documentclass{ximera}
\title{The Cosine and Sine Fucntions}

\newcommand{\pskip}{\vskip 0.1 in}

\begin{document}
\begin{abstract}
An introduction to circular trigonometry.
\end{abstract}
\maketitle

\section{The Position Problem}

The fundamental idea of this class gives a way to describe how your position changes if you walk a given distance in a fixed direction. For example, you might walk $100$ meters in the direction $40^\circ$ east of north and ask how far north and how far east you end up relative to your starting point.

Here's a problem like this from the last chapter, but expressed in a more general way.

\pskip

\begin{question} \label{Q2435r6:Cosine}

You stand at the origin facing the direction of the postive $x$-axis. Estimate your coordinates after you turn counterclockwise about the origin through an angle of 4 radians and then walk $r$ meters directly away from the origin.

\begin{explanation}
The idea is to solve the same problem for a circle of some specific radius and then scale the solution by the appropriate factor. We'll choose a radius of $r=7$ meters, but any radius will work.

We use the radian protractor to estimate our coordinates $(x, y)$ if we had walked $7$ meters directly away from the origin.  It looks like the coordinates of the point $P$ at the 4 radian mark (see below) are about 
\[
   x\sim -4.7 \text{ meters}
\]
and 
\[
  y\sim -5.3 \text{ meters}.
\]

To approximate our coordinates if we walked $r$ meters instead of $7$ meters, we just need to scale these coordinates by a factor of $(r\text{ meters})/(7\text{ meters}) = r/7$. Then our coordinates $(x^\prime,y^\prime)$ would be about
\[
   x^\prime \sim (-4.7\text{ m}) \left( \frac{r\text{ meters}}{7\text{ meters}} \right) = (r \text{ meters})\left( \frac{-4.7\text{ m}}{7\text{ m}}\right) \sim  -0.67r   \text{ meters}.
\]
and
\[
   y^\prime   \sim (-5.3\text{ m}) \left( \frac{r\text{ meters}}{7\text{ meters}} \right) = (r \text{ meters})\left( \frac{-5.7\text{ m}}{7\text{ m}}\right) \sim   -0.76 r \text{ meters}.
\]

\pdfOnly{
Access Desmos interactives through the online version of this text at
 
\href{https://www.desmos.com/calculator/vb1kvzkzlq}.
}
 
\begin{onlineOnly}
    \begin{center}
\desmos{vb1kvzkzlq}{900}{600}
\end{center}
\end{onlineOnly}
\end{explanation}

\pskip

So our coordinates were determined how far we walked and by the two {\bf dimensionless} ratios
\[
    \frac{x}{r} \sim \frac{-4.7\text{ m}}{7\text{ m}}
\] 
and
\[
   \frac{y}{r} \sim \frac{-5.7\text{ m}}{7\text{ m}}
\]
that we approximated from the protractor. These ratios depend on the angle $\theta$ through which we turn but not at all on the radius of the circle that we choose. 

The angle $\theta$, called the {\bf polar angle}, is measured in radians counterclockwise from the direction of the postive $x$-axis. Choosing a point $P$ at the polar angle $\theta$ with coordinates $(x,y)$ cm on any circle of radius $r$ centimeters, these two \emph{dimensionless} ratios are written as

\[
   \cos \theta = \frac{x \text{ cm}}{r \text{ cm}} = \frac{x}{r} .
\]
and as
\[
   \sin \theta = \frac{y \text{ cm}}{r \text{ cm}} = \frac{y}{r} .
\]

There is a third \emph{dimensionless} ratio 
\[
  \tan \theta = \frac{y \text{ cm}}{x \text{ cm}} = \frac{y}{x}
\]
that gives the slope of our path. 


\end{question}



\begin{question} \label{Q1:Cosine}
Explain why the values of $\cos \theta$, $\sin\theta$, and $\tan\theta$ do {\bf not} depend on the radius $r$ of our circle. 
\end{question}



Many of you are probably familiar with the unit circle. But working with the unit circle makes it seem like the outputs to these functions are lengths. It obscures the key point that the outputs are dimensionless. So do yourself a favor and {\bf forget about the unit circle.} It is also not necessary to waste time memorizing anything about the unit circle. Instead, knowing the definitions of the cosine and sine functions is enough to figure out all you need.

It's one thing to define these functions, but quite another to compute or approximate their values. Much effort and ingenuity has gone into this problem over many centuries. But we can start to get a feel for the sine and cosine functions by using a radian protractor to approximate some of their outputs.

Returning to our problem that started this chapter where the polar angle was $\theta=4$, we can use a caclulator to find that
\[
  \cos 4 \sim -0.6536
\]
and
\[
   \sin 4 \sim -0.7568
\]
So our coordinates after walking $100$ meters directly away from the origin are 
\[
     x = r\cos \theta = (100 \text{ m})\cos 4 \sim -65.36\text{ m}
\]
and
\[
     y = r\cos \theta = (100 \text{ m})\sin 4 \sim -75.68\text{ m} .
\]

\begin{question}  \label{QWEREWR:Cosine}
Go back to our work from the opening question using the radian protractor and find our approximate coordinates had we walked $100$ meters directly away from the origin. How do these estimates compare with the estimates from the calculator we just computed?
\end{question}


\begin{exploration}\label{Exp1:CF}
Use the radian protractor below with a radius of $5$ cm to approximate each of the following expressions as best you can. Do {\bf not} change the radius. Show a screenshot and give a brief explanation for each. Include all units in your computations. 

Then use a calculator and compare your estimates with the true values.
%Then approximate each ratio to the nearest hundredth. You may use a {\bf four-function calculator} for the last two of these questions to do arithmetic.

(a) $\cos 3.6$, $\sin 3.6$  %(no calculator)

\begin{explanation}
First move point $P$ to the $3.6$ radian mark on the protractor below on a circle of radius $5$ cm. Then it looks like $P(x,y)$ has approximate coordinates
\[
     x \sim -4.5 \text{ cm}
\]
\[
    y \sim -2.2 \text{ cm} .
\]
So 
\[
   \cos (4.5) = \frac{x}{r} \sim \frac{-4.5 \text{ cm}}{5 \text{ cm}} = -0.9
\]
and 
\[
   \sin (4.5) = \frac{y}{r} \sim \frac{-2.2 \text{ cm}}{5 \text{ cm}} = -0.44 .
\]

\pdfOnly{
Access Desmos interactives through the online version of this text at
 
\href{https://www.desmos.com/calculator/lbkveixdno}.
}
 
\begin{onlineOnly}
    \begin{center}
\desmos{lbkveixdno}{900}{600}
\end{center}
\end{onlineOnly}



\end{explanation}

(b) $\cos (-3.6)$,  $\sin (-3.6)$ %(no calculator)

(c) $\cos 0.1$, $\sin 0.1$ %(no calculator)

%(d) $\cos (83\pi + 2.7)$, $\sin (83\pi-2.7)$ (no calculator)

(e) $\cos 360$, $\sin 360$ (use a four-function calculator to help)

(f) $\cos 200^\circ$, $\sin 200^\circ$ (use a four-function calculator to first convert these angles to radians)

\pdfOnly{
Access Desmos interactives through the online version of this text at
 
\href{https://www.desmos.com/calculator/mpx0pu8w7i}.
}
 
\begin{onlineOnly}
    \begin{center}
\desmos{mpx0pu8w7i}{900}{600}
\end{center}
\end{onlineOnly}
\end{exploration}

\begin{question} \label{Q2:Cosine}
Approximate the value of $\cos (\pi^\circ)$. Explain your reasoning and include a picture to help with your explanation.
\end{question}


\begin{question}  \label{Q346fr:Cosine}
1. Find the exact values of each of the following working on a circle of radius $5$cm. Sketch a picture for each to help explain your reasoning. 

(a) $\cos (2\pi)$

(b) $\sin(3\pi/2)$

(c) $\cos(\pi)$

(d) $\cos(31\pi)$

(e) $\sin (233\pi/2)$

(f) $\sin(-3\pi/2)$

(g) $\sin(3\pi/4)$   (work on a circle of a different radius for this one, but {\bf not} on the unit circle)

(h) $\tan(5\pi/4)$

(i) $\cos(-\pi/4)$

(j) $\sin(-\pi/4)$

(k) $\sin(33\pi/4)$

\pskip

2. How would your answers to the above questions have changed if you had worked on a circle of radius $10$ cm?

\end{question}


\begin{question}  \label{Q2354234:Cosine}
1. Give exact answers to each of the following {\bf without} using a cacluator. All coordinates are measured in meters. Include pictures to help with your explanations.

(a) Starting from the point $(80,0)$ you walk counterclockwise around the circle centered at the origin. Find your coordinates after you walk $176$ meters.

(b) Starting from the point $(80,0)$ you walk clockwise around the circle centered at the origin. Find your coordinates after you walk $176$ meters.

(c)  Starting from the point $(80,0)$ you run counterclockwise around the circle centered at the origin, turning about the origin at the constant rate of $0.04$ rad/sec. Find your coordinates after two minutes.

(d) Starting from the point $(-80,0)$ you run counterclockwise around the circle centered at the origin, turning about the origin at the constant rate of $0.04$ rad/sec. Find your coordinates after two minutes.

\pskip

2. Use the radian protractor from Exploration 4 above to find the approximate coordinates in each of the above questions.

3. Use a calculator to find better approximations (to the nearest hundredth of a meter) to the coordinates in each of the above questions.


\end{question}



\begin{question}  \label{Qsdg53:Cosine}
1. Give exact answers to each of the following {\bf without} using a cacluator. All coordinates are measured in meters. Include pictures to help with your explanations.

(a) You start at the origin facing the positive $x$-axis and turn counterclockise through an angle of $2.7$ radians. You then walk $100$ meters directly away from the origin. Find an equation of your path. Include the appropriate domain. Then graph the path on the radian protractor in Exploration 4 to make sure you are correct.

(b) Repeat part (b) but this time turning clockwise through the same angle. 


\end{question}



\begin{question} \label{Qet433:Cosine}
Give exact answers to each of the following {\bf without} using a cacluator. Include pictures to help with your explanations.

(a) As a beetle crawls directly away from the origin at a constant speed of $5$ cm/sec its $x$-coordinate increases at the rate of $3$ cm/sec. At what rate is the $y$-coordinate of the beetle changing? Find all possiblities. 

(b) As a beetle crawls directly away from the origin at a constant speed of $8$ cm/sec its $y$-coordinate decreases at the rate of $5$ cm/sec. At what rate is the $x$-coordinate of the beetle changing? Find all possiblities. 

(c) As a beetle walks directly away from the point $(6,1)$ its $x$-coordinate increases at the rate of $3$ cm/sec and its $y$-
coordinate decreases at the rate of $5$ cm/sec. 

\pskip

\hskip 0.2 in (i) Find the beetle's speed. 

(ii) Find a Cartesian equation of the beetle's path path (include the domain). Assume the beetle passed the point $(6,1)$ at $t=7$ seconds and crawled between times $t=0$ and $t=12$ seconds. 


\end{question}




\section{Parameterizing Motions}
From the definitions of $\cos\theta$ and $\sin \theta$, we know
\[
    x = r \cos \theta 
\]
and
\[
      y=r\sin \theta .
\]
These equations tell us how to compute the coordinates of point given its distance to the orign $r$ and its \emph{polar angle} $\theta$.

Regarding $\theta$ as a constant lets us parameterize a segment from the origin that makes the angle $\theta$ with the postive $x$-axis.

\begin{example} \label{Ex11:Cosine}
Starting at the origin facing the positive $x$-axis, you turn counterclockwise through an angle of 2 radians. You then walk $10$ meters directly away from the origin. Express the coordinates of a point along your path in terms of its distance $r$ from the origin.

\begin{explanation}
The coordinate functions
\[
     x = f(r) = r\cos 2  , \,\,   0\leq r \leq 10
\]
and
\[
  y =g(r) = r \sin 2  , \,\,   0\leq r \leq 10
\]
parameterize this path in terms of the distance to the origin.
\end{explanation}
\end{example}



Regarding $r$ as a constant, they also tell us how to paramterize the circle $x^2 + y^2 = r^2$ in terms of the polar angle $\theta$.

\begin{example} \label{Ex10:Cosine}
Parameterize the circle of radius $10$ centered at the point $(4,9)$.

\begin{explanation}
The coordinate functions
\[
     x = f(\theta) = 10\cos \theta  , \, \,   0\leq \theta \leq 2\pi
\]
and
\[
  y =g(\theta) = 10\sin \theta   , \,\, 0\leq \theta \leq 2\pi
\]
parameterize the circle of radius $10$ centered at the origin. Adding $4$ to the $x$-coordinates and $9$ to the $y$-coordinates translates this circle centered at $(0,0)$ to the circle of radius 10 centered at $(4,9)$. So the latter has parameterization
\[
     x = f(\theta) = 4+ 10\cos \theta ,  \,\,  0\leq \theta \leq 2\pi
\]
and
\[
  y = g(\theta) = 9 + 10\sin \theta  , \,\,   0\leq \theta \leq 2\pi .
\]
\end{explanation}
\end{example}


\begin{example} \label{Ex12:Cosine}
Parameterize a motion that rotates counterclockwise about the circle of radius $10$ centered at the point $(4,9)$ with a speed of of $3$ cm/s. The motion runs from time $t=0$ to time $t=20$ seconds and passes the point $(14,9)$ at time $t=12$ seconds. The coordinates are measure in centimeters.

\begin{explanation}
We need only to express the polar angle $\theta$ of the motion as a function of $t$ and substitute this expression for $\theta$ in parameterization in the previous example.

Since the motion rotates about the point $(4,9)$ at a rate of
\[
    v = \frac{3 \text{ cm/s}}{10 \text{cm}} = 0.3 \text{ rad/sec} 
\]
and $\theta = 0$ when $t=12$, 
\[
   \theta = 0.3(t-12) , \,\, 0\leq t \leq 20 .
\]
So the coordinate functions of the motion are
\[
     x = f(t) = 4+ 10\cos (0.3(t-12)) ,  \,\,  0\leq \theta \leq 20
\]
and
\[
  y = g(t) = 9 + 10\sin (0.3(t-12))  , \,\,   0\leq t\leq 20 .
\]
\end{explanation}
\end{example}



\begin{question}  \label{Q12:Cosine}
Parameterize each of the following motions from time $t=0$ seconds to time $t=20$ seconds.  All coordinates are measured in meters. Use the point-slope form of a linear function when finding the function that expresses the polar angle in terms of time. Check that your parameterizations are correct in the Desmos worksheet  below. 

(a) A motion that rotates counter-clockwise at a constant rate of $20^\circ/\text{sec}$ around the circle centered at the point $(-2,10)$ and passes the point $(5,10)$ at time $t=13$ seconds. 

(b) A motion that rotates clockwise around the circle centered at the point $(-2,10)$ at a constant speed of $3$m/s  and passes the point $(-2,0)$ at time $t=13$ seconds. 

(c) A motion that moves along a path inclined at the angle of $5\pi/8$ radians to the postive $x$-axis. The motion has a constant speed of $1/4$ m/s and leaves the origin at time $t=0$ seconds.

(d) A motion in a fixed direction with a constant speed that passes the point $A(3,-2)$ at time $t=6$ seconds and the point $B(-1,8)$ at time $t=10$ seconds.

(e) A clockwise motion about the ellipse
\[
   \frac{x^2}{25} + \frac{y^2}{15} = 1 .
\]
Does your motion rotate about the origin at a constant rate?

(f) A motion along the semi-circle
\[
    x = \sqrt{25 - y^2}
\]
that oscillates between the points $(0,5)$ and $(0,-5)$ and returns to its starting position $(0,5)$ every 4 seconds.

(g) A counterclockwise motion around the circle $x^2+y^2=r^2$ that rotates about the point $(a,0)$ at a constant rate. Assume $-r<a<r$.

\pdfOnly{
Access Desmos interactives through the online version of this text at
 
\href{https://www.desmos.com/calculator/3r6ncnf3fc}.
}
 
\begin{onlineOnly}
    \begin{center}
\desmos{3r6ncnf3fc}{900}{600}
\end{center}
\end{onlineOnly}

\end{question}

\begin{question} \label{Q3:Cosine}
Partial information is given below regarding three uniform circlular motions. Describe possible motions for each by giving

(a) the coordinates of the center and the radius of the path,

(b) the rate and sense of rotation about the path's center, and

(c) the position of the motion at a specific time.

Then find the missing coordinate function for each of your descriptions.

(a) $x=f(t) = 5 + 3 \cos \left( \frac{2}{5}t \right)$

(b) $y = g(t) = 5 - 3 \sin \left( \frac{2\pi}{5}t \right)$

(c) $x=f(t) = 5 - 3 \cos \left( \frac{\pi}{12}\left( t+5 \right) \right)$


\end{question}


\end{document}
