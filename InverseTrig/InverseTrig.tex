\documentclass{ximera}
\title{Inverse Trigonometric Functions}

\newcommand{\pskip}{\vskip 0.1 in}

\begin{document}
\begin{abstract}
Introduction to inverse trig.
\end{abstract}
\maketitle

Suppose I pick a number, square it, and get 25. Then there are two possibilities for my original number, either $5$ or $-5$. This says the squaring function $f(x)=x^2$ is \emph{not} one-to-one and implies that its inverse is \emph{not} a function. 

Now you might object and claim that the function $h(x)=\sqrt{x}$ is the inverse of $f$. But that would not be correct. Rather, the square root function $h$ is the inverse of the  function
\[
    g(x) = x^2 , x\geq 0 .
\]
This new function $g$ is \emph{not} the same as $f$. The function $g$ is the squaring function with a restricted domain, $x\geq 0$. On this restricted domain, $g$ is one-to-one and its inverse $g^{-1}(x) = \sqrt{x}$ is a function. This is the familiar square-root function. Now if I pick a non-negative number and square it to get 25, there is only one possiblity for my original number, namely $5$. So it is not enough to say, for example, that $\sqrt{19}$ is the number whose square is 19. Rather it is the \emph{positive} number whose square is 19.

The sine and cosine functions work similarly. They are also not one-to-one and so their inverses are not functions. It is only by restricting the domains of these functions that we get \emph{new} functions that are one-to-one. And the inverses of these new functions are what we call the \emph{inverse sine and cosine functions}.

How to restrict the domains? The idea is to restrict the domains of the sine and cosines function so that 
\begin{itemize}

\item{over these restricted domains the new functions are one-to-one and}

\item{over these restricted domains the new functions give all the possible outputs (ie. all the numbers between $-1$ and $1$, endpoints included.)}

\end{itemize}

There are infinitley many ways to make such domain restrictions, but by convention mathematicians have arbitrarily decided how.

For the sine function, the new function with restricted domain is 
\[
    y  = f(\theta) = \sin \theta , -\pi/2 \leq \theta \leq \pi/2 . 
\]

\begin{question}  \label{Q1:Inverse}
(a) Graph the function $y=f(\theta)$ above.

(b) Explain why the function $f$ is one-to-one and has a range 
\[
   \{y | -1 \leq y \leq 1\}. 
\]
%Include a picture of a circle or a graph to help with your explantion.
\end{question}

The function
\[
   \theta = f^{-1}(y) = \arcsin y = \sin^{-1} y , -1\leq y \leq 1 ,
\]
is the inverse of this new sine function $f$ with the above restricted domain. So, for example, 
\[
    \arcsin (1/3) = \sin^{-1}(1/3)
\]
is the angle \emph{between} $-\pi/2$ and $\pi/2$ whose sine is equal to $1/3$. 

\emph{Note:} Just like it is not enough to say that $\sqrt{19}$ is the number whose square is $19$ (since there are two numbers whose squares are $19$), it is also \emph{not} enough to say that $\arcsin (1/3)$ is the angle whose sine is $1/3$. The reason is that there are \emph{infinitely} many angles with a sine equal to $1/3$. The arcsine function returns the \emph{unique} angle between $-\pi/2$ and $\pi/2$ whose sine is $1/3$.



To get the inverse cosine function we cannot use the same domain restriction.

\begin{question}  \label{Q2:Inverse}
Explain why not.
\end{question}

To define the inverse cosine function, we first define a \emph{new} cosine function with the restricted domain $\{\theta \, | \, 0\leq \theta \leq \pi \}$ as shown here
\[
    x  = g(\theta) = \cos \theta , 0 \leq \theta \leq \pi . 
\]

\begin{question}  \label{Q3:Inverse}
Explain why the function $g$ is one-to-one and has a range 
\[
   \{x | -1 \leq x \leq 1\}. 
\]
Include a picture of a circle or a graph to help with your explantion.
\end{question}


\begin{exploration}  \label{Esdgtrhnh}
The function
\[
   \theta = g^{-1}(x) = \arccos y = \cos^{-1} x , -1\leq x \leq 1 ,
\]
is the inverse of this new cosine function $g$ with the restricted domain above. So, for example, 
\[
   \theta =  \arccos(1/4) = \cos^{-1}(1/4)
\]
is the angle \emph{between} $0$ and $\pi$ whose cosine is equal to $1/4$. 

We can use the radian protractor below to approximate the measure of this angle. 
To do this, we first need to recognize that we're working on a circle of radius $r=5$cm. Now we'll let $P(x,y)$ be the point on this circle with polar angle $\theta = \arccos(1/4)$. To find where $P$ lies on the circle, we'll compute the $x$-coordinate of $P$.

To start, remember that 
\[
      \cos \theta = \frac{x}{r} = \frac{x}{5}.
\]
And for our particular angle $\theta = \arccos(1/4)$,  
\[
   \cos \theta = \cos (\arccos(1/4)) = 1/4.
\]
So we know that 
\[
    \frac{x}{5} = \frac{1}{4} .
\] 
Solve this equation for $x$ to get 
\[
  x = \frac{5}{4} = 1.25 .
\]
So $P$ has $x$-coordinate $x=1.25$. 

To find where $P$ is on the circle, we draw the line $x=1.25$. This line intersects the circle in two points, $P$ and $Q$, where $P$ is in the first quadrant and $Q$ in the fourth (see demonstration below). Now remember that the arccosine function returns radian measures of angles between $0$ and $\pi$. Such angles must be in the first or second quadrants.  So we reject $Q$ and the unique polar angle of $P$ that lies between $0$ and $\pi$.  Using the protractor we estimate this angle to be about $1.3$ radians.

Our conclusion is that 
\[
   \cos^{-1}(1/4)  = \arccos(1/4) \sim 1.3.
\]


\begin{onlineOnly}
    \begin{center}
\desmos{pzftbzlfkn}{900}{600}
\end{center}
\end{onlineOnly}

Access Desmos interactives through the online version of this text at
 
\href{https://www.desmos.com/calculator/pzftbzlfkn}{142: Radian Protractor 3B}.

\end{exploration}




\begin{question} \label{Q5:Inverse}
(a) Explain the meaning of $\cos^{-1}(-2/5) = \arccos(-2/5)$. Draw an annotated picture of a circle to help with your explanation. Would you expect the value of $\cos^{-1}(-2/5)$ to be positive or negative? Explain.

(b)  Explain the meaning of $\sin^{-1}(-2/5) = \arcsin(-2/5)$. Draw an annotated picture of a circle to help with your explanation. Would you expect the value of $\sin^{-1}(-2/5)$ to be positive or negative? Explain.

\end{question}


\begin{question} \label{Ex1dsfsdafgt4hh:Inverse}
Use the radian protractor below to approximate each of the following. Explain your reasoning. Show a screenshot for each.

(a) $\cos^{-1}(0.3)$

(b) $\cos^{-1}(-0.3)$

(c) $\sin^{-1}(0.3)$

(d) $\sin^{-1}(-0.3)$

(e) $\sin^{-1}(-3)$


\begin{exploration}\label{Exp3:Comp}

\pdfOnly{
Access Desmos interactives through the online version of this text at
 
\href{https://www.desmos.com/calculator/kf29wdiej0}.
}
 
\begin{onlineOnly}
    \begin{center}
\desmos{kf29wdiej0}{900}{600}
\end{center}
\end{onlineOnly}
\end{exploration}

\end{question}



\begin{question} \label{Q8:Inverses}

Listen to the short radio segment below:

\href{https://www.npr.org/2023/04/02/1167645463/sunday-puzzle-correction-a-
lesson-in-trigonometry}{Inverse Trig}


(a) The segment is about correcting an error from a previous episode. What was the error?


(b) Describe your perception of the announcers’ reactions to the professor’s explanation. Do you think this says anything about how our society might view mathematics?

(c) What did you like about the professor’s explanation? Be specific.

(d) The professor makes at least two errors regarding both the concept of inverse functions in general and the inverse sine function in particular. Identify and correct these errors. Be thorough.

\end{question}

\section{The Tangent and Inverse Tangent Functions}

The \emph{tangent} function 
\[
   f(\theta) = \tan \theta = y/x
\]
takes as an input the dimensionless angle $\theta$ measured in radians and returns as an output the dimensionless ratio $y/x$, where $(x,y)$ are the coordinates of a point (other than $(0,0)$) on the ray emanating from the origin and inclined at the polar angle $\theta$ to the positive $x$-axis. The ratio $y/x$ has a familiar geometric interpretation. It is the slope of the line through $(0,0)$ and $(x,y)$. 

\begin{exploration}\label{Exp3:Comp}
We can visualize $\tan(\theta)$ as the $y$-coordinate of point $P$ in the demonstration below. 
\begin{question} \label{Qetgfgyyy5} 
(a) Explain why.

(b) Move the slider $u$ to get a sense of how quickly the tangent function increases near certain inputs. What are these inputs?
\end{question}


\pdfOnly{
Access Desmos interactives through the online version of this text at
 
\href{https://www.desmos.com/calculatorvq7ka8mg13}.
}
 
\begin{onlineOnly}
    \begin{center}
\desmos{vq7ka8mg13}{900}{600}
\end{center}
\end{onlineOnly}
\end{exploration}



\begin{question}   \label{Q9:InverseTrig}
Evaluate each of the following without a calculator. Explain your reasoning. Include a picture for each problem to help with your explanation. Work on a circle of radius 10 cm for parts (a)-(c).

(a) $\tan (4\pi)$

(b) $\tan (5\pi)$

(c) $\tan (\pi/2)$

(d) $\tan (\pi/4)$. For this, choose the coordinates of a point on the appropriate ray instead of working on a circle of radius 10 cm.

\end{question}


\begin{question}   \label{Q11:InverseTrig}
Suppose that $\tan\theta = -2/3$.

(a) Find the value of $\tan (\theta + \pi)$. Explain your reasoning. Include a picture to help with your explanation. Work on a circle. Do not refer to the graph of the function $f(\theta) = \tan\theta$.

(b) Use part (a) to express $\tan(\theta + \pi)$ in terms of $\tan\theta$. 

(c) What does part (b) tell you about the period of the tangent function?

(d) Explain why the inverse of the function $f(\theta) = \tan\theta$ is not a function.

\end{question}

The inverse tangent function is the inverse of the tangent function
\[
   m = g(\theta) = \tan \theta \, , -\pi/2 < \theta < \pi/2 ,
\]
with the restricted domain above. The inverse as written as
\[
    \theta = g^{-1}(m) = \arctan (m) = \tan^{-1}(m) .
\]

\begin{question} \label{Q12:InverseTrig}

(a) Sketch by hand a graph of the function $g$ above. What is the significance of the above domain restriction $-\pi/2 < \theta < \pi/2$  in defining the inverse tangent function?

(b) State the domain and range of the function $g^{-1}(m) = \arctan (m)$.

(c) Explain the meaning of $\tan^{-1}(-3)$.

(d) Would you expect the value of $\tan^{-1}(-3)$ to be positive or negative? Explain. Include a picture to help with your explanation. Do {\bf not} rely on the graph of either the tangent function or the inverse tangent function, but rather on the definition of the inverse tangent function. 

(e) Sketch by hand a graph of the function $\theta =  \tan^{-1}(m) $.

\end{question}

\begin{question} \label{Q10:InverseTrig}
Use the radian protractor below to approximate each of the following. Explain your reasoning. Include a picture for each problem to help with your explanation. Do not use a calculator.

(a) $\tan 3$

\emph{Hint: } Move $P(x,y)$ so that it has polar angle $\theta =3$ radians. Then estimate the coordinates $(x,y)$ of $P$ and use these to approximate $\tan 3 = y/x$.

(b) $\tan (-3)$

(c) $\tan (3+53\pi)$.

(d) $\tan 1.5$

(e) $\tan^{-1}(10/7)$

\emph{Hint: } Let $\theta = \tan^{-1}(10/7)$. Then we know two things about the angle $\theta$.
\begin{itemize}
\item{$\tan (\theta) = 10/7$ and}

\item{$-\pi/2 < \theta <\pi/2$}

\end{itemize}

We now want to position point $P(x,y)$ so that its polar angle is $\theta = \arctan(10/7)$. Because $\tan\theta = 10/7 = y/x$, $P(x,y)$ must lie on the line $y=(10/7)x$. There are many choices for $P$, some in the first quadrant others in the third. But for 
the polar angle of $P$ to lie strictly between $-\pi/2$ and $\pi/2$, $P$ must lie in the first or fourth quadrant. This condition forces $P$ to be in the first quadrant.

%Now move $P$ to be on the line $y=(10/7)x$ and in the first quadrant. 

\begin{question}  \label{Q35445fg}
Which of the following are possible choices for the coordinates of $P$?
\begin{multipleChoice}  
\choice{$(-10,-7)$}  
\choice{$(10,7)$}  
\choice{$(-7,10)$}  
\choice[correct]{$(7,10)$} 
\end{multipleChoice}   
\end{question}

Now move $P$ and use the protractor to estimate its polar angle  $\theta = \arctan(10/7)$. 



(f) $\tan^{-1}(-10/7)$

(g) $\tan^{-1}(8)$

(h) $\tan^{-1}(-1/8)$

\begin{exploration}\label{Exp3:Comp}

\pdfOnly{
Access Desmos interactives through the online version of this text at
 
\href{https://www.desmos.com/calculator/zpjfxxl23e}.
}
 
\begin{onlineOnly}
    \begin{center}
\desmos{zpjfxxl23e}{900}{600}
\end{center}
\end{onlineOnly}
\end{exploration}

\end{question}


\section{Back to Bearings}
We can use the inverse trig functions to compute exact bearings instead of relying on the radian protractor to approximate them.

\begin{example}  \label{Ed4t5t4dt44}
Find the exact bearing of the path that goes directly from the point $A(3,1)$ to the point $B(-2,4)$. Find the least positive bearing, measure counterclockwise from due east (the positive $x$-axis).

\begin{explanation}
The question is asking us to find the polar angle $\theta$ of the vector
\[
      \overrightarrow{AB} = \langle -5, 3  \rangle ,
\]
where $0\leq \theta \leq 2\pi$.

We could use any of the three inverse trigonometric functions arccosine, arcsine, and arctangent, and we'll try all three.

(a) The inverse cosine is the easiest because the polar angle $\theta$ of $\overrightarrow{AB}$ is between $0$ and $\pi$. To find the angle we first find the distance between $A$ and $B$. This is 
\[
   r =      | \overrightarrow{AB} | = | \langle -5, 3  \rangle | = \sqrt{34} .
\]
Now since
\[
   \cos\theta = \frac{x}{r} = \frac{-5}{\sqrt{34}}
\]
{\bf and}
\[
      0< \theta < \pi ,
\]
we know that to go directly from $A$ to $B$ we should walk at a bearing of
\[
     \theta = \arccos \left(  \frac{-5}{\sqrt{34}}  \right) .
\]
Using a calculator gives the bearing to be approximately $2.6$ radians.

(b) For the inverse sine function, we know that
\[
   \sin\theta = \frac{y}{r} = \frac{3}{\sqrt{34}}
\]
and it would be tempting to think that 
\[
  \theta = \arcsin \left(  \frac{3}{\sqrt{34}}  \right) .
\]
But this would \emph{not} be correct. The reason is that the arcsine function returns angles between $-\pi/2$ and $\pi/2$ radians. And because the vector $\overrightarrow{AB}$ has a negative $x$-component and a postive $y$-component, we know that its polar angle lies between $\pi/2$ and $\pi$ radians.

\end{explanation}

\end{example}


\begin{question} \label{Qetfg8uijjg}


 \begin{exploration}
\begin{onlineOnly}
    \begin{center}
\desmos{balmyutbjj}{900}{600}
\end{center}
\end{onlineOnly}


Access Desmos interactives through the online version of this text at
 
\href{https://www.desmos.com/calculator/balmyutbjj}{PolarAngle}.

\end{exploration}

\end{question}





\section{Some Review Questions}

\begin{question}   \label{Q17:InverseTrig}
Explain the meaning of each expression. Then evaluate each expression without using technology and without relying on any trigonometric identies. Explain your reasoning thoroughly and include pictures to help with your explanations.

\pskip

(a) $\cos (\arccos(-3/4))$ 

(b) $\arccos(\cos (5\pi/13))$

(c) $\arccos(\cos (42\pi/13))$

(d) $\sin(\arctan (-4))$

(e) $\cos(\arctan (u))$

(f) Express $\cos (\theta)$ in terms of $u$  if $\pi/2 < \theta < \pi$ and $\tan \theta=u$.

\end{question} 


\begin{question} \label{Q18:InverseTrig}
(a) The point $P$ has coordinates $(x,y)$ and is in the second quadrant (where $x<0$ and $y>0$). Find three expressions for the polar angle $\theta$ of $P$, with $0 < \theta < 2\pi$. Use the inverse cosine function for one expression, the inverse sine function for another, and the inverse tangent function for the third. 

(b) Repeat part (a) for a point $P$ in the third quadrant (where $x,y<0$).

(c) Repeat part (a) for a point $P$ in the fourth quadrant (where $x>0$ and $y<0$).
\end{question}



\begin{question} \label{Q19:InverseTrig}
Let $P$ be the point where the line through $(1,0)$ with slope $m$ intersects the circle $x^2+y^2=1$ a second time. Find a function 
\[
      \theta = f(m) , m \in \mathbb{R}
\] 
that expresses the polar angle $\theta$ of $P$, with $0 \leq  \theta < 2\pi$, in terms of the slope $m$. 
\end{question}

\begin{question} \label{Q20:InverseTrig}
Let $P$ be a point different from the origin having coordinates $(x,y)$. Find a function 
\[
      \theta = f(x,y) , (x,y) \in \mathbb{R}^2 \setminus {(0,0)} ,
\] 
that expresses the polar angle $\theta$ of $P$, with $0 \leq  \theta < 2\pi$, in terms of the coordinates of $P$.

\end{question}

\begin{question} \label{Q23:InverseTrig}
(a) Find the solution sets of the following equations. Give exact solutions without using a calculator. No need for explanations.

\pskip

(i) $5 + 13 \cos \left( \frac{\pi}{8}t \right) = 2$

(ii) $5 + 13 \sin \left( \frac{\pi}{8}t \right) = 2$

\pskip

(b) Use a calculator to approximate (to the nearest hundredth) all solutions $t\in [100, 116]$ of the equation in part (i) above. Some explanation required here.

\end{question}


\begin{question} \label{Q25:InverseTrig}
The function
\[
  h = f(t) = 2500 - 200 \cos \left(  \frac{\pi}{12}t  \right) \, , 0\leq t \leq 10 ,
\]
expresses the height (in feet) of a balloon in terms of the number of minutes past noon.

\pskip

(a) Find a function $T=u(h)$ that takes a height (in feet) as an input and returns as an output the time (measured in minutes past noon) when the balloon is at that height and on its way up. Include the correct domain.

(b) Find a function $t=d(h)$ that  takes a height (in feet) as an input and returns as an output the time (measured in minutes past noon) when the balloon is at that height and on its way down. Include the correct domain.

(c) Explain the meaning of the composition $u\circ f$. What does the function take as an input? What does it return as an output? Then simplify the composition as much as possible (there should be no trig or inverse trig functions in your final result). Finally graph the compostion over its domain.

(d) Repeat part (d) for the composition $d\circ f$.

\end{question}



\begin{question} \label{Q28:InverseTrig}
The function
\[
  h = f(t) = 2500 - 200 \cos \left(  \frac{\pi}{6} \sqrt{t}  \right) \, , 0\leq t \leq 81 ,
\]
expresses the height (in feet) of a balloon in terms of the number of minutes past noon.

\pskip

Answer parts (a)-(d) of the previous question for this new height function.

\end{question}


\begin{question}   \label{Q30:InverseTrig}
This question models the motion of a simple pendulum of length $L$ meters. 

The pendulum is released from rest at time $t=0$ seconds from at the intial angular displacement $\theta_0$ radians, measured counterclockwise from the negative $y$-axis in the demonstration below (ie. $\theta_0$ is the angle $\angle EOA$.)  We assume that for small oscillations, the angular displacment $\theta = \angle EOP$ is a sinusoidal function of time with period $T=2\pi \sqrt{L/g}$ seconds (here $g$ is the graviational acceleration, measured in $m/sec^2$). This is not actually true, but only approximately so for small initial displacements $\theta_0$. 

\pskip

(a) First find a function 
\[
   \theta = a(t) 
\]
that expresses the angluar displacement in terms of the number of seconds since the pendulum was released. Enter this function on Line 25 of the Desmos Worksheet below. But first click off the folder in Line 1.

(b) The second step is to use the angular displacement function to parameterize the pendulum's motion. As usual, use the cosine function for the $x$-coordinate and the sine function for the $y$-coordinate. Note 

i) the motion is a long a circle of radius $L$ meters centered at $(0,-1)$, and

ii) the angle $\theta = a(t)$, being measured counterclockwise from the negative $y$-axis,  is {\bf not} the polar angle. So your first step should be to use the function $a(t)$ to find an expression for the polar angle.

(c) Follow the directions in the Desmos activity. Experiment with the sliders and compare how closely your graph of the angular displacement function matches the true angular displacement graphed in red. Summarize your observations.


\begin{exploration}\label{Exp3:Comp}

\pdfOnly{
Access Desmos interactives through the online version of this text at
 
\href{https://www.desmos.com/calculator/kkmarshsyk}.
}
 
\begin{onlineOnly}
    \begin{center}
\desmos{kkmarshsyk}{900}{600}
\end{center}
\end{onlineOnly}
\end{exploration}


\end{question}

\end{document}

