\documentclass{ximera}
\title{Inverse Trigonometric Functions}

\newcommand{\pskip}{\vskip 0.1 in}

\begin{document}
\begin{abstract}
Introduction to inverse trig.
\end{abstract}
\maketitle

Suppose I pick a number, square it, and get 25. Then there are two possibilities for my original number, either $5$ or $-5$. This says the squaring function $f(x)=x^2$ is \emph{not} one-to-one and implies that its inverse is \emph{not} a function. 

Now you might object and claim that the function $h(x)=\sqrt{x}$ is the inverse of $f$. But that would not be correct. Rather, the square root function $h$ is the inverse of the  function
\[
    g(x) = x^2 , x\geq 0 .
\]
This new function $g$ is \emph{not} the same as $f$. The former is the squaring function with a restricted domain, $x\geq 0$. On this restricted domain, $g$ is one-to-one and its inverse $g^{-1}(x) = \sqrt{x}$ is a function. This is the familiar square-root function. Now if I pick a non-negative number and square it to get 25, there is only one possiblity for my original number, namely $5$. So it is not enough to say, for exampole, that $\sqrt{19}$ is the number whose square is 19. Rather it is the \emph{positive} number whose square is 19.

The sine and cosine functions work similarly. They are also not one-to-one and so their inverses are not functions. It is only by restricting the domains of these functions that we get \emph{new} functions that are one-to-one. And the inverses of these new functions are what we call the \emph{inverse sine and cosine functions}.

How to restrict the domains? The idea is to restrict the domains of the sine and cosines function so that over these restricted domains the new functions are one-to-one and give all the possible outputs (ie. all the numbers between $-1$ and $1$, endpoints included.) There are infinitley many ways to make such domain restrictions, but by convention mathematicians have arbitrarily decided how.

For the sine function, the new function with restricted domain is 
\[
    y  = f(\theta) = \sin \theta , -\pi/2 \leq \theta \leq \pi/2 . 
\]

\begin{question}  \label{Q1:Inverse}
Explain why the function $f$ is one-to-one and has a range 
\[
   \{y | -1 \leq y \leq 1\}. 
\]
Include a picture of a circle or a graph to help with your explantion.
\end{question}

The function
\[
   \theta = f^{-1}(y) = \arcsin y = \sin^{-1} y , -1\leq y \leq 1 ,
\]
is the inverse of this new sine function $f$ with the above restricted domain. So, for example, 
\[
    \arcsin (1/3) = \sin^{-1}(1/3)
\]
is the angle \emph{between} $-\pi/2$ and $\pi/2$ whose sine is equal to $1/3$.

To get the inverse cosine function we cannot use the same domain restriction.

\begin{question}  \label{Q2:Inverse}
Explain why not.
\end{question}

Rather, the new cosine function with restricted domain is
\[
    x  = g(\theta) = \cos \theta , 0 \leq \theta \leq \pi . 
\]

\begin{question}  \label{Q3:Inverse}
Explain why the function $g$ is one-to-one and has a range 
\[
   \{x | -1 \leq x \leq 1\}. 
\]
Include a picture of a circle or a graph to help with your explantion.
\end{question}

The function
\[
   \theta = g^{-1}(x) = \arccos y = \cos^{-1} x , -1\leq x \leq 1 ,
\]
is the inverse of this new cosine function $g$ with the restricted domain above. So, for example, 
\[
    \arccos(1/3) = \cos^{-1}(1/3)
\]
is the angle \emph{between} $0$ and $\pi$ whose cosine is equal to $-1/3$.


\begin{question} \label{Q5:Inverse}
(a) Explain the meaning of $\cos^{-1}(-2/5) = \arccos(-2/5)$. Draw a picture to help with your explanation. Would you expect the value of $\cos^{-1}(-2/5)$ to be positive or negative? Explain.

(b)  Explain the meaning of $\sin^{-1}(-2/5) = \arcsin(-2/5)$. Draw a picture to help with your explanation. Would you expect the value of $\sin^{-1}(-2/5)$ to be positive or negative? Explain.

\end{question}


\begin{example} \label{Ex1:Inverse}
Use the radian protractor below to approximate each of the following. Explain your reasoning. Show a screenshot for each.

(a) $\cos^{-1}(0.3)$

(b) $\cos^{-1}(-0.3)$

(c) $\sin^{-1}(0.3)$

(d) $\sin^{-1}(-0.3)$

(e) $\sin^{-1}(-3)$


\begin{exploration}\label{Exp3:Comp}

\pdfOnly{
Access Desmos interactives through the online version of this text at
 
\href{https://www.desmos.com/calculator/kf29wdiej0}.
}
 
\begin{onlineOnly}
    \begin{center}
\desmos{kf29wdiej0}{900}{600}
\end{center}
\end{onlineOnly}
\end{exploration}

\end{example}

\begin{question} \label{Q8:Inverses}

Listen to the short radio segment below:

\href{https://www.npr.org/2023/04/02/1167645463/sunday-puzzle-correction-a-
lesson-in-trigonometry}{Inverse Trig}


(a) The segment is about correcting an error from a previous episode. What was the error?


(b) Describe your perception of the announcers’ reactions to the professor’s explanation. Do you think this says anything about how our society might view mathematics?

(c) What did you like about the professor’s explanation? Be specific.

(d) The professor makes at least two errors regarding both the concept of inverse functions in general and the inverse sine function in particular. Identify and correct these errors. Be thorough.

\end{question}


\end{document}
