\documentclass{ximera}
\title{Simple Harmonic Motion with Vectors}

\newcommand{\pskip}{\vskip 0.1 in}

\begin{document}
\begin{abstract}
Using vectors to paramaterize simple harmonic motion.
\end{abstract}
\maketitle


\begin{question} \label{Qoer333fr}
Suppose that between consecutive noons on July 21 and July 22 the depth of the water at the Edmonds pier oscillates in simple harmonic motion between $9$ feet and $17$ feet with a period of $12$ hours. Suppose also that at 3am on July 21 the water is $11$ feet deep and falling.

Use vectors to find a function
\[
  h = f(t) \, , \, 0\leq t \leq 24
\]
that expresses the depth of the water in terms of the number of hours since noon on July 21.

Solution:

Remember the key idea. That a constant speed motion around a circle drives simple harmonic motion along a diameter of that circle. So to model the depth of the water we need only choose the right circle and a diameter of that circle.

I like to imagine the water oscillating vertically, so I'll label the upward-pointing axis as the positive $h$-axis and keep the $x$-axis horizontal.

%In the standard $xy$-coordinate system with the positive $y$-axis pointing upward, we'll relabel the $y$-axis as the $h$-axis and imagine the water to oscillate vertically.

We now describe our circle of motion.


Its center is at a point with $h$-coordinate
\[
   h  =  \frac{1}{2}\left( 17 \text{ ft} + 9 \text{ ft} \right) = 13 \text{ ft}
\]
equal to the average depth of the water over a 12-hour period.

It has radius
\[
   \frac{1}{2} \left( 17 \text{ ft} - 9 \text{ ft} \right) = 8\text{ ft} ,
\]
equal to the maximum deviation from the average depth.

The $x$-coordinate of this circle is immaterial and we'll suppose the circle is centered at the point $C$ with coordinates $(0,13)$ (all coordinates in ft) as shown below.


\begin{onlineOnly}
    \begin{center}
\desmos{mo7ekjkp2s}{900}{600}
\end{center}
\end{onlineOnly}

\href{https://www.desmos.com/calculator/mo7ekjkp2s}{142: SHM Vectors}



Now let $A$ be one of the two points on the circle of motion with $h$ coordinate $h=11$. We'll arbitrarily choose the point in the first quadrant and focus our attention on the vector $\overrightarrow{CA}$. Since $C$ has coordinates $(0,13)$, $\overrightarrow{CA}$ has $h$-component
\[
    h = 11 - 13 = -2 .
\]
Let $x$ be the $x$-component of $\overrightarrow{CA}$. Then since $|\overrightarrow{CA}| = 8$,
\[
x^2 + (-2)^2 = 8^2
\]
and 
\[
  x = \pm \sqrt{60}.
\]
But because $A$ is in the first quadrant 
\[
    x = \sqrt{60}
\]
and
\[
\overrightarrow{CA} = \langle \sqrt{60},-2\rangle.
\]

We'll ignore time for now and express the components of the vector $\overrightarrow{CP}$ in terms of the clockwise angle from $\overrightarrow{CA}$ to $\overrightarrow{CP}$. 


Since the period of oscillation is $12$ hours, the point $P$ that drives the oscillation rotates about the center of its circle at the constant rate of
\[
      \frac{2\pi \text{ rad}}{12\text{ hrs}} = \frac{\pi}{6} \frac{\text{rad}}{\text{hr}} .
\]


\end{question}


\end{document}

