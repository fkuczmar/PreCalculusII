\documentclass{ximera}
\title{The Cosine Function Classwork}

\newcommand{\pskip}{\vskip 0.1 in}

\begin{document}
\begin{abstract}
An introduction to circular trigonometry.
\end{abstract}
\maketitle


\section{The Position Problem}

\begin{question} \label{Q9df33f}
Driving north on I-5 at a constant speed of $40$ miles/hour, you pass Exit 176 at 12:12pm. Where are you (what exit) at 12:45pm? \emph{Note:} The exit numbers measure the number of miles from the WA-OR border as measured along I-5.
\end{question}

\begin{question}  \label{Q9dfedferyghnfetf4}

Between noon and 2:00 pm a beetle crawls counterclockwise around the circle of radius $50$ meters centered at the origin at a constant speed of $4$ meters/min. It passes the point $(50,0)$ at 12:17 pm.

Use the protractor below to answer the following questions. Do not change its radius.
\begin{enumerate}

\item Estimate the beetle's coordinates at 12:50. 

\item Estimate the first clock time after 12:17pm when the beetle is in the third quadrant with $x$-coordinate $x=-35$.

\begin{onlineOnly}
    \begin{center}
\desmos{lbkveixdno}{900}{600}
\end{center}
\end{onlineOnly}

\href{https://www.desmos.com/calculator/lbkveixdno}{142: Radian Protractor 2C}

\end{enumerate}
\end{question}


\begin{question}\label{Exp1:CFsdfdfdseeFFR}
Use the radian protractor below with a radius of $5$ cm to approximate each of the following expressions. Pretend the protractor is on a piece of paper. Do not change its radius or gather any other information than what is printed below.

Include all units in your computations. Then use a calculator and compare your estimates with the true values. 

\pdfOnly{
Access Desmos interactives through the online version of this text at
 
\href{https://www.desmos.com/calculator/lbkveixdno}.
}
 
\begin{onlineOnly}
    \begin{center}
\desmos{enlot2bcb7}{900}{600}
\end{center}
\end{onlineOnly}


\begin{enumerate}

\item $\cos 5.2$, $\sin 5.2$  


\item $\cos (-3.5)$,  $\sin (-3.5)$ %(no calculator)

\item $\cos (2\pi + 4.4)$, $\sin (2\pi+4.4)$ 

\item $\cos (83\pi + 4.4)$, $\sin (83\pi+4.4)$ 

\item $\cos 100^\circ$, $\sin 100^\circ$ (use a four-function calculator, arithmetic only, to first convert these angles to radians)

\item $\cos 180$, $\sin 180$ (use a four-function calculator, arithmetic only, to help.)

\end{enumerate}

\end{question}

\begin{question} \label{Q2:Cosine454}
Approximate the value of $\cos (\pi^\circ)$. Explain your reasoning and include a picture to help with your explanation.
\begin{hint}
Keep in mind that the angle is measured in degrees.
\end{hint}
\end{question}


\begin{question} \label{Q99f3332}
Go back to Question 1(a) and find the exact coordinate of the beetle at 12:50pm. Then use a calculator to approximate the coordinates and compare with your previous approximation.
\end{question}

\begin{question} \label{Q9df3rFDD}
Suppose $\cos\theta =2/3$. Use geometric reasoning (no formulas) to find the value(s) of each of the following expressions. Draw pictures. Do \emph{not} work on the unit circle.

\begin{enumerate}
\item $\cos(-\theta)$

\item $\cos(\theta+\pi)$

\item $\cos(\theta+2\pi)$

\item $\cos(\theta+\pi/2)$

\item  $\cos(\pi/2 - \theta)$

\item $\sin\theta$


\end{enumerate}
\end{question}








\begin{example}  \label{Exp89dsfrer94444}

Let $B$ be the point on the circle of radius $50$ meters that is $90$ meters from the point $A(50,0)$  (coordinates measured in meters). The distance is \emph{measured counterclockwise around the circle} from $A$ to $B$. 

Starting from $B$, you walk  $220$ meters \emph{clockwise} around the circle and stop.

\begin{enumerate}

\item Find your exact coordinates after you stop. Do \emph{not} use a calculator.

\item Use the radian protractor below to approximate your coordinates after you walk $220$ meters.
 
\item Use your answer from part (a) and a calculator to find a better approximation of your coordinates. Compare this with your approximation from part (b).

\begin{onlineOnly}
    \begin{center}
\desmos{lbkveixdno}{900}{600}
\end{center}
\end{onlineOnly}

\href{https://www.desmos.com/calculator/lbkveixdno}{142: Radian Protractor 2C}

\end{enumerate}
\end{example}

\begin{question}\label{QPPdfeer333}
Let $A$, $B$, $C$, and $D$ be points with respective coordinates $(70,0)$, $(0,70)$, $(-70,0)$, and $(0,-70)$, all measured in centimeters. \emph{Without using a calculator}, find the exact coordinates of each of the following points. Then use a calculator to approximate the coordinates to the nearest hundredth of a centimeter. Start each solution by drawing a reasonably accurate picture and then computing the exact polar angle of the point so described.

\begin{enumerate}
\item The point $P$ on the circle of radius $70$ cm centered at the origin that is $130$ cm from $A$. The distance is measured counterclockwise around the circle.

%\item The point $Q$ on the same circle thta is also $130$ cm from $A$, but here the distance is measured clockwise around the circle.

\item The point $Q$ on the same circle that is $200$ cm from $C$, the distance being measured counterclockwise around the circle.

\item The point $R$ on the circle that is $70$ cm from $B$, but this time the distance is measured clockwise around the circle.

\item The point $S$ on the circle that is twice as far from $D$ as from $A$. The distances are measure along the circle. Find all possibilities.

\item The point $W$ on the circle through the point $E(-10,10)$cm that is $29$cm from $E$. The distance is measured counterclockwise around the circle.



\end{enumerate}
\end{question}



\end{document}
