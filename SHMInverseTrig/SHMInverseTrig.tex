\documentclass{ximera}
\title{Simple Harmonic Motion with Inverse Trig}

\newcommand{\pskip}{\vskip 0.1 in}

\begin{document}
\begin{abstract}
Using inverse trigonometry to paramaterize simple harmonic motion.
\end{abstract}
\maketitle

Here's a problem we solved earlier using vectors. 

\href{https://ximera.osu.edu/precalc2/PreCalculus2/SHMVectors/SHMVectors}{Simple Harmonic Motion with Vectors}

Now we'll solve the same problem using vectors and inverse trigonometry.

You should compare the two solutions.

\begin{question} \label{Qoer333fr}
Suppose that between consecutive noons on July 21 and July 22 the depth of the water at the Edmonds pier oscillates in simple harmonic motion between $9$ feet and $17$ feet with a period of $12$ hours. Suppose also that at 3am on July 21 the water is $11$ feet deep and falling.

Use inverse trigonometry to help find a function
\[
  h = f(t) \, , \, 0\leq t \leq 24
\]
that expresses the depth of the water in terms of the number of hours since noon on July 21.

Solution:

Remember the key idea. That a constant speed motion around a circle drives simple harmonic motion along a diameter of that circle. So to model the depth of the water we start by choosing the circle of motion and a diameter of that circle.

I like to imagine the water oscillating vertically, so I'll label the upward-pointing axis as the positive $h$-axis and keep the $x$-axis horizontal. Both are measured in feet.

%In the standard $xy$-coordinate system with the positive $y$-axis pointing upward, we'll relabel the $y$-axis as the $h$-axis and imagine the water to oscillate vertically.

We now describe our circle of motion.


Its center is at a point with $h$-coordinate
\[
   h  =  \frac{1}{2}\left( 17 \text{ ft} + 9 \text{ ft} \right) = 13 \text{ ft}
\]
equal to the average depth of the water over a 12-hour period.

It has radius
\[
   \frac{1}{2} \left( 17 \text{ ft} - 9 \text{ ft} \right) = 8\text{ ft} ,
\]
equal to the maximum deviation from the average depth.

The $x$-coordinate of this circle is immaterial and we'll suppose the circle is centered at the point $C$ with coordinates $(0,13)$ (all coordinates in ft) as shown below.


\begin{onlineOnly}
    \begin{center}
\desmos{mytmutzt73}{900}{600}
\end{center}
\end{onlineOnly}

\href{https://www.desmos.com/calculator/mytmutzt73}{142: SHM Inverse Trig}



Now let $A$ be one of the two points on the circle of motion with $h$ coordinate $h=11$. We'll arbitrarily choose the point in the first quadrant and focus our attention on the vector $\overrightarrow{CA}$. Since $C$ has coordinates $(0,13)$, $\overrightarrow{CA}$ has $h$-component
\[
    h = 11 - 13 = -2 .
\]
Let $x$ be the $x$-component of $\overrightarrow{CA}$. Then $\overrightarrow{CA} = \langle x, -2 \rangle$. And since $|\overrightarrow{CA}| = 8$,
\[
x^2 + (-2)^2 = 8^2
\]
and 
\[
  x = \pm \sqrt{60}.
\]
But because $A$ is in the first quadrant 
\[
    x = \sqrt{60}
\]
and
\[
\overrightarrow{CA} = \langle \sqrt{60},-2\rangle.
\]


Up to this point, the solution has been identical to what we did before. But here's where the change comes. We'll let $P$ be the point on the circle that drives the oscillation of the tides. Now we'll express the components of the vector $\overrightarrow{CP}$ in terms of the  \emph{polar} angle $\theta$ of the vector $\overrightarrow{CP}$, measured in the usual way from $\overrightarrow{CW}$ (pointing in the direction of the positive $x$-axis) to $\overrightarrow{CP}$. 

How is this strategy different from our prior solution?
\begin{freeResponse}
\end{freeResponse}

This is much easier.

For this, we need to find a polar angle for the vector $\overrightarrow{OA}$. We'll call this angle $\alpha$. From the picture we can take $-\pi/2 < \alpha < 0$. Then since
\[
    \sin \alpha = -\frac{2}{8} ,
\] 
\[
 \alpha = \arctan \left(-\frac{1}{4}\right) .
\]

Now since the tide is falling at 3:00am, the point $P$ that drives the oscillation rotates clockwise about the center of its circle. And since the period of oscillation is $12$ hours, $P$ rotates about the center of its circle at the signed rate of
\[
     - \frac{2\pi \text{ rad}}{12\text{ hrs}} = - \frac{\pi}{6} \frac{\text{rad}}{\text{hr}} .
\]
And since 
\[
  \theta=\arctan \left(-\frac{1}{4}\right)
\]
at time $t=3$ hours past midnight,
\[
  \theta = a(t) = \arctan \left(-\frac{1}{4}\right) -  \frac{\pi}{6} \left(t-3\right) \, , \, 0\leq t \leq 24.
\]

Because 


So the function that 
\begin{align*}
    \overrightarrow{OP} & = \langle \sqrt{60}\cos\theta - 2 \sin \theta,  13-2\cos\theta -\sqrt{60}\sin \theta \rangle \\
                                  & = \langle \sqrt{60}\cos\left(  \frac{\pi}{6} \left(t-3\right)\right) - 2 \sin \left(  \frac{\pi}{6} \left(t-3\right)\right),  \\
                                  & 13-2\cos\left(  \frac{\pi}{6} \left(t-3\right)\right) -\sqrt{60}\sin \left(  \frac{\pi}{6} \left(t-3\right)\right) \rangle .
\end{align*}

But remember, it's the $h$-component of this vector that expresses the depth of the water as a function of time. So the function 
\[
   h = f(t) = 13-2\cos\left(  \frac{\pi}{6} \left(t-3\right)\right) -\sqrt{60}\sin \left(  \frac{\pi}{6} \left(t-3\right)\right) \, , \, 0\leq t \leq  24,
\]
expresses the depth of the water (in feet) in terms of the number of hours past midnight, July 21.

You should check this by 
\begin{enumerate}

\item checking that $f(3) = 11$ and

\item by inputting this function (named $f_1(t)$ in Line 4 of the worksheet above.

\end{enumerate}
\end{question}


\end{document}

