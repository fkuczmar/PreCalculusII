\documentclass{ximera}
\title{Law of Sines, CW}

\newcommand{\pskip}{\vskip 0.1 in}

\begin{document}
\begin{abstract}
The law of sines.
\end{abstract}
\maketitle

\section{Circles and the Law of Sines}

\begin{question} \label{QPkdf9fM}

In the figure below, the marked angles at $C$ and $Z$ are congruent and $\angle ABC = \pi/2$.

Express the length of $AZ$ (the diameter of the circle through $A$, $B$, and $C$) in terms of $\angle C$ and length $c$ of segment $\overline{AB}$.

\begin{onlineOnly}
    \begin{center}
\desmos{kxhdw1ha0p}{900}{600}
\end{center}
\end{onlineOnly}

\href{https://www.desmos.com/calculator/kxhdw1ha0p}{142: Law of Sines Circle}

\end{question}

\emph{Takeaway:} In any triangle $\Delta ABC$, 
\[
     \frac{a}{\sin A} = \frac{b}{sin B} = \frac{c}{\sin C} = d ,
\]
where $d$ is the diameter of the circle through the vertices of $\Delta ABC$.


\section{Addition Formula for Sine}

\begin{question} \label{QKefefe}

When the circle through the vertices of $\Delta ABC$ has unit diameter, then $a=\sin A$, $b=\sin B$, and $c=\sin C$.

Now let $D$ be the foot of the altitude from $C$ to $\overline{AB}$ as shown below.

\begin{onlineOnly}
    \begin{center}
\desmos{p78xpmbkq1}{900}{600}
\end{center}
\end{onlineOnly}

\href{https://www.desmos.com/calculator/p78xpmbkq1}{142: Law of Sines and Addition Formula}

\begin{enumerate}

\item Express the lengths $BD$ and $CD$ in terms of $\sin A$, $\sin B$, $\cos A$, and $\cos B$.

\item Express the length $BC = BD + DC$ in two different ways.

\end{enumerate}

\end{question}


\end{document}