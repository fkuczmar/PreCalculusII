\documentclass{ximera}
\title{Law of Sines, CW}

\newcommand{\pskip}{\vskip 0.1 in}

\begin{document}
\begin{abstract}
The law of sines.
\end{abstract}
\maketitle

\section{Circles and the Law of Sines}

\begin{question} \label{QPkdf9fM}

In the figure below, the marked angles at $C$ and $Z$ are congruent and $\angle ABC = \pi/2$.

\begin{enumerate}
\item Express the length of $AZ$ (the diameter of the circle through the vertices of $\Delta ABC$) in terms of $\angle C$ and length $c$ of segment $\overline{AB}$.

\item Find similar expressions for the diameter in terms of $\angle A$ and side length $a$, and also $\angle B$ and $b$.

\item Equate your three expressions. This gives the law of sines.
\end{enumerate}


\begin{onlineOnly}
    \begin{center}
\desmos{kxhdw1ha0p}{900}{600}
\end{center}
\end{onlineOnly}

\href{https://www.desmos.com/calculator/kxhdw1ha0p}{142: Law of Sines Circle}

\end{question}

\emph{Takeaway:} In any triangle $\Delta ABC$, 
\[
     \frac{a}{\sin A} = \frac{b}{sin B} = \frac{c}{\sin C} = d ,
\]
where $d$ is the diameter of the circle through the vertices of $\Delta ABC$.

We can also write this equality as
\[
   a: b: c: = \sin A : \sin B : sin C .
\]
This means the lengths of the sides of a triangle are in the same ratio as the lengths of the \emph{sines} of their opposite angles. 

%So if one angle of a triangle is twice another, the side oppos

\begin{question}  \label{QLDFer3r}
A loading ramp from the ground to the bed of a truck making an angle of $0.2$ radians with the ground is replaced with another ramp making an angle of $0.1$ radians with the ground.

Compare the lengths of the two ramps in a meaningful way.

Do this problem twice:

\begin{enumerate}
\item once using the law of sines and

\item again using right triangle trigonometry.
\end{enumerate}
\end{question}


https://www.desmos.com/calculator/4mtiavrshh


\begin{question} \label{QKefeeefe}

In $\Delta PQR$, $r=PQ = 6$ cm, $\angle P = 0.8$ radians, and $\angle Q = 1.4$ radians.


\begin{onlineOnly}
    \begin{center}
\desmos{4mtiavrshh}{900}{600}
\end{center}
\end{onlineOnly}

\href{https://www.desmos.com/calculator/4mtiavrshh}{142: Law of Sines P2}

\begin{enumerate}

\item Drag point $R$ in the worksheet above to make triangle $\Delta PQR$ satisfy the above conditions. Then  approximate the lengths $p$ and $q$ of the respective sides $\overline{QR}$ and $\overline{PR}$.

\item Use the law of sines to find the exact lengths of sides $p=QR$ and $q=PR$ without using a calculator.  Then use a calculator to approximate these lengths and compare them with your estimates from part (a).

\item Use right triangle trigonometry to find the exact lengths of sides $p=QR$ and $q=PR$ without using a calculator. \emph{Hint:} Activate the folder \emph{Triangle Solution} in Line 4 of the above worksheet.

\end{enumerate}
\end{question}


\section{Colliding Boats}

\begin{question}  \label{Q343f4eer4fd}
At noon a rowboat is $10$ km due east of a sailboat. The sailboat travels at a consant speed at a fixed bearing of $\alpha = \arcsin(0.4)$ radians (measured counterclockwise from the east). The rowboat travels half as fast as the sailboat at some fixed bearing. Sometime later the boats collide.

\pdfOnly{
Access Desmos interactives through the online version of this text at
 
\href{https://www.desmos.com/calculator/xjiwflhe2o}.
}
 
\begin{onlineOnly}
    \begin{center}
\desmos{gymps9za9i}{900}{600}
\end{center}
\end{onlineOnly}

Access Desmos interactive at

\href{https://www.desmos.com/calculator/gymps9za9i}{142: Sailboat and Rowboat 55}

\begin{enumerate}

\item Adjust the slider $\beta$ in Line 2 of the worksheet above to change the bearing of the rowboat. Then drag the slider $u$ (the number of hours past noon) to see if the boats collide. If not, adjust the bearing of the rowboat to make the boats collide. Approximate these bearings. Consider all possibilities.

\item Use trigonometry to find the exact bearing(s) of the rowboat. Do not use a calculator.

\item Use a calculator to approximate the bearing(s) of the rowboat. Compare these with your approximations from part (a).

\item Determine the exact time(s) of collision if the sailboat's speed is $4$ km/hour. Do not use a calculator. 

\item Use a calculator to approximate the time(s) in part (d). Then check these times with the slider in Line 4.


\end{enumerate}



\end{question}



\section{Addition Formula for Sine}

\begin{question} \label{QKefefe}

When the circle of Question 1 through the vertices of $\Delta ABC$ has unit diameter, then $a=\sin A$, $b=\sin B$, and $c=\sin C$.

Now let $D$ be the foot of the altitude from $C$ to $\overline{AB}$ as shown below.

\begin{onlineOnly}
    \begin{center}
\desmos{p78xpmbkq1}{900}{600}
\end{center}
\end{onlineOnly}

\href{https://www.desmos.com/calculator/p78xpmbkq1}{142: Law of Sines and Addition Formula}

\begin{enumerate}

\item Express the lengths $AD$ and $DB$ in terms of $\sin A$, $\sin B$, $\cos A$, and $\cos B$.

\[
    AD = \answer{\sin B \cos A}
\]
and
\[
    DB = \answer{\sin A \cos B}
\]

\item Use the results of part (a) to express the length $AB$ in terms of $\sin A$, $\sin B$, $\cos A$, and $\cos B$.

\[
     AB = \answer{\sin A \cos B + \sin B \cos A} .
\]

\item On the other hand, since
\[
    A + B + C = \answer{\pi} ,
\]
\[
     C = \answer{\pi - (A + B)}.
\]
So
\begin{align*}
   AB &= \sin C  \\
       &= \sin (\pi - (A+B)) \\
       &= \sin (A+B). \\
\end{align*}

\item Draw a picture of a circle to explain why $\sin (\pi - \theta) = \sin \theta$ for any angle $\theta$. You may assume $0<\theta <\pi$ for your picture.

\item Our conclusion is that
\[
  \sin (A+B) = \answer{\sin A \cos B + \sin B \cos A} .
\]

\item Use part (e) to find an expression for $\sin(A-B)$.

\begin{align*}
      \sin(A-B) &= \sin(A + (-B))  \\
                    &=\sin A \cos (-B) + \sin (-B) \cos A \\
                    &= \sin A \cos B - \sin B \cos A.
\end{align*}

\item Draw pictures of circles to explain why
\[
  \sin (-\theta)  = - \sin\theta
\]
and
\[
   \cos (-\theta) = \cos \theta
\]
for any angle $\theta$.

\end{enumerate}

\end{question}


\section{The Ambiguous Case}

\begin{question} \label{Q44tdfdsHHERER}
In $\Delta VUW$, side $\overline{VU}$ has length $10$cm, side $\overline{UW}$ has length $6$cm, and  $\angle V$ has measure $\arccos(9/10)$.

\begin{onlineOnly}
    \begin{center}
\desmos{a2m3iexwfx}{800}{600}         
\end{center}
\end{onlineOnly}

\href{https://www.desmos.com/calculator/a2m3iexwfx}{142: Law of Cosinses 53}

\begin{enumerate}
\item Drag slider $u$ (the length of side $\overline{VW}$) in Line 4 above to change the position of $W$. Keep your eye on Line 6 (showing the length of $\overline{UW}$) to first determine how many triangles satisfy the given conditions and second to approximate the possible lengths of side $\overline{VW}$.

\item Use right triangle trigonometry to determine the \emph{exact} lengths of $\overline{VW}$  in \emph{all} possible triangles satisfying the given conditions. Then use a calculator to approximate the lengths to the nearest hundredth of a centimeter. Activate the \emph{Hint} folder on Line 7 (click the circle at the left of the line) for a hint.

\item Work directly with $\Delta VUW$ and use the law of sines to solve part (b) again. Do not use a calculator.

\item Use the result of Question 5 to simplify your expressions from part (c). Then compare with part (b). Do \emph{not} use a calculator.

 
\end{enumerate}

\end{question}



\end{document}