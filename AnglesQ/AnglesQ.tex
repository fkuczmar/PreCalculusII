\documentclass{ximera}
\title{Measuring Angles - Discussion Questions}

\newcommand{\pskip}{\vskip 0.1 in}

\begin{document}
\begin{abstract}
Some questions about measuring angles in radians.
\end{abstract}
\maketitle



\begin{question} \label{QPodferhmmre}
\begin{enumerate}
\item Explain what it means for an angle to measure $10$ radians.

\item Sketch an accurate picture of an angle measuring $10$ radians. 

\end{enumerate}
\end{question}


\begin{question} \label{QOIERder3333}

You run $520$ meters around a circular track of radius $40$ meters. 

\begin{enumerate}

\item Through how many radians did you turn about the track's center?

\item How many laps (revolutions about the track's center) did you make? Give an exact number (without a calculator) and then an approximation to the nearest tenth.

\end{enumerate} 
\end{question}


\begin{question}  \label{Q9df3DDgZZ}
\begin{enumerate}
\item Find the exact number of radians in 25 revolutions.

\item Find the exact number of radians in $1/4$ of a revolution. Then the approximate number to the nearest tenth.

\item Find the exact number of radians in $1/6$ of a revolution. Then the approximate number to the nearest tenth.

\item Find the exact number of revolutions in $10$ radians. Then the approximate number to the nearest tenth.




\end{enumerate}
\end{question}

\begin{question} \label{Q5fggtg4t4tggdet4}
\begin{enumerate}
\item Use the radian protractor below to measure the marked angle $\angle ABC$.

\item Turn off the folder Angle 1 in Line 4. Then activate the folder Angle 2 in Line 6 and measure the new marked angle $\angle ABC$.

\item Turn off the folder Angle 2 in Line 6. Then activate the folder Angle 4 in Line 8 and measure the new marked angle $\angle ABC$.

\end{enumerate}

\begin{onlineOnly}
    \begin{center}
\desmos{dqedbpsohm}{900}{600}
\end{center}
\end{onlineOnly}

\href{https://www.desmos.com/calculator/dqedbpsohm}{142: Radian Protractor 21}
\end{question}


\begin{question} \label{QPPPdfrrr3r54g}
Take the radius of the earth to be $4000$ miles for these questions.

\begin{enumerate}
\item Find the distance between two ships on the equator at longitudes $20^\circ$W and $50^\circ$W. Measure the distance along the shorter arc of the equator between the points. %Take the radius of the earth to be $4000$ miles.

\item Use the radian protractor below to approximate the radius of the circle of latitude $70^\circ$N on the earth. Do \emph{not} use any trigonometric functions. Think proportionately instead.

\item Two ships are at longitudes $20^\circ$W and $50^\circ$W on the same circle of latitude $70^\circ$N . Use the radian protractor below to help approximate the distance between these ships as measured along the shorter arc of the circle of latitude between them.

\item Two ships are on the same circle of longitude $3500$ miles apart. One ship is at a latitude of $0.5$ radians north of the equator. Find all possible latitudes for the second ship. 

\item Two ships are on the same circle of latitude $400$ miles apart, at longitudes $80^\circ$W and $130$W. Use the protractor below to estimate their latitude. Find all possibilities.  
 
 
\end{enumerate}

\begin{onlineOnly}
    \begin{center}
\desmos{fmjbqszyge}{900}{600}
\end{center}

\href{https://www.desmos.com/calculator/fmjbqszyge}{142: Ships 22}

\end{onlineOnly}
\end{question}



\end{document}