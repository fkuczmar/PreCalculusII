\documentclass{ximera}
\title{Measuring Distances on the Earth}

\newcommand{\pskip}{\vskip 0.1 in}

\begin{document}
\begin{abstract}
Shortest paths on the earth.
\end{abstract}
\maketitle

\section{Latitude and Longitude}

The position of a point on the earth can be described by its latitude and longitude.

The marked red angle below is the longitude $\theta$ ($0\leq \theta < 2\pi$) of the blue point $P$. The longitude is measured relative to the prime meridian (the red semicircle, through the poles and Greenwich, England).

Turn off the folder \emph{Sphere} in Line 8 and you can another angle (marked red) that also measures the longitude of $P$.

Turn off the folder \emph{Longitude} in Line 25 and activate the folder \emph{latitude} in Line 12. The marked blue angles each measure the latidude of $P$. It is the angle that the radius from the center of the earth to the point $P$ makes with the plane of the equator. The latitude $\phi$ is positive for points in the northern hemisphere, negative for points in the southern hemisphere.
We measure the latitude to be between $=-\pi/2$ and $\pi/2$.


\begin{onlineOnly}
   \begin{center}
\desmosThreeD{jjaqozozvi}{900}{600}
\end{center}
\end{onlineOnly}

\href{https://www.desmos.com/3d/jjaqozozvi}{142: Latitude and Longitude}


\begin{question} \label{QPidferr3}

\begin{enumerate}
\item Express the radius of the circle of latitude $\phi$ in terms of $\phi$ and the radius of the earth $R$.

The radius is
\[
   r = \answer{R\cos\phi} .
\]

\item Shoreline is at a latitude of about $47.75^\circ$ N. Approximate the radius of our circle of latitude to the nearest mile. Take the earth to be a perfect sphere of radius $3960$ miles.

\item How far do we travel in one day due to the rotation of the earth about its axis?

\item How fast are we moving due to the rotation of the earth about its axis? How does this compare with the speed at the equator?

%\item Two points on the same circle of latitude $\phi$ have longitudes $\theta_1$ and $\theta_2$. We'll suppose  

\end{enumerate}
\end{question}


\begin{question} \label{QLkdf3FERE}

Cities $A$ and $B$ (blue points in the figure below) lie on the circle of latitude $\phi$ and have repsective longitudes $\theta_1$  and $\theta_2$ west of the prime meridian, measured in radians.
 
\begin{onlineOnly}
   \begin{center}
\desmosThreeD{5e5rrpkccb}{900}{600}
\end{center}
\end{onlineOnly}

\href{https://www.desmos.com/3d/5e5rrpkccb}{142: Latitude and Longitude 2}

\begin{enumerate}
\item Suppose the earth to be a sphere of radius $R$ miles and find an expression for the distance between $A$ and $B$ along the shorter arc of their circle of latitude. Assume 
\[
      \Delta \theta = \theta_2 - \theta_1
\]
is between $0$ and $\pi$.

\item Approximate the distance between Seattle and Spokane as measured along their common circle of latitude. Assume the cities have latitude $47.75^\circ$N and respective longitudes $122^\circ$W and $117^\circ$W. Take $R=3960$ miles.

\item Two cities at the same latitude have longitudes $130^\circ$W and $80^\circ$W. The distance between them is $1000$ miles as measured along their common circle of latitude. How far are the cities from the north pole? Find all possibilities.


\end{enumerate}

\end{question}


\section{Trigonometry: Along a Circle of Latitude}

\begin{question} \label{Q34dgbnhhtrg}

Pick points $A$ and $B$ at the same latiitude $\phi$ on the earth's surface and let $\Delta \theta$ (measured in radians) be the difference in their longitudes, with $0<\Delta \theta \leq \pi$.

\begin{onlineOnly}
   \begin{center}
\desmosThreeD{0fc0f7f1ea}{900}{600}
\end{center}
\end{onlineOnly}

\href{https://www.desmos.com/3d/0fc0f7f1ea}{151: Distances on Earth}

%hhpog6ijnr

The purpose of this problem is to compare two distances in traveling from $A$ to $B$, one along their common circle of latitude, the other along the great circle through $A$ and $B$. We'll assume the earth is a sphere of radius $R$ miles.

To compute each distance, we need to see inside the earth. You can do this by deactivating the \emph{Sphere} folder in Line 3.

\begin{enumerate}
\item First the distance along the circle of latitude.

\begin{enumerate}
\item Start by expressing the radius of the circle of latitude in terms of the latitude $\phi$ and the radius of earth $R$. Work in general, not with the specific values in the worksheet above.

\item Then express the distance between $A$ and $B$ along the circle of latitude in terms of $\phi$, $R$, and $\Delta \theta$.
\end{enumerate}

\item Next we'll compute the distance along the great circle through $A$ and $B$. The center of this circle coincides with the center of the sphere.

\begin{enumerate}
\item Try to do this on your own. Here are two triangles (from inside the sphere as illustrated above) shown in two dimensions to help.

\begin{onlineOnly}
   \begin{center}
\desmos{wkdhhbojii}{900}{600}
\end{center}
\end{onlineOnly}

\href{https://www.desmos.com/calculator/wkdhhbojii}{151: Distances Earth 2D}


\end{enumerate}

\item Use the results of parts (a) and (b) to compare these two distances betweeen San Francisco (latitude $38^\circ$N, longtitude $122^\circ$W) and Washington DC  (latitude $38^\circ$N, longtitude $77^\circ$W). Take the radius of the earth to be $3960$ miles.

\item Find two other locations at approximately the same latitude and compare the two distances.

\end{enumerate}

\end{question}

\section{Calculus: Along a Circle of Latitude}
This is a continuation of the previous problem.

The question is this: Fix the difference in longitude $\theta$ between two points at the same latitude and determine an expression for the latitude at which the postive difference in the two distances (one along the circle of latitude, the other along the great circle through the points) is a maximum.

\begin{onlineOnly}
   \begin{center}
\desmos{muhahdawza}{900}{600}
\end{center}
\end{onlineOnly}

\href{https://www.desmos.com/calculator/muhahdawza}{151: Distances Earth 2D Part 2}

The difference in the two distances is a maximum at latitude
\[
   \phi = \answer{\arccos\left( \sqrt{\csc^2 \left(\frac{\theta}{2}\right) - \frac{4}{\theta^2}} \right)} .
\]

\end{document}

