\documentclass{ximera}
\title{Quizzes}

\newcommand{\pskip}{\vskip 0.1 in}

\begin{document}
\begin{abstract}
Quiz Solutions
\end{abstract}
\maketitle

\section{Quiz1A}

\pskip

%\emph{Directions:} Include units for each number in each computation and use only what we have learned so far in this class.

%\item Answer each question with a concluding sentence. 

A mass oscillates along the $x$-axis in simple harmonic motion between $x=-10$ cm and $x=10$ cm with a period of $2.5$ seconds. The mass is released from rest at position $x=10$ cm.

Use the protractor below to approximate the second time when the mass is at position $x=-3$ cm. Explain your reasoning thoroughly in complete sentences. Annotate the figure to help with your explanation. Show all computations, no matter how simple. Include units for each number in each computation. use only what we have learned so far in this class.

\begin{onlineOnly}
    \begin{center}
\desmos{5fuh8n6ef1}{900}{600}
\end{center}
\end{onlineOnly}

\href{https://www.desmos.com/calculator/5fuh8n6ef1}{142: Quiz1AS}


{\bf Solution:} We suppose the point $P$ on the circle that drives the simple harmonic motion starts at point $(10,0)$cm at time $t=0$ seconds and rotates counterclockwise around the circle. Then when the mass is at position $x=-3$ for the second time, $P$ is in the third quadrant on the vertical line $x=-3$ as shown above.

At this time, $P$ has made approximately
\[
     \frac{14}{20}\text{ revolutions} = \frac{7}{10} \text{ revolutions}
\]
around the circle. 

Since each revolution takes $2.5 = 5/2$ seconds and $P$ moves at a constant speed, the mass has position $x=-3$ at approximately time
\[
 t = \left( \frac{7}{10} \text{ rev}  \right) \left(\frac{5}{2} \frac{\text{ sec}}{\text{rev}}  \right) = \frac{7}{4} \text{ sec}
\]
after being released.
\end{document}