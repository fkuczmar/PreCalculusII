\documentclass{ximera}
\title{Rotation Rates}

\newcommand{\pskip}{\vskip 0.1 in}

\begin{document}
\begin{abstract}
Measuring rates of rotation.
\end{abstract}
\maketitle




\section{Rotation Rates}
\begin{question}  \label{Q245: Angles}
Find the exact rotation rates of the following:

(a) the minute hand of a clock: $\answer{\frac{1}{60}} \text{ rev/min} = \answer{\frac{\pi}{30}}$ rad/min. 

(b) the hour hand of a clock: $\answer{\frac{1}{720}} \text{ rev/min} =\answer{\frac{\pi}{360}}$ rad/min. 

(c) the second hand of a clock: $\answer{1} \text{ rev/min} =\answer{2\pi}$ rad/min. 

(d) the earth about its axis (assume the earth takes 24 hours to rotate once about its axis) : $\answer{\frac{1}{24}} \text{ rev/hour} =\answer{\frac{\pi}{12}}$ rad/hr.

\end{question}

\section{Running in Circles}
\begin{exploration}
\pdfOnly{
Access Desmos interactives through the online version of this text at
 
\href{https://www.desmos.com/calculator/mjs5mwp827}.
}
 
\begin{onlineOnly}
    \begin{center}
\desmos{mjs5mwp827}{900}{600}
\end{center}
\end{onlineOnly}


\begin{question} \label{Q72:Angles}
Which of the points above is rotating about $O$ at the greatest rate? Explain your reasoning.
\begin{multipleChoice}  
\choice{$A$}  
\choice{$B$}  
\choice{$C$}  
\choice[correct]{They are all rotating at the same rate.}
\end{multipleChoice}  
\end{question}

\begin{question} \label{Q74353452:Angles}
Which of the points above has the greatest speed? Explain your reasoning.
\begin{multipleChoice}  
\choice{$A$}  
\choice{$B$}  
\choice[correct]{$C$}  
\choice{They are all have the same speed.}
\end{multipleChoice}  
\end{question}
\end{exploration}


\begin{question}  \label{Q2675: Angles}
You run around a circular track with radius $30$ meters while turning about the track's center at a constant rate of $0.2$ rad/sec. What is your speed?

\begin{explanation}
All we really need to do is to convert the rotation rate of $0.2$ rad/sec to a speed. And for this, we just need to know that 
as you turn through an angle of 1 radian about the center of the track you run a distance of 
\[
     s = (1 \text{ radian}) \left( 30 \frac{\text{ meters}}{\text{ radian}}\right) = 30 \text{ meters}
\]
So your speed is
\[
   v = \left( 0.2 \frac{\text{ radians}}{\text{ sec}}\right) \left( 30 \frac{\text{ meters}}{\text{ radian}}\right) = 6 \frac{\text{ meters}}{\text{ sec}}.
\]

The same reasoning shows that as you run around a circlular track of radius $r$ meters while turning about its center at the constant rate of $\omega$ rad/sec, your speed is
\[
   v = \left( \omega \frac{\text{ radians}}{\text{ sec}}\right) \left( r \frac{\text{ meters}}{\text{ radian}}\right) = \omega r \frac{\text{ meters}}{\text{ sec}}.
\]


\end{explanation}
\end{question}


\begin{question} \label{Qdft677788877}
\begin{enumerate}

\item You jog around a circular track with radius $50$ feet at the constant speed of $10$ ft/sec. At what rate are you rotating about the track's center?

\item You jog around a circular track at the constant speed of $15$ ft/sec while rotating about the track's center at the rate of $0.10$ rad/sec. Find the radius of the track.


\end{enumerate}
\end{question}

\begin{question}  \label{Q5g4yth4yghy}
\begin{enumerate}

\item A sleeping beetle rests on the second hand of a clock, $4$ feet from the clock's center. Find the beetle's speed.

\item A sleeping ant rests on the minute hand of a clock, $3$ feet from the clock's center. Find the ant's speed.

\end{enumerate}
\end{question}




\begin{question}  \label{Q255: Angles}
At what speed are people at the equator moving due to the rotation of the earth about its axis? Assume the earth to be a perfect sphere of radius $3960$ miles. Assume also that the earth takes 24 hours to rotate once about its axis.

Do this in two ways:

\begin{enumerate}
\item Without using the above formula or thinking about radians in any way.

\begin{hint}
Because the speed is constant, %we can compute the speed as
\[
     \text{speed} = \frac{\text{distance}}{{time}} ,
\]
over any time interval. Pick a time interval of 24 hours to compute the speed.
\end{hint}

\item By using the above formula

\end{enumerate}
\end{question}

\begin{question}  \label{Q2455: Angles}
Back to the formula $v=\omega r$ that expresses your speed (in ft/sec) running around a circluar track in terms of the track's radius (in feet) and your rotation rate about the track's center (measured in rad/sec).

\begin{enumerate}
\item Rewrite this formula if the rotation rate is measured in degrees/sec instead of rad/sec. Explain your reasoning.

\item Which choice gives the simpler formula, measuring the rotation rate in rad/sec or in deg/sec?
\end{enumerate}
\end{question}




\section{Rolling Wheels}

\begin{exploration}\label{exp:angles2}
Two wheels with the same radius rotate at the same rate. The first turns about its center, the second rolls on a road without slipping.

\begin{enumerate}
\item Use the animation below to compare the speed of point $P$ (attached to the first wheel) and and the speed of point $C$ (the center of the second wheel). Explain your reasoning. Include a screenshot to help with your explanation.

\item If the wheels turn at the constant rate of $0.45$ rad/s and the wheel's have a radius of 10 cm, find the speed of the center of the rolling wheel. Do not use a  formula. Instead, do this logically and explain your reasoning. 

\item Turn on the folder in Line 21 by clicking on the circle at the left of the line and play the animation. Then sketch graphs (on the same coordinate system) that show the speeds of points $P^\prime$ and $C$ as functions of time. No need for scales on the axes, but pay particular attention to the ratios of the speeds in drawing  your graphs.

\end{enumerate}

\pdfOnly{
Access Desmos interactives through the online version of this text at
 
\href{https://www.desmos.com/calculator/cwaasmghns}.
}
 
\begin{onlineOnly}
    \begin{center}
\desmos{ehvwrubxuv}{900}{600}
\end{center}
\end{onlineOnly}
\end{exploration}

\href{https://www.desmos.com/calculator/ehvwrubxuv}{142:Rolling Stationary Wheels 2}

\begin{question} \label{Qdsfdsf4rt5yy55t}
A pebble is stuck in the tire of a bicycle wheel with a radius of $0.8$ feet. 
\begin{enumerate}
\item Find the speed of the pebble if you hold the axle of the wheel in place and turn the wheel at the constant rate of $5$ rev/sec.

\item At what speed does the bike move if the wheels rotate at a constant rate of $5$ rev/sec as you ride the bike down the street?

\item At what rate are the wheels rotating if you ride at a constant speed of $20$ ft/sec?

\end{enumerate}

\end{question}








\begin{exploration}\label{exp:angles2}
Experiment with the desmos activity below to determine an expression in $a$ and $b$ that gives the radian measure of the angle through which the rolling circle turns during the time it takes point $P$ to return to its intial position. Try to explain the logic behind the expression.


\pdfOnly{
Access Desmos interactives through the online version of this text at
 
\href{https://www.desmos.com/calculator/g8rjobapit}.
}
 
\begin{onlineOnly}
    \begin{center}
\desmos{g8rjobapit}{900}{600}
\end{center}
\end{onlineOnly}
\end{exploration}


\section{Thinking Proportionately, Part 1}

\begin{question}  \label{Q6521:Angles}
You replace  your truck's $25$ inch diameter tires with $35$ inch diameter tires without recalibrating the odometer and speedometer.

(a) You go on a trip that your odometer records as $200$ miles. How far did you actually drive.

(b) On the way you get pulled over for speeding. The officer clocked you at going $70$ miles/hour in a $60$ mile/hour zone. What did your speedometer read?

\begin{hint}
(a) The key point is that the car records, either mechanically or electrically, the angle through which the wheels turn. It then converts the change in this angle to a distance by using the radius of the wheels. Let the radian measure of the change in this angle over the course of the trip be $\theta$. Let the radius of the $25$-inch diameter wheels be $r_1$ miles, and let the radius of the $35$-inch diameter wheels be $r_2$ miles. 

Then the trip odometer records a trip distance (in miles) of 
\[
   s_1 = r_1\theta =200.
\]
But the actual distance (in miles) was
\[
  s_2 = r_2\theta .
\]
This tells us that
\[
    \frac{s_2}{s_1} = \frac{r_2\theta}{r_1 \theta} = \frac{r_2}{r_1} = \frac{35}{25}.
\]
So the length of the trip was actually
\[
   s_2 = \left( \frac{35}{25}\right) s_1 = \left(\frac{35\text{ in}}{25\text{ in}} \right) (200 \text{ miles}) = 280 \text{ miles}.
\]

\pskip

(b) This part works much the same way as part (a). The key point is that the car measures the rotation rate of the wheels and converts this to a speed using the radius of the wheels. Let this rotation rate be $\omega$ rad/sec at the moment you were clocked at $70$ miles/hour. Also let $v_1$ and $v_2$ be  your recorded (as on the speedometer) and actual speeds, respectively, both measured in miles/hour.

Then since
\[
  v_1 = \omega r_1
\]
and 
\[
  v_2 = \omega r_2 ,
\]
we know that
\[
    \frac{v_1}{v_2} =  \frac{r_1}{r_2} = \frac{25}{35}. 
\]
So when you were clocked going $70$ miles/hour, your speedometer showed a speed of
\[
   v_1 = \left( \frac{25}{35}\right) v_2 = \left(\frac{25\text{ in}}{35\text{ in}} \right) (70 \text{ miles/hr}) = 50 \text{ miles/hr}.
\]

\end{hint}

\end{question}


\begin{question} \label{Q6721:Angles}
Kryptonites living at a latitude of $20^\circ$N move at a speed of $2000$ miles/hour due to the rotation of the planet about its axis.

Use the radian protractor below to help find approximate answers to the following questions.

\begin{enumerate}
\item  At what speed do the Kryptonites living at a latitude of $40^\circ$S move due to the rotation of the planet about its axis?
\begin{hint}
Start by converting the angle $40^\circ$ to radians.
\end{hint}

\item At what latitude(s) is the speed $500$ miles/hour?
\end{enumerate}

\begin{exploration}
\pdfOnly{
Access Desmos interactives through the online version of this text at
 
\href{https://www.desmos.com/calculator/kdakzcloqr}.
}
 
\begin{onlineOnly}
    \begin{center}
\desmos{kdakzcloqr}{900}{600}
\end{center}
\end{onlineOnly}
\end{exploration}

\end{question}


\section{Measuring the Radius of the Earth}

\begin{question} \label{Q793:Angles}
You are due west of your friend on the spring eqiunox. Both of you are on the equator. 

On the spring equinox at the equator the sun rises due east at 6:00am local time, passes directly overhead at noon, and sets due west at 6:00pm local time. You both watch the sun set into the ocean. Being west of your friend, you see the sun set later. How much later? Well it depends on the distance between you and your friend.

(a) Experiment with the sliders $\phi$ and $n$ in the demonstration below.

(i) Explain what the demonstration illustrates.

(ii) Which of the two points represents you? How do you know?

(iii) When $\phi=0$, what is your local time? 

\pdfOnly{
Access Desmos interactives through the online version of this text at
 
\href{https://www.desmos.com/calculator/jmslo0laqz}.
}
 
\begin{onlineOnly}
    \begin{center}
\desmos{jmslo0laqz}{900}{600}
\end{center}
\end{onlineOnly}



(b) Find a function 
\[
   T = f(s) \, , \, 0\leq s \leq 1000 ,
\]
that expresses the time difference in the observed sunsets (measured in minutes) in terms of the distance (measured in miles) between you and your friend. Take the earth to be a perfect sphere of radius $R$ miles.

(c) Suppose you are $100$ miles west of your friend and that the radius of the earth is $3960$ miles. How much later do you see the sun set? Round your answer to the nearest second.

(d) This problem suggests a way to measure the radius of the earth. How?

\end{question}







\begin{question} \label{Q35yt6344bnxx}
\begin{enumerate}
\item Find the distance between two ships on the equator at longitudes $20^\circ$W and $50^\circ$W. Measure the distance along the shorter arc of the equator between the points. Take the radius of the earth to be $3960$ miles.

\item Use the radian protractor above to approximate the distance between two ships on the circle of latitude $55^\circ$N at longitudes $20^\circ$W and $50^\circ$W. Measure the distance along the shorter arc of the circle of latitude through the ships.  Take the radius of the earth to be $3960$ miles.
\end{enumerate}

\pdfOnly{
Access Desmos interactives through the online version of this text at
 
\href{https://www.desmos.com/calculator/kdakzcloqr}.
}
 
\begin{onlineOnly}
    \begin{center}
\desmos{vb1kvzkzlq}{900}{600}
\end{center}
\end{onlineOnly}


\end{question}







\section{Rotating Gears}

\begin{question}  \label{Q3242:Angles}
Experiment with the desmos demonstration below. 

Given the radii $r_1$, $r_2$ cm of the blue and red wheels respectively, and the rotation rate $\omega_1$ rad/sec of the blue gear, find an expression for the rotation rate (say $\omega_2$ rad/sec) of the red gear. Explain your logic.

\pdfOnly{
Access Desmos interactives through the online version of this text at
 
\href{https://www.desmos.com/calculator/4vuc86pxwo}.
}
 
\begin{onlineOnly}
    \begin{center}
\desmos{4vuc86pxwo}{900}{600}
\end{center}
\end{onlineOnly}


\begin{hint}
The key point is that the gears are assumed to roll on each other without slipping. Because of this, the speed of two points on the outer rims of the two gears are equal (to see this compare the distances traveled by the two black tick marks on the gears over some time interval). These speeds (in cm/sec) are 
\[
   v_1 = \omega_1 r_1
\]
for the blue wheel, and
\[
   v_2 = \omega_2 r_2
\]
for the red wheel.

Now since the two speeds are equal,
\[
   \omega_1 r_1 = \omega_2 r_2
\]
and so the rotation rate of the red wheel (in rad/sec) is
\[
  \omega_2 = \left( \frac{r_1}{r_2} \right) \omega_1 .
\]
\end{hint}

\end{question}




\begin{question}  \label{Q1764:Angles}
Experiment with the desmos demonstration below where the black gear is welded to the red. The black gear pushes a chain.  

Given the radii of the three gears and the rotation rate of the blue gear, how can you determine the speed of the chain? Explain your logic.

\pdfOnly{
Access Desmos interactives through the online version of this text at
 
\href{https://www.desmos.com/calculator/6nxze8ikhg}.
}
 
\begin{onlineOnly}
    \begin{center}
\desmos{6nxze8ikhg}{900}{600}
\end{center}
\end{onlineOnly}
\end{question}







\begin{question}  \label{Q323342:Angles}
The wheels of a car have a diameter of two feet. A gear mechanism with four gears connects one of the car's wheels to the wheel that spins the tenth of a mile reading on the odometer. %When you mow the lawn, the spindle turns four times as fast as the wheels. 

(a) What gear ratio does this?

(b) Design such a gear train.

(c) Investigate how the other digits of the odometer reading turn.
\end{question}


\begin{question}    \label{Q850:Angles}
The photo below shows a mechanical mower. The wheels have a radius $AQ$ of 9 inches. The distance $AB$ between the center of each wheel and the center of the spindle is 2.25 inches. The radius of the blade wheel is also approximately 2.25 inches.

\pdfOnly{
Access Desmos interactives through the online version of this text at
 
\href{https://www.desmos.com/calculator/ryxfilsnef}.
}
 
\begin{onlineOnly}
    \begin{center}
\desmos{ryxfilsnef}{900}{600}
\end{center}
\end{onlineOnly}

The diagram below shows the inner workings of the mower.
\begin{onlineOnly}
\begin{center}
\desmos{lsxarkvo6r}{900}{600}
\end{center}
\end{onlineOnly}

\href{https://www.desmos.com/calculator/lsxarkvo6r}{142: Mechanical Mower}

The animation below shows the  mower in motion.

\begin{onlineOnly}
\begin{center}
\desmos{3yjpzc2s1s}{900}{600}
\end{center}
\end{onlineOnly}

\href{https://www.desmos.com/calculator/3yjpzc2s1s}{142: Push Mower}



\begin{enumerate}
\item Relate the walking speed to the speed of the cutting edge of the blade as observed in the reference frame of the moving mower. Start by defining the appropriate parameters.
\end{enumerate}

\end{question}



\iffalse

*****************************************************************************************
******************************************************************************************

\section{Clock Problems}


\begin{question}   \label{Q54: Angles}
(a) Find an increasing function $\theta = a(t)$ that expresses the radian measure of the angle between the minute and hour hands of a clock in terms of the number of hours past noon. Measure the angle from the hour hand to the minute hand, taking the clockwise sense to be positive. %Note that for a function to be \emph{monotonic} means that it is always increasing or always decreasing.

(b) Use the clock below to estimate the radian measure of the \emph{acute} angle between the minute and hour hands at 12:53pm. An \emph{acute} angle has a measure between $0$ and $\pi/2$ radians.

(c) Use your function from part (a) to help find the exact radian measure of the \emph{acute} angle between the minute and hour hands at 12:53pm. Then find the approximate radian measure, correct to the nearest hundredth. Compare with your estimate from part (b).

(d) Use the clock below to estimate the first two times after 12:00pm when the minute and hour hands are perpendicular.

(e) Use your function from part (a) to help find the first two times after 12:00pm when the minute and hour hands of a clock are perpendicular. Round these times to the nearest second and compare them with your estimate from part (d).

\pdfOnly{
Access Desmos interactives through the online version of this text at
 
\href{https://www.desmos.com/calculator/vt6utnkfve}.
}
 
\begin{onlineOnly}
    \begin{center}
\desmos{vt6utnkfve}{900}{600}
\end{center}
\end{onlineOnly}

\end{question}


\begin{question}   \label{Q54B: Angles}
(a) Find an increasing function $\theta = a(t)$ that expresses the radian measure of the angle between the minute and hour hands of a clock in terms of the number of hours past 3:00pm. Measure the angle from the hour hand to the minute hand, taking the clockwise sense to be positive. %Note that for a function to be \emph{monotonic} means that it is always increasing or always decreasing.

(b) Use the clock above to estimate the first two times after 3:00pm when the minute and hour hands make an angle of $2\pi/3$ radians with each other.

(c) Use your function from part (a) to help find the first two times after 3:00pm when the minute and hour hands of a clock make an angle of $2\pi/3$ radians with each other. Round these times to the nearest second and compare them with your estimate from part (b).

\end{question}






\section{Thinking Proportionately, Part 2}

\begin{question}  \label{Q4P:Angles}
Use the radian protractor below to help you find approximate answers to the following questions. Keep the radius of the protractor at $7$ m.

\pskip

(a) You stand at the origin facing the direction of the postive $x$ axis. Estimate your coordinates after you turn counterclockwise through an angle of 4 radians and then walk 100 meters directly away from the origin.

\begin{explanation}
We first use the radian protractor to estimate our coordinates if we had walked $7$ meters directly away from the origin. To do this, we frist drag point $P$ on the radian protractor to the 4 radian mark as shown. Next we approximate the coordinates of $P$ by estimating where the vertical and horizontal lines through $P$ intersect the coordinate axes. It looks like the coordinates of $P$ are about $(-4.7, -5.3){\text m}$.

So if we had walked $7$ meters directly away from the origin, we would have ended up at the point $P$ with coordinates about $(-4.7, -5.3)\text{ meters}$. To approximate our coordinates if we walked $100$ meters instead of $7$ meters, we just need to scale these coordinates by a factor of $(100\text{m})/7(\text{m}) = 100/7$. Then our coordinates $(x,y)$ would be about
\[
   x \sim (-4.7\text{ m}) \left( \frac{100\text{ m}}{7\text{ m}} \right) = (100 \text{ m})\left( \frac{-4.7\text{ m}}{7\text{ m}}\right) \sim  -67.1 \text{ m}.
\]
and
\[
   y   \sim (-5.3\text{ m}) \left( \frac{100\text{ m}}{7\text{ m}} \right) = (100 \text{ m})\left( \frac{-5.7\text{ m}}{7\text{ m}}\right) \sim   -75.7 \text{ m}.
\]

\pdfOnly{
Access Desmos interactives through the online version of this text at
 
\href{https://www.desmos.com/calculator/vb1kvzkzlq}.
}
 
\begin{onlineOnly}
    \begin{center}
\desmos{vb1kvzkzlq}{900}{600}
\end{center}
\end{onlineOnly}


\end{explanation}

(b) You start from the point with coordinates $(100,0)$ meters and walk counterclockwise around the circle of radius $100$ meters centered at the origin. Estimate your coordinates after you have walked $230$ meters. {\bf Click the Hint tab at the beginning of this question for a hint.}

\begin{hint}
First compute the radian measure of the angle through which you turn about the circle's center.
\end{hint}

(c) Use your estimate from part (b) to approximate your coordinates if you had walked $230$ meters clockwise around the same circle.

(d) Repeat part (b) if you walk $2$ meters instead of $230$ meters.

(e) You start on the postive $x$-axis and walk counterclockwise around a circle centered at the origin until you reach the point with coordinates $(7,-24)$ meters. Estimate the distance you walked.  



(f) A ferris wheel has a radius of $50$ feet and its center is $60$ feet above the ground. The wheel rotates at a constant rate, making one revolution every two minutes.

(i) Approximate your height above the ground $10$ seconds after you get on the ferris wheel. 

(ii) Approximate your height above the ground $70$ seconds after you get on the ferris wheel. 

(iii) Approximately how long after you board are you $100$ feet above the ground for the first time.

(iv) Use your approximation from part (iii) to approximate the second time you are $100$ feet above the ground.


\begin{exploration}
\pdfOnly{
Access Desmos interactives through the online version of this text at
 
\href{https://www.desmos.com/calculator/kdakzcloqr}.
}
 
\begin{onlineOnly}
    \begin{center}
\desmos{vb1kvzkzlq}{900}{600}
\end{center}
\end{onlineOnly}
\end{exploration}

\end{question}


\begin{question}  \label{Q941F:Angles}
(a) Find the distance between two ships on the equator at longitudes $20^\circ$W and $50^\circ$W. Measure the distance along the shorter arc of the equator between the points. Take the radius of the earth to be $3960$ miles.

(b) Use the radian protractor above to approximate the distance between two ships on the circle of latitude $55^\circ$N at longitudes $20^\circ$W and $50^\circ$W. Measure the distance along the shorter arc of the circle of latitude through the ships.  Take the radius of the earth to be $3960$ miles.
\end{question}

















\section{Measuring the Polar Angle}


%\begin{exploration}  \label{Q57:Radians}

%\pdfOnly{
%Access Desmos interactives through the online version of this text at
 
%\href{https://www.desmos.com/calculator/2dlgnpeqsm}.
%}
 
%\begin{onlineOnly}
%    \begin{center}
%\geogebra{cgpad5nc}{900}{600}
%\end{center}
%\end{onlineOnly}
%\end{exploration}



\begin{exploration}\label{exp:angles1}
Suppose you stand in place at point $A$ and face point $B$ in the desmos activity below. 

(a) Now turn counterclockwise until you face $C$ for the first time. Use the radian protractor to approximate the angle through which you turn during this time. Show a screenshot.

(b) Now turn back to face $B$ again (sitll standing at $A$). Then turn clockwise and stop when you face $C$ for the fifth time. Through approximately what angle do you turn during this time? 

(c) Turn back to face $B$ again (still standing at $A$). Then turn counterclockwise at the constant rate of $0.45$ rad/sec until you face $C$ for the tenth time. Approximately how long does this take? 


\pdfOnly{
Access Desmos interactives through the online version of this text at
 
\href{https://www.desmos.com/calculator/mwh3wwfrqv}.
}
 
\begin{onlineOnly}
    \begin{center}
\desmos{mwh3wwfrqv}{900}{600}
\end{center}
\end{onlineOnly}
\end{exploration}









\begin{example} \label{Ex1:Angles}
At noon an ant is in the second quadrant at the point on the circle $x^2+y^2=400$ (the coordinates $(x,y)$ measured in centimeters) that is 45 cm from $(20,0)$, as measured along the circle. It crawls counterclockwise around the circle at a constant speed of $5$ cm/sec.

At noon a beetle is in the first quardrant on the same circle at a point that is 10 cm from $(20,0)$, this distance also measured along the circle. It crawls clockwise around the circle at a constant speed of $8$ cm/sec.

The insects crawl for 100 seconds.

\pskip

(a) Find a function
\[
    \theta = a(t) , 0\leq t \leq 100,
\]
that expresses the angle (in radians) from the postive $x$-axis to the segment $OA$ running from the origin to the ant in terms of the number of seconds past noon. Take the counterclockwise sense of rotation to correspond to a postive angle.

\pskip

(b) Find a function
\[
    \theta = b(t) , 0\leq t \leq 100,
\]
that expresses the angle (in radians) from the postive $x$-axis to the segment $OB$ running from the origin to the beetle in terms of the number of seconds past noon. Take the counterclockwise sense of rotation to correspond to a postive angle.

\pskip


(c) Use your functions from (a) and (b) to determine the first time the insects pass each other. Give an exact time, then an approximation to the nearest tenth of a second.

\pskip

(d) Use your functions from (a) and (b) to find an expression for the $n$th time the insects pass each other. 

\pskip

(e) Find a function $c(t)$ that counts the number of times the insects pass each other during the first $t$ seconds of their motions.

\pskip

(f) When is the last time the insects pass each other? Give an exact time, then an approximation to the nearest tenth of a second.

\begin{exploration}\label{exp:angles2}
Use the demonstration below to check your work.


\pdfOnly{
Access Desmos interactives through the online version of this text at
 
\href{https://www.desmos.com/calculator/btopul9ji9}.
}
 
\begin{onlineOnly}
    \begin{center}
\desmos{btopul9ji9}{900}{600}
\end{center}
\end{onlineOnly}
\end{exploration}


\end{example}


\begin{example} \label{Ex2:Angles}
Make up and solve your own version of Example 3. Use the exploration below to check your work, but not to help with the computations in any way. 


\begin{exploration}\label{exp:angles2}

\pdfOnly{
Access Desmos interactives through the online version of this text at
 
\href{https://www.desmos.com/calculator/632iayap2b}.
}
 
\begin{onlineOnly}
    \begin{center}
\desmos{632iayap2b}{900}{600}
\end{center}
\end{onlineOnly}
\end{exploration}


\end{example}



****************************************************************************************
*************************************************************************************

\fi


\end{document}
