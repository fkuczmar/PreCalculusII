\documentclass{ximera}
\title{Circular Interpolation}

\newcommand{\pskip}{\vskip 0.1 in}

\begin{document}
\begin{abstract}
Circular Interpolation
\end{abstract}
\maketitle


What is usually called \emph{sinusoidal modeling} is kind of a misnomer. First, it's usually easier to model osciallatory behavior with a cosine function. Second, and more importantly, the name \emph{sinusuoidal modeling} fails to convey the main idea of replacing linear interpolation with circular interpolation.

Here's an example.


\begin{example}  \label{E888b0bbdsdsf}
Suppose that over the course of a 24-hour period, from midnight October 29 to midnight October 30, the depth of the water at the Edmonds Pier is a sinusoidal function of time. Suppose also that a high tide of 21 feet occurs at 2:00am and the following low tide of 5 feet occurs at 8:00am. 


\begin{explanation}
Our aim is to find a function
\[
    h = f(t) , 0\leq t \leq 24, 
\]
that expresses the depth of the water (in feet) in terms of the number of hours past midnight, October 29. 

\pskip

Note that to say ``sinusoidal function'' means that the graph of $f$ is generated by uniform circular motion. But the graphs of the sine and cosine functions are both generated this way, so it is ok to express $f(t)$ in terms of the cosine function, and we will do just that.

Here are the steps.

\begin{enumerate}

\item  First we'll use the information above to sketch by hand a graph of the function $f$. Label the axes with the appropriate variable names and units. Label the coordinates of two key points on the graph.

Desmos activity available at:

\href{https://www.desmos.com/calculator/x2kocpkcfm}{142: Edmonds Pier}.

 
\begin{onlineOnly}
    \begin{center}
\desmos{x2kocpkcfm}{900}{600}
\end{center}
\end{onlineOnly}



\item Next we'll compute the mean depth of the water over the 24 hours and the maximum deviation of the depth from this mean. Including units in our computation, the mean depth is 
\[
    h_{avg} = 0.5 ( 21 \text{ ft } + 5 \text{ ft }) = 13 \text{ ft} . 
\]
And the maximum deviation from the mean is
\[
     21 \text{ ft } - 13 \text{ ft } = 8 \text{ ft}.
\]

Activate the Amplitude folder on Line 5 of the above demonstration to draw the horizontal line showing the average depth. Note this line is labeled with its equation. There is also a vertical line that shows the the maximum deviation from the mean.

\item Next we'll use the graph to find the period of oscillation. Since high tide occurs at 2:00am and low tide at 8:00am, the period (the time between succesive) high (or low) tides is
\[
    2(8 \text{ hours } - 2 \text{ hours }) = 12 \text{ hours}
\]
Activate the Period folder on Line 11 to show the period on the graph.

\item Now we'll compute the rotation rate of a uniform circular motion that generates the sinusoidal variation in the depth of the water. This rotation rate is
\[
  \omega = \frac{2\pi \text{ radians}}{12 \text{ hours}} = \frac{\pi}{6} \text{ radians/hr} .
\]
Activate the Uniform Circular Motion folder on Line 18 to show the period on the graph.

\item Next we'll use parts (a)-(d) to find an expression for the function 
\[
   h = f(t) ,  0\leq t \leq 24, 
\]
that gives the depth of the water (in feet) at time $t$ hours past midnight. Use the cosine function. Keeping in mind that high tide occrs at 2:00am, our function is 
\[
   h  = f(t) = 13 + 8 \cos \left(  \frac{\pi}{6} \left( t - 2 \right) \right),  0\leq t \leq 24.
\]



\item To check that our function is correct, we'll use the given information that the depth of the water is $21$ feet at 2:00am and $5$ feet at 8:00am.

Substituting $t=2$ gives the depth at 2:00am (in feet) as
\begin{align*}
   f(2)   & = 13 + 8 \cos \left(  \frac{\pi}{6} \left( 2 - 2 \right) \right) \\ 
           & = 13 + 8 \cos (0)  \\ 
           & = 13 + 8 \\
           & = 21 .
\end{align*}

Substituting $t=2$ gives the depth at 2:00am (in feet) as
\begin{align*}
   f(8)   & = 13 + 8 \cos \left(  \frac{\pi}{6} \left(8 - 2 \right) \right) \\ 
           & = 13 + 8 \cos (\pi)  \\ 
           & = 13 + 8(-1) \\
           & = 5 .
\end{align*}
These check out.

\item Finally, we'll use our function to estimate the depth (in feet) of the water at 5:30am, October 29 to be
\begin{align*}
   f(5.5)   & = 13 + 8 \cos \left(  \frac{\pi}{6} \left(5.5 - 2 \right) \right) \\ 
           & = 13 + 8 \cos (7\pi / 12)  \\ 
           &  \sim  10.92 \\
         \end{align*}

\item Enter the coordinates of the appropriate point in Line 27 of the Desmos Activity above to check that the depth at 5:30am is reasonable.

\end{enumerate}

\end{explanation}

\end{example}


\begin{example}  \label{Q4vdtggbghh}
This is a continuation of the previous example, and we'll start with the function
\[
   h  = f(t) = 13 + 8 \cos \left(  \frac{\pi}{6} \left( t - 2 \right) \right),  0\leq t \leq 24,
\]
that expresses the depth of the water (in feet) at the Edmonds Pier in terms of the number of hours past midnight.

Our question now is to find the rate at which the depth of the water is changing the \emph{second} time the water is $10$ feet deep. 

One approach would be to first determine when the water is $10$ feet deep for the second time. But this is not necessary. We can determine the rate without finding the time. Here's how.

\begin{enumerate}

\item We'll first take time out of the picture by letting
\[
   \theta = \frac{\pi}{6} \left( t - 2 \right)
\]
Then we'll let $t_0$ be the second time when the water is $10$ feet deep and define the angle $\theta_0$ as 
\[
   \theta_0 = \frac{\pi}{6} \left( t_0 - 2 \right)
\]

\item Our aim is to evaluate the derivative
\begin{align*}
  \frac{dh}{dt}\Big|_{t=t_0} &= -\frac{4\pi}{3} \sin\left( \frac{\pi}{6} \left( t_0 - 2 \right)   \right)   \\
                                          &= -\frac{4\pi}{3} \sin\left( \theta_0   \right) ,
\end{align*} 
where I've left the computation of the derivative $dh/dt$ to you.

Note the key point. To evaluate this derivative we need \emph{not} know the value of $\theta_0$, but only the value of $\sin\theta_0$.

\item To determine the value of $\sin\theta_0$, we go back to the height function, which we now write in the form (since $\theta = \pi(t-2)/6$)
\[
     h = g(\theta) = 13 + 8\cos \theta .
\]

When the water is $10$ feet deep for the second time,
\[
  10 = 13 + 8\cos \theta_0 
\]
and
\[
   \cos\theta_0 = -3/8 .
\]

\item Now since
\[
     \cos^2\theta_0 + \sin^2\theta_0 = 1 ,
\]
we find that
\[
    \sin\theta_0 = \pm \frac{\sqrt{55}}{8} .
\]

\item To choose the correct sign, we go back to the graph.

Desmos activity available at:

\href{https://www.desmos.com/calculator/fs9xy1lf0e}{142: Edmonds Pier 2}.

 
\begin{onlineOnly}
    \begin{center}
\desmos{fs9xy1lf0e}{900}{600}
\end{center}
\end{onlineOnly}

Here we can see that at the point $Q$ (with coordinates $(t_0, f(t_0))$, 
\[
          8 < t_0 < 14.
\]  
This tells us that 
\[
 \theta_ 0 =  \frac{\pi}{6} \left( t_0 - 2 \right)
\]
is between
\[
       \frac{\pi}{6} \left(8 - 2 \right)  <   \theta_0  < \frac{\pi}{6} \left(14 - 2 \right)
\]
or that
\[
     \pi < \theta_0 < 2\pi .
\]  
In other words, the angle $\theta_0$ is in the third or fourth quadrant. This means that $\sin\theta_0 < 0$ so 
\[
 \sin\theta_0 = - \frac{\sqrt{55}}{8} .
\]

\item Finally, we evaluate the derivative
\begin{align*}
 \frac{dh}{dt}\Big|_{t=t_0} &=-\frac{4\pi}{3} \sin\left( \theta_0   \right)  \\
                                         & = \left(- \frac{4\pi}{3}\right)  \left( - \frac{\sqrt{55}}{8} \right) \\
                                        &= \frac{\pi \sqrt{55}}{6} \\
\end{align*}

So when the water is $10$ feet deep for the second time, the depth is increasing at the rate of 
\[
  \frac{\pi \sqrt{55}}{6} \text{ ft/hr} \sim 3.88\text{ ft/hr} .
\]

\end{enumerate}
\end{example}








\end{document}