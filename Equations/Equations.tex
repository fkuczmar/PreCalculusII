\documentclass{ximera}
\title{Solving Trigonometric Equations}

\newcommand{\pskip}{\vskip 0.1 in}

\begin{document}
\begin{abstract}
Introduction to solving trigonometric equations.
\end{abstract}
\maketitle



\begin{example}  \label{Ex:43733g3e}
Approximate two consecutive solutions to the equation
\[
    \cos\theta =   -\frac{2}{7} .
\]

\begin{explanation}
 
Since
\[
   \cos \theta = \frac{x}{r} = -\frac{2}{7} ,
\]
we first draw a circle of radius $r=7$ and then the vertical line $x=-2$. The line intersects the circle in two points. Let $G$ be the intersection point in the second quadrant. Its polar angle is one solution to the above equation. The exact measure of this angle is 
\[
  \theta = \arccos(-2/7).
\]
Then using the radian protractor below, we find that one solution to the equation is 
\[
   \theta  = \arccos(-2/7) \sim 1.87 .
\]

The polar angle of second point $K$ where the line $x=-2$ intersects the circle gives another solution to the equation $\cos\theta = -2/7$. We can measure its exact and approximate values as
\[
   \theta = -\arccos(-2/7) \sim -1.87 
\]
or as
\[
       \theta = 2\pi - arccos(-2/7) \sim 6.28 - 1.87 \sim 4.41 .
\]


\pdfOnly{
Access Geogebra interactives through the online version of this text at
 
\href{https://www.geogebra.org/classic/qnhtd4h8}.
}
 
\begin{onlineOnly}
    \begin{center}
\geogebra{qnhtd4h8}{900}{600}
\end{center}
\end{onlineOnly}

Access Geogebra interactives through the online version of this text at
 
\href{https://www.geogebra.org/classic/qnhtd4h8}{142: Solving Trigonometric Equations}.


\end{explanation}

(b) Find all solutions to the equation in part (a). Give exact solutions without using a caculator.

\pskip \pskip

See the Panopto Recording \emph{Solving Trigonometric Equations, Part 1}.


\end{example}



\end{document}


