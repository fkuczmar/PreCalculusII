\documentclass{ximera}
\title{Solving Trigonometric Equations}

\newcommand{\pskip}{\vskip 0.1 in}

\begin{document}
\begin{abstract}
Introduction to solving trigonometric equations.
\end{abstract}
\maketitle

\section{Some Examples}

\begin{example}  \label{Ex:43733g3e}
(a) Find two consecutive solutions to the equation
\[
    7 \cos\theta +5 =  3 .  
\]

(b) Use the the radian protractor below to approximate these solutions.

(c) Find the solution set of all exact solutions to the above equation without using a calculator.

\begin{explanation}
(a)-(b) Solving the above equation for $\cos\theta$ gives 
\[
   \cos \theta = \frac{x}{r} = -\frac{2}{7}.
\]
To visualize the solutions of this equation, we draw a circle of radius $r=7$ and the vertical line $x=-2$. The line intersects the circle in two points $G$ and $K$. The set of all possible radian measures of their polar angles is the solution set to the original equation. There are two families of solutions, one that gives all possible polar angles for $G$, the other all possible polar angles for $K$
 
\pdfOnly{
Access Geogebra interactives through the online version of this text at
 
\href{https://www.geogebra.org/classic/qnhtd4h8}.
}
 
\begin{onlineOnly}
    \begin{center}
\geogebra{qnhtd4h8}{900}{600}
\end{center}
\end{onlineOnly}


To find these angles, we first look at the intersection point $G$ in the second quadrant. Here we can use the inverse cosine function to find one possible measure for the polar angle. This exact radian measure, the one between $0$ and $\pi$, is 
\[
  \theta = \arccos(-2/7).
\]
Then using the radian protractor below, we find that
\[
   \theta  = \arccos(-2/7) \sim 1.87 .
\]



A polar angle of the second point $K$ where the line $x=-2$ intersects the circle gives another solution to the equation $\cos\theta = -2/7$. Two possibilities for the exact and approximate radian measures for this angle are
\[
   \theta = -\arccos(-2/7) \sim -1.87 
\]
or 
\[
       \theta = 2\pi - \arccos(-2/7) \sim 6.28 - 1.87 \sim 4.41 .
\]


(c) We can find the set of all possible solutions by adding integral multiples of $2\pi$ to the angles $\theta  =\arccos(-2/7)$ and either the angle $\theta = -\arccos(-2/7)$ or $\theta = 2\pi - \arccos(-2/7)$. With the first of the two latter choices we can write  the solution set to the equation
\[
     7 \cos\theta +5 =  3 
\]
as
\[
     \{  \theta \, | \, \theta = \pm \arccos(-2/7) + 2\pi n \, , \, n \in \mathbb{Z}    \}.
\]
 


Access Geogebra interactives through the online version of this text at
 
\href{https://www.geogebra.org/classic/qnhtd4h8}{142: Solving Trigonometric Equations}.


\end{explanation}


\pskip \pskip

See the Panopto Recording \emph{Solving Trigonometric Equations, Part 1}.


\end{example}



\begin{example}  \label{Exer67u8gg}
The function
\[
    h  = f(t) = 7 + 5 \cos \left( \frac{\pi}{6.1}t  \right) \, , \, 0\leq t \leq 96 ,
\]
expresses the depth of the water (in feet) at the Edmonds pier in terms of the number of hours since noon of February 23rd.

\pskip

(a) Describe the uniform circular motion that generates the sinusoidal variation in the depth of the water.

(b) Graph two periods of the function $h=f(t)$.

(c) Use the graph in part (b) to approximate the first two times when the water is $3$ feet deep. 

(d) Find the exact values of the first two times when the water is $3$ feet deep.

(e) Find all (exact) times when the depth of the water is $3$ feet.

(f) During what fraction of a period of oscillation is the water at most $3$ feet deep?


\begin{explanation}
(a) The uniform circular motion that drives the sinusoidal variation

\begin{itemize}

\item{is about a circle centered at $h=7$ feet.}

\item{is about a circle with radius $r=5$ feet}

\item{rotates about the center at a constant rate of 
\[
     \omega = \frac{\pi}{6.1} \frac{ \text{ rad}}{\text{hr}} .
\]}

\item{has a period of
\[
   T =  \frac{2\pi \text{ rad}}{\frac{\pi}{6.1} \frac{ \text{ rad}}{\text{hr}}} = 12.2 \text{ hours and} 
\]}

\item{reaches its maximum of 
\[
   h  = f(0) = 7 + 5 \cos(0) = 7 + 5(1) = 12 \text{ feet}
\]
at time $t=0$ hours.}

\end{itemize}

\pskip

(c) Using the graph below, the first two times when the water at the pier is $3$ feet deep are approximately
\[
  t \sim 4.9, 7.3 .
\]

\pdfOnly{
Access Desmos interactives through the online version of this text at
 
\href{https://www.desmos.com/calculator/3w56sayz2h}.
}
 
\begin{onlineOnly}
    \begin{center}
\desmos{bab6apch7i}{900}{600}
\end{center}
\end{onlineOnly}
  
Access this desmos activity at
 
\href{https://www.desmos.com/calculator/bab6apch7i}{142: Edmonds Pier 58}


\pskip \pskip

(d), (e) The water is $3$ feet deep when
\[
   h = f(t) =3.
\]

To solve this equation and find all times when the water is $3$ feet deep, we'll make the substitution
\[
    \theta = \frac{\pi}{6.1}t ,
\]
so that
\[
     h = 7 + 5 \cos \left( \frac{\pi}{6.1}t  \right) = 7 + 5\cos\theta.
\]  
Then our first step is to solve the equation 
\[
  7 + 5\cos\theta = 3 ,
\]
or equivalently the equation
\[
   \cos \theta = -\frac{4}{5} .
\]

One solution to this equation is the angle (the one between $0$ and $\pi$)
\[
   \theta = \arccos(-4/5) .
\]
This angle generates a family of angles
\[
    \theta = \arccos(-4/5)  + 2\pi k, k\in \mathbb{Z}
\]
that are solutions to the equation 
\[
   \cos\theta = -\frac{4}{5} .
\]

To find a member of the second family of solutions, we reason as in Example 1, remembering that since
\[
\cos\theta = \frac{x}{r}   = -\frac{4}{5} ,
\] 
we can visualize the solutions as the polar angles of the two points where the line $x=4$ intersects a circle of radius $r=5$ feet centered at the origin. 

\begin{onlineOnly}
    \begin{center}
\geogebra{ydhvebag}{900}{600}
\end{center}
\end{onlineOnly}

Because the intersection points $G$ and $K$ are symmetric about the $x$-axis, one possible choice for the polar angle of point $K$ is
\[
   \theta=  - \arccos(-4/5) \sim -2.5 .
\]
This angle generates the second family of solutions
\[
  \theta = -\arccos(-4/5)  + 2\pi k, k\in \mathbb{Z}
\]
to the equation $\cos\theta = -4/5$.

Remembering that
\[
  \theta =  \frac{\pi}{6.1}t ,
\]
we now solve the equation
\[
     \frac{\pi}{6.1}t = \pm \arccos(-4/5)  + 2\pi k , k\in \mathbb{Z},
\]
getting
\[
   t = \pm \frac{6.1}{\pi} \arccos(-4/5) + 12.2 k , k\in \mathbb{Z}.
\]
So
\[
      t  \sim \pm 4.85 + 12.2 k , k\in \mathbb{Z}.
\]
The final step is to choose the appropriate values for $k$ so that $0\leq t \leq 96$.

For the first family, we need
\[
         0 \leq 4.85 + 12.2 k \leq 96
\]
so that
\[
      -0.40 \leq k \leq 7.45.
\]
Remembering that $k$ is an integer tells us
\[
   k = 0, 1, 2, \ldots ,7.
\]
For the second family,
\[
   0 \leq  - 4.85 + 12.2 k \leq 96
\]
and
\[
       0.40\leq k \leq 8.26 .
\]
So for the second family
\[
     k = 1, 2, \ldots , 8 .
\]

Summary:
\begin{itemize}
\item{There are 16 times (four each day) during the time interval $t\in [0,96]$ when the water is $3$ feet deep. The times are
\[
   t = \frac{6.1}{\pi}\arccos(-4/5) + 12.2k  \sim 4.85 + 12.2k ,  k=0, 1, 2, \ldots , 7
\]
or
\[
      t = -  \frac{6.1}{\pi}\arccos(-4/5) + 12.2k  \sim -4.85 + 12.2k ,  k=1, 1, 2, \ldots , 8.
\]}

\item{
The first two times (measured in hours past noon of February 23) are
\[
    t =  \frac{6.1}{\pi} \arccos(-4/5) \sim 4.85
\]
and 
\[
    t = - \frac{6.1}{\pi}\arccos(-4/5) + 12.2 \sim 7.15 .
\]
}
\end{itemize}

\pskip

(f) During a period of $12.2$ hours, the water is at most $3$ feet deep during a time interval of about
\[
   (7.15 - 4.85) \text { hours} = 2.23 \text{ hours}
\]
or about
\[
    \frac{2.23\text{ hrs}}{12.2 \text{ hrs}} \sim 18.85\%
\]
of the time.
\end{explanation}

\end{example}


\begin{question}  \label{Qert7u8gg}
The function
\[
    s  = f(t) = 12 - 4 \sin \left( \frac{2\pi}{3}t  \right) \, , \, 0\leq t \leq 48 ,
\]
expresses the height of a block (in feet) attached to a spring in terms of the number of seconds past noon.

\pskip

(a) Describe the uniform circular motion that generates the sinusoidal variation in the height of the block.

(b) Graph two periods of the function $h=f(t)$. Label the axes with the appropriate variable names and units

(c) Use the graph in part (b) to approximate the first two times when the block is $13$ feet high. 

(d) Find the exact values of the first two times when the block is $13$ feet high. 

(e) Find all (exact) times when the block is $13$ feet high. 

(f) During what fraction of a period of oscillation is the block at most $13$ feet above the ground?

\end{question}



\section{Discussion Questions}

\begin{question}  \label{Q:4dDFEFR33g3e}

\pdfOnly{
Access Geogebra interactives through the online version of this text at
 
\href{https://www.geogebra.org/classic/qnhtd4h8}.
}
 
\begin{onlineOnly}
    \begin{center}
\geogebra{qnhtd4h8}{900}{600}
\end{center}
\end{onlineOnly}

\begin{enumerate}
\item Use the radian protractor above to approximate two consecutive solutions to the equation
\[
   7 \cos\theta +11 =  8 .  
\]

\item Find the exact solutions in part (a)

\item Find the solution set of all exact solutions to the above equation without using a calculator.

\item Find the solution set of all exact solutions that lie between $24\pi$ and $26\pi$ radians.

\end{enumerate}
\end{question}




\begin{question}  \label{QdfEEFR33g3e}

\pdfOnly{
Access Geogebra interactives through the online version of this text at
 
\href{https://www.geogebra.org/classic/qnhtd4h8}.
}
 
\begin{onlineOnly}
    \begin{center}
\geogebra{qnhtd4h8}{900}{600}
\end{center}
\end{onlineOnly}

\begin{enumerate}
\item Use the radian protractor above to approximate two consecutive solutions to the equation
\[
   7 \sin\theta +11 =  8 .  
\]

\item Find the exact solutions in part (a)

\item Find the solution set of all exact solutions to the above equation without using a calculator.

\item Find the solution set of all exact solutions that lie between $12\pi$ and $14\pi$ radians.

\end{enumerate}
\end{question}



\begin{question}  \label{QDEFklEFR33g3e}

\pdfOnly{
Access Geogebra interactives through the online version of this text at
 
\href{https://www.geogebra.org/classic/qnhtd4h8}.
}
 
\begin{onlineOnly}
    \begin{center}
\geogebra{qnhtd4h8}{900}{600}
\end{center}
\end{onlineOnly}

\begin{enumerate}
\item Use the radian protractor above to approximate two consecutive solutions to the equation
\[
   7 \tan\theta +11 =  8 .  
\]

\item Find the exact solutions in part (a)

\item Find the solution set of all exact solutions to the above equation without using a calculator.

\item Find the solution set of all exact solutions that lie between $24\pi$ and $26\pi$ radians.

\end{enumerate}
\end{question}

 


\end{document}


