\documentclass{ximera}
\title{Solving Trigonometric Equations}

\newcommand{\pskip}{\vskip 0.1 in}

\begin{document}
\begin{abstract}
Introduction to solving trigonometric equations.
\end{abstract}
\maketitle



\begin{example}  \label{Ex:43733g3e}
(a) Find two consecutive solutions to the equation
\[
    7 \cos\theta +5 =  3 .  
\]

(b) Use the the radian protractor below to approximate these solutions.

(c) Find the solution set of all exact solutions to the above equation without using a calculator.

\begin{explanation}
(a)-(b) Solving the above equation for $\cos\theta$ gives 
\[
   \cos \theta = \frac{x}{r} = -\frac{2}{7}.
\]
To visualize the solutions of this equation, we draw a circle of radius $r=7$ and the vertical line $x=-2$. The line intersects the circle in two points $G$ and $K$. The set of all possible radian measures of their polar angles is the solution set to the original equation. There are two families of solutions, one that gives all possible polar angles for $G$, the other all possible polar angles for $K$
 
\pdfOnly{
Access Geogebra interactives through the online version of this text at
 
\href{https://www.geogebra.org/classic/qnhtd4h8}.
}
 
\begin{onlineOnly}
    \begin{center}
\geogebra{qnhtd4h8}{900}{600}
\end{center}
\end{onlineOnly}


To find these angles, we first look at the intersection point $G$ in the second quadrant. Here we can use the inverse cosine function to find one possible measure for the polar angle. This exact radian measure, the one between $0$ and $\pi$, is 
\[
  \theta = \arccos(-2/7).
\]
Then using the radian protractor below, we find that
\[
   \theta  = \arccos(-2/7) \sim 1.87 .
\]



A polar angle of the second point $K$ where the line $x=-2$ intersects the circle gives another solution to the equation $\cos\theta = -2/7$. Two possibilities for the exact and approximate radian measures for this angle are
\[
   \theta = -\arccos(-2/7) \sim -1.87 
\]
or 
\[
       \theta = 2\pi - \arccos(-2/7) \sim 6.28 - 1.87 \sim 4.41 .
\]


(c) We can find the set of all possible solutions by adding integral multiples of $2\pi$ to the angles $\theta  =\arccos(-2/7)$ and either the angle $\theta = -\arccos(-2/7)$ or $\theta = 2\pi - \arccos(-2/7)$. With the first of the two latter choices we can write  the solution set to the equation
\[
     7 \cos\theta +5 =  3 
\]
as
\[
     \{  \theta \, | \, \theta = \pm \arccos(-2/7) + 2\pi n \, , \, n \in \mathbb{Z}    \}.
\]
 


Access Geogebra interactives through the online version of this text at
 
\href{https://www.geogebra.org/classic/qnhtd4h8}{142: Solving Trigonometric Equations}.


\end{explanation}

(b) Find all solutions to the equation in part (a). Give exact solutions without using a caculator.

\pskip \pskip

See the Panopto Recording \emph{Solving Trigonometric Equations, Part 1}.


\end{example}



\end{document}


