\documentclass{ximera}
\title{Right Triangle Trigonometry}

\newcommand{\pskip}{\vskip 0.1 in}

\begin{document}
\begin{abstract}
Right triangle trigonometry.
\end{abstract}
\maketitle


Right triangle trigonometry is a special case of circular trigonometry, where the polar angle $\theta$ is one of the acute angles in a right triangle and therefore has a measure strictly between $0$ and $\pi/2$ radians. In this case, we can interpret the trigonometric ratios geometrically as %$\cos\theta$, $\sin\theta$, and $\tan\theta$ respectively as the ratios of the leg adjato the hypotenuse
\[
   \cos \theta = \frac{x}{r}  = \ \frac{\text{leg adjacent the angle } \theta}{\text{hypotenuse}}
\]
\[
   \sin \theta = \frac{y}{r}  = \ \frac{\text{leg opposite the angle } \theta}{\text{hypotenuse}}
\]
\[
   \cos \theta = \frac{y}{x}  = \ \frac{\text{leg opposite the angle } \theta}{\text{leg adjacent the angle } \theta}
\]

These ratios, as well as the angle $\theta$, all encode the \emph{shape} of a right triangle. 

\begin{question}   \label{Q45:RightTriangle}
Draw a right triangle in the appropriate postion in an $xy$-coordinate system and explain the interpretations of the three ratios above.
\end{question}

The most important thing to remember about right triangle trigonometry is that you can only use it for right triangles.

\begin{question} \label{Q1:RightTriangle}
You stand $a$ meters from the base of a tall building and measure the angle of  elevation to a point $P$ on the building. You then use that measurement to compute the height of the point $P$ above eye-level.

Let 
\[
    h = f(\theta) \, , 0\leq \theta < \pi/2 ,
\]
be the function that expresses that computed height (in feet) in terms of the measured angle (in radians).

(a) Draw a picture that captures the scenario. Include the ground, you, the building, the point $P$, the line of sight to $P$, and the angle $\theta$.

(b) Use your picture from part (a) to find an expression for $f(\theta)$.

(c) There will be an error $\Delta \theta$ in your measurement of the angle $\theta$ and this will lead to an error in computing the height of point $P$.  To approximate this error, we average the two errors
\[
     e_1 = f(\theta + \Delta\theta) - f(\theta)
\]
and 
\[
   e_2 = f(\theta) - f(\theta - \Delta\theta).
\]
This gives the approximate error in the computed height as
\begin{equation}
   e =   0.5 (e_1 + e_2) = 0.5 \left(  f(\theta + \Delta\theta) - f(\theta - \Delta\theta)     \right) . \label{Eq:Error65}
\end{equation}
Experiment with the Desmos activity below and describe qualitatively how the error varies with the measured angle of elevation.

(d) You stand $20$ meters from the base of a building and measure the angle of elevation to the top of the building. You use this measurement to compute the height of the building above eye level to be $80$ meters. 

You then walk away from the building, measure the angle of elevation to the top to be $0.6$ radians, and for the second time compute the height of the building above eye level.

Use the above expression (\ref{Eq:Error65}) and the function $f$ from part (b) with the appropriate values of $\theta$, $\Delta\theta$ and $a$ to approximate the two errors in computing the height. Assume a maximum error of $\pm 0.1$ radians in measuring the two angles of elevation. 


\begin{exploration}

\pdfOnly{
Access Desmos interactives through the online version of this text at
 
\href{https://www.desmos.com/calculator/15px2vpxz1}.
}
 
\begin{onlineOnly}
    \begin{center}
\desmos{15px2vpxz1}{900}{600}
\end{center}
\end{onlineOnly}
\end{exploration} 

\end{question}


\begin{question}  \label{Q34ftt4g}
You stand an unknown distance from the base of a building and measure the angle of elevation to the top of the building to be $\theta_1$ radians. You then walk an {\bf additional} 20 meters directly away from the building (so that you are now {\bf more than} 20 meters from the building) and measure the angle of elevation to its top to be $\theta_2$ radians. Express the height of the building above eye-level in terms of $\theta_1$ and $\theta_2$.

%From the top of a building you measure the angle of depression to the bottom of a nearby building to be $\theta_1$ radians. You measure the angle of elevation to the top of the same building to be $\theta_2$ radians. Express the distance $s$ between the buildings (measured in feet) in terms of the height $h$ (also measured in feet) of the nearby building.

\pskip


A complete solution will include:

\begin{itemize}
\item{A picture accurately depicting the situation with the angles and distance labeled accordingly.}

\item{Complete sentences that define any new variables (defined with single letters) that you introduce. Include units in the definitions. Do {\bf not} use notation like $AB$ or $\overline{AB}$ to denote lengths. Define variables instead.}

\item{The introduced variable(s) depicted clearly on your picture.}

\item{All your work, shown clearly and without skipping any steps.}

\item{A concluding sentence that answers the question.}

\end{itemize}

\emph{Suggestion:} Suppose instead you were given the height of the building above eye level and the angles $\theta_1$, $\theta_2$, but \emph{not} the distance ($20$ feet) that you walked. Try to find an equation that expresses the distance you walked, call it $s$ (mesaured in feet), in terms of the height  $h$ of the building (in feet) above eye level. Then solve this equation for $h$ in terms of $s$.

\end{question}


\begin{question} \label{Q3:RightTriangle}
A ferris wheel has radius $80$ meters and the center of the wheel is $90$ meters above the ground. You ride the wheel for one revolution.

Let 
\[
    h = f(\theta) \, , 0\leq \theta \leq 2\pi ,
\]
be the function that expresses your height above the ground in terms of the wheel's angle of rotation, measured from the time you  got on.

(a)  Assume $0 < \theta < \pi/2$ and draw a picture that captures the scenario. Include the ground, the ferris wheel, a point $P$ on the wheel representing your position after the wheel has turned through $\theta$ radians, and other lines as necessary. Label the angle $\theta$.

(b) Use your picture from part (a) to find an expression for $f(\theta)$.

(c) Suppose the wheel stops when you are at the angular position $\theta$ and that it then turns through a small angle $\Delta \theta$ and stops again.  Use the exploration below to describe qualitatively how the small change
\[
    \Delta h = f(\theta + \Delta \theta) - f(\theta)
\]
in your height depends on the angle $\theta$. 

(d) Fix $\Delta \theta \sim 0$ and use the exploration to sketch by hand a graph of the function 
\[
     \Delta h = g(\theta) \, , 0\leq \theta \leq 2\pi ,
\]
that takes as an input the wheel's angle of rotation $\theta$ and returns as an output the change $\Delta h$
in your height as the wheel turns through the small angle $\Delta \theta$.

\begin{exploration}

\pdfOnly{
Access Desmos interactives through the online version of this text at
 
\href{https://www.desmos.com/calculator/8swp20zond}.
}
 
\begin{onlineOnly}
    \begin{center}
\desmos{8swp20zond}{900}{600}
\end{center}
\end{onlineOnly}
\end{exploration} 
\end{question}


\begin{question} \label{Q4:RightTriangle}
You stand on the top of a building $50$ feet from a 200-foot tall building across the street. You measure the angle of depression to the base of the nearby building to be $\theta_1$ radians. You then measure the angle of elevation to the top of the same building to be $\theta_2$ radians. Express $\theta_2$ in terms of $\theta_1$.
\end{question}


\begin{question} \label{Q5:RightTriangle}
On the vernal equinox at a location on the equator, the sun rises due east at 6am, passes directly overhead at noon, and sets due west at 6pm. 
(a) Find a function 
\[
    s = f(t) \, , 0\leq t <6 ,
\]
that expresses the length of the shadow cast by a 100-foot tall building in terms of the number of hours past noon.

(b) Find the average speed of the tip of the shadow between 1:00pm and 5:00pm.

(c) Find a function 
\[
     v = g(t) 
\]
that gives the average speed of the tip of the shadow between times $t - \Delta t$ and $t+\Delta t$ hours past noon. 

(d) Follow the directions in the exploration below to approximate the speed of the tip of the shadow at 4pm, 5pm, and 5:45pm. Note the slider $w$ is used for $\Delta t$.

\begin{exploration}

\pdfOnly{
Access Desmos interactives through the online version of this text at
 
\href{https://www.desmos.com/calculator/aiqlac1gpf}.
}
 
\begin{onlineOnly}
    \begin{center}
\desmos{aiqlac1gpf}{900}{600}
\end{center}
\end{onlineOnly}
\end{exploration} 

\end{question}


\begin{question} \label{Q6:RightTriangle}
On the vernal equinox at a location on the equator, the sun rises due east at 6am, passes directly overhead at noon, and sets due west at 6pm. 

Imagine you and a friend are at the beach watching the sunset from a point on the equator on the vernal equinox. You lie on the beach and your friend stands on the balcony of your hotel, $h$ feet above the ground. As illustrated below, your friend will see the sunset later than you. 

(a) Find a function
\[
   t = f(h) \, , h\geq 0 ,
\]
that takes as an input the height $h$ of your friend above the ground (measured in feet) and returns as an output the difference in time (measured in seconds) between the moments the two of you see the sun dip below the horizon. Take the radius of the earth to be $R=3960$ miles. Note that there are 5,280 feet in one mile.

(b) How much later does your friend see the sun set if she is $100$ feet above the ground?

(c) Find a function 
\[
   h = f^{-1}(t)
\]
that express the height of your friend above the ground in terms of the time difference between the observed sunsets. This suggests a way to approximate the radius of the earth. How?

(d) How high above the ground is your friend if she sees the sun set two minutes later than you do?

(e) (optional) As the sun dips below the horizon, the earth casts a shadow on the building. Use the method of Question 5(d) to approximate the rate at which the shadow moves up the building when the shadow is 100 feet long. Explain your reasoning.
\begin{exploration}

\pdfOnly{
Access Desmos interactives through the online version of this text at
 
\href{https://www.desmos.com/calculator/2z5ai4ctfc}.
}
 
\begin{onlineOnly}
    \begin{center}
\desmos{2z5ai4ctfc}{900}{600}
\end{center}
\end{onlineOnly}
\end{exploration}

\end{question}



\begin{question} \label{Q7:RightTriangle}
In the animation below, point $Q$ moves counterclockwise around a circle of radius $r$ meters at a constant speed of $v$ m/sec. Segment $\overline{PQ}$ has length $L$ meters. Point $O$ is the center of the circle. The motion begins at time $t=0$ seconds when $Q$ is closest to $P$ and ends when $Q$ has completed one revolution about $O$.

(a) How long does it take $Q$ to turn once about $O$? Explain your reasoning. Work in general with the parameters $v$ and $r$, \emph{not} with their particular values in the animation below.

(b) Find a function 
\[
    s = f(t) \, , t \in \mathbb{R} ,
\] 
that expresses the distance $OP$ in terms of the number of seconds since time $t=0$. Include a picture to help with your explanation. State the appropriate domain for $f$. Work in general with the parameters $L$, $v$ and $r$, \emph{not} with their particular values in the animation below.

\pskip

Find the function as follows:

\begin{itemize}

\item{Express the s}

\end{itemize}



\pskip \pskip

(c) Use the animation below with $r=2$ meters, $L=5$ meters, and $v=4$ m/sec to sketch \emph{by hand} a graph of the function $s=f(t)$. Label the axes with the appropriate variable names and units. Label the exact coordinates of the graph's turning points. Briefly explain your reasoning and show how you computed the coordinates of the turning points. 

(d) Use the animation below with $r=2$ meters, $L=5$ meters, and $v=4$ m/sec to sketch \emph{by hand} a graph of the function $v=g(t)$ that expresses the speed of $P$ as a function of the number of seconds since time $t=0$. Label the exact coordinates of the graph's turning points. Briefly explain your reasoning. 

\pskip \pskip

(e) Find a function
\[
    w = g(t)
\]
that gives the average speed of $P$ between times $t-\Delta t$ and $t+\Delta t$ seconds.

 
\begin{exploration}

\pdfOnly{
Access Desmos interactives through the online version of this text at
 
\href{https://www.desmos.com/calculator/hhppprq7dn}.     %omxanb12qd
}
 
\begin{onlineOnly}
    \begin{center}
\desmos{hhppprq7dn}{900}{600}
\end{center}
\end{onlineOnly}
\end{exploration}

\end{question}


\begin{question}  \label{Qdegb574}
(a) At what speed are we moving due to the rotation of the earth about its axis? Assume Shoreline is at a latitude of $0.8$ radians
north (of the equator) and take the radius of the earth to be $3960$ miles.

(b) At what latitudes are people moving half as fast as we are?

(c) At what latitudes are people moving twice as fast as we are?

\end{question}

\begin{question} \label{Q34ft45rtg}
Assume for this question that Seattle and Spokane are at latitude $0.8$ radians north of the equator and that they have respective longitudes $122^\circ$W and $117^\circ$W.

(a) Find the distance between the cities along their common circle of latitude.

(b) Find the distance between the cities along the shortest path between them on the earth's surface.

\end{question}


\begin{question}  \label{Qewrdfg67}
From the top of a building you meaure the angle of evelation to the top of a neighboring building to be $\theta_1$ radians and the angle of depression to the base of the same building to be $\theta_2$ radians.

(a) Express the height $h$ (measured in feet) of the neighboring building in terms of the distance $s$ (measured in feet) between the buildings and the angles $\theta_1$, $\theta_2$.

(b) Express the distance between the buildings in terms of the height of the neighboring building and the angles $\theta_1$, $\theta_2$.

\end{question}

\begin{question}  \label{Qerdft543}
You measure the angle of elevation to the top of a tree to be $\theta_1$ radians. From your current position, you then walk $s$ feet directly away from the tree and measure the angle of elevation to the top of the tree to be $\theta_2$ radians.

(a) Express the distance $s$ in terms of the height $h$ of the tree above eye level (measured in feet) and the angles $\theta_1$, $\theta_2$.

(b) Express the height of the tree above eye level in terms of $s$ and the angles $\theta_1$, $\theta_2$.
\end{question}


\begin{question}  \label{Qtgb677432}
One end of an 8-ft long string is attached to the top of a 10-ft pole, the other end to a hat. Someone, a child, a giant, or perhaps a giraffe, puts the hat on their head and walks away from the pole until the string is taut.

(a) Find a function $h=f(\theta)$ that expresses the height of the hat (above the ground) in terms of the angle the string makes with the pole.

b) Find the domain and range of the function $f$.

c) Does $f$ have an inverse? If not, explain why not. If so, find an expression for the inverse, along with its domain and range.
\end{question}




\begin{question} \label{Q8:RightTriangle}
Phases of the moon.

\begin{exploration}

\pdfOnly{
Access Desmos interactives through the online version of this text at
 
\href{https://www.desmos.com/calculator/omxanb12qd}.
}
 
\begin{onlineOnly}
    \begin{center}
\desmos{omxanb12qd}{900}{600}
\end{center}
\end{onlineOnly}
\end{exploration}

\end{question}



\end{document}