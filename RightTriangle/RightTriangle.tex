\documentclass{ximera}
\title{Right Triangle Trigonometry}

\newcommand{\pskip}{\vskip 0.1 in}

\begin{document}
\begin{abstract}
Right triangle trigonometry.
\end{abstract}
\maketitle


Right triangle trigonometry is a special case of circular trigonometry, where the polar angle $\theta$ is one of the acute angles in a right triangle and therefore has a measure strictly between $0$ and $\pi/2$ radians. In this case, we can interpret the trigonometric ratios geometrically as %$\cos\theta$, $\sin\theta$, and $\tan\theta$ respectively as the ratios of the leg adjato the hypotenuse
\[
   \cos \theta = \frac{x}{r}  = \ \frac{\text{leg adjacent the angle } \theta}{\text{hypotenuse}}
\]
\[
   \sin \theta = \frac{y}{r}  = \ \frac{\text{leg opposite the angle } \theta}{\text{hypotenuse}}
\]
\[
   \cos \theta = \frac{y}{x}  = \ \frac{\text{leg opposite the angle } \theta}{\text{leg adjacent the angle } \theta}
\]

These ratios, as well as the angle $\theta$, all encode the \emph{shape} of a right triangle. 

\begin{question}   \label{Q45:RightTriangle}
Draw a right triangle in the appropriate postion in an $xy$-coordinate system and explain the interpretations of the three ratios above.
\end{question}

The most important thing to remember about right triangle trigonometry is that you can only use it for right triangles.

\begin{question} \label{Q1:RightTriangle}
You stand $a$ meters from the base of a tall building and measure the angle of  elevation to a point $P$ on the building. You then use that measurement to compute the height of the point $P$ above eye-level.

Let 
\[
    h = f(\theta) \, , 0\leq \theta < \pi/2 ,
\]
be the function that expresses that computed height (in feet) in terms of the measured angle (in radians).

(a) Draw a picture that captures the scenario. Include the ground, you, the building, the point $P$, the line of sight to $P$, and the angle $\theta$.

(b) Use your picture from part (a) to find an expression for $f(\theta)$.

(c) There will be an error $\Delta \theta$ in your measurement of the angle $\theta$ and this will lead to an error in computing the height of point $P$.  To approximate this error, we average the two errors
\[
     e_1 = f(\theta + \Delta\theta) - f(\theta)
\]
and 
\[
   e_2 = f(\theta) - f(\theta - \Delta\theta).
\]
This gives the approximate error in the computed height as
\begin{equation}
   e =   0.5 (e_1 + e_2) = 0.5 \left(  f(\theta + \Delta\theta) - f(\theta - \Delta\theta)     \right) . \label{Eq:Error65}
\end{equation}
Experiment with the Desmos activity below and describe qualitatively how the error varies with the measured angle of elevation.

(d) You stand $20$ meters from the base of a building and measure the angle of elevation to the top of the building. You use this measurement to compute the height of the building to be $80$ meters. 

You then walk away from the building, measure the angle of elevation to the top to be $0.6$ radians, and compute the height of the building again.

 Use the above expression (\ref{Eq:Error65}) to approximate the two errors in computing the height. Assume a maximum error of $\pm 0.1$ radians in measuring the two angles of elevation. 

\begin{exploration}

\pdfOnly{
Access Desmos interactives through the online version of this text at
 
\href{https://www.desmos.com/calculator/15px2vpxz1}.
}
 
\begin{onlineOnly}
    \begin{center}
\desmos{15px2vpxz1}{900}{600}
\end{center}
\end{onlineOnly}
\end{exploration} 

\end{question}


\begin{question} \label{Q3:RightTriangle}
A ferris wheel has radius $80$ meters and the center of the wheel is $90$ meters above the ground. You ride the wheel for one revolution.

Let 
\[
    h = f(\theta) \, , 0\leq \theta \leq 2pi ,
\]
be the function that expresses your height above the ground in terms of the wheel's angle of rotation, measured from the time you  got on.

(a)  Assume $0 < \theta < \pi/2$ and draw a picture that captures the scenario. Include the ground, the ferris wheel, a point $P$ on the wheel representing your position after the wheel has turned through $\theta$ radians, and other lines as necessary. Label the angle $\theta$.

(b) Use your picture from part (a) to find an expression for $f(\theta)$.

(c) Suppose the wheel stops when you are at the angular position $\theta$ and that it then turns through a small angle $\Delta \theta$ and stops again.  Use the exploration below to describe qualitatively how the small change
\[
    \Delta h = f(\theta + \Delta \theta) - f(\theta)
\]
in your height depends on the angle $\theta$. 

(d) Fix $\Delta \theta \sim 0$ and use the exploration to sketch by hand a graph of the function 
\[
     \Delta h = g(\theta) \, , 0\leq \theta \leq 2\pi ,
\]
that takes as an input the wheel's angle of rotation $\theta$ and returns as an output the change $\Delta h$
in your height as the wheel turns through the small angle $\Delta \theta$.

\begin{exploration}

\pdfOnly{
Access Desmos interactives through the online version of this text at
 
\href{https://www.desmos.com/calculator/8swp20zond}.
}
 
\begin{onlineOnly}
    \begin{center}
\desmos{8swp20zond}{900}{600}
\end{center}
\end{onlineOnly}
\end{exploration} 





\end{question}

\end{document}