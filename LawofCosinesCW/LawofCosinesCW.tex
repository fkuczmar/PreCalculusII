\documentclass{ximera}
\title{Law of Cosines, CW}

\newcommand{\pskip}{\vskip 0.1 in}

\begin{document}
\begin{abstract}
The law of cosines.
\end{abstract}
\maketitle



\section{The Law of Cosines and Relative Position}
Earlier in this class we solved problems about relative position by resolving vectors into their $x$ and $y$ components. Here we show how to take a different approach by working with triangles instead.

\begin{question}  \label{Exedfrgt43tgr}
Point $B$ is $10$ miles due east of point $A$. Point $C$ is $8$ miles from $A$ at a bearing of $0.6$ radians. We'll describe the position of $C$ relative to $B$ by giving a distance and a bearing.

\pdfOnly{
Access Geogebra interactives through the online version of this text at
 
\href{https://www.geogebra.org/classic/vsdrcnqm}.
}
 
\begin{onlineOnly}
    \begin{center}
\geogebra{vsdrcnqm}{900}{600}
\end{center}
\end{onlineOnly}


Geogebra activity available at

\href{https://www.geogebra.org/classic/vsdrcnqm}{142: Relative Position 4b}

\begin{enumerate}
\item Find an expression for the exact distance from $B$ to $C$ by resolving vectors into their components. Simplify your expression as much as possible. Do \emph{not} use a calculator.

\item Find the exact distance from $B$ to $C$ by using the law of cosines in $\Delta ABC$ instead. Do not use a calculator. Compare with part (a).

\item Find the exact measure of $\angle ACB$ without using a calculator except to do arithmetic. Do this twice.

\begin{enumerate}
\item First using the law of cosines.

\item Again using the law of sines. Explain the problem you run into here.

\end{enumerate}

\item Find a way to use the law of sines in $\Delta ABC$ to find the exact measure of $\angle ACB$.

\item Use the previous results to find two different expressions for the exact bearing of the direct path from $B$ to $C$, measured counterclockwise from the east. 

\item Use a calculator to approximate the bearing in part (e).

\item Use the protractor in the worksheet above to check the bearing is correct.

\end{enumerate}

\end{question}


\section{The Ladder and the Tree}

\begin{question} \label{Q00:Si44neCosine}
A tree leans precariously (or maybe impossibly) with its trunk making at an angle of $\alpha = \arcsin(1/2)$ radians with the ground. The bottom end of a 10-foot ladder is 16 feet from the base of the trunk and the top end rests against the trunk. 

\begin{enumerate}

\item Drag the slider $u$ in Line 2 in the exploration below to slide the ladder and approximate the distance from its top end ($T$) to the trunk's base ($A$) when its bottom end ($B$) is 16 feet from the trunk's base. Find all possibilities.

\item Use the law of cosines in $\Delta ABT$ to find the exact distance(s) from the top of the ladder to the base of the tree when its bottom end is $16$ feet from the trunk's base. Do this by completing the square. Do \emph{not} use the quadratic formula.

\item Use a calculator to approximate the distance(s) in part (b). Compare with your estimates in part (a).

\item Use the law of sines in $\Delta ABT$ to find the exact distance(s) from the top of the ladder to the base of the tree. Then use a calculator to approximate the distance(s).

\item Use right triangle trigonometry to find the exact distance(s) from the top of the ladder to the base of the tree. Activate the folder \emph{Right Triangle Trig} in Line 3 for a hint.

\end{enumerate}

\begin{exploration}

\pdfOnly{
Access Desmos interactives through the online version of this text at
 
\href{https://www.desmos.com/calculator/a1lr8gkfgz}.
}
 
\begin{onlineOnly}
    \begin{center}
\desmos{a1lr8gkfgz}{900}{600}
\end{center}
\end{onlineOnly}
\end{exploration} 

\href{https://www.desmos.com/calculator/a1lr8gkfgz}{142: Ladder and Tree 2C}

\end{question}


\end{document}