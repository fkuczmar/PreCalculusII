\documentclass{ximera}
\title{Using Vectors to Parameterize Circles}

\newcommand{\pskip}{\vskip 0.1 in}

\begin{document}
\begin{abstract}
Vectors and uniform circular motion.
\end{abstract}
\maketitle

\section{Circles not Centered at the Origin}


\begin{question}  \label{ExKdfeKR}
We'll parameterize a circle with radius $3$ cm centered at the point $C(5,4)$ (coordinates measured in cm) in terms of the polar angle of the vector from $C$ to a point $P(x,y)$ on the circle. The steps are outlined below.

\begin{onlineOnly}
    \begin{center}
\desmos{pcndel98wg}{900}{600}
\end{center}
\end{onlineOnly}

\href{https://www.desmos.com/calculator/pcndel98wg}{142: Translating Circles}


\begin{explanation} 

Here are the steps, with the details left for you to fill in.

We'll start by letting $P$ be a point on the circle with coordinates $(x,y)$ and $\theta$ the polar angle of the vector $\overrightarrow{CP}$ giving the position of $P$ relative to $C$.

\begin{enumerate}
\item First express the components of the vector $\overrightarrow{CP}$ in terms its polar angle $\theta$.

\item Next express the vector $\overrightarrow{OP}$ giving the position of $P$ relative to the origin in terms of $\overrightarrow{CP}$ and the vector $\overrightarrow{OC}$.

\item Use the results of parts (a) and (b) to express the components of the vector $\overrightarrow{OP}$ in terms of $\theta$.

\item Use the result of part (c) to find functions
\[
    x = f(\theta)  \hskip 0.5 in \text{and} \hskip 0.5in  y=g(\theta) ,
\]
that express the coordinates of $P$ in terms of $\theta$. Include appropriate domains for these functions. 

\item Use part (d) to find the coordinates of $P$ when $\theta=\pi/2, 3\pi/4, \pi$. Check that these results are correct.

\item Use part (d) to find the exact and approximate coordinates of $P$ when $\theta =4$. Use the worksheet above to check that the coordinates are reasonable (the slider $u$ in Line 1 is another name for the polar angle $\theta$.) 


\item One last question. Find equations in Cartesian coordinates of 

\begin{enumerate}
\item the unit circle centered at the origin and 

\item The circle of radius $3$ centered at  $C(5,4)$.

These equations express relationships between the $x$ and $y$ coordinates of points on the circles. They do not involve the parameter $\theta$. 

\end{enumerate}
\end{enumerate}

\end{explanation}
\end{question}


\begin{question}  \label{QDF09555dfdfLKDD}
\begin{enumerate}

\item Parametrize the circle in Question 1 by the signed arclength from $A$ of a point $P(x,y)$ on the circle. This means to express the coordinates $(x,y)$ of $P$ (measured in cm) in terms of its signed distance from $A$. Measure the signed arclength $s$ (in cm)  from the point $A(8,0)$ along the circle and take the counterclockwise direction to be positive. 

\begin{onlineOnly}
    \begin{center}
\desmos{llrzcv5ckf}{900}{600}
\end{center}
\end{onlineOnly}

\href{https://www.desmos.com/calculator/llrzcv5ckf}{142: Circle Arclength Parameterization}



The coordinate functions (in cm) are 
\[
  x = f_1(s) = \answer{5 + 3 \cos(s/3)} \, , \, s\in \mathbb{R}
\]
and
\[
   y = g_1(s) = \answer{4 + 3 \sin(s/3)} \, , \, s\in \mathbb{R} .
\]

\item Use part (a) to find the exact coordinates of the point on the circle  $20$ cm from $A$ as measured counterclockwise along the circle. 

\item Use the worksheet above to approximate these coordinates and check if your coordinates look reasonable.

\item Between noon and 12:02pm a beetle crawls counterclockwise around the circle at a constant speed of $6$cm/sec, passing the point $A(8,0)$ at $23$ seconds past noon. Find coordinate functions
\[
   x = f_2(t)  \hskip 0.5 in \text{and} \hskip 0.5in  y=g_2(t) ,
\] 
that express the beetle's coordinates (in cm) in terms of the number of seconds past noon.

\end{enumerate}
\end{question}



\begin{question}  \label{QDfg554egsE}
A beetle starts from the point $A(8,3)$, crawls $s$ meters counterclockwise around the circle centered at $Q(2,3)$ and stops at point $B$. All coordinates measured in meters.

\begin{enumerate}
\item Express the components of the vector $\overrightarrow{QB}$ in terms of $s$.

\item Express the vector $\overrightarrow{OB}$ in terms of the vectors $\overrightarrow{OQ}$ and $\overrightarrow{QB}$, where $O$ is the origin.

\item Express the coordinates of $B$ in terms of $s$.

\item Find the exact coordinates of $B$ when $s=10$ meters. Then use a calculator to approximate these coordinates. Finally, use the worksheet below to check that your answer is reasonable.
\end{enumerate}
\end{question}



\section{Vector Arithmetic}

\begin{question} \label{QLfderr33r3}

\begin{onlineOnly}
    \begin{center}
\desmos{jg9co800dy}{900}{600}
\end{center}
\end{onlineOnly}

\href{https://www.desmos.com/calculator/jg9co800dy}{142: Vector Arithmetic}

\begin{enumerate}
\item Given the vectors ${\bf u}$ and ${\bf v}$ above, drag point $Q$ to sketch the vectors

\begin{enumerate}
\item $2{\bf u}$

\item $-3{\bf v}$

\item ${\bf u} + {\bf v}$

\item $2{\bf u} + 3{\bf v}$

\item $2{\bf u} - 3{\bf v}$

\item $0.5{\bf u} + 0.5{\bf v}$

\item ${\bf  u} - {\bf v}$

\end{enumerate}

\item Express vector ${\bf r}$ in terms of ${\bf u}$ and ${\bf v}$.

\end{enumerate}
\end{question}


\begin{question} \label{QLf45445rr33r3}

The vectors ${\bf u}$ and ${\bf v}$ below are perpendicular and have the same length.

\begin{onlineOnly}
    \begin{center}
\desmos{gah1mliasn}{900}{600}
\end{center}
\end{onlineOnly}

\href{https://www.desmos.com/calculator/gah1mliasn}{142: Vector Arithmetic 3}


\begin{enumerate}
\item Express vector ${\bf q}$ in terms of ${\bf u}$ and ${\bf v}$. 

\item Use the figure to approximate the values of $\cos\theta$ and $\sin\theta$ for the marked angle $\theta$ from ${\bf u}$ to ${\bf q}$. How are these values related to your answer in part (a)? 

For this question, forget about vectors. Think about coordinates and the definitions of these functions instead. The only thing new here is that the coordinate axes are rotated. Which vector, ${\bf u}$ or ${\bf v}$, should you take to point in the direction of the positive $x$-axis?

\item Turn off the folder \emph{Vector $q$} in Line 1. Then activate the folder \emph{Vector $z$} in Line 5 and repeat parts (a) and (b).

\item Turn off the folder \emph{Vector $z$} in Line 5. Then activate the folder \emph{Vector $j$} in Line 9 and repeat parts (a) and (b).


\item Turn off the Folder \emph{Vectors $j$} in Line 9 and activate the folder \emph{General Vector} in Line 11. Then express vector ${\bf p}$ in terms of the vectors ${\bf u}$ and ${\bf v}$ and the angle $\theta$. Do \emph{not} read off any numbers from the worksheet. Work in general.

\end{enumerate}
\end{question}


\section{General Starting Points}

\begin{question} \label{QIDdfdseIFDfeD}
A beetle starts at point $A(3,5)$ (coordinates in cm), crawls $s$ meters counterclockwise around the circle through $A$ centered at the origin $O$ and stops at $P$.

\begin{onlineOnly}
    \begin{center}
\desmos{ufyt7kkxc0}{900}{600}
\end{center}
\end{onlineOnly}

\href{https://www.desmos.com/calculator/ufyt7kkxc0}{142: Vectors and Circular Motion 1}


\begin{enumerate}
\item Express the vector $\overrightarrow{OP}$ in terms of $s$. Do this as follows.

\begin{enumerate}
\item Find the components of the vector $\overrightarrow{OA}$ from the origin to $A$.

\item Find the coordinates of the point $B$ on the circle $\pi/2$ radians counterclockwise from $A$. Then find the components of the vector $\overrightarrow{OB}$.

\item Express the radian measure of the angle $\theta = \angle AOP$ in terms of $s$.
\[
   \theta = \answer{s/\sqrt{34}} .
\]

\item Express the vector $\overrightarrow{OP}$ in terms $s$ and the vectors $\overrightarrow{OA}$ and $\overrightarrow{OB}$.
\[
      \overrightarrow{OP} = \answer{\cos (s/\sqrt{34})} \overrightarrow{OA} + \answer{\sin (s/\sqrt{34})} \overrightarrow{OB} . 
\]

\end{enumerate}

\item Use the result of the previous part to find functions
\[
 x = f(s)  \text{ and } y=g(s)
\]
that express the coordinates of $P$ in terms of $s$.

The coordinate functions are 
\[
    x = f(t) = \answer{3\cos(s/\sqrt{34}) - 5\sin(s/\sqrt{34})}
\]
and
\[
    y = g(t) = \answer{5\cos(s/\sqrt{34}) + 3\sin(s/\sqrt{34})}.
\]

\item Enter the expressions for the coordinate functions $f$ and $g$ in Lines 34 and 35 of the worksheet above as a check.

\item Find the exact (without a calculator) and approximate coordinates of the point on the circle $20$cm from point $A$. The $20$cm is measured counterclockwise along the circle.
\end{enumerate}

\emph{Explanation:}

Since the point $A$ has coordinates $(3,5)$ and $O$ is the origin,
\[
    \overrightarrow{OA} = \langle 3, 5 \rangle.
\]
Similarly,
\[
    \overrightarrow{OB} = \langle -5,3 \rangle.
\]

Note this key point. The vectors $\overrightarrow{OA}$ and $\overrightarrow{OB}$ are perpendicular and have the same length. And the sense of rotation from $\overrightarrow{OA}$ to $\overrightarrow{OB}$ is counterclockwise.

So because the angle $\theta$ measures the counterclockwise angle from $\overrightarrow{OA}$ to $\overrightarrow{OP}$,
\begin{align*}
  \overrightarrow{OP} &= (\cos \theta) \overrightarrow{OA} + (\sin\theta) \overrightarrow{OB} \\
                                 &=\cos\theta \langle 3, 5 \rangle + \sin \theta \langle -5, 3 \rangle \\
                                 &=\langle 3 \cos\theta , 5\cos\theta \rangle + \langle -5\sin \theta, 3 \sin\theta \rangle \\
                                 &= \langle 3 \cos\theta - 5\sin\theta , 5\cos\theta + 3\sin\theta \rangle.
\end{align*}

This tells us (since $O$ is the origin) that $P$ has coordinates
\[
  x = 3 \cos\theta - 5\sin\theta
\]
and
\[ 
    y = 5\cos\theta + 3\sin\theta .
\]

To express these coordinates in terms of the counterclockwise arclength $s$ from $A$ to $P$, note that
But 
\[
  \theta =    \frac{s}{\left|  \overrightarrow{OA} \right|}  = \frac{s}{\sqrt{34}} .
\]

So $P$ has coordinates
\[
      x = f(s) = 3 \cos\left( \frac{s}{\sqrt{34}} \right) - 5\sin\left( \frac{s}{\sqrt{34}} \right)
\]
and
\[
      y = g(s) = 5 \cos\left( \frac{s}{\sqrt{34}} \right) +3\sin\left( \frac{s}{\sqrt{34}} \right) .
\]
 

\end{question}


Before parameterizing a general circle by arclength, we need to know how to rotate a vector counterclockwise by $\pi/2$ radians.
That's the goal of the following problem.


\begin{question} \label{Q77ee333}


\begin{enumerate}
\item Drag point $R$ in the worksheet below so that
\begin{enumerate}
\item $\left|\overrightarrow{OR}\right| = \left|\overrightarrow{OQ}\right|$ and
\item the sense of rotation from $\overrightarrow{OQ}$ to $\overrightarrow{OR}$ is $\pi/2$ radians counterclockwise.
\end{enumerate}
Then record the components of $\overrightarrow{OQ}$ and $\overrightarrow{OR}$.

\item Move point $Q$ to the second, third, and then fourth quadrants, each time repeating part (a).

\item What do you notice about the relationship between the components of the vectors $\overrightarrow{OQ}$ and $\overrightarrow{OR}$?

\item The conclusion is this:

Rotating the vector 
\[
  \overrightarrow{OQ} = \langle a, b\rangle
\]
counterclockwise by $\pi/2$ radians gives the vector $\overrightarrow{OR}$ with components
\[
  \overrightarrow{OQ} = \langle \answer{-b}, \answer{a}\rangle .
\] 

\end{enumerate}

\begin{onlineOnly}
    \begin{center}
\desmos{in7whvqoay}{900}{600}
\end{center}
\end{onlineOnly}

\href{https://www.desmos.com/calculator/in7whvqoay}{142: Vectors Arithmetic 5}


\end{question}


\begin{question} \label{QWEercfe341}

The aim of this problem is to parameterize a circle by arclength.

This particular circle has center at the point $C$ with coordinates $(5,4)$cm and it passes through the point $A$ with coordinates $(9,2)$cm. 

We wish to express the coordinates of a point $P$ on the circle in terms of its distance $s$ from $A$, measured counterclockwise around the circle in cm. In the worksheet below, for example, $P$ is $s=11$cm from $A$. But we'll work in general and assume the (signed) distance between $P$ and $A$ is $s$ cm. The signed distance is measured negatively in the clockwise direction from $A$.

\begin{onlineOnly}
    \begin{center}
\desmos{kuwm2chgyu}{900}{600}
\end{center}
\end{onlineOnly}

\href{https://www.desmos.com/calculator/kuwm2chgyu}{142: Circle Arc Length Parameterization}


We'll take two steps:

\begin{enumerate}
\item First express the coordinates $(x,y)$ of $P$ (measured in cm) in terms of the marked angle $\theta$ from $\overrightarrow{CA}$ to $\overrightarrow{CP}$.

\item Then express the coordinates $(x,y)$ of $P$ (measured in cm) in terms of the signed arclength $AP$ measured counterclockwise around the circle from $A$ in cm. 
\end{enumerate}

You can try these problems on your own, but click the arrow at the bottom right if you need help.

\begin{expandable}

\begin{enumerate}
\item The first step is to find the components of the vector $\overrightarrow{CP}$ that gives the position of $P(x,y)$ relative to $C$. For this do the following.

\begin{enumerate}
\item Find the components of the vectors $\overrightarrow{CA}$ and $\overrightarrow{CB}$.  Here $B$ is the point on the circle rotated counterclockwise from $A$ through the angle $\pi/2$ radians. Use the result of Question 7 to find the components of $\overrightarrow{CB}$. But if you need help, activate the folder \emph{vector $CB$} in Line 1.

\item Next express the vector $\overrightarrow{CP}$ in terms of the angle $\theta$ and the vectors $\overrightarrow{CA}$, $\overrightarrow{CB}$.
\[
       \overrightarrow{CP} = (\answer{\cos \theta}) \overrightarrow{CA} + (\answer{\sin \theta}) \overrightarrow{CB} .
\]

\item Then express the angle $\theta$ in terms of the arclength $s$ from $A$ to $P$.
\[
     \theta  =\answer{s/\sqrt{20}} .
\]

\item Finally, express  the vector $\overrightarrow{CP}$ in terms of $s$ and the vectors $\overrightarrow{CA}$, $\overrightarrow{CB}$.

\[
       \overrightarrow{CP} = (\answer{\cos (s/\sqrt{20})}) \overrightarrow{CA} + (\answer{\sin (s/\sqrt{20})}) \overrightarrow{CB} .
\]
\end{enumerate}

\item The second step is to express the vector  $\overrightarrow{OP}$ giving the position of $P$ relative to the origin, in terms of
the vectors $\overrightarrow{CA}$, $\overrightarrow{CB}$ and $s$.

First, since
\[
    \overrightarrow{OP} = \overrightarrow{\answer{OC}} + \overrightarrow{\answer{CP}},
\]
we know
\[
  \overrightarrow{OP} = \langle \answer{5}, \answer{4}\rangle +   \cos \left( \frac{s}{\sqrt{20}} \right) \langle \answer{4} , \answer{-2} \rangle + \sin \left( \frac{s}{\sqrt{20}} \right) \langle \answer{2} , \answer{4} \rangle .
\]

\item Finally, use the result of part (b) to find functions
\[
    x = f(s) \text{ and } y=g(s) , s\in \mathbb{R}
\]
that express the coordinates $(x,y)$ of $P$ in terms of $s$.

The coordinate functions are
\[
    x = f(s) = \answer{5 + 4\cos (s/\sqrt{20}) +2\sin (s/\sqrt{20}) }
\]
and
\[
  x = g(s) = \answer{4 - 2\cos (s/\sqrt{20}) +4\sin (s/\sqrt{20}) } .
\]

\item Use part (c) to find the exact (without a calculator) and then approximate (with a calculator) coordinates of the point $P$ above 11 cm from $A$. 
\end{enumerate}

\end{expandable}

\end{question}


\begin{question} \label{QLdferr444}
Between noon and 12:15pm an ant turns counterclockwise in a circle about the point $C(2,-3)$ at the constant rate of $3/4$ rad/sec. It passes the point $A(-1,4)$ at 12:05m. All coordinates are measured in cm.

\begin{onlineOnly}
    \begin{center}
\desmos{genxpc7blg}{900}{600}
\end{center}
\end{onlineOnly}

\href{https://www.desmos.com/calculator/genxpc7blg}{142: Ant General Circle}


\begin{enumerate}
\item Find functions
\[
   x = f(t) \text{ and } y=g(t) \, , \, 0\leq t \leq 15,
\]
that express the ant's coordinates (in cm) in terms of the number of minutes past noon.

\item Enter the coordinate functions in Lines 42 and 43 of the worksheet above. Then drag the slider $t_1$ (another name for $t$) in Line 2 to check your work.

\item Find (without a calculator) the ant's exact coordinates at noon. Then (with a calculator) the approximate coordinates. Use the worksheet above to check your work.

\item How would your coordinate functions change if the ant crawled clockwise around the circle instead (all other given information being identical)?

\end{enumerate}
\end{question}


\end{document}