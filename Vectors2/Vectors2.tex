\documentclass{ximera}
\title{Using Vectors to Parameterize Circles}

\newcommand{\pskip}{\vskip 0.1 in}

\begin{document}
\begin{abstract}
Vectors and uniform circular motion.
\end{abstract}
\maketitle

\section{Circles not Centered at the Origin}


\begin{question}  \label{ExKdfeKR}
We'll parameterize a circle with radius $3$ cm centered at the point $C(5,4)$ (coordinates measured in cm) in terms of the polar angle of the vector from $C$ to a point $P(x,y)$ on the circle. The steps are outlined below.

\begin{onlineOnly}
    \begin{center}
\desmos{pcndel98wg}{900}{600}
\end{center}
\end{onlineOnly}

\href{https://www.desmos.com/calculator/pcndel98wg}{142: Translating Circles}


\begin{explanation} 

Here are the steps, with the details left for you to fill in.

We'll start by letting $P$ be a point on the circle with coordinates $(x,y)$ and $\theta$ the polar angle of the vector $\overrightarrow{CP}$ giving the position of $P$ relative to $C$.

\begin{enumerate}
\item First express the components of the vector $\overrightarrow{CP}$ in terms its polar angle $\theta$.

\item Next express the vector $\overrightarrow{OP}$ giving the position of $P$ relative to the origin in terms of $\overrightarrow{CP}$ and the vector $\overrightarrow{OC}$.

\item Use the results of parts (a) and (b) to express the components of the vector $\overrightarrow{OP}$ in terms of $\theta$.

\item Use the result of part (c) to find functions
\[
    x = f(\theta)  \hskip 0.5 in \text{and} \hskip 0.5in  y=g(\theta) ,
\]
that express the coordinates of $P$ in terms of $\theta$. Include appropriate domains for these functions. 

\item Use part (d) to find the coordinates of $P$ when $\theta=\pi/2, 3\pi/4, \pi$. Check that these results are correct.

\item Use part (d) to find the exact and approximate coordinates of $P$ when $\theta =4$. Use the worksheet above to check that the coordinates are reasonable (the slider $u$ in Line 1 is another name for the polar angle $\theta$.) 


\item One last question. Find equations in Cartesian coordinates of 

\begin{enumerate}
\item the unit circle centered at the origin and 

\item The circle of radius $3$ centered at  $C(5,4)$.

These equations express relationships between the $x$ and $y$ coordinates of points on the circles. They do not involve the parameter $\theta$. 

\end{enumerate}
\end{enumerate}

\end{explanation}
\end{question}


\begin{question}  \label{QDF09555dfdfLKDD}
\begin{enumerate}

\item Parametrize the circle in Question 1 by the signed arclength from $A$ of a point $P(x,y)$ on the circle. This means to express the coordinates $(x,y)$ of $P$ (measured in cm) in terms of its signed distance from $A$. Measure the signed arclength $s$ (in cm)  from the point $A(8,0)$ along the circle and take the counterclockwise direction to be positive. 

\begin{onlineOnly}
    \begin{center}
\desmos{llrzcv5ckf}{900}{600}
\end{center}
\end{onlineOnly}

\href{https://www.desmos.com/calculator/llrzcv5ckf}{142: Circle Arclength Parameterization}



The coordinate functions (in cm) are 
\[
  x = f_1(s) = \answer{5 + 3 \cos(s/3)} \, , \, s\in \mathbb{R}
\]
and
\[
   y = g_1(s) = \answer{4 + 3 \sin(s/3)} \, , \, s\in \mathbb{R} .
\]

\item Use part (a) to find the exact coordinates of the point on the circle  $20$ cm from $A$ as measured counterclockwise along the circle. 

\item Use the worksheet above to approximate these coordinates and check if your coordinates look reasonable.

\item Between noon and 12:02pm a beetle crawls counterclockwise around the circle at a constant speed of $6$cm/sec, passing the point $A(8,0)$ at $23$ seconds past noon. Find coordinate functions
\[
   x = f_2(t)  \hskip 0.5 in \text{and} \hskip 0.5in  y=g_2(t) ,
\] 
that express the beetle's coordinates (in cm) in terms of the number of seconds past noon.

\end{enumerate}
\end{question}



\begin{question}  \label{QDfg554egsE}
A beetle starts from the point $A(8,3)$, crawls $s$ meters counterclockwise around the circle centered at $Q(2,3)$ and stops at point $B$. All coordinates measured in meters.

\begin{enumerate}
\item Express the components of the vector $\overrightarrow{QB}$ in terms of $s$.

\item Express the vector $\overrightarrow{OB}$ in terms of the vectors $\overrightarrow{OQ}$ and $\overrightarrow{QB}$, where $O$ is the origin.

\item Express the coordinates of $B$ in terms of $s$.

\item Find the exact coordinates of $B$ when $s=10$ meters. Then use a calculator to approximate these coordinates. Finally, use the worksheet below to check that your answer is reasonable.
\end{enumerate}
\end{question}



\section{Vector Arithmetic}

\begin{question} \label{QLfderr33r3}

\begin{onlineOnly}
    \begin{center}
\desmos{jg9co800dy}{900}{600}
\end{center}
\end{onlineOnly}

\href{https://www.desmos.com/calculator/jg9co800dy}{142: Vector Arithmetic}

\begin{enumerate}
\item Given the vectors ${\bf u}$ and ${\bf v}$ above, drag point $Q$ to sketch the vectors

\begin{enumerate}
\item $2{\bf u}$

\item $-3{\bf v}$

\item ${\bf u} + {\bf v}$

\item $2{\bf u} + 3{\bf v}$

\item $2{\bf u} - 3{\bf v}$

\item $0.5{\bf u} + 0.5{\bf v}$

\item ${\bf  u} - {\bf v}$

\end{enumerate}

\item Express vector ${\bf r}$ in terms of ${\bf u}$ and ${\bf v}$.

\end{enumerate}
\end{question}


\begin{question} \label{QLf45445rr33r3}

The vectors ${\bf u}$ and ${\bf v}$ below are perpendicular and have the same length.

\begin{onlineOnly}
    \begin{center}
\desmos{prtepbs3ul}{900}{600}
\end{center}
\end{onlineOnly}

\href{https://www.desmos.com/calculator/prtepbs3ul}{142: Vector Arithmetic 2}


\begin{enumerate}
\item Express vector ${\bf q}$ in terms of ${\bf u}$ and ${\bf v}$. 

\item Use the figure to approximate the values of $\cos\theta$ and $\sin\theta$ for the marked angle $\theta$ from ${\bf u}$ to ${\bf q}$. How are these values related to your answer in part (a)?

\item Turn off the folder \emph{Vector $q$} in Line 1. Then activate the folder \emph{Vector $z$} in Line 5 and repeat parts (a) and (b).

\item Turn off the folder \emph{Vector $z$} in Line 5. Then activate the folder \emph{Vector $j$} in Line 9 and repeat parts (a) and (b).


\item Turn off the Folder \emph{Vectors $j$} in Line 9 and activate the folder \emph{General Vector} in Line 11. Then express vector ${\bf p}$ in terms of the vectors ${\bf u}$ and ${\bf v}$ and the angle $\theta$. Do \emph{not} read off any numbers from the worksheet. Work in general.

\end{enumerate}
\end{question}


\section{General Starting Points}

\begin{question} \label{QIDdfdseIFDfeD}
A beetle starts at point $A(3,5)$ (coordinates in cm), crawls $s$ meters counterclockwise around the circle through $A$ centered at the origin $O$ and stops at $P$.

\begin{onlineOnly}
    \begin{center}
\desmos{ufyt7kkxc0}{900}{600}
\end{center}
\end{onlineOnly}

\href{https://www.desmos.com/calculator/ufyt7kkxc0}{142: Vectors and Circular Motion 1}


\begin{enumerate}
\item Express the vector $\overrightarrow{OP}$ in terms of $s$. Do this as follows.

\begin{enumerate}
\item Find the components of the vector $\overrightarrow{OA}$ from the origin to $A$.

\item Find the coordinates of the point $B$ on the circle $\pi/2$ radians counterclockwise from $A$. Then find the components of the vector $\overrightarrow{OB}$.

\item Express the radian measure of the angle $\theta = \angle AOP$ in terms of $s$.
\[
   \theta = \answer{s/\sqrt{34}} .
\]

\item Express the vector $\overrightarrow{OP}$ in terms $s$ and the vectors $\overrightarrow{OA}$ and $\overrightarrow{OB}$.
\[
      \overrightarrow{OP} = \answer{\cos (s/\sqrt{34})} \overrightarrow{OA} + \answer{\sin (s/\sqrt{34})} \overrightarrow{OB} . 
\]

\end{enumerate}

\item Use the result of the previous part to find functions
\[
 x = f(s)  \text{ and } y=g(s)
\]
that express the coordinates of $P$ in terms of $s$.

The coordinate functions are 
\[
    x = f(t) = \answer{3\cos(s/\sqrt{34}) - 5\sin(s/\sqrt{34})}
\]
and
\[
    y = g(t) = \answer{5\cos(s/\sqrt{34}) + 3\sin(s/\sqrt{34})}.
\]

\item Enter the expressions for the coordinate functions $f$ and $g$ in Lines 34 and 35 of the worksheet above as a check.

\item Find the exact (without a calculator) and approximate coordinates of the point on the circle $20$cm from point $A$. The $20$cm is measured counterclockwise along the circle.



\end{enumerate}
\end{question}




\end{document}