\documentclass{ximera}
\title{Simple Harmonic Motion, Part 2}

\newcommand{\pskip}{\vskip 0.1 in}

\begin{document}
\begin{abstract}
Parameterizing motion around a circle at a constant speed.
\end{abstract}
\maketitle



\begin{example}  \label{Exp89dfe94tf4}

Between 12:00pm and 1:00pm a beetle crawls counterclockwise around the circle of radius $30$ meters centered at the origin at a constant speed, making one revolution every $14$ minutes.  It leaves the point $(30,0)$ at noon.

\begin{enumerate}

\item At what rate does the beetle rotate about the origin?

\item Through what angle does the beetle turn about the origin between 12:00pm and 12:34pm?

\item Find a possible value for the beetle's polar angle at 12:34pm. This is the angle the segment $\overline{OB}$ from the origin to the beetle makes with the postive $x$-axis.

\item Drag point $B$ in the worksheet below to the beetle's approximate position at 12:34pm. Do this by finding another possible polar angle for the beetle at this time, one measured between $0$ and $2\pi$.

\item Use part (c) to find the beetle's exact coordinates at 12:34pm \emph{without} using a calculator.

\item Use a calculator to approximate the beetle's coordinates at 12:34pm. Use the worksheet below to check that these coordinates make sense.



\begin{onlineOnly}
    \begin{center}
\desmos{qjrantrk6e}{900}{600}
\end{center}
\end{onlineOnly}

\href{https://www.desmos.com/calculator/qjrantrk6e}{142: Radian Protractor 2D}



\end{enumerate}

\end{example}


\begin{example} \label{Edfhg]hghgdfg}
A beetle crawls counterclockwise around the circle of radius $40$ meters centered at the origin at a constant speed of $8$ meters/min, starting from the point $(40,0)$ (coordinates in meters) at noon and making $5$ complete revolutions before stopping.

\begin{enumerate}

\item Find a function
\[
    \theta = b(t)
\]
that expresses the beetle's polar angle in terms of the number of minutes past noon. Include the appropriate domain. Start by sketching a graph of this function.

The polar angle function is
\[
   \theta = b(t) = \answer{t/5} \, , \, 0 \leq t \leq \answer{50\pi} .
\]

\item At what rate does the beetle turn about the origin?  $\answer{1/5}$ rad/min



\item Find functions
\[
   x = f(t) \text{ and } y=g(t)
\]
that express the beetle's coordinates (in meters)  in terms of the number of minutes past noon. Include the appropriate domains.

The coordinate functions are
\[
    x = f(t) = \answer{40\cos(\frac{1}{5}t)} \, , \, 0 \leq t \leq \answer{50\pi} 
\]
and 
\[
    y = g(t) = \answer{40\sin(\frac{1}{5}t)} \, , \, 0 \leq t \leq \answer{50\pi} .
\]

\item Enter these coordinate functions on Lines 3 and 4 of the worksheet below. Then play the slider $u$ (the number of minutes past noon) to check if they are correct.

\begin{onlineOnly}
    \begin{center}
\desmos{7uewzebyjt}{900}{600}
\end{center}
\end{onlineOnly}

\href{https://www.desmos.com/calculator/7uewzebyjt}{142: Beetle 11}

\item Use your functions from part (c) to find the beetle's exact coordinates 35 minutes after it starts to crawl. Then use a calculator to approximate these coordinates. Use the animation to check that these coordinates are reasonable.

\end{enumerate}
\end{example}


\begin{example} \label{ExODferre3}
A mass released from rest at position $x=20$ cm oscillates in simple harmonic motion about the origin with a period of $5$ seconds.

\begin{onlineOnly}
    \begin{center}
\desmos{hunfxq5fuz}{900}{600}
\end{center}
\end{onlineOnly}

\href{https://www.desmos.com/calculator/dcbahunfxq5fuz}{151: Simple Harmonic Motion 24}

\begin{enumerate}

\item Describe the uniform circular motion that drives the oscillation by stating
\begin{enumerate}
\item the center and radius of the circle, and

\item the rotation rate of the uniform circular motion. 
\end{enumerate}


\item Now let $P$ be the point moving around a circle that drives the oscillation. Find a function
\[
    \theta = a(t)
\]
that expresses the polar angle of $P$ in terms of the number of seconds since the mass was released. Start by graphing this function. 

\item Use your polar angle function to find a function 
\[
   x= f(t)
\]
that expresses the position of the oscillating mass in terms of the number of seconds since being released.

\item Use the animation above to approximate  the position (ie. $x$-coordinate) of the mass $1.75$ seconds after being released.

\item Find the exact position (ie. $x$-coordinate) of the mass $1.75$ seconds after being released. Then use a calculator to approximate this position to the nearest hundredth of a centimeter.

\item Use the animation to approximate the position of the mass $33.5$ seconds after being released. 

\item Find the exact position (ie. $x$-coordinate) of the mass $33.5$ seconds after being released. Then use a calculator to approximate this position to the nearest hundredth of a centimeter.

\item How would your answers to the questions above have changed if instead the mass were released from the point $(0,10)$?

\end{enumerate}

\end{example}


\begin{example} \label{ExLKDF3rerdf}
Between times $t=0$ and $t=60$ seconds a mass oscillates in simple harmonic motion along the $y$-axis between $y=-20$cm and $y=20$cm. It has position $y=20$ at time $t=3$ seconds and reaches the origin for the first time after that when $t=3.2$ seconds. 

\begin{enumerate}

\item Describe the uniform circular motion that drives the oscillation by stating
\begin{enumerate}
\item the center and radius of the circle, and

\item the rotation rate of the uniform circular motion. 
\end{enumerate}


\item Now let $P$ be the point moving around a circle that generates the oscillation. Find a function
\[
    \theta = a(t)
\]
that expresses the polar angle of $P$ in terms of the number of seconds since the mass was released. Start by graphing this function. 

\item Use your polar angle function to find a function 
\[
   y= f(t)
\]
that expresses the position of the oscillating mass in terms of the number of seconds since being released.

%\item Use the animation above to approximate  the position (ie. $y$-coordinate) of the mass $4.1$ seconds after being released.

\item Find the exact position (ie. $y$-coordinate) of the mass $1.75$ seconds after being released. Then use a calculator to approximate this position to the nearest hundredth of a centimeter.

%\item Use the animation to approximate the position of the mass $33.5$ seconds after being released. 

\item Find the exact position (ie. $y$-coordinate) of the mass $11.5$ seconds after being released. Then use a calculator to approximate this position to the nearest hundredth of a centimeter.

\end{enumerate}
\end{example}

\begin{example} \label{ExLKDr45533}
Between times $t=0$ and $t=12$ seconds a mass oscillates in simple harmonic motion along the $s$-axis between positions $s=-6$ m and $s=6$ m. with a period of four seconds. At time $t=7$ seconds the mass has position $s=3$ and is moving toward its equilibrium position $s=0$. 

Find a function
\[
     s = f(t) \, , \, 0\leq t \leq 12,
\]
that expresses the position of the mass (in meters) in terms of the number of seconds since time $t=0$ seconds.

Start by sketching a possible graph of the polar angle function.
\end{example}

\begin{example} \label{ExLKDrefr33}
Between times $t=0$ and $t=12$ seconds a mass oscillates in simple harmonic motion along the $s$-axis between positions $s=-2$ m and $s=12$ m with a period of five seconds. The mass has position $s=-2$ at time $t=3$ seconds. %It takes five seconds for the mass to make one oscillation. 

Find a function
\[
     s = f(t) \, , \, 0\leq t \leq 12,
\]
that expresses the position of the mass (in meters) in terms of the number of seconds since time $t=0$ seconds.
\end{example}


\end{document}
