\documentclass{ximera}
\title{Simple Harmonic Motion}

\newcommand{\pskip}{\vskip 0.1 in}

\begin{document}
\begin{abstract}
Simple harmonic motion, an introduction.
\end{abstract}
\maketitle

\section{Mass on a Spring}

We'll start the class with one of its most important topics, simple harmonic motion. The classic example starts with a mass attached to the end of a spring. Stretch the spring beyond its relaxed length and relase the mass. Then then mass, assumed free to slide without friction on a horizontal surface, oscillates between its extremes in emphasis \emph{simple harmonic motion}.

Such oscillatory motion is common and one primary aim of this class is to describe the motion mathematially. While we do not yet have the tools to do this, we can still describe the motion geometrically. And it's this geometric description as illustrated in the following example that forms the basis not only for simple harmonic motion but for much of this course.

{\bf Note:} If you already had trigonometry, forget what you know for now. In particular, forget about the sine and cosine functions. The purpose of these examples is to think geometrically, not algebraically.



\begin{example}  \label{Ex:LLL}
A mass at the end of a spring oscillates in simple harmonic motion on a frictionless surface as shown below. The spring is stretched two meters from beyond its relaxed (ie. unstretched)(length and then released from rest, completing one oscillation about the origin every $5$ seconds.

\begin{onlineOnly}
    \begin{center}
\desmos{dcba538898}{900}{600}
\end{center}
\end{onlineOnly}

\href{https://www.desmos.com/calculator/dcba538898}{151: Simple Harmonic Motion}

\begin{enumerate}
\item Is the speed of the mass constant? If not, describe how it varies.

\item What is the spring's relaxed length?

\item Activate the folder \emph{Circle} in Line 2 above. Describe what you see. In particular,
\begin{enumerate}
\item Describe the relationship between point $P$ and the mass.

\item Describe the motion of $P$ around the circle. Does its speed appear to be constant?

\item Explain what it means for a mass to oscillate in simple harmonic motion. 
\end{enumerate}

\item Use the animation above to apprimate the \emph{displacement} of the mass from the origin at times $t=0, 1, 2, 3, 4$ seconds. The displacement (relative position) is positive when the mass is to the right of the origin, negative to the left.

\item Use the animation to approximate the displacement from the origin at time $t=26.5$.

\item Use the animation to approximate the first four times when the mass is $3$ meters to the left of the origin.

\item Use the animation to approximate when the mass is $3$ meters to the left of the origin for the $103$rd time.
\end{enumerate}
\end{example}

The key takeaway from the previous example is that \emph{the projection of constant speed motion around a circle gives simple harmonic motion along a diameter of that circle}.  

The next example takes the first step toward a mathematical description of simple harmonic motion by replacing an animation with a protractor.

\begin{example}  \label{Ex:PDoEr3rdfsx}
A mass oscillates along the the $x$-axis in simple harmonic motion  about the origin  with a period of $3$ seconds and an amplitude of $10$ cm. The mass is released from rest at time $t=0$ seconds from position $x=10$ cm.

\begin{onlineOnly}
    \begin{center}
\desmos{rqdenajviu}{900}{600}
\end{center}
\end{onlineOnly}

\href{https://www.desmos.com/calculator/rqdenajviu}{142: Mass on Spring 2 }

\begin{enumerate}

\item Use \emph{only} the slider $v$ in Line 1 in the worksheet above and simple arithmetic (\emph{no trig functions})  to answer the following questions.

\begin{enumerate}
\item What is the approximate postion of the mass at time $t=1.25$ seconds after being released?

\item What is the aproximate position of the mass $t=13.8$ seconds after being released?

\item When is the mass $8$cm to the right of the origin? Approximate the first two times. Then approximate the 25th time.

\item When is the mass  at position $x = -3$? Approximate the first four times.  
\end{enumerate}

\item Use the protractor above to answer the questions in part (a) if instead the mass oscillates about the origin with an amplitude of $20$ cm and a period of $3$ seconds. \emph{Do not change the protractor's radius.}

%\item Use only the relevant information from your solution to Question (c) of Example 1 to answer part (ii) above. \emph{Hint:} Think proportionally.

%\item Use only the information from your solution to Question (e) of Example 1 to answer part (iv) above.

\end{enumerate}
\end{example}


\begin{example}  \label{Ex:FEfe344trgfd}

Suppose for this problem that the earth is a ball with uniform density of radius 4000 miles. Now imagine drilling a straight tunnel through the earth from the North Pole to the South Pole. A rock dropped from rest at the north pole falling through the tunnel would then oscillate in simple harmonic motion between the poles and return to the north pole every 84  minutes as illustrated below. 

\begin{onlineOnly}
    \begin{center}
\desmos{z5ya8iy2e4}{900}{600}
\end{center}
\end{onlineOnly}

\href{https://www.desmos.com/calculator/z5ya8iy2e4}{142: Tunnel Throught the Earth }


\emph{Use only the protractor in Example 2} (without changing its radius) and four-function arithmetic to answer the following questions. Activate the \emph{Protractor} folder in Line 15 of the worksheet above \emph{only if you get stuck.} 
\begin{enumerate}
\item Estimate the distance of the rock from the South Pole $16.8$ minutes after the rock is released.

\item Estimate the first two times when the rock is $1000$ miles from the South Pole.

\item During one oscillation, what fraction of the time is the rock at least $1200$ miles from the South Pole?
\end{enumerate}
\end{example}

\begin{example} \label{Ex:8dferGGDEE}
A mass oscillates in simple harmonic motion along the $x$-axis between positions $x=-3$ meter and $x=11$ meters with a period of $8$ seconds. It is released from rest at $x=11$ at time $t=0$ seconds.

\emph{Use only the protractor in Example 2} (without changing its radius) and four-function arithmetic to answer the following questions.

\begin{enumerate}
\item Approximate the position of the mass at time $t=38$ seconds.

\item Approximate the first four times when the mass passes $x=0$.
\end{enumerate}

\end{example}




\end{document}
