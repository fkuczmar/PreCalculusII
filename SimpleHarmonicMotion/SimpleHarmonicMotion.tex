\documentclass{ximera}
\title{Simple Harmonic Motion}

\newcommand{\pskip}{\vskip 0.1 in}

\begin{document}
\begin{abstract}
Simple harmonic motion, an introduction.
\end{abstract}
\maketitle

\section{Circular Interpolation with a Protractor}

\begin{example}  \label{Ex:LLL}
We just completed the first third of spring and you've probably noticed how many more hours of daylight we get than we did than just a few months ago. On the vernal equinox (the first day of spring, usually March 21), Seattle gets 12 hours of daylight/day. And on the summer solstice (the first day of summer) we'll get just over 16 hours of daylight/day. 

The question is this. On April 21st, one-third of the way into spring, about how many hours of daylight per day do we get? Or more to the point, thinking proportionally and asking a question that would have the almost same answer for all latitudes, when we're one-third of the way through spring, through what fraction of the way from 12 hours/day to the maximum number of hours of daylight/day are we? One-third? Halfway? Two-thirds? What do you think?

The simplest way to approach this problem would be with a linear interpolation. Then when we're one-third of the way through spring in Seattle, we would be 1/3 of the way from $12$ to $16$ hours of daylight/day. So on April 21st we would expect to get about
\[
    12 \text{ hrs/day} + \frac{1}{3}(4 \text{ hrs/day}) = 13 \frac{1}{3}\text{ hrs/day}
\]
hours of daylight/day as illustrated in the graph below.


\begin{onlineOnly}
    \begin{center}
\desmos{dcba538898}{900}{600}
\end{center}
\end{onlineOnly}

\href{https://www.desmos.com/calculator/dcba538898}{151: Simple Harmonic Motion}

\end{example}

\end{document}
