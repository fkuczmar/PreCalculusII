\documentclass{ximera}
\title{Simple Harmonic Motion}

\newcommand{\pskip}{\vskip 0.1 in}

\begin{document}
\begin{abstract}
Simple harmonic motion, an introduction.
\end{abstract}
\maketitle

\section{An Oscillating Mass}

We'll start the class with one of its most important topics, simple harmonic motion. The classic example starts with a mass attached to the end of a spring. Stretch the spring beyond its relaxed length and relase the mass. Then then mass, assumed free to slide without friction on a horizontal surface, oscillates between its expremes in emphasis \emph{simple harmonic motion}.

Such oscillatory motion is common and one primary aim of this class is to describe this type of motion mathematially. While we do not yet have the tools to do this, we can describe the motion geometrically. And it's this geometric description as illustrated in the following example that forms the basis not only for simple harmonic motion but for much of this course.



\begin{example}  \label{Ex:LLL}
A mass at the end of a spring oscillates in simple harmonic motion on a frictionless surface as shown below. The spring is stretched two meters from beyond its relaxed (ie. unstretched)(length and then released from rest, completing one oscillation about the origin every $5$ seconds.

\begin{onlineOnly}
    \begin{center}
\desmos{dcba538898}{900}{600}
\end{center}
\end{onlineOnly}

\href{https://www.desmos.com/calculator/dcba538898}{151: Simple Harmonic Motion}

\begin{enumerate}
\item What is the spring's relaxed length?

\item Activate the folder \emph{Circle} in Line 2 above. Describe what you see. In particular,
\begin{enumerate}
\item Describe the relationship between point $P$ and the mass.

\item Describe the motion of $P$ around the circle.

\item Explain what it means for a mass to oscillate in simple harmonic motion. 
\end{enumerate}

\item Use the animation above to apprimate the \emph{displacement} of the mass from the origin at times $t=0, 1, 2, 3, 4$ seconds. The displacement (relative position) is positive when the mass is to the right of the origin, negative to the left.

\item Use the animation to approximate the displacement from the origin at time $t=26.5$.

\item Use the animation to approximate the first four times when the mass is $4$ meters to the left of the origin.

\item Use the animation to approximate when the mass is $4$ meters to the left of the origin for the $103$rd time.
\end{enumerate}

\end{example}

\end{document}
