\documentclass{ximera}
\title{Circular Interpolation, Part 2}

\newcommand{\pskip}{\vskip 0.1 in}

\begin{document}
\begin{abstract}
Circular Interpolation
\end{abstract}
\maketitle


\section{Circular Interpolation with a Protractor}



\begin{example}  \label{E888GGGbdsdsf}
Suppose that over the course of a 24-hour period, from midnight October 29 to midnight October 30, the depth of the water at the Edmonds Pier is a sinusoidal function of time. Suppose also that a high tide of 21 feet occurs at 2:00am and the following low tide of 5 feet occurs at 8:00am. 

For the following four questions, forget all that you've learned in this class.  Do \emph{not} use any trigonometry.

\begin{enumerate}
\item Use linear interpolation to approximate the depth of the water at 6:00am on October 29.

\item Use circular interpolation with the protractor below to approximate the depth of the water at 6:00am on October 29.

\item Use linear interpolation to approximate the first two times after midnight when the water is $8$ feet deep.

\item Use the circular protractor to approximate the first two times after midnight when the water is $8$ feet deep.

Explain your reasoning and include screenshots to help with your explanations. For parts (b) and (d) the key idea is to model the oscillation of the tides with uniform circular motion. \emph{Keep in mind that high tide occurs at 2:00am and low tide at 8:00am.}% \emph{without} using your knowledge of trigonometry.


\begin{freeResponse}
\end{freeResponse}

\begin{onlineOnly}
    \begin{center}
\desmos{0wxwmkzvky}{900}{600}
\end{center}
\end{onlineOnly}

\href{https://www.desmos.com/calculator/0wxwmkzvky}{142: Circular Interpolation Tides}



\end{enumerate}
\end{example}









\section{Circular Interpolation using the Cosine Function}

\begin{example} \label{EODGrgrgrFDSFSD}


This is a continuation of Example 1.

Our aim is to find a sinusoidal function
\[
    h = f(t) , 0\leq t \leq 24, 
\]
that expresses the depth of the water (in feet) in terms of the number of hours past midnight, October 29. As before we suppose that over the course of a 24-hour period, from midnight October 29 to midnight October 30, the depth of the water at the Edmonds Pier is a sinusoidal function of time. And also that a high tide of 21 feet occurs at 2:00am and the following low tide of 5 feet occurs at 8:00am. 

\pskip

Note that to say \emph{sinusoidal function}, an unfortunate name, means that the graph of $f$ is generated by uniform circular motion. But the graphs of the sine and cosine functions are both generated this way, so it is ok to express $f(t)$ in terms of the cosine function, and you should do just that.

\begin{explanation}
Here are the steps.

\begin{enumerate}

\item  First we'll use the information above to sketch by hand a graph of the function $f$. Label the axes with the appropriate variable names and units. Label the coordinates of two key points on the graph.

Desmos activity available at:

\href{https://www.desmos.com/calculator/cxybepr3od}{142: Edmonds Pier 2}.

 
\begin{onlineOnly}
    \begin{center}
\desmos{cxybepr3od}{900}{600}
\end{center}
\end{onlineOnly}

\item To find an expression for the function $h=f(t)$ do the following.

\begin{enumerate}
\item Find the mean depth of the water over a period of oscillation and then find the maximum deviation of the depth from this mean.

\item Express the depth of the water in terms of the marked polar angle $\theta$ (measured counterclockwise from the upward vertical).

\item Express the polar angle $\theta$ as a function of the number of hours past midnight, October 29. Draw a graph of this function to help with your explanation

\item Use parts (i) and (ii) to find an expression for the function $h=f(t)$.
\end{enumerate}


\item Check that your function is correct by using the given information that the depth of the water is $21$ feet at 2:00am and $5$ feet at 8:00am.


\item Use your function to estimate the depth of the water at 6:00am, October 29.

\item Enter the coordinates of the appropriate point in Line 27 of the Desmos Activity above to check that the depth at 6:00am is reasonable.

\item Use your function to determine the first two exact times when the water is $8$ feet deep. Then use a calclulator to approximate these times to the nearest minute. Then use the demonstration above to check your work.

\end{enumerate}

\end{explanation}

\end{example}


\begin{example} \label{EXPODFefeRE}
The graph below shows the actual variation of the tides at the Edmonds pier for Sunday, June 1, 2025.


Desmos activity available at:

\href{https://www.desmos.com/calculator/kbjvcy5bs1}{142: Edmonds Tide Graph Real}.

 
\begin{onlineOnly}
    \begin{center}
\desmos{kbjvcy5bs1}{900}{600}
\end{center}
\end{onlineOnly}

\begin{enumerate}
\item Use the graph to find a sinusoidal function $h=f(t)$ that models the depth of the water (in feet) between 9am and 3:30pm, where $t$ is the number of hours past noon. Use the cosine function.

\item Input your expression for $f(t)$ on Line 7 above.

\item Use the graph above and  your graph to approximate the depth of the water at noon.

\item Use your function to predict the depth of the water at noon. Compare your prediction with the graphs.

\item Use the graph above and  your graph to approximate when the water is two feet deep.


\item Use your function to find the exact times when the water is two feet deep. Then use a calculator to approximate these times.  Compare these with your estimates in part (e).

\item Use your function to predict the depth of the water at 11:00pm. Compare this prediction with the actual depth.
\end{enumerate}



\end{example}



\section{Exercises}
\begin{exercise} \label{ExPOfdgryreer}

Suppose that over the course of a $24$-hour period in Seattle, from midnight July 12 to midnight July 13, the temperature is a sinusoidal function of time. Suppose also that the minimum temperature of $50^\circ$F occurs at 5am and the maximum temperature of $80^\circ$F occurs at 5pm. 

\begin{enumerate}
\item Use the protractor below and arithmetic to find approximate answers the following questions. Do \emph{not} use any trigonometry.

\begin{enumerate}
\item Find the temperature at noon.

\item When is the temperature $64^\circ$F?
\end{enumerate}


\begin{onlineOnly}
    \begin{center}
\desmos{0bywoftlld}{900}{600}
\end{center}
\end{onlineOnly}

\href{https://www.desmos.com/calculator/0bywoftlld}{142: Circular Interpolation 3}


\item Find a function
\[
 T = f(h) \, , \, 0\leq h \leq 24,
\]
that expresses the temperature (in Fahrenheit degrees) in terms of the number of hours since midnight, July 12. {\bf Use the cosine function, not the sine function.}

\item Use your function to approximate the temperature at noon.

\item Use your function to determine the exact times when the temperature is $64^\circ$F. Then use a calculator to approximate these times to the nearest minute.

\end{enumerate}
\end{exercise}



\begin{exercise} \label{EPODRerDFGe}
Assume for this problem that each month has 30 days. Assume also that on the summer solstice (the first day of summer) Shoreline gets about 16 hours of daylight/day. And on the winter solsctice (December 21) Shoreline gets about $8$ hours of daylight/day.


\begin{enumerate}
\item Use the protractor below and arithmetic to find approximate answers the following questions. Do \emph{not} use any trigonometry.

\begin{enumerate}
\item How many hours of daylight/day do we get on November 21st?

\item On what days of the year do we get $14$ hours of daylight/day. Thirteen hours of daylight/day?
\end{enumerate}

\begin{onlineOnly}
    \begin{center}
\desmos{nfsifbppzz}{900}{600}
\end{center}
\end{onlineOnly}

\href{https://www.desmos.com/calculator/nfsifbppzz}{142: Circular Interpolation}



\item Make an assumption and find a function $F=f(t)$, $0\leq t \leq 12$, that expresses the number of hours of daylight/day in terms of the number of months since the summer solstice. Be sure to state your assumption.

\item Use your function to questions  (i) and (ii) above.

\end{enumerate}
\end{exercise}



\section{Discussion Questions}
\begin{question} \label{QLDfFFdRE}

\begin{enumerate}

\item Parameterize the circle of radius $7$ meters centered at the point $C(4,-4)$ in terms of the polar angle of the vector $\overrightarrow{CP}$ from the center $C$ to a point $P$ on the circle.

This means to find functions
\[
    x = f(\theta) \hskip 0.5 in \text{and} \hskip 0.5 in y=g(\theta)
\]
that express the coordinates of $P$ in terms of the polar angle. Include an appropriate domain for these functions.

\item Use the geometry of the circle to sketch by hand a graph of the $x$-coordinate function $x=f(\theta)$.

\item The animation below shows a motion around this circle. Watch the animation by playing the slider $u$ (another name for time $t$)

\begin{onlineOnly}
    \begin{center}
\desmos{esr4pacqae}{900}{600}
\end{center}
\end{onlineOnly}

\href{https://www.desmos.com/calculator/esr4pacqae}{142: Watch this motion}

\begin{enumerate}
\item Use the animation to graph by hand the function
\[
  h = k(t) \, , \, 0\leq t \leq 24,
\]
that expresses the $x$-coordinate of the beetle (in meters) in terms of the number of seconds since the motion began. If you need help activate the folders in Lines 3 and 9.

\item Activate the folder in Line 17 to see how you did.

\item Find an expression for the function $h=k(t)$. Do this by first finding a function 
\[
 \theta = a(t), 0\leq t \leq 24, 
\]
that expresses the polar angle (in radians) of the vector $\overrightarrow{CP}$ in terms of $t$.



\end{enumerate}
\end{enumerate}


\end{question} 



\end{document}