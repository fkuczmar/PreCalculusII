\documentclass{ximera}
\title{Measuring Angles}

\newcommand{\pskip}{\vskip 0.1 in}

\begin{document}
\begin{abstract}
Measuring angles in radians.
\end{abstract}
\maketitle

The defintion of a meter is arbitrary and if we lived on a flat world there would be no natural unit of length. But on a perfect sphere, the radius serves as a natural length. A distance equal to the radius measured along the equator of a planet would cover the same fraction of the equator no matter the radius.

The \emph{nautical mile} was originally defined as the distance along the earth's equator subtended by an angle of $1/60$ degree with its vertex at the earth's center. It is an example of a naturally defined length on an earth assumed to be a perfect sphere.

Just like the meter, the definition of what it means for an angle to have a measure of one degree ($1^\circ$) is arbitrary. It means that an angle measuring $1^\circ$ with its vertex at the center of any circle, cuts out an arclength equal to $1/360$ of the circle's circumference. Why $1/360$? Why are there 360 degrees in one revolution? We don't really know.

\begin{question}  \label{Q0:Radians}
A lost penguin swims 786 miles due south from the north pole on a perfectly spherical earth assumed to have a radius of $3960$ miles. A disoriented ostrich walks $2871$ miles due south from the north pole on Planet X, assumed to have a radius of $14,500$ miles. Which animal ends up at the greater latitude on their respective planet? Explain your reasoning.

Note: Latitude is measured as an angle between the equatorial plane and a spherical radius. For example, Shoreline is at a latitude of about $47.75^\circ$ North of the equator. This means that the radius of earth drawn to Shoreline makes an angle of  $47.75^\circ$ with the equatorial plane. But you need not think about degrees to answer this question.
\end{question}

A less arbitrary way to measure angles uses \emph{radians}. The word comes from radius and an angle of 1 radian with its vertex at the center of a circle subtends (cuts out) an arclength equal to the circle's radius. An angle with a measure of 2 radians similarly placed subtends an arclength equal to twice the circle's radius. 

Many formulas that use the radian measure of an angle are more simple than they would be if the angle were measured in degrees.

To measure an angle in radians, draw a circle of any radius centered at the angle's vertex. Then measure the arclength along the circumference subtended by the angle. Take this length and divide it by the radius (measured in the same units). The result gives the measure of the angle in radians. Being the ratio of two lengths, the radian measure is dimensionless.


\begin{question} \label{Q1:Radians}
What does it mean for an angle to have a measure of $4.5$ radians? Sketch such an angle directly, {\bf without} converting the radian measure to degrees.
\end{question}



\begin{exploration}  \label{Q111:Radians}
(a) Use the radian protractor below to measure the five exterior angles of pentagon $ABCDE$ below.

(b) Find the sum of the angles in part (a).

(c) Using your result from part (a) and without making any additional measurements, find the sum of the five interior angles of pentagon $ABCDE$.

(d) Move vertex $C$. Which of the exterior angles does this change? Use the protractor to measure these angles and then compute the sum of the five exterior angles.

(e) How do your sums from parts (b) and (d) compare? What do you think the exact sums are? Explain why.

(f) Suppose you walk counterclockwise one time around the pentagon. When you reach a vertex you change your direction by turning through the exterior ange at a constant rate of $2/5$ rad/sec. How much time do you spend turning during your walk? Explain. 

\pdfOnly{
Access Desmos interactives through the online version of this text at
 
\href{https://www.desmos.com/calculator/2dlgnpeqsm}.
}
 
\begin{onlineOnly}
    \begin{center}
\desmos{2dlgnpeqsm}{900}{600}
\end{center}
\end{onlineOnly}
\end{exploration}






\begin{exploration}\label{exp:angles1}
Suppose you stand in place at point $A$ and face point $B$ in the desmos activity below. 

(a) Now turn counterclockwise until you face $C$ for the first time. Use the radian protractor to approximate the angle through which you turn during this time. Show a screenshot.

(b) Now turn back to face $B$ again (sitll standing at $A$). Then turn clockwise and stop when you face $C$ for the fifth time. Through approximately what angle do you turn during this time? 

(c) Turn back to face $B$ again (still standing at $A$). Then turn counterclockwise at the constant rate of $0.45$ rad/sec until you face $C$ for the tenth time. Approximately how long does this take? 


\pdfOnly{
Access Desmos interactives through the online version of this text at
 
\href{https://www.desmos.com/calculator/mwh3wwfrqv}.
}
 
\begin{onlineOnly}
    \begin{center}
\desmos{mwh3wwfrqv}{900}{600}
\end{center}
\end{onlineOnly}
\end{exploration}







\begin{exploration}\label{exp:angles2}
Two wheels with the same radius rotate at the same rate. One turns about its center, the other rolls on a road without slipping.

(a) Compare the speeds of points $P$ and $C$ below. Explain your reasoning. Include a screenshot to help with your explanation.

(b) If the wheels turn at the constant rate of $0.45$ rad/s and the wheel's have a radius of 10 cm, find the speed of the center of the rolling wheel. Do not use a  formula. Instead, do this logically and explain your reasoning. 

(c) Turn on the folder in Line 19 by clicking on the circle at the left of the line and play the animation. Then sketch graphs (on the same coordinate system) that show the speeds of points $P^\prime$ and $C$ as functions of time. No need for scales on the axes, but pay particular attention to the ratios of the speeds in drawing  your graphs.


\pdfOnly{
Access Desmos interactives through the online version of this text at
 
\href{https://www.desmos.com/calculator/cwaasmghns}.
}
 
\begin{onlineOnly}
    \begin{center}
\desmos{cwaasmghns}{900}{600}
\end{center}
\end{onlineOnly}
\end{exploration}



\begin{exploration}\label{exp:angles2}
Experiment with the desmos activity below to determine an expression in $a$ and $b$ that gives the radian measure of the angle through which the rolling circle turns during the time it takes point $P$ to return to its intial position. Try to explain the logic behind the expression.


\pdfOnly{
Access Desmos interactives through the online version of this text at
 
\href{https://www.desmos.com/calculator/g8rjobapit}.
}
 
\begin{onlineOnly}
    \begin{center}
\desmos{g8rjobapit}{900}{600}
\end{center}
\end{onlineOnly}
\end{exploration}



\begin{example} \label{Ex1:Angles}
At noon an ant is in the second quadrant at the point on the circle $x^2+y^2=400$ (the coordinates $(x,y)$ measured in centimeters) that is 45 cm from $(20,0)$, as measured along the circle. It crawls counterclockwise around the circle at a constant speed of $5$ cm/sec.

At noon a beetle is in the first quardrant on the same circle at a point that is 10 cm from $(20,0)$, this distance also measured along the circle. It crawls clockwise around the circle at a constant speed of $8$ cm/sec.

The insects crawl for 100 seconds.

\pskip

(a) Find a function
\[
    \theta = a(t) , 0\leq t \leq 100,
\]
that expresses the angle (in radians) from the postive $x$-axis to the segment $OA$ running from the origin to the ant in terms of the number of seconds past noon. Take the counterclockwise sense of rotation to correspond to a postive angle.

\pskip

(b) Find a function
\[
    \theta = b(t) , 0\leq t \leq 100,
\]
that expresses the angle (in radians) from the postive $x$-axis to the segment $OB$ running from the origin to the beetle in terms of the number of seconds past noon. Take the counterclockwise sense of rotation to correspond to a postive angle.

\pskip


(c) Use your functions from (a) and (b) to determine the first time the insects pass each other. Give an exact time, then an approximation to the nearest tenth of a second.

\pskip

(d) Use your functions from (a) and (b) to find an expression for the $n$th time the insects pass each other. 

\pskip

(e) Find a function $c(t)$ that counts the number of times the insects pass each other during the first $t$ seconds of their motions.

\pskip

(f) When is the last time the insects pass each other? Give an exact time, then an approximation to the nearest tenth of a second.

\begin{exploration}\label{exp:angles2}
Use the demonstration below to check your work.


\pdfOnly{
Access Desmos interactives through the online version of this text at
 
\href{https://www.desmos.com/calculator/btopul9ji9}.
}
 
\begin{onlineOnly}
    \begin{center}
\desmos{btopul9ji9}{900}{600}
\end{center}
\end{onlineOnly}
\end{exploration}


\end{example}


\begin{example} \label{Ex2:Angles}
Make up and solve your own version of Example 3. Use the exploration below to check your work, but not to help with the computations in any way. 


\begin{exploration}\label{exp:angles2}

\pdfOnly{
Access Desmos interactives through the online version of this text at
 
\href{https://www.desmos.com/calculator/632iayap2b}.
}
 
\begin{onlineOnly}
    \begin{center}
\desmos{632iayap2b}{900}{600}
\end{center}
\end{onlineOnly}
\end{exploration}


\end{example}


\end{document}
