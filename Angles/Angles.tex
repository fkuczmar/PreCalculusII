\documentclass{ximera}
\title{Measuring Angles}

\newcommand{\pskip}{\vskip 0.1 in}

\begin{document}
\begin{abstract}
Measuring angles in radians.
\end{abstract}
\maketitle

\section{Turning in Place}

The defintion of a meter is arbitrary and if we lived on a flat world there would be no natural unit of length. But on a perfect sphere, the radius serves as a natural length. A distance equal to the radius measured along the equator of a planet would cover the same fraction of the equator on all planets.

The \emph{nautical mile} was originally defined as the distance along the earth's equator subtended by an angle of $1/60$ degree with its vertex at the earth's center. It is an example of a naturally defined length on an earth assumed to be a perfect sphere.

Just like the meter, the definition of what it means for an angle to have a measure of one degree ($1^\circ$) is arbitrary. It means that an angle measuring $1^\circ$ with its vertex at the center of any circle, cuts out an arclength equal to $1/360$ of the circle's circumference. Why $1/360$? Why are there 360 degrees in one revolution? Perhaps because there are about 360 days in one year, but we don't know for sure.

To me it seems like the most natural way to measure an angle would be in revolutions. For example, you might stand in one place and turn through $4.5$ revolutions. This would be a lot clearer than saying you turned through
\[
   4.5 \text{ rev} = (4.5 \text{ rev} ) \left( \frac{360 \text{ degrees}}{1 \text{ rev}} \right) = 1620 \text{ degrees}.
\]

But it turns out that there is an even easier way to measure angles. 

\begin{question}   \label{Q00:Radians}
You walk $32$ meters around a circulular track with a radius of $50$ meters. Your friend walks $42$ meters around a circular track with radius $65$ meters. Who turns through the greater angle about the center of their respective track?

\begin{hint}
Think proportionally.
\end{hint}

\end{question}

Here's another version of this same type of question.


\begin{question}  \label{Q0:Radians}
A lost penguin swims 786 miles due south from the north pole on a perfectly spherical earth assumed to have a radius of $3960$ miles. A disoriented ostrich walks $2871$ miles due south from the north pole on Planet X, assumed to have a radius of $14,500$ miles. Which animal ends up at the greater latitude on their respective planet? Explain your reasoning.

\pskip

{\bf Note:} Latitude is measured as an angle between the equatorial plane and a spherical radius. For example, Shoreline is at a latitude of about $47.75^\circ$ North of the equator. This means that the radius of the earth drawn to Shoreline makes an angle of  $47.75^\circ$ with the equatorial plane. Or it migh be easier to think of the radius to Shoreline making an angle of $90^\circ - 47.75^\circ = 42.25^\circ$ with the radius to the north pole. %But you need not think about degrees to answer this question.
\end{question}

Question 1 suggests a natural way to compare turning angles around the centers of different circular tracks by comparing the ratios of the distances travelled along the tracks to the respective radii of the tracks. In that question, for you the ratio is
\[
    \theta_1 = \frac{\text{distance}}{\text{radius}} = \frac{32 \text{ meters}}{50 \text{ meters}} = 0.64.
\]
This means that you walked a distance equal to $0.64$ radii along the track. The ratio, being the quotient of two length, is dimensionless. It measures the angle you turned about the tracks center and we say the angle has a measure of $0.64$ \emph{radians}. This means that the arclength along the track subtended (cut out) by the angle is $64/100$ the radius of the track.  


So an angle measuring 1 \emph{radian} (from the word radius) placed with its vertex at the center of a circle subtends (cuts out) an arclength equal to the circle's radius. An angle with a measure of 2 radians similarly placed subtends an arclength equal to twice the circle's radius. 

To measure an angle in radians do the following. First draw a circle of any radius centered at the angle's vertex. Then measure the arclength along the circumference subtended by the angle. Finally take this arclength and divide it by the radius (measured in the same units). The result gives the measure of the angle in radians. Being the ratio of two lengths, the radian measure is dimensionless.

Many formulas that use the radian measure of an angle are more simple than they would be if the angle were measured in degrees. We'll see a few examples of this shortly and you'll see more next quarter in calculus.


\begin{question} \label{Q1:Radians}
What does it mean for an angle to have a measure of $4.5$ radians? Sketch such an angle directly, {\bf without} converting the radian measure to degrees.
\end{question}

\begin{question} \label{Q01:Radians}
(a) You run $725$ meters around a circlular track of radius $50$ meters. Through what angle did you turn about the track's center?
\[
    \answer{14.5} \text{ radians}
\]

(b) In running around a circular track of radius $50$ meters you turn through an angle of $21$ radians about its center. How far did you run?
\[
    \answer{1050} \text{ meters}
\]

(c) In running $1000$ meters around a circular track you turn through an angle of $16$ radians about its center. What is the track's radius?
\[
   \answer{62.5} \text{ meters}
\]

\end{question}



\begin{exploration}  \label{E33:Angles}
To convert between radians and revolutions, we just need to know how many radians are in one revolution.


%How many radians are in one revolution?


%Suppose you stand in place and turn through one revolution. Through how many radians did you turn?

To figure this out, imagine drawing a circle of say radius $r$ meters about yourself. Now stand in place and turn through one revolution as your gaze sweeps out the entire circle. To compute the number of radians in one revolution, divide the circumference of the circle by its radius. That dimensionless ratio gives you the radian measure of one revolution. It measures the number of radii that wrap exactly once around a circle's circumference.


\begin{question}
Drag the slider for $\phi$ in the demonstration below to wrap the string around the circle. About how many radii wrap around the circumference of a circle (measure to the nearest integer). 

There are a little more than $\answer{6}$ radians in one revolution.

\end{question}

It turns out that there are exactly $2\pi \sim 6.28$ radians in one revolution. This follows directly from the definintion of $\pi \sim 3.14$ as the dimensionless ratio of the circumference of a circle to its diameter. There are exactly $\pi$ radians in half a revolution and $2\pi$ radians in a complete revolution.


\pdfOnly{
Access Desmos interactives through the online version of this text at
 
\href{https://www.desmos.com/calculator/24sbrdmpjx}.
}
 
\begin{onlineOnly}
    \begin{center}
\desmos{24sbrdmpjx}{900}{600}
\end{center}
\end{onlineOnly}
\end{exploration}


\begin{question}  \label{Q43:Angles}
How many radians are in 4.5 revolutions?

\begin{explanation}
Since there are $2\pi$ radians in one revolution,
\[
   4 \text{ rev} = (4 \text{ rev})  \left( \frac{2\pi \text{ rad}}{1 \text{ rev}}  \right)  = 9\pi \text{ rad} .
\]
\end{explanation}

\end{question}




\begin{exploration}  \label{Q111:Radians}
(a) Use the radian protractor below to measure the five exterior angles of pentagon $ABCDE$ below.

(b) Find the sum of the angles in part (a).

(c) Using your result from part (a) and without making any additional measurements, find the sum of the five interior angles of pentagon $ABCDE$.

(d) Move vertex $C$. Which of the exterior angles change? Use the protractor to measure these angles and then compute the sum of the five exterior angles.

(e) How do your sums from parts (b) and (d) compare? What do you think the exact sums are? Explain why.

%(f) Suppose you walk counterclockwise one time around the pentagon. When you reach a vertex you change your direction by turning through the exterior angle there at a constant rate of $2/5$ rad/sec. How much time do you spend turning during your walk? Explain. 

\pdfOnly{
Access Desmos interactives through the online version of this text at
 
\href{https://www.desmos.com/calculator/2dlgnpeqsm}.
}
 
\begin{onlineOnly}
    \begin{center}
\desmos{2dlgnpeqsm}{900}{600}
\end{center}
\end{onlineOnly}
\end{exploration}





\begin{question} \label{Q999:Radians}
The area $A$ of a spherical triangle on a sphere of radius $r$ with angles $\alpha$, $\beta$, $\gamma$, measured in radians, is given by the formula
\[
 A = r^2 (\alpha+\beta+\gamma - \pi) .
\]

Click the link below to see an equilateral triangle on a sphere of radius $r$

\href{https://www.desmos.com/3d/91881c3088}{Spherical Triangle}

\pskip

(a) What is the approximate radian measure of the three equal angles in the triangle?

(b) What fraction of the sphere's surface area does the triangle cover?

(c) Use the formula above to find an expression for the area of triangle. Then use part (b) to find an expression for the surface area of a sphere.

(d) What are the lengths of the sides of the triangle?

(e) Use the formula $A=s^2\sqrt{3}/4$ for the area of an equilateral triangle in the plane to compute the area of such a triangle with the side length from part (d). How does the area of this planar triangle compare with the area of the spherical triangle?

(f) Experiment with the slider $\alpha$ in the desmos demonstration. Then find the area of a spherical triangle with angle measures $\alpha$, $\phi/2$, and $\pi/2$ (in radians). What fraction of the area of the hemisphere northern hemisphere does this triangle cover? Does this agree with what you see in the demonstration? Explain.

\end{question}


\section{Rotation Rates}
\begin{question}  \label{Q245: Angles}
Find the exact rotation rates of the following:

(a) the minute hand of a clock: $\answer{\frac{1}{60}} \text{ rev/min} = \answer{\frac{\pi}{30}}$ rad/min. 

(b) the hour hand of a clock: $\answer{\frac{1}{720}} \text{ rev/min} =\answer{\frac{\pi}{360}}$ rad/min. 

(c) the second hand of a clock: $\answer{1} \text{ rev/min} =\answer{2\pi}$ rad/min. 

(d) the earth about its axis: $\answer{\frac{1}{24}} \text{ rev/hour} =\answer{\frac{\pi}{12}}$ rad/hr.

\end{question}


\begin{question}   \label{Q54: Angles}
(a) Find an increasing function $\theta = a(t)$ that expresses the radian measure of the angle between the minute and hour hands of a clock in terms of the number of hours past noon. Measure the angle from the hour hand to the minute hand, taking the clockwise sense to be positive. %Note that for a function to be \emph{monotonic} means that it is always increasing or always decreasing.

(b) Use the clock below to estimate the radian measure of the \emph{acute} angle between the minute and hour hands at 12:53pm. An \emph{acute} angle has a measure between $0$ and $\pi/2$ radians.

(c) Use your function from part (a) to help find the exact radian measure of the \emph{acute} angle between the minute and hour hands at 12:53pm. Then find the approximate radian measure, correct to the nearest hundredth. Compare with your estimate from part (b).

(d) Use the clock below to estimate the first two times after 12:00pm when the minute and hour hands are perpendicular.

(e) Use your function from part (a) to help find the first two times after 12:00pm when the minute and hour hands of a clock are perpendicular. Round these times to the nearest second and compare them with your estimate from part (d).

\pdfOnly{
Access Desmos interactives through the online version of this text at
 
\href{https://www.desmos.com/calculator/1v6wvugh8g}.
}
 
\begin{onlineOnly}
    \begin{center}
\desmos{1v6wvugh8g}{900}{600}
\end{center}
\end{onlineOnly}

\end{question}


\begin{question}   \label{Q54B: Angles}
(a) Find an increasing function $\theta = a(t)$ that expresses the radian measure of the angle between the minute and hour hands of a clock in terms of the number of hours past 3:00pm. Measure the angle from the hour hand to the minute hand, taking the clockwise sense to be positive. %Note that for a function to be \emph{monotonic} means that it is always increasing or always decreasing.

(b) Use the clock above to estimate the first two times after 3:00pm when the minute and hour hands make an angle of $2\pi/3$ radians with each other.

(c) Use your function from part (a) to help find the first two times after 3:00pm when the minute and hour hands of a clock make an angle of $2\pi/3$ radians with each other. Round these times to the nearest second and compare them with your estimate from part (b).

\end{question}



\begin{question} \label{Q793:Angles}
You are due west of your friend on the spring eqiunox. Both of you are on the equator. 

On the spring equinox at the equator the sun rises due east at 6:00am local time, passes directly overhead at noon, and sets due west at 6:00pm local time. You both watch the sun set into the ocean. Being west of your friend, you see the sun set later. How much later? Well it depends on the distance between you and your friend.

(a) Experiment with the sliders $\phi$ and $n$ in the demonstration below.

(i) Explain what the demonstration illustrates.

(ii) Which of the two points represents you? How do you know?

(iii) When $\phi=0$, what is your local time? 

\pdfOnly{
Access Desmos interactives through the online version of this text at
 
\href{https://www.desmos.com/calculator/jmslo0laqz}.
}
 
\begin{onlineOnly}
    \begin{center}
\desmos{jmslo0laqz}{900}{600}
\end{center}
\end{onlineOnly}



(a) Find a function 
\[
   T = f(s) \, , \, 0\leq s \leq 1000 ,
\]
that expresses the time difference in the observed sunsets (measured in minutes) in terms of the distance (measured in miles) between you and your friend. Take the earth to be a perfect sphere of radius $R$ miles.

(b) Suppose you are $100$ miles west of your friend and that the radius of the earth is $3960$ miles. How much later do you see the sun set? Round your answer to the nearest second.

(c) This problem suggests a way to measure the radius of the earth. How?

\end{question}



\section{Thinking Proportionately, Part 1}

\begin{question}  \label{Q4P:Angles}
Use the radian protractor below to help you find approximate answers to the following questions. Keep the radius of the protractor at $7$ cm.

\pskip

(a) You stand at the origin facing the direction of the postive $x$ axis. Estimate your coordinates after you turn counterclockwise through an angle of 4 radians and then walk 100 meters directly away from the origin.

(b) You start from the point with coordinates $(100,0)$ meters and walk counterclockwise around the circle of radius $100$ meters centered at the origin. Estimate your coordinates after you have walked $230$ meters.

(c) Use your estimate from part (b) to approximate your coordinates if you had walked $230$ meters clockwise around the same circle.

(d) Repeat part (b) if you walk $2$ meters instead of $230$ meters.

(e) A ferris wheel has a radius of $50$ feet and its center is $60$ feet above the ground. The wheel rotates at a constant rate, making one revolution every two minutes.

(i) Approximate your height above the ground $10$ seconds after you get on the ferris wheel. 

(ii) Approximate your height above the ground $70$ seconds after you get on the ferris wheel. 

(iii) Approximately how long after you board are you $100$ feet above the ground for the first time.

(iv) Use your approximation from part (iii) to approximate the second time you are $100$ feet above the ground.


\begin{exploration}
\pdfOnly{
Access Desmos interactives through the online version of this text at
 
\href{https://www.desmos.com/calculator/kdakzcloqr}.
}
 
\begin{onlineOnly}
    \begin{center}
\desmos{kdakzcloqr}{900}{600}
\end{center}
\end{onlineOnly}
\end{exploration}

\end{question}




\section{Running in Circles}
\begin{exploration}
\pdfOnly{
Access Desmos interactives through the online version of this text at
 
\href{https://www.desmos.com/calculator/mjs5mwp827}.
}
 
\begin{onlineOnly}
    \begin{center}
\desmos{mjs5mwp827}{900}{600}
\end{center}
\end{onlineOnly}


\begin{question} \label{Q72:Angles}
Which of the points above is rotating about $O$ at the greatest rate? Explain?
\begin{multipleChoice}  
\choice{$A$}  
\choice{$B$}  
\choice{$C$}  
\choice[correct]{They are all rotating at the same rate.}
\end{multipleChoice}  
\end{question}

\begin{question} \label{Q72:Angles}
Which of the points above has the greatest speed? Explain.
\begin{multipleChoice}  
\choice{$A$}  
\choice{$B$}  
\choice[correct]{$C$}  
\choice{They are all have the same speed.}
\end{multipleChoice}  
\end{question}


\end{exploration}




\begin{question}  \label{Q255: Angles}
At what speed are people at the equator moving due to the rotation of the earth about its axis? Assume the earth to be a perfect sphere of radius $3960$ miles.
\end{question}

\begin{question}  \label{Q2675: Angles}
You run around
\end{question}


\begin{exploration}  \label{Q57:Radians}

\pdfOnly{
Access Desmos interactives through the online version of this text at
 
\href{https://www.desmos.com/calculator/2dlgnpeqsm}.
}
 
\begin{onlineOnly}
    \begin{center}
\geogebra{cgpad5nc}{900}{600}
\end{center}
\end{onlineOnly}
\end{exploration}



\begin{exploration}\label{exp:angles1}
Suppose you stand in place at point $A$ and face point $B$ in the desmos activity below. 

(a) Now turn counterclockwise until you face $C$ for the first time. Use the radian protractor to approximate the angle through which you turn during this time. Show a screenshot.

(b) Now turn back to face $B$ again (sitll standing at $A$). Then turn clockwise and stop when you face $C$ for the fifth time. Through approximately what angle do you turn during this time? 

(c) Turn back to face $B$ again (still standing at $A$). Then turn counterclockwise at the constant rate of $0.45$ rad/sec until you face $C$ for the tenth time. Approximately how long does this take? 


\pdfOnly{
Access Desmos interactives through the online version of this text at
 
\href{https://www.desmos.com/calculator/mwh3wwfrqv}.
}
 
\begin{onlineOnly}
    \begin{center}
\desmos{mwh3wwfrqv}{900}{600}
\end{center}
\end{onlineOnly}
\end{exploration}







\begin{exploration}\label{exp:angles2}
Two wheels with the same radius rotate at the same rate. One turns about its center, the other rolls on a road without slipping.

(a) Compare the speeds of points $P$ and $C$ below. Explain your reasoning. Include a screenshot to help with your explanation.

(b) If the wheels turn at the constant rate of $0.45$ rad/s and the wheel's have a radius of 10 cm, find the speed of the center of the rolling wheel. Do not use a  formula. Instead, do this logically and explain your reasoning. 

(c) Turn on the folder in Line 19 by clicking on the circle at the left of the line and play the animation. Then sketch graphs (on the same coordinate system) that show the speeds of points $P^\prime$ and $C$ as functions of time. No need for scales on the axes, but pay particular attention to the ratios of the speeds in drawing  your graphs.


\pdfOnly{
Access Desmos interactives through the online version of this text at
 
\href{https://www.desmos.com/calculator/cwaasmghns}.
}
 
\begin{onlineOnly}
    \begin{center}
\desmos{cwaasmghns}{900}{600}
\end{center}
\end{onlineOnly}
\end{exploration}



\begin{exploration}\label{exp:angles2}
Experiment with the desmos activity below to determine an expression in $a$ and $b$ that gives the radian measure of the angle through which the rolling circle turns during the time it takes point $P$ to return to its intial position. Try to explain the logic behind the expression.


\pdfOnly{
Access Desmos interactives through the online version of this text at
 
\href{https://www.desmos.com/calculator/g8rjobapit}.
}
 
\begin{onlineOnly}
    \begin{center}
\desmos{g8rjobapit}{900}{600}
\end{center}
\end{onlineOnly}
\end{exploration}



\begin{example} \label{Ex1:Angles}
At noon an ant is in the second quadrant at the point on the circle $x^2+y^2=400$ (the coordinates $(x,y)$ measured in centimeters) that is 45 cm from $(20,0)$, as measured along the circle. It crawls counterclockwise around the circle at a constant speed of $5$ cm/sec.

At noon a beetle is in the first quardrant on the same circle at a point that is 10 cm from $(20,0)$, this distance also measured along the circle. It crawls clockwise around the circle at a constant speed of $8$ cm/sec.

The insects crawl for 100 seconds.

\pskip

(a) Find a function
\[
    \theta = a(t) , 0\leq t \leq 100,
\]
that expresses the angle (in radians) from the postive $x$-axis to the segment $OA$ running from the origin to the ant in terms of the number of seconds past noon. Take the counterclockwise sense of rotation to correspond to a postive angle.

\pskip

(b) Find a function
\[
    \theta = b(t) , 0\leq t \leq 100,
\]
that expresses the angle (in radians) from the postive $x$-axis to the segment $OB$ running from the origin to the beetle in terms of the number of seconds past noon. Take the counterclockwise sense of rotation to correspond to a postive angle.

\pskip


(c) Use your functions from (a) and (b) to determine the first time the insects pass each other. Give an exact time, then an approximation to the nearest tenth of a second.

\pskip

(d) Use your functions from (a) and (b) to find an expression for the $n$th time the insects pass each other. 

\pskip

(e) Find a function $c(t)$ that counts the number of times the insects pass each other during the first $t$ seconds of their motions.

\pskip

(f) When is the last time the insects pass each other? Give an exact time, then an approximation to the nearest tenth of a second.

\begin{exploration}\label{exp:angles2}
Use the demonstration below to check your work.


\pdfOnly{
Access Desmos interactives through the online version of this text at
 
\href{https://www.desmos.com/calculator/btopul9ji9}.
}
 
\begin{onlineOnly}
    \begin{center}
\desmos{btopul9ji9}{900}{600}
\end{center}
\end{onlineOnly}
\end{exploration}


\end{example}


\begin{example} \label{Ex2:Angles}
Make up and solve your own version of Example 3. Use the exploration below to check your work, but not to help with the computations in any way. 


\begin{exploration}\label{exp:angles2}

\pdfOnly{
Access Desmos interactives through the online version of this text at
 
\href{https://www.desmos.com/calculator/632iayap2b}.
}
 
\begin{onlineOnly}
    \begin{center}
\desmos{632iayap2b}{900}{600}
\end{center}
\end{onlineOnly}
\end{exploration}


\end{example}


\end{document}
