\documentclass{ximera}
\title{Motion Quiz}

\newcommand{\pskip}{\vskip 0.1 in}

\begin{document}
\begin{abstract}
An introduction to parameterizing uniform cirular motion and motion with a constant speed.
\end{abstract}
\maketitle

\section{Uniform Circlular Motion}

\begin{question} \label{Qdbyp44:Motiondfdf}
Between 12:00pm and 12:05pm a beetle crawls clockwise with a constant speed around a circle of radius $40$cm centered at the origin. At time $t=10$ seconds past noon, the beetle passes the point $Q$ in the second quadrant $70$cm from $A(40,0)$ as measured along the circle. At 12:01pm, without ever having completed a full lap, the beetle passes the point $R$ in the fourth quardrant $20$cm from $A$, the distance also measured along the circle. 

Find functions
\[
   x = f(t) \text{ and } y=g(t) , \text{ for } 0\leq t \leq 300 ,
\]
that express the beetle's coordinates (measured in centimeters) in terms of the number of seconds past noon. 

Start by graphing the function
\[
    \theta = a(t) \, , \, 0\leq t \leq 300 
\] 
that expresses the ant's polar angle (measured in radians) in terms of the number of minutes past noon. Then find an expression for this function.


\pdfOnly{
Access Desmos interactives through the online version of this text at
 
\href{https://www.desmos.com/calculator/3iah0ue1ts}.
}
 
\begin{onlineOnly}
    \begin{center}
\desmos{3iah0ue1ts}{900}{600}
\end{center}
\end{onlineOnly}

\end{question}

\begin{question} \label{Qgpnd0:Motiondfdsfeee}
\begin{enumerate}
\item Find functions
\[
   x = f_1(t) \text{ and } y=g_1(t)
\]
that express the coordinates (in inches) of the tip of the minute hand on a clock in terms of the number of hours after 12:00pm. Assume the minute hand is 5 inches long and that the clock is centered at the origin as shown in the demonstration below. Use the cosine function for the $x$-coordinate and the sine function for the $y$-coordinate.

\item Find functions
\[
   x = f_2(t) \text{ and } y=g_2(t)
\]
that express the coordinates (in inches) of the tip of the hourhand on a clock in terms of the number of hours after 12:00pm. Assume the hour hand is 3 inches long.

\item Enter your coordinate functions in the demonstration below (correct the functions there instead of adding more lines). Then play the slider $u$ as a check.


\pdfOnly{
Access Desmos interactives through the online version of this text at
 
\href{https://www.desmos.com/calculator/ogxka9nfmi}.
}
 
\begin{onlineOnly}
    \begin{center}
\desmos{ogxka9nfmi}{900}{600}
\end{center}
\end{onlineOnly}

\end{enumerate}

\end{question}

\begin{question} \label{Q43fgde5r4tggfr}
An ant and a beetle crawl counterclockwise around circles centered at the origin with respective radii $20$ meters and $30$ meters.  They pass the positive $x$-axis simultaneously at noon.  The ant crawls at a speed of $5$ meters/min, the beetle at a speed of $4$ meters/min. 

\begin{onlineOnly}
    \begin{center}
\desmos{ie75j2a9ky}{900}{600}  
\end{center}
\end{onlineOnly}

\href{https://www.desmos.com/calculator/ie75j2a9ky}{142: Concentric Circles 4}

\begin{enumerate}

\item Find possible functions
\[
   \theta = a_1(t) \text{ and } \theta = a_2(t) \, , \, t\geq 0 ,
\]
that express the respective polar angles (in radians) of the ant and beetle in terms of the number of minutes since noon.

\item Use the demonstration above to approximate when and where the insects pass each other for the first time .

\item Use your polar angle functions from part (a) to find the exact time when they pass each other for the first time.

%\item Find the exact time (no calculator) when the insects pass each other for the first time.

\item Find their exact coordinates (no calculator) when they pass each other for the first time.

\item What about for the second time? For the tenth time?
\end{enumerate}
\end{question}

\begin{question}  \label{Q9dfdfhdferrefe94tf4}

Between noon and 12:16 pm an ant and a beetle crawl around the circle of radius $40$ meters centered at the origin.  

The graphs of the functions
\[
  \theta = a(t) \, , \, 0\leq t \leq 16
\]
and
\[
   \theta = b(t) \, , \, 0\leq t \leq 16
\]
expressing their respective polar angles (in radians) in terms of the number of minutes past noon are shown below.

\begin{onlineOnly}
    \begin{center}
\desmos{dilqmqqrt4}{900}{600}
\end{center}
\end{onlineOnly}

\href{https://www.desmos.com/calculator/dilqmqqrt4}{142: Ant and Beetle Polar Angles}


\begin{enumerate}

\item Use the graphs above and the slider $u$ to approximate the first time when the bugs pass one another. 

\item Use algebra to find the \emph{exact} time when the bugs first pass one another.

\item Find \emph{where} the bugs pass one another for the first time. Give the exact coordinates..

\end{enumerate}

\end{question}




\end{document}
