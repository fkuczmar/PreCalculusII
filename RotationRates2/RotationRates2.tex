\documentclass{ximera}
\title{Rotating Gears}

\newcommand{\pskip}{\vskip 0.1 in}

\begin{document}
\begin{abstract}
Rotating gears.
\end{abstract}
\maketitle

%\section{Rotating Gears}

\begin{question}  \label{Q3242:Angles}
Experiment with the desmos demonstration below. 

Given the radii $r_1$, $r_2$ cm of the blue and red wheels respectively, and the rotation rate $\omega_1$ rad/sec of the blue gear, find an expression for the rotation rate (say $\omega_2$ rad/sec) of the red gear. Explain your logic.

\pdfOnly{
Access Desmos interactives through the online version of this text at
 
\href{https://www.desmos.com/calculator/4vuc86pxwo}.
}
 
\begin{onlineOnly}
    \begin{center}
\desmos{4vuc86pxwo}{900}{600}
\end{center}
\end{onlineOnly}


\begin{hint}
The key point is that the gears are assumed to roll on each other without slipping. Because of this, the speed of two points on the outer rims of the two gears are equal (to see this compare the distances traveled by the two black tick marks on the gears over some time interval). These speeds (in cm/sec) are 
\[
   v_1 = \omega_1 r_1
\]
for the blue wheel, and
\[
   v_2 = \omega_2 r_2
\]
for the red wheel.

Now since the two speeds are equal,
\[
   \omega_1 r_1 = \omega_2 r_2
\]
and so the rotation rate of the red wheel (in rad/sec) is
\[
  \omega_2 = \left( \frac{r_1}{r_2} \right) \omega_1 .
\]
\end{hint}

\end{question}




\begin{question}  \label{Q1764:Angles}
Experiment with the desmos demonstration below where the black gear is welded to the red. The black gear pushes a chain.  

Given the radii of the three gears and the rotation rate of the blue gear, how can you determine the speed of the chain? Explain your logic.

\pdfOnly{
Access Desmos interactives through the online version of this text at
 
\href{https://www.desmos.com/calculator/6nxze8ikhg}.
}
 
\begin{onlineOnly}
    \begin{center}
\desmos{6nxze8ikhg}{900}{600}
\end{center}
\end{onlineOnly}
\end{question}







\begin{question}  \label{Q323342:Angles}
The wheels of a car have a diameter of two feet. A gear mechanism with four gears connects one of the car's wheels to the wheel that spins the tenth of a mile reading on the odometer. %When you mow the lawn, the spindle turns four times as fast as the wheels. 

(a) What gear ratio does this?

(b) Design such a gear train.

(c) Investigate how the other digits of the odometer reading turn.
\end{question}


\begin{question}    \label{Q850:Angles}
The photo below shows a mechanical mower. The wheels have a radius $AQ$ of 9 inches. The distance $AB$ between the center of each wheel and the center of the spindle is 2.25 inches. The radius of the blade wheel is also approximately 2.25 inches.

\pdfOnly{
Access Desmos interactives through the online version of this text at
 
\href{https://www.desmos.com/calculator/ryxfilsnef}.
}
 
\begin{onlineOnly}
    \begin{center}
\desmos{ryxfilsnef}{900}{600}
\end{center}
\end{onlineOnly}

The diagram below shows the inner workings of the mower.
\begin{onlineOnly}
\begin{center}
\desmos{lsxarkvo6r}{900}{600}
\end{center}
\end{onlineOnly}

\href{https://www.desmos.com/calculator/lsxarkvo6r}{142: Mechanical Mower}

The animation below shows the  mower in motion.

\begin{onlineOnly}
\begin{center}
\desmos{3yjpzc2s1s}{900}{600}
\end{center}
\end{onlineOnly}

\href{https://www.desmos.com/calculator/3yjpzc2s1s}{142: Push Mower}



\begin{enumerate}
\item Relate the walking speed to the speed of the cutting edge of the blade as observed in the reference frame of the moving mower. Start by defining the appropriate parameters.
\end{enumerate}

\end{question}

\end{document}