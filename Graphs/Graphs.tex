\documentclass{ximera}
\title{Graphing the Cosine and Sine Functions}

\newcommand{\pskip}{\vskip 0.1 in}

\begin{document}
\begin{abstract}
Graphing.
\end{abstract}
\maketitle


To graph the sine and cosine functions, it helps to work on the unit circle. Then we can interpret the polar angle (ie. the input to the functions $\cos\theta$ and $\sin\theta$) 
\[
   \theta = s/r = s
\]
as a signed arclength measured counterclockwise around the unit circle from $(1,0)$. And we can also interpret the outputs 
\[
   \cos\theta = x/r = x
\]
and
\[
   \sin\theta = y/r = y
\]
as lengths. But it is  important to keep in mind that both the input and outputs to these functions are really ratios of lengths, and hence dimensionless.

Work through the activity below, paying careful attention to the input and outputs as they appear on the graphs and on the unit circle. Notice also the labels on the axis for the unit circle and on the coordinate axes for the graphs.

\begin{exploration}
\pdfOnly{
Access Desmos interactives through the online version of this text at
 
\href{https://www.desmos.com/calculator/7glpwswts1}.
}
 
\begin{onlineOnly}
    \begin{center}
\desmos{7glpwswts1}{900}{600}
\end{center}
\end{onlineOnly}
\end{exploration}



\begin{example}
\begin{itemize}

\item{Parameterize each of the following motions.}

\item{Then use the {\bf description} of each motion to graph the coordinate functions.}

\item{Then check your graphs are correct in the demonstration below.}

\end{itemize}

(a) A counterclockwise motion about the circle of radius 2 cm centered at the origin that moves at a constant speed of 2 cm/sec and passes the point $(2,0)$ at time $t=0$ seconds.

(b) A counterclockwise motion about the circle of radius 2 cm centered at the origin that moves at a constant speed of 4 cm/sec and passes the point $(2,0)$ at time $t=0$ seconds.

(c) A counterclockwise motion about the circle of radius 2 cm centered at the origin that rotates about the origin at a constant rate of $\pi/6$ rad/sec and passes the point $(2,0)$ at time $t=0$ seconds.

(d) A counterclockwise motion about the circle of radius 2 cm centered at the origin that rotates about the origin at a constant rate of $\pi/6$ rad/sec and passes the point $(2,0)$ at time $t=1$ second.

\begin{exploration}
\pdfOnly{
Access Desmos interactives through the online version of this text at
 
\href{https://www.desmos.com/calculator/hijhhpu8q5}.
}
 
\begin{onlineOnly}
    \begin{center}
\desmos{hijhhpu8q5}{900}{600}
\end{center}
\end{onlineOnly}
\end{exploration}


\end{example}









\end{document}