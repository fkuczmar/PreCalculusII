\documentclass{ximera}
\title{Graphing the Cosine and Sine Functions}

\newcommand{\pskip}{\vskip 0.1 in}

\begin{document}
\begin{abstract}
Graphing.
\end{abstract}
\maketitle


%To graph the sine and cosine functions, it helps to work on the unit circle. Then we can interpret the polar angle (ie. the input to the functions $\cos\theta$ and $\sin\theta$) 
%\[
%   \theta = s/r = s
%\]
%as a signed arclength measured counterclockwise around the unit circle from $(1,0)$. And we can also interpret the outputs 
%\[
%   \cos\theta = x/r = x
%\]
%and
%\[
%   \sin\theta = y/r = y
%\]
%as lengths. But it is  important to keep in mind that both the input and outputs to these functions are really ratios of lengths, and hence dimensionless.

Consider a counterclockwise motion at a constant speed of 1 m/sec about the circle of radius 1 meters centered at the origin that passes the point $(1,0)$ at time $t=0$ seconds.

\begin{question}
(a) At what rate does the motion turn about the origin?

(b) Find a function $\theta = a(t)$ that expresses the polar angle (in radians) in terms of time (measured in seconds).
\end{question}

Use the definitions of the sine and cosine functions and your answer to Question (b) above to parameterize the motion. Then graph the coordinate functions.

The demonstration below gives the option of showing the graph of the $x$ or the $y$ coordinate functions for the above motion (plotted in the $tx$ or $ty$ planes). It also shows the path of the motion (plotted in the $xy$-pane). Follow the directions to see the motion along with the graphs of the coordinate functions. Note in particular how the orientation of the $xy$-coordinate system changes when you change the coordinate function. Why do you suppose that it is?

\begin{exploration}
\pdfOnly{
Access Desmos interactives through the online version of this text at
 
\href{https://www.desmos.com/calculator/6dptkb49vf}.
}
 
\begin{onlineOnly}
    \begin{center}
\desmos{6dptkb49vf}{900}{600}
\end{center}
\end{onlineOnly}
\end{exploration}



\begin{example}     \label{Ex1:Graphs}
\begin{itemize}

\item{Parameterize each of the following motions.}

\item{Compute the period of the motion. Show all units in the computation. The period is the time it takes the motion to make one complete revolution around its path.}

\item{Use the {\bf description} of each motion to graph the coordinate functions by hand. Label the coordinates of 5 points on each graph at time intervals equal to 1/4 of the period.}

\item{Check your graphs are correct in the demonstration below. Adjust the sliders as needed.}

\item{Describe, in the correct order, the transformations that take the graphs of the parent functions $x=\cos t$ or $y=\sin t$ to the graphs of the coordinate functions.}

\end{itemize}

(a) A counterclockwise motion about the circle of radius 2 m centered at the origin that moves at a constant speed of 2 m/sec and passes the point $(2,0)$ at time $t=0$ seconds.

(b) A counterclockwise motion about the circle of radius 2 m centered at the origin that moves at a constant speed of 4 m/sec and passes the point $(2,0)$ at time $t=0$ seconds.

(c) A counterclockwise motion about the circle of radius 2 m centered at the origin that rotates about the origin at a constant rate of $\pi/6$ rad/sec and passes the point $(2,0)$ at time $t=0$ seconds.

(d) A counterclockwise motion about the circle of radius 2 m centered at the origin that rotates about the origin at a constant rate of $\pi/6$ rad/sec and passes the point $(2,0)$ at time $t=1$ second.

(e) A counterclockwise motion about the circle of radius 2 m centered at the origin with a constant speed. At time $t=1$ second the motion passes the point $3\pi/8$ m from $(2,0)$ as measured counteclcockwise along the path. Four seconds later it passes the point $5\pi/8$ m from $(2,0)$ (also measured counteclcockwise along the path).

\begin{exploration}
\pdfOnly{
Access Desmos interactives through the online version of this text at
 
\href{https://www.desmos.com/calculator/lwaitp4xof}.
}
 
\begin{onlineOnly}
    \begin{center}
\desmos{lwaitp4xof}{900}{600}
\end{center}
\end{onlineOnly}
\end{exploration}
\end{example}






\begin{example} \label{Ex2:Graphs}
The graphs below show the coordinate functions for a motion around a circle at a constant speed. 

\pskip


(a) Make an assumption about which graph is for the $x$-coordinate function and which is for the $y$-coordinate. Then describe the motion precisely by giving the center and radius of the path, the rotation rate about the center, the sense of rotation, and the position at at time $t=0$.

(b) Use your description and the graphs to find expressions for each coordinate function. Use the cosine function for the $x$-coordinate and the sine function for the $y$-coordinate.

(c) Reverse  your assumption in part (a) and repeat parts (a) and (b). Be sure to still use the cosine function for the $x$-coordinate and the sine function for the $y$-coordinate.


\begin{exploration}
\pdfOnly{
Access Desmos interactives through the online version of this text at
 
\href{https://www.desmos.com/calculator/a5rgc9gyaa}.
}
 
\begin{onlineOnly}
    \begin{center}
\desmos{a5rgc9gyaa}{900}{600}
\end{center}
\end{onlineOnly}
\end{exploration}

\end{example}



\begin{example} \label{Ex3:Graphs}
The graphs below show the coordinate functions for a motion around a circle at a constant speed. 

\pskip


(a) Make an assumption about which graph is for the $x$-coordinate function and which is for the $y$-coordinate. Then describe the motion precisely by giving the center and radius of the path, the rotation rate about the center, the sense of rotation, and the position at at time $t=0$.

(b) Use your description and the graphs to find expressions for each coordinate function. Use the cosine function for the $x$-coordinate and the sine function for the $y$-coordinate.

(c) Reverse  your assumption in part (a) and repeat parts (a) and (b). Be sure to still use the cosine function for the $x$-coordinate and the sine function for the $y$-coordinate.


\begin{exploration}
\pdfOnly{
Access Desmos interactives through the online version of this text at
 
\href{https://www.desmos.com/calculator/7c5fetslgv}.
}
 
\begin{onlineOnly}
    \begin{center}
\desmos{7c5fetslgv}{900}{600}
\end{center}
\end{onlineOnly}
\end{exploration}

\end{example}


\begin{example} \label{Ex4:Graphs}
Suppose that over the course of a 24-hour period, from midnight October 29 to midnight October 30, the depth of the water at the Edmonds Pier is a sinusoidal function of time. Suppose further that a high tide of 21 feet occurs at 2:00am and the following low tide of 5 feet occurs six hours later. 

Our aim is to find a function
\[
    h = f(t) , 0\leq t \leq 24, 
\]
that expresses the depth of the water (in feet) in terms of the number of hours past midnight, October 29. 

\pskip

Note that to say ``sinusoidal function'' means that the graph of $f$ has the shape of some sine curve. But the graphs of the sine and cosine functions have the same shape, so it is ok to express $f(t)$ in terms of the cosine function, and we will do just that.


%(a) Describe a uniform circular motion with an $x$-coordinate function that matches the graph below. Be sure to give the center and radius of the path, the rotation rate about the center, the sense of rotation, and the $x$-coordinate at some particular time. 

(a) Use the information above to sketch by hand a graph of the function $f$. Label the axes with the appropriate variable names and units. Label the coordinates of two key points on the graph.

(b) Compute the mean depth of the water over the 24 hours and the maximum deviation of the depth from this mean. Include all units in your computations. Draw on your graph the horizontal line that shows the average depth. Label this line with its equation. Draw also a vertical line that shows the the maximum deviation from the mean.

(c) Use the graph to find the period of oscillation. Then compute the rotation rate of a uniform circular motion that generates the sinusoidal variation in the depth of the water. Include all units.

(d Use parts (a)-(c) to find an expression for the function 
\[
   h = f(t) ,  0\leq t \leq 24, 
\]
that gives the depth of the water (in feet) at time $t$ hours past midnight. Use the cosine function.

(e) Check that your function is correct by using the information given in the problem.

(f) Use your function to estimate the depth of the water at midnight, October 29 to the nearest tenth of a foot.


\end{example}


\begin{example} \label{Ex5:Graphs}
In desribing the transformations that take the graph of the function $f(t) = \sin t$ to the function
\[
  g(t) = 20 + 4 \sin \left( \frac{1}{5}(t-3) \right) 
\]
a student says the following:

(a) Translate the graph of $h=f(t)$ 20 feet vertically in the direction of the postive $h$ axis and $3$ seconds horizontally in the direction of the postiive $t$-axis.

(b) Then dilate vertically by a factor of 4 about the $t$-axis and horizontally by a factor of $5$ about the $h$ aixs.
\end{example}

\begin{itemize}
\item{What is wrong with the above description?}

\item{Sketch by hand a graph of the function $h=w(t)$ that the student described.}

\item{Find an expression for $w(t)$.}

\item{Describe the transformations that take the graph of $h=g(t)$ to the graph of $h=w(t)$.}

\end{itemize}




\end{document}