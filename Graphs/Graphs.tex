\documentclass{ximera}
\title{Graphing the Cosine and Sine Functions}

\newcommand{\pskip}{\vskip 0.1 in}

\begin{document}
\begin{abstract}
Graphing.
\end{abstract}
\maketitle


%To graph the sine and cosine functions, it helps to work on the unit circle. Then we can interpret the polar angle (ie. the input to the functions $\cos\theta$ and $\sin\theta$) 
%\[
%   \theta = s/r = s
%\]
%as a signed arclength measured counterclockwise around the unit circle from $(1,0)$. And we can also interpret the outputs 
%\[
%   \cos\theta = x/r = x
%\]
%and
%\[
%   \sin\theta = y/r = y
%\]
%as lengths. But it is  important to keep in mind that both the input and outputs to these functions are really ratios of lengths, and hence dimensionless.

Consider a counterclockwise motion at a constant speed of 1 m/sec about the circle of radius 1 meters centered at the origin that passes the point $(1,0)$ at time $t=0$ seconds.

\begin{question}
(a) At what rate does the motion turn about the origin?

(b) Find a function $\theta = a(t)$ that expresses the polar angle (in radians) in terms of time (measured in seconds).
\end{question}

Use the definitions of the sine and cosine functions and your answer to Question (b) above to parameterize the motion. Then graph the coordinate functions.

The demonstration below gives the option of showing the graph of the $x$ or the $y$ coordinate functions for the above motion (plotted in the $tx$ or $ty$ planes). It also shows the path of the motion (plotted in the $xy$-pane). Follow the directions to see the motion along with the graphs of the coordinate functions. Note in particular how the orientation of the $xy$-coordinate system changes when you change the coordinate function. Why do you suppose that it is?

\begin{exploration}
\pdfOnly{
Access Desmos interactives through the online version of this text at
 
\href{https://www.desmos.com/calculator/6dptkb49vf}.
}
 
\begin{onlineOnly}
    \begin{center}
\desmos{6dptkb49vf}{900}{600}
\end{center}
\end{onlineOnly}
\end{exploration}



\begin{example}     \label{Ex1:Graphs}
\begin{itemize}

\item{Parameterize each of the following motions.}

\item{Compute the period of the motion. Show all units in the computation. The period is the time it takes the motion to make one complete revolution around its path.}

\item{Use the {\bf description} of each motion to graph the coordinate functions by hand. Label the coordinates of 5 points on each graph at time intervals equal to 1/4 of the period.}

\item{Check your graphs are correct in the demonstration below. Adjust the sliders as needed.}

\item{Describe, in the correct order, the transformations that take the graphs of the parent functions $x=\cos t$ or $y=\sin t$ to the graphs of the coordinate functions.}

\end{itemize}

(a) A counterclockwise motion about the circle of radius 2 m centered at the origin that moves at a constant speed of 2 m/sec and passes the point $(2,0)$ at time $t=0$ seconds.

(b) A counterclockwise motion about the circle of radius 2 m centered at the origin that moves at a constant speed of 4 m/sec and passes the point $(2,0)$ at time $t=0$ seconds.

(c) A counterclockwise motion about the circle of radius 2 m centered at the origin that rotates about the origin at a constant rate of $\pi/6$ rad/sec and passes the point $(2,0)$ at time $t=0$ seconds.

(d) A counterclockwise motion about the circle of radius 2 m centered at the origin that rotates about the origin at a constant rate of $\pi/6$ rad/sec and passes the point $(2,0)$ at time $t=1$ second.

(e) A counterclockwise motion about the circle of radius 2 m centered at the origin with a constant speed. At time $t=1$ second the motion passes the point $3\pi/8$ m from $(2,0)$ as measured counteclcockwise along the path. Four seconds later it passes the point $5\pi/8$ m from $(2,0)$ (also measured counteclcockwise along the path).

\begin{exploration}
\pdfOnly{
Access Desmos interactives through the online version of this text at
 
\href{https://www.desmos.com/calculator/lwaitp4xof}.
}
 
\begin{onlineOnly}
    \begin{center}
\desmos{lwaitp4xof}{900}{600}
\end{center}
\end{onlineOnly}
\end{exploration}
\end{example}






\begin{example} \label{Ex2:Graphs}
The graphs below show the coordinate functions for a motion around a circle at a constant speed. 

\pskip

\begin{itemize}
\item{Make an assumption about which graph is for the $x$-coordinate function. Then describe the motion precisely by giving the center and radius of the path, the rotation rate about the center, the sense of rotation, and the position at at time $t=0$.}

\item{Use your description and the graphs to find expressions for each coordinate function.}

%\item{Reverse  your assumption about which graph is for the $x$-coordinate function and find new expressions for each coordinate function.}
\end{itemize}

\begin{exploration}
\pdfOnly{
Access Desmos interactives through the online version of this text at
 
\href{https://www.desmos.com/calculator/a5rgc9gyaa}.
}
 
\begin{onlineOnly}
    \begin{center}
\desmos{a5rgc9gyaa}{900}{600}
\end{center}
\end{onlineOnly}
\end{exploration}

\end{example}







\end{document}